\documentclass{article}
\usepackage[margin=1in]{geometry}

\usepackage{amsmath}
\usepackage{amssymb}
\usepackage{amsthm}
\usepackage{comment}
\usepackage{enumitem}
\usepackage{fancyhdr}
\usepackage{hyperref}
\usepackage{mathrsfs}
\usepackage{mathtools}
\usepackage{standalone}
\usepackage{tcolorbox}
\usepackage{tikz}
\usepackage{xcolor}

\usetikzlibrary{decorations.pathreplacing}
\tcbuselibrary{skins}

\DeclareMathOperator{\lcm}{lcm}
\DeclareMathOperator{\proj}{proj}
\DeclareMathOperator{\vspan}{span}
\DeclareMathOperator{\im}{im}
\DeclareMathOperator{\range}{range}
\DeclareMathOperator{\Diff}{Diff}
\DeclareMathOperator{\Int}{Int}
\DeclareMathOperator{\fcn}{fcn}
\DeclareMathOperator{\id}{id}

\newcommand{\N}{\ensuremath{\mathbb{N}}}
\newcommand{\Z}{\ensuremath{\mathbb{Z}}}
\newcommand{\Q}{\ensuremath{\mathbb{Q}}}
\newcommand{\R}{\ensuremath{\mathbb{R}}}
\newcommand{\C}{\ensuremath{\mathbb{C}}}
\newcommand{\F}{\ensuremath{\mathbb{F}}}
\newcommand{\lam}{\ensuremath{\lambda}}
\newcommand{\nab}{\ensuremath{\nabla}}
\newcommand{\eps}{\ensuremath{\varepsilon}}
\newcommand{\es}{\ensuremath{\varnothing}}

\newcommand{\abs}[1]{\ensuremath{\left\lvert #1 \right\rvert}}
\newcommand{\bigpar}[1]{\ensuremath{\left( #1 \right)}}
\newcommand{\norm}[1]{\ensuremath{\lVert #1\rVert}}
\newcommand{\set}[1]{\ensuremath{\left\{#1\right\}}}
\newcommand{\tuple}[1]{\ensuremath{\left\langle #1 \right\rangle}}
\newcommand{\floor}[1]{\ensuremath{\left\lfloor #1 \right\rfloor}}
\newcommand{\ceil}[1]{\ensuremath{\left\lceil #1 \right\rceil}}

\newcommand{\chapternum}{}
\newcommand{\ex}[1]{\noindent\textbf{Exercise \chapternum.{#1}.}}

\newcommand{\dx}[1]{\,\mathrm{d}#1}
\newcommand{\inv}{\ensuremath{^{-1}}}
\newcommand{\sm}{\setminus}
\newcommand{\sse}{\subseteq}
\newcommand{\ceq}{\coloneqq}

\definecolor{problemBackground}{RGB}{212,232,246}
\newenvironment{problem}[1]
  {
    \begin{tcolorbox}[
      boxrule=.5pt,
      titlerule=.5pt,
      sharp corners,
      colback=problemBackground
    ]
    \ifx &#1& \textbf{Problem. }
    \else \textbf{Problem #1.} \fi
  }
  {
    \end{tcolorbox}
  }

\definecolor{exampleBackground}{RGB}{255,249,248}
\definecolor{exampleAccent}{RGB}{158,60,14}
\newenvironment{example}[1]
  {
    \begin{tcolorbox}[
      boxrule=.5pt,
      sharp corners,
      colback=exampleBackground,
      colframe=exampleAccent,
    ]
    \color{exampleAccent}\textbf{Example.} \emph{#1}\color{black}
  }
  {
    \end{tcolorbox}
  }

\definecolor{theoremBackground}{RGB}{234,243,251}
\definecolor{theoremAccent}{RGB}{0,116,183}
\newenvironment{theorem}[1]
  {
    \begin{tcolorbox}[
      boxrule=.5pt,
      titlerule=.5pt,
      sharp corners,
      colback=theoremBackground,
      colframe=theoremAccent
    ]
      \color{theoremAccent}\textbf{Theorem --- }\emph{#1}\\\color{black}
  }
  {
    \end{tcolorbox}
  }

\definecolor{noteBackground}{RGB}{244,249,244}
\definecolor{noteAccent}{RGB}{34,139,34}
\newenvironment{note}[1]
  {
  \begin{tcolorbox}[
    enhanced,
    boxrule=0pt,
    frame hidden,
    sharp corners,
    colback=noteBackground,
    borderline west={3pt}{-1.5pt}{noteAccent}
    ]
    \ifx &#1& \color{noteAccent}\textbf{Note. }\color{black}
    \else \color{noteAccent}\textbf{Note (#1). }\color{black} \fi
    }
    {
  \end{tcolorbox}
  }

\definecolor{definitionBackground}{RGB}{246,246,246}
\newenvironment{definition}[1]
  {
    \begin{tcolorbox}[
      enhanced,
      boxrule=0pt,
      frame hidden,
      sharp corners,
      colback=definitionBackground,
      borderline west={3pt}{-1.5pt}{black}
    ]
    \textbf{Definition. }\emph{#1}\\
  }
  {
    \end{tcolorbox}
  }
\newenvironment{amatrix}[2]{
    \left[
      \begin{array}{*{#1}{c}|*{#2}c}
  }
  {
      \end{array}
    \right]
  }
\date{\the\year-\the\month-\the\day}
\author{Kyle Chui}


\fancyhf{}
\setlength{\headheight}{24pt}
\lhead{Math 61 \\Worksheet 4}
\rhead{Kyle Chui \\Page \thepage}
\pagestyle{fancy}
\pagenumbering{gobble}

\title{Worksheet 4}

\begin{document}
  \maketitle
  \newpage
  \pagenumbering{arabic}
  \begin{problem}{1}
    \begin{enumerate}[label=(\alph*)]
      \item Show that $\binom{m+n}{r} = \sum_{k=0}^r \binom{m}{k}\binom{n}{r-k}$. You can get this result by using the binomial theorem on $(x+y)^{m+n}=(x+y)^m(x+y)^n$ or with a counting argument (think of choosing $r$ things from $m+n$ things where you keep track of what you choose from the first $m$ things and from the last $n$ things).
      \item Show that $\sum_{k=0}^n \binom{n}{k}^2 = \binom{2n}{n}$.
    \end{enumerate}
  \end{problem}
  \begin{enumerate}[label=(\alph*)]
    \item 
    \begin{proof}
      Observe that choosing $r$ items from a set with $m+n$ items is the same as first choosing $k$ items from a set of $m$, and then choosing $r-k$ items from a set of $n$. We need a summation in order to capture all possible different pairs of subsets of size $k$ and $r-k$.
    \end{proof}
    \item 
    \begin{proof}
      Observe that the left hand side is the same as $\sum_{k=0}^n \binom{n}{k}\binom{n}{n-k}$ because $\binom{n}{k} = \binom{n}{n-k}$. The left hand side of this formula counts the number of ways to choose a $k$ element subset from a set of $n$ elements, and a $n-k$ element subset from another $n$ element subset. This is the same as choosing an $n$ element subset from a set of $2n$ elements, which is represented on the right hand side.
    \end{proof}
  \end{enumerate}
  \newpage
  \begin{problem}{2}
    Show that $n 2^{n-1} = \sum_{i=1}^{n}i \binom{n}{i}$. See if you can do this by using the binomial theorem and also with a counting argument (count the number of teams with a captain from a set $n$ people).
  \end{problem}
  \begin{proof}
    Observe that the left hand side is the number of ways to choose a team with a captain from $n$ people. We have $n$ choices for the captain, and then $2^{n-1}$ different subsets of the remaining $n-1$ people. The right hand side counts how many teams with a captain can be made with $1$ person, then $2$ people, then $3$ people, etc. There are $\binom{n}{i}$ sets of $i$ people from the total set of $n$. Then we choose any of those team members to be the captain, so we multiply by $i$. Thus they count the same thing and are equal.
  \end{proof}
  \begin{note}{Alternate Equation}
    If you first chose the captain, and then the other $i-1$ team members, you would have $\sum_{i=1}^{n}n\cdot \binom{n-1}{i-1}$.
  \end{note}
\end{document}
