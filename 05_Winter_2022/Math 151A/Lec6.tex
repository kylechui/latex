\documentclass[class=article, crop=false]{standalone}
% Import packages
\usepackage[margin=1in]{geometry}

\usepackage{amssymb, amsthm}
\usepackage{comment}
\usepackage{enumitem}
\usepackage{fancyhdr}
\usepackage{hyperref}
\usepackage{import}
\usepackage{mathrsfs, mathtools}
\usepackage{multicol}
\usepackage{pdfpages}
\usepackage{standalone}
\usepackage{transparent}
\usepackage{xcolor}

\usepackage{minted}
\usepackage[many]{tcolorbox}
\tcbuselibrary{minted}

% Declare math operators
\DeclareMathOperator{\lcm}{lcm}
\DeclareMathOperator{\proj}{proj}
\DeclareMathOperator{\vspan}{span}
\DeclareMathOperator{\im}{im}
\DeclareMathOperator{\Diff}{Diff}
\DeclareMathOperator{\Int}{Int}
\DeclareMathOperator{\fcn}{fcn}
\DeclareMathOperator{\id}{id}
\DeclareMathOperator{\rank}{rank}
\DeclareMathOperator{\tr}{tr}
\DeclareMathOperator{\dive}{div}
\DeclareMathOperator{\row}{row}
\DeclareMathOperator{\col}{col}
\DeclareMathOperator{\dom}{dom}
\DeclareMathOperator{\ran}{ran}
\DeclareMathOperator{\Dom}{Dom}
\DeclareMathOperator{\Ran}{Ran}
\DeclareMathOperator{\fl}{fl}
% Macros for letters/variables
\newcommand\ttilde{\kern -.15em\lower .7ex\hbox{\~{}}\kern .04em}
\newcommand{\N}{\ensuremath{\mathbb{N}}}
\newcommand{\Z}{\ensuremath{\mathbb{Z}}}
\newcommand{\Q}{\ensuremath{\mathbb{Q}}}
\newcommand{\R}{\ensuremath{\mathbb{R}}}
\newcommand{\C}{\ensuremath{\mathbb{C}}}
\newcommand{\F}{\ensuremath{\mathbb{F}}}
\newcommand{\M}{\ensuremath{\mathbb{M}}}
\renewcommand{\P}{\ensuremath{\mathbb{P}}}
\newcommand{\lam}{\ensuremath{\lambda}}
\newcommand{\nab}{\ensuremath{\nabla}}
\newcommand{\eps}{\ensuremath{\varepsilon}}
\newcommand{\es}{\ensuremath{\varnothing}}
% Macros for math symbols
\newcommand{\dx}[1]{\,\mathrm{d}#1}
\newcommand{\inv}{\ensuremath{^{-1}}}
\newcommand{\sm}{\setminus}
\newcommand{\sse}{\subseteq}
\newcommand{\ceq}{\coloneqq}
% Macros for pairs of math symbols
\newcommand{\abs}[1]{\ensuremath{\left\lvert #1 \right\rvert}}
\newcommand{\paren}[1]{\ensuremath{\left( #1 \right)}}
\newcommand{\norm}[1]{\ensuremath{\left\lVert #1\right\rVert}}
\newcommand{\set}[1]{\ensuremath{\left\{#1\right\}}}
\newcommand{\tup}[1]{\ensuremath{\left\langle #1 \right\rangle}}
\newcommand{\floor}[1]{\ensuremath{\left\lfloor #1 \right\rfloor}}
\newcommand{\ceil}[1]{\ensuremath{\left\lceil #1 \right\rceil}}
\newcommand{\eclass}[1]{\ensuremath{\left[ #1 \right]}}

\newcommand{\chapternum}{}
\newcommand{\ex}[1]{\noindent\textbf{Exercise \chapternum.{#1}.}}

\newcommand{\tsub}[1]{\textsubscript{#1}}
\newcommand{\tsup}[1]{\textsuperscript{#1}}

\renewcommand{\choose}[2]{\ensuremath{\phantom{}_{#1}C_{#2}}}
\newcommand{\perm}[2]{\ensuremath{\phantom{}_{#1}P_{#2}}}

% Include figures
\newcommand{\incfig}[2][1]{%
    \def\svgwidth{#1\columnwidth}
    \import{./figures/}{#2.pdf_tex}
}

\definecolor{problemBackground}{RGB}{212,232,246}
\newenvironment{problem}[1]
  {
    \begin{tcolorbox}[
      boxrule=.5pt,
      titlerule=.5pt,
      sharp corners,
      colback=problemBackground,
      breakable
    ]
    \ifx &#1& \textbf{Problem. }
    \else \textbf{Problem #1.} \fi
  }
  {
    \end{tcolorbox}
  }
\definecolor{exampleBackground}{RGB}{255,249,248}
\definecolor{exampleAccent}{RGB}{158,60,14}
\newenvironment{example}[1]
  {
    \begin{tcolorbox}[
      boxrule=.5pt,
      sharp corners,
      colback=exampleBackground,
      colframe=exampleAccent,
    ]
    \color{exampleAccent}\textbf{Example.} \emph{#1}\color{black}
  }
  {
    \end{tcolorbox}
  }
\definecolor{theoremBackground}{RGB}{234,243,251}
\definecolor{theoremAccent}{RGB}{0,116,183}
\newenvironment{theorem}[1]
  {
    \begin{tcolorbox}[
      boxrule=.5pt,
      titlerule=.5pt,
      sharp corners,
      colback=theoremBackground,
      colframe=theoremAccent,
      breakable
    ]
      % \color{theoremAccent}\textbf{Theorem --- }\emph{#1}\\\color{black}
    \ifx &#1& \color{theoremAccent}\textbf{Theorem. }\color{black}
    \else \color{theoremAccent}\textbf{Theorem --- }\emph{#1}\\\color{black} \fi
  }
  {
    \end{tcolorbox}
  }
\definecolor{noteBackground}{RGB}{244,249,244}
\definecolor{noteAccent}{RGB}{34,139,34}
\newenvironment{note}[1]
  {
  \begin{tcolorbox}[
    enhanced,
    boxrule=0pt,
    frame hidden,
    sharp corners,
    colback=noteBackground,
    borderline west={3pt}{-1.5pt}{noteAccent},
    breakable
    ]
    \ifx &#1& \color{noteAccent}\textbf{Note. }\color{black}
    \else \color{noteAccent}\textbf{Note (#1). }\color{black} \fi
    }
    {
  \end{tcolorbox}
  }
\definecolor{lemmaBackground}{RGB}{255,247,234}
\definecolor{lemmaAccent}{RGB}{255,153,0}
\newenvironment{lemma}[1]
  {
    \begin{tcolorbox}[
      enhanced,
      boxrule=0pt,
      frame hidden,
      sharp corners,
      colback=lemmaBackground,
      borderline west={3pt}{-1.5pt}{lemmaAccent},
      breakable
    ]
    \ifx &#1& \color{lemmaAccent}\textbf{Lemma. }\color{black}
    \else \color{lemmaAccent}\textbf{Lemma #1. }\color{black} \fi
  }
  {
    \end{tcolorbox}
  }
\definecolor{definitionBackground}{RGB}{246,246,246}
\newenvironment{definition}[1]
  {
    \begin{tcolorbox}[
      enhanced,
      boxrule=0pt,
      frame hidden,
      sharp corners,
      colback=definitionBackground,
      borderline west={3pt}{-1.5pt}{black},
      breakable
    ]
    \textbf{Definition. }\emph{#1}\\
  }
  {
    \end{tcolorbox}
  }

\newenvironment{amatrix}[2]{
    \left[
      \begin{array}{*{#1}{c}|*{#2}c}
  }
  {
      \end{array}
    \right]
  }

\definecolor{codeBackground}{RGB}{253,246,227}
\renewcommand\theFancyVerbLine{\arabic{FancyVerbLine}}
\newtcblisting{cpp}[1][]{
  boxrule=1pt,
  sharp corners,
  colback=codeBackground,
  listing only,
  breakable,
  minted language=cpp,
  minted style=material,
  minted options={
    xleftmargin=15pt,
    baselinestretch=1.2,
    linenos,
  }
}

\author{Kyle Chui}


\fancyhf{}
\lhead{Kyle Chui}
\rhead{Page \thepage}
\pagestyle{fancy}

\begin{document}
  \section{Lecture 6}
  \subsection{Fixed Point Method (Finding Roots)}
  \begin{definition}{Fixed Point}
    We say $p$ is a \emph{fixed point} of $g(x)$ if $g(p) = p$.
  \end{definition}
  \begin{itemize}
    \item We first need to translate our root-finding problem into a fixed point problem.
    \item We want to solve $f(x) = 0$, so we should find $p$ such that $f(p) = 0$.
    \item We rewrite $f(x) = 0$ as $g(x)\ceq f(x) + x = x$.
    \item Consider the relation between $f$ and $g$:
    \begin{itemize}
      \item Assume $p$ is a root of $f(x)$, so $g(p) = f(p) + p = 0 + p = p$.
      \item Thus if $p$ is a root of $f$, then it must be a fixed point of $g$.
    \end{itemize}
  \end{itemize}
  \begin{note}{}
    The fixed point function $g(x)$ is \emph{not} unique! In general, we have that $g(x)\ceq cf(x) + x$ is a fixed point function of $f$, given that $c\neq 0$.
  \end{note}
  How can we find the fixed point $p$ of $g(x)$?
  \begin{itemize}
    \item Choose an initial approximation $p_0$ (of $p$).
    \item We have $p_1 = g(p_0), p_2 = g(p_1), \dotsc$, so compute $p_n = g(p_{n - 1})$ for $n\geq 1$.
    \item If $\abs{p_n - p_{n - 1}} < \eps$ or $\frac{\abs{p_n - p_{n - 1}}}{\abs{p_n}} < \eps$, then stop.
  \end{itemize}
  \begin{note}{}
    Some fixed point functions can diverge! You should choose a fixed point that converges.
  \end{note}
  \begin{theorem}{Fixed Point Theorem}
    \begin{enumerate}[label=(\arabic*)]
      \item If $g\in C[a, b]$ and $g(x)\in [a, b]$ for all $x\in [a, b]$, then $g(x)$ has at least one fixed point in $[a, b]$.
      \item If, in addition, $g'(x)$ exists on $(a, b)$ and there exists a positive constant $k < 1$ such that
      \[
        \abs{g'(x)}\leq k, \quad \text{for all }x\in (a, b),
      \]
      then there is exactly one fixed point in $[a, b]$.
    \end{enumerate}
    \begin{enumerate}[label=(\arabic*)]
      \item
      \begin{proof}
        If $g(a) = a$ or $g(b) = b$ then we are done. Hence we may assume that $g(a)\neq a, g(b)\neq b$. Let us define a function $h(x)\ceq g(x) - x$, which is continuous. Then we know that since $g(a) > a$, then $h(a) > 0$. Similarly, as $g(b) < b$, we have $h(b) < 0$. Hence by Intermediate Value Theorem there exists a point $c\in (a, b)$ such that $h(c) = 0$. In other words, there exists a point such that $g(c) - c = 0$, or $g(c) = c$.
      \end{proof}
      \item
      \begin{proof}
        We have that $\abs{g'(x)}\leq k < 1$. Suppose towards a contradiction that there are two fixed points $p, q$ of $g(x)$ on $[a, b]$ where $p\neq q$. Thus we have
        \begin{align*}
          \abs{p - q} &= \abs{g(p) - g(q)} \\
                      &= \abs{p - q}\cdot \frac{g(p) - g(q)}{\abs{p - q}} \\
                      &= \abs{p - q}\cdot g'(c) \tag{MVT, $c\in (a, b)$}
        \end{align*}
        Dividing both sides by $\abs{p - q}$, we have $g'(c) = 1$, a contradiction. Hence $g(x)$ has a \emph{unique} fixed point on $[a, b]$.
      \end{proof}
    \end{enumerate}
  \end{theorem}
  \subsubsection{Convergence of Fixed Point Iteration}
  A fixed point of $g(x)$ is the point where $g(x)$ intersects the line $y = x$.
\end{document}
