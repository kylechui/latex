\documentclass[class=article, crop=false]{standalone}
\usepackage[margin=1in]{geometry}

\usepackage{amsmath}
\usepackage{amssymb}
\usepackage{amsthm}
\usepackage{comment}
\usepackage{enumitem}
\usepackage{fancyhdr}
\usepackage{hyperref}
\usepackage{mathrsfs}
\usepackage{mathtools}
\usepackage{standalone}
\usepackage{tcolorbox}
\usepackage{tikz}
\usepackage{xcolor}

\usetikzlibrary{decorations.pathreplacing}
\tcbuselibrary{skins}

\DeclareMathOperator{\lcm}{lcm}
\DeclareMathOperator{\proj}{proj}
\DeclareMathOperator{\vspan}{span}
\DeclareMathOperator{\im}{im}
\DeclareMathOperator{\range}{range}
\DeclareMathOperator{\Diff}{Diff}
\DeclareMathOperator{\Int}{Int}
\DeclareMathOperator{\fcn}{fcn}
\DeclareMathOperator{\id}{id}

\newcommand{\N}{\ensuremath{\mathbb{N}}}
\newcommand{\Z}{\ensuremath{\mathbb{Z}}}
\newcommand{\Q}{\ensuremath{\mathbb{Q}}}
\newcommand{\R}{\ensuremath{\mathbb{R}}}
\newcommand{\C}{\ensuremath{\mathbb{C}}}
\newcommand{\F}{\ensuremath{\mathbb{F}}}
\newcommand{\lam}{\ensuremath{\lambda}}
\newcommand{\nab}{\ensuremath{\nabla}}
\newcommand{\eps}{\ensuremath{\varepsilon}}
\newcommand{\es}{\ensuremath{\varnothing}}

\newcommand{\abs}[1]{\ensuremath{\left\lvert #1 \right\rvert}}
\newcommand{\bigpar}[1]{\ensuremath{\left( #1 \right)}}
\newcommand{\norm}[1]{\ensuremath{\lVert #1\rVert}}
\newcommand{\set}[1]{\ensuremath{\left\{#1\right\}}}
\newcommand{\tuple}[1]{\ensuremath{\left\langle #1 \right\rangle}}
\newcommand{\floor}[1]{\ensuremath{\left\lfloor #1 \right\rfloor}}
\newcommand{\ceil}[1]{\ensuremath{\left\lceil #1 \right\rceil}}

\newcommand{\chapternum}{}
\newcommand{\ex}[1]{\noindent\textbf{Exercise \chapternum.{#1}.}}

\newcommand{\dx}[1]{\,\mathrm{d}#1}
\newcommand{\inv}{\ensuremath{^{-1}}}
\newcommand{\sm}{\setminus}
\newcommand{\sse}{\subseteq}
\newcommand{\ceq}{\coloneqq}

\definecolor{problemBackground}{RGB}{212,232,246}
\newenvironment{problem}[1]
  {
    \begin{tcolorbox}[
      boxrule=.5pt,
      titlerule=.5pt,
      sharp corners,
      colback=problemBackground
    ]
    \ifx &#1& \textbf{Problem. }
    \else \textbf{Problem #1.} \fi
  }
  {
    \end{tcolorbox}
  }

\definecolor{exampleBackground}{RGB}{255,249,248}
\definecolor{exampleAccent}{RGB}{158,60,14}
\newenvironment{example}[1]
  {
    \begin{tcolorbox}[
      boxrule=.5pt,
      sharp corners,
      colback=exampleBackground,
      colframe=exampleAccent,
    ]
    \color{exampleAccent}\textbf{Example.} \emph{#1}\color{black}
  }
  {
    \end{tcolorbox}
  }

\definecolor{theoremBackground}{RGB}{234,243,251}
\definecolor{theoremAccent}{RGB}{0,116,183}
\newenvironment{theorem}[1]
  {
    \begin{tcolorbox}[
      boxrule=.5pt,
      titlerule=.5pt,
      sharp corners,
      colback=theoremBackground,
      colframe=theoremAccent
    ]
      \color{theoremAccent}\textbf{Theorem --- }\emph{#1}\\\color{black}
  }
  {
    \end{tcolorbox}
  }

\definecolor{noteBackground}{RGB}{244,249,244}
\definecolor{noteAccent}{RGB}{34,139,34}
\newenvironment{note}[1]
  {
  \begin{tcolorbox}[
    enhanced,
    boxrule=0pt,
    frame hidden,
    sharp corners,
    colback=noteBackground,
    borderline west={3pt}{-1.5pt}{noteAccent}
    ]
    \ifx &#1& \color{noteAccent}\textbf{Note. }\color{black}
    \else \color{noteAccent}\textbf{Note (#1). }\color{black} \fi
    }
    {
  \end{tcolorbox}
  }

\definecolor{definitionBackground}{RGB}{246,246,246}
\newenvironment{definition}[1]
  {
    \begin{tcolorbox}[
      enhanced,
      boxrule=0pt,
      frame hidden,
      sharp corners,
      colback=definitionBackground,
      borderline west={3pt}{-1.5pt}{black}
    ]
    \textbf{Definition. }\emph{#1}\\
  }
  {
    \end{tcolorbox}
  }
\newenvironment{amatrix}[2]{
    \left[
      \begin{array}{*{#1}{c}|*{#2}c}
  }
  {
      \end{array}
    \right]
  }
\date{\the\year-\the\month-\the\day}
\author{Kyle Chui}


\fancyhf{}
\lhead{Kyle Chui}
\rhead{Page \thepage}
\pagestyle{fancy}

\begin{document}
  \section{Lecture 12}
  \begin{note}{}
    Using Lagrange interpolation yields the same result as solving the system of linear equations, but without the need to solve a matrix equation.
  \end{note}
  \begin{theorem}{}
    Suppose $x_0,x_1,\dotsc,x_n$ are distinct numbers in the interval $[a, b]$ and $f\in C^{n + 1}[a, b]$. Then for each $x\in [a, b]$, there exists $\xi_x\in (a, b)$ such that
    \[
      f(x) = P_n(x) + \frac{f^{(n + 1)}(\xi_x)}{(n + 1)!}(x - x_0)\dotsb(x - x_n)
    \]
    where $P_n(x)$ is the interpolation polynomial.
    \begin{proof}
      Let $x\in [a, b]$. Then we have two cases: either $x = x_i$ for some $i = 0,1,\dotsc,n$, or $x\neq x_i$ for any $i = 0,1,\dotsc,n$. In the former case we have $(x - x_0)\dotsb(x - x_n) = 0$ so for any $\xi_x\in (a, b)$, we have
      \[
        \frac{f^{(n + 1)}(\xi_x)}{(n + 1)!}(x - x_0)\dotsb(x - x_n) = 0.
      \]
      Furthermore, since $f(x_i) = P(x_i)$ for all $i = 0,1,\dotsc,n$, the statement holds true for any $\xi_x\in (a, b)$. \par
      In the latter case, we define a new function $g(t)\colon [a, b]\to\R$ by
      \[
        g(t)\ceq f(t) - P_n(t) - (f(x) - P_n(x))\cdot \prod_{i = 0}^n \frac{t - x_i}{x - x_i}.
      \]
      Since $f(t), P(x)\in C^{n + 1}[a, b]$, we have that $g(t)\in C^{n + 1}[a, b]$. Observe that $g(x_i) = 0$ for all $i = 0,1,\dotsc,n$, so $g$ has at least $n + 1$ distinct roots. Furthermore, we also have $g(x) = 0$, so $g$ has a total of $n + 2$ distinct roots. Hence we may apply generalized Rolle's Theorem to get that there exists some $\xi_x\in (a, b)$ such that $g^{(n + 1)}(\xi_x) = 0$. Therefore
      \begin{align*}
        g^{(n + 1)}(\xi_x) &= f^{(n + 1)}(\xi_x) - 0 - (f(x) - P_n(x))\cdot (n + 1)! = 0 \\
        f(x) &= P_n(x) + \frac{f^{(n + 1)}(\xi_x)}{(n + 1)!}(x - x_0)\dotsb(x - x_n).
      \end{align*}
    \end{proof}
  \end{theorem}
\end{document}
