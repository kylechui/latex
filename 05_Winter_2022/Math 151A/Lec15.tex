\documentclass[class=article, crop=false]{standalone}
\usepackage[margin=1in]{geometry}

\usepackage{amsmath}
\usepackage{amssymb}
\usepackage{amsthm}
\usepackage{comment}
\usepackage{enumitem}
\usepackage{fancyhdr}
\usepackage{hyperref}
\usepackage{mathrsfs}
\usepackage{mathtools}
\usepackage{standalone}
\usepackage{tcolorbox}
\usepackage{tikz}
\usepackage{xcolor}

\usetikzlibrary{decorations.pathreplacing}
\tcbuselibrary{skins}

\DeclareMathOperator{\lcm}{lcm}
\DeclareMathOperator{\proj}{proj}
\DeclareMathOperator{\vspan}{span}
\DeclareMathOperator{\im}{im}
\DeclareMathOperator{\range}{range}
\DeclareMathOperator{\Diff}{Diff}
\DeclareMathOperator{\Int}{Int}
\DeclareMathOperator{\fcn}{fcn}
\DeclareMathOperator{\id}{id}

\newcommand{\N}{\ensuremath{\mathbb{N}}}
\newcommand{\Z}{\ensuremath{\mathbb{Z}}}
\newcommand{\Q}{\ensuremath{\mathbb{Q}}}
\newcommand{\R}{\ensuremath{\mathbb{R}}}
\newcommand{\C}{\ensuremath{\mathbb{C}}}
\newcommand{\F}{\ensuremath{\mathbb{F}}}
\newcommand{\lam}{\ensuremath{\lambda}}
\newcommand{\nab}{\ensuremath{\nabla}}
\newcommand{\eps}{\ensuremath{\varepsilon}}
\newcommand{\es}{\ensuremath{\varnothing}}

\newcommand{\abs}[1]{\ensuremath{\left\lvert #1 \right\rvert}}
\newcommand{\bigpar}[1]{\ensuremath{\left( #1 \right)}}
\newcommand{\norm}[1]{\ensuremath{\lVert #1\rVert}}
\newcommand{\set}[1]{\ensuremath{\left\{#1\right\}}}
\newcommand{\tuple}[1]{\ensuremath{\left\langle #1 \right\rangle}}
\newcommand{\floor}[1]{\ensuremath{\left\lfloor #1 \right\rfloor}}
\newcommand{\ceil}[1]{\ensuremath{\left\lceil #1 \right\rceil}}

\newcommand{\chapternum}{}
\newcommand{\ex}[1]{\noindent\textbf{Exercise \chapternum.{#1}.}}

\newcommand{\dx}[1]{\,\mathrm{d}#1}
\newcommand{\inv}{\ensuremath{^{-1}}}
\newcommand{\sm}{\setminus}
\newcommand{\sse}{\subseteq}
\newcommand{\ceq}{\coloneqq}

\definecolor{problemBackground}{RGB}{212,232,246}
\newenvironment{problem}[1]
  {
    \begin{tcolorbox}[
      boxrule=.5pt,
      titlerule=.5pt,
      sharp corners,
      colback=problemBackground
    ]
    \ifx &#1& \textbf{Problem. }
    \else \textbf{Problem #1.} \fi
  }
  {
    \end{tcolorbox}
  }

\definecolor{exampleBackground}{RGB}{255,249,248}
\definecolor{exampleAccent}{RGB}{158,60,14}
\newenvironment{example}[1]
  {
    \begin{tcolorbox}[
      boxrule=.5pt,
      sharp corners,
      colback=exampleBackground,
      colframe=exampleAccent,
    ]
    \color{exampleAccent}\textbf{Example.} \emph{#1}\color{black}
  }
  {
    \end{tcolorbox}
  }

\definecolor{theoremBackground}{RGB}{234,243,251}
\definecolor{theoremAccent}{RGB}{0,116,183}
\newenvironment{theorem}[1]
  {
    \begin{tcolorbox}[
      boxrule=.5pt,
      titlerule=.5pt,
      sharp corners,
      colback=theoremBackground,
      colframe=theoremAccent
    ]
      \color{theoremAccent}\textbf{Theorem --- }\emph{#1}\\\color{black}
  }
  {
    \end{tcolorbox}
  }

\definecolor{noteBackground}{RGB}{244,249,244}
\definecolor{noteAccent}{RGB}{34,139,34}
\newenvironment{note}[1]
  {
  \begin{tcolorbox}[
    enhanced,
    boxrule=0pt,
    frame hidden,
    sharp corners,
    colback=noteBackground,
    borderline west={3pt}{-1.5pt}{noteAccent}
    ]
    \ifx &#1& \color{noteAccent}\textbf{Note. }\color{black}
    \else \color{noteAccent}\textbf{Note (#1). }\color{black} \fi
    }
    {
  \end{tcolorbox}
  }

\definecolor{definitionBackground}{RGB}{246,246,246}
\newenvironment{definition}[1]
  {
    \begin{tcolorbox}[
      enhanced,
      boxrule=0pt,
      frame hidden,
      sharp corners,
      colback=definitionBackground,
      borderline west={3pt}{-1.5pt}{black}
    ]
    \textbf{Definition. }\emph{#1}\\
  }
  {
    \end{tcolorbox}
  }
\newenvironment{amatrix}[2]{
    \left[
      \begin{array}{*{#1}{c}|*{#2}c}
  }
  {
      \end{array}
    \right]
  }
\date{\the\year-\the\month-\the\day}
\author{Kyle Chui}


\fancyhf{}
\lhead{Kyle Chui}
\rhead{Page \thepage}
\pagestyle{fancy}

\begin{document}
  \section{Lecture 15}
  \begin{definition}{Divided Difference}
    The zeroth divided difference is written as
    \[
      f[x_i] = f(x_i).
    \]
    The first divided difference is written as
    \[
      f[x_i, x_{i + 1}] = \frac{f[x_{i + 1}] - f[x_i]}{x_{i + 1} - x_i}.
    \]
    The second divided difference is written as
    \[
      f[x_i, x_{i + 1}, x_{i + 2}] = \frac{f[x_{i + 1, x_{i + 2}}] - f[x_i, x_{i + 1}]}{x_{i + 2} - x_i}.
    \]
    Hence the $k$\tsup{th} divided difference is
    \[
      f[x_i,\dotsc,x_{i + k}] = \frac{f[x_{i + 1},\dotsc,x_{i + k}] - f[x_i,\dotsc,x_{i + k - 1}]}{x_{i + k} - x_i}.
    \]
  \end{definition}
  \begin{note}{}
    The value of the divided differences is \emph{independent} of the order of $x_0,\dotsc,x_k$.
  \end{note}
  \subsection{Newton's Divided Difference Form of the Interpolation Polynomial}
  When we take $n = 1$, we have $P_1(x) = f(x_0) + a_1(x - x_0)$. To find $a_1$ such that $P_1(x_1) = f(x_1)$, we have
  \begin{align*}
    f(x_0) + a_1(x_1 - x_0) &= f(x_1) \\
    a_1 &= \frac{f(x_1) - f(x_0)}{x_1 - x_0} \\
    a_1 &= f[x_0, x_1].
  \end{align*}
  When we have $n = 2$, we have
  \[
    P_2(x) = f(x_0) + a_1(x - x_0) + a_2(x - x_0)(x - x_1).
  \]
  Hence we solve for $a_2$ such that $P_2(x_2) = f(x_2)$. If we continue these computations we find that
  \[
    P_n(x) = f[x_0] + f[x_0, x_1](x - x_0) + \dotsb + f[x_0,\dotsc,x_n](x - x_0)\dotsb(x - x_{n - 1}).
  \]
  \begin{note}{}
    A benefit for using the Divided difference algorithm is that we can continually factor out $x - x_i$ terms, without expanding the polynomial.
  \end{note}
  \begin{theorem}{Uniqueness of Interpolation Polynomials}
    If $x_0,x_1,\dotsc,x_n$ are $n + 1$ distinct numbers and $f$ is a function whose values are gives at these inputs, then a unique polynomial $P$ of degree at most ``$n$'' exists with $P(x_i) = f(x_i)$ for all $i\in 0,\dotsc,n$.
    \begin{proof}
      We first show existence, by defining
      \[
        P(x) = \sum_{i=0}^{n} \ell_i(x)f(x_i),
      \]
      where $\ell_i(x)$ is our base function from before. We now show uniqueness. Let $P(x)$ and $Q(x)$ be two interpolation polynomials such that $P(x_i) = Q(x_i) = f(x_i)$ for all $i=0,\dotsc,n$. We define a difference function $D(x) = P(x) - Q(x)$, so $D(x_i) = 0$ for all $i = 0,\dotsc,n$ and $D(x)$ has degree at most ``$n$''. Since it has more distinct roots than it has degree, it must be identically zero. Therefore $P(x) = Q(x)$ and our interpolation polynomial is unique.
    \end{proof}
  \end{theorem}
\end{document}
