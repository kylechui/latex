\documentclass[class=article, crop=false]{standalone}
\usepackage[margin=1in]{geometry}

\usepackage{amsmath}
\usepackage{amssymb}
\usepackage{amsthm}
\usepackage{comment}
\usepackage{enumitem}
\usepackage{fancyhdr}
\usepackage{hyperref}
\usepackage{mathrsfs}
\usepackage{mathtools}
\usepackage{standalone}
\usepackage{tcolorbox}
\usepackage{tikz}
\usepackage{xcolor}

\usetikzlibrary{decorations.pathreplacing}
\tcbuselibrary{skins}

\DeclareMathOperator{\lcm}{lcm}
\DeclareMathOperator{\proj}{proj}
\DeclareMathOperator{\vspan}{span}
\DeclareMathOperator{\im}{im}
\DeclareMathOperator{\range}{range}
\DeclareMathOperator{\Diff}{Diff}
\DeclareMathOperator{\Int}{Int}
\DeclareMathOperator{\fcn}{fcn}
\DeclareMathOperator{\id}{id}

\newcommand{\N}{\ensuremath{\mathbb{N}}}
\newcommand{\Z}{\ensuremath{\mathbb{Z}}}
\newcommand{\Q}{\ensuremath{\mathbb{Q}}}
\newcommand{\R}{\ensuremath{\mathbb{R}}}
\newcommand{\C}{\ensuremath{\mathbb{C}}}
\newcommand{\F}{\ensuremath{\mathbb{F}}}
\newcommand{\lam}{\ensuremath{\lambda}}
\newcommand{\nab}{\ensuremath{\nabla}}
\newcommand{\eps}{\ensuremath{\varepsilon}}
\newcommand{\es}{\ensuremath{\varnothing}}

\newcommand{\abs}[1]{\ensuremath{\left\lvert #1 \right\rvert}}
\newcommand{\bigpar}[1]{\ensuremath{\left( #1 \right)}}
\newcommand{\norm}[1]{\ensuremath{\lVert #1\rVert}}
\newcommand{\set}[1]{\ensuremath{\left\{#1\right\}}}
\newcommand{\tuple}[1]{\ensuremath{\left\langle #1 \right\rangle}}
\newcommand{\floor}[1]{\ensuremath{\left\lfloor #1 \right\rfloor}}
\newcommand{\ceil}[1]{\ensuremath{\left\lceil #1 \right\rceil}}

\newcommand{\chapternum}{}
\newcommand{\ex}[1]{\noindent\textbf{Exercise \chapternum.{#1}.}}

\newcommand{\dx}[1]{\,\mathrm{d}#1}
\newcommand{\inv}{\ensuremath{^{-1}}}
\newcommand{\sm}{\setminus}
\newcommand{\sse}{\subseteq}
\newcommand{\ceq}{\coloneqq}

\definecolor{problemBackground}{RGB}{212,232,246}
\newenvironment{problem}[1]
  {
    \begin{tcolorbox}[
      boxrule=.5pt,
      titlerule=.5pt,
      sharp corners,
      colback=problemBackground
    ]
    \ifx &#1& \textbf{Problem. }
    \else \textbf{Problem #1.} \fi
  }
  {
    \end{tcolorbox}
  }

\definecolor{exampleBackground}{RGB}{255,249,248}
\definecolor{exampleAccent}{RGB}{158,60,14}
\newenvironment{example}[1]
  {
    \begin{tcolorbox}[
      boxrule=.5pt,
      sharp corners,
      colback=exampleBackground,
      colframe=exampleAccent,
    ]
    \color{exampleAccent}\textbf{Example.} \emph{#1}\color{black}
  }
  {
    \end{tcolorbox}
  }

\definecolor{theoremBackground}{RGB}{234,243,251}
\definecolor{theoremAccent}{RGB}{0,116,183}
\newenvironment{theorem}[1]
  {
    \begin{tcolorbox}[
      boxrule=.5pt,
      titlerule=.5pt,
      sharp corners,
      colback=theoremBackground,
      colframe=theoremAccent
    ]
      \color{theoremAccent}\textbf{Theorem --- }\emph{#1}\\\color{black}
  }
  {
    \end{tcolorbox}
  }

\definecolor{noteBackground}{RGB}{244,249,244}
\definecolor{noteAccent}{RGB}{34,139,34}
\newenvironment{note}[1]
  {
  \begin{tcolorbox}[
    enhanced,
    boxrule=0pt,
    frame hidden,
    sharp corners,
    colback=noteBackground,
    borderline west={3pt}{-1.5pt}{noteAccent}
    ]
    \ifx &#1& \color{noteAccent}\textbf{Note. }\color{black}
    \else \color{noteAccent}\textbf{Note (#1). }\color{black} \fi
    }
    {
  \end{tcolorbox}
  }

\definecolor{definitionBackground}{RGB}{246,246,246}
\newenvironment{definition}[1]
  {
    \begin{tcolorbox}[
      enhanced,
      boxrule=0pt,
      frame hidden,
      sharp corners,
      colback=definitionBackground,
      borderline west={3pt}{-1.5pt}{black}
    ]
    \textbf{Definition. }\emph{#1}\\
  }
  {
    \end{tcolorbox}
  }
\newenvironment{amatrix}[2]{
    \left[
      \begin{array}{*{#1}{c}|*{#2}c}
  }
  {
      \end{array}
    \right]
  }
\date{\the\year-\the\month-\the\day}
\author{Kyle Chui}


\fancyhf{}
\lhead{Kyle Chui}
\rhead{Page \thepage}
\pagestyle{fancy}

\begin{document}
  \section{Lecture 4}
  \subsection{Rational Points on the Circle}
  \textbf{Observation.} If $a^2 + b^2 = c^2$, then
  \[
    \paren{\frac{a}{c}}^2 + \paren{\frac{b}{c}}^2 = 1.
  \]
  Hence rational points on the unit circle correspond to Pythagorean triples.
  \subsection{Chord-Tangent Construction}
  \begin{itemize}
    \item The idea is most likely due to Diophantos, circa $250$ CE.
    \item Worked out in detail in the 17\tsup{th} century (Lagrange, Euler, etc.).
    \item Take a line through $(-1, 0)$ to $(x, y)$ with slope $t$. The equation of this line is thus $y = t(x + 1)$.
    \item We claim that $R$ has rational coordinates if and only if $t$ is rational.
    \begin{proof}
      ($\Rightarrow$) Suppose $R = (x_0, y_0)$, with $x_0, y_0\in\Q$. The slope of the line can be computed to be $\frac{y_0}{1 + x_0} = t$. Thus $t$ is rational. \par
      ($\Leftarrow$) Suppose $t$ is rational. Since the line intersects the unit circle, we have
      \begin{align*}
        x^2 + y^2 &= 1 \\
        x^2 + t(x + 1)^2 &= 1 \\
        x^2 + tx^2 + 2tx + t &= 1 \\
        (1 + t)x^2 + (2t)x + (t - 1) &= 0.
      \end{align*}
      \textbf{Observation:} If $x_0$, $x_1$ are solutions of $ax^2 + bx + c = 0$, then we know that we can factor into $a(x - x_0)(x - x_1) = 0$. Thus by equating coefficients, we see that $x_0$ is rational if $x_1$ is rational. \par
      We already know that $x_1 = -1$ is a solution to our equation, so we know that $x$ must be rational. Since $y = t(x_0 + 1)$, it must be rational as well, so $R$ has rational coordinates.
    \end{proof}
    If we actually solve the equation, we get the point $R$ has coordinates $(-1, 0)$ or $\displaystyle \paren{\frac{1-t^2}{1+t^2}, \frac{2t}{1 + t^2}}$. If we write $t = \frac{p}{q}$ for integers $p, q$, then we have
    \[
      x_0 = \frac{p^2 - q^2}{p^2 + q^2}, y_0 = \frac{2pq}{p^2 + q^2}.
    \]
    Thus we may convert this into a true Pythagorean triple via
    \begin{align*}
      a &= (p^2 - q^2)r \\
      b &= 2pqr \\
      c &= (p^2 + q^2)r
    \end{align*}
  \end{itemize}
  \begin{note}{}
    The general takeaway from this proof is that you can use ``known'' points to help you deduce properties of certain ``unknown'' points. In this case, we found an intersection point between a line and a circle, using that we are given the other intersection point.
  \end{note}
\end{document}
