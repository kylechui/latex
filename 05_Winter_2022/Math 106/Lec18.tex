\documentclass[class=article, crop=false]{standalone}
\usepackage[margin=1in]{geometry}

\usepackage{amsmath}
\usepackage{amssymb}
\usepackage{amsthm}
\usepackage{comment}
\usepackage{enumitem}
\usepackage{fancyhdr}
\usepackage{hyperref}
\usepackage{mathrsfs}
\usepackage{mathtools}
\usepackage{standalone}
\usepackage{tcolorbox}
\usepackage{tikz}
\usepackage{xcolor}

\usetikzlibrary{decorations.pathreplacing}
\tcbuselibrary{skins}

\DeclareMathOperator{\lcm}{lcm}
\DeclareMathOperator{\proj}{proj}
\DeclareMathOperator{\vspan}{span}
\DeclareMathOperator{\im}{im}
\DeclareMathOperator{\range}{range}
\DeclareMathOperator{\Diff}{Diff}
\DeclareMathOperator{\Int}{Int}
\DeclareMathOperator{\fcn}{fcn}
\DeclareMathOperator{\id}{id}

\newcommand{\N}{\ensuremath{\mathbb{N}}}
\newcommand{\Z}{\ensuremath{\mathbb{Z}}}
\newcommand{\Q}{\ensuremath{\mathbb{Q}}}
\newcommand{\R}{\ensuremath{\mathbb{R}}}
\newcommand{\C}{\ensuremath{\mathbb{C}}}
\newcommand{\F}{\ensuremath{\mathbb{F}}}
\newcommand{\lam}{\ensuremath{\lambda}}
\newcommand{\nab}{\ensuremath{\nabla}}
\newcommand{\eps}{\ensuremath{\varepsilon}}
\newcommand{\es}{\ensuremath{\varnothing}}

\newcommand{\abs}[1]{\ensuremath{\left\lvert #1 \right\rvert}}
\newcommand{\bigpar}[1]{\ensuremath{\left( #1 \right)}}
\newcommand{\norm}[1]{\ensuremath{\lVert #1\rVert}}
\newcommand{\set}[1]{\ensuremath{\left\{#1\right\}}}
\newcommand{\tuple}[1]{\ensuremath{\left\langle #1 \right\rangle}}
\newcommand{\floor}[1]{\ensuremath{\left\lfloor #1 \right\rfloor}}
\newcommand{\ceil}[1]{\ensuremath{\left\lceil #1 \right\rceil}}

\newcommand{\chapternum}{}
\newcommand{\ex}[1]{\noindent\textbf{Exercise \chapternum.{#1}.}}

\newcommand{\dx}[1]{\,\mathrm{d}#1}
\newcommand{\inv}{\ensuremath{^{-1}}}
\newcommand{\sm}{\setminus}
\newcommand{\sse}{\subseteq}
\newcommand{\ceq}{\coloneqq}

\definecolor{problemBackground}{RGB}{212,232,246}
\newenvironment{problem}[1]
  {
    \begin{tcolorbox}[
      boxrule=.5pt,
      titlerule=.5pt,
      sharp corners,
      colback=problemBackground
    ]
    \ifx &#1& \textbf{Problem. }
    \else \textbf{Problem #1.} \fi
  }
  {
    \end{tcolorbox}
  }

\definecolor{exampleBackground}{RGB}{255,249,248}
\definecolor{exampleAccent}{RGB}{158,60,14}
\newenvironment{example}[1]
  {
    \begin{tcolorbox}[
      boxrule=.5pt,
      sharp corners,
      colback=exampleBackground,
      colframe=exampleAccent,
    ]
    \color{exampleAccent}\textbf{Example.} \emph{#1}\color{black}
  }
  {
    \end{tcolorbox}
  }

\definecolor{theoremBackground}{RGB}{234,243,251}
\definecolor{theoremAccent}{RGB}{0,116,183}
\newenvironment{theorem}[1]
  {
    \begin{tcolorbox}[
      boxrule=.5pt,
      titlerule=.5pt,
      sharp corners,
      colback=theoremBackground,
      colframe=theoremAccent
    ]
      \color{theoremAccent}\textbf{Theorem --- }\emph{#1}\\\color{black}
  }
  {
    \end{tcolorbox}
  }

\definecolor{noteBackground}{RGB}{244,249,244}
\definecolor{noteAccent}{RGB}{34,139,34}
\newenvironment{note}[1]
  {
  \begin{tcolorbox}[
    enhanced,
    boxrule=0pt,
    frame hidden,
    sharp corners,
    colback=noteBackground,
    borderline west={3pt}{-1.5pt}{noteAccent}
    ]
    \ifx &#1& \color{noteAccent}\textbf{Note. }\color{black}
    \else \color{noteAccent}\textbf{Note (#1). }\color{black} \fi
    }
    {
  \end{tcolorbox}
  }

\definecolor{definitionBackground}{RGB}{246,246,246}
\newenvironment{definition}[1]
  {
    \begin{tcolorbox}[
      enhanced,
      boxrule=0pt,
      frame hidden,
      sharp corners,
      colback=definitionBackground,
      borderline west={3pt}{-1.5pt}{black}
    ]
    \textbf{Definition. }\emph{#1}\\
  }
  {
    \end{tcolorbox}
  }
\newenvironment{amatrix}[2]{
    \left[
      \begin{array}{*{#1}{c}|*{#2}c}
  }
  {
      \end{array}
    \right]
  }
\date{\the\year-\the\month-\the\day}
\author{Kyle Chui}


\fancyhf{}
\lhead{Kyle Chui}
\rhead{Page \thepage}
\pagestyle{fancy}

\begin{document}
  \section{Lecture 18}
  \subsection{Pell's Equation in India}
  Recall that both the ancient Chinese and Indians found integer solutions to linear equations using Chinese remainder theorem
  \begin{itemize}
    \item Chinese found general (approximate) solutions to ``more complicated equations'' (systems of polynomial equations)
    \item Indians found integer solutions to other equations
  \end{itemize}
  Pell's equation takes the form
  \[
    x^2 - ny^2 = 1,
  \]
  where $n$ is a positive integer and not a square.
  \textbf{Observation.} We may rewrite this equation to be
  \begin{align*}
    x^2 - ny^2 &= 1 \\
    (x + y\sqrt{n})(x - y\sqrt{n}) &= 1
  \end{align*}
  Hence if $\sqrt{n}$ is an integer, then we have the trivial solution $x = \pm 1, y = 0$. If $\sqrt{n}\notin \Z$, there could be ``non-trivial'' solutions to this equation.
  \paragraph{Why this equation?} \par
  Solutions to this equation can give good approximations for $\sqrt{n}\approx \frac{x}{y}$.
  \begin{note}{}
    This equation with $n = 2$ appears already in Greece, China, etc.
  \end{note}
  \paragraph{Brahmagupta's Discovery} \par
  Suppose we have $(x_1^2 - ny_1^2)(n_2^2 - ny_2^2) = (x_1x_2 + ny_1y_2)^2 - n(x_1y_2 + x_2y_1)^2 = 1$. Hence we can define a new pair $(x_3, y_3)$ to be $x = x_1x_2 + ny_1y_2, y = x_1y_2 + x_2y_1$. Furthermore, if $x_1^2-ny_1^2 = k_1$ and $x_2^2-ny_2^2 = k_2$, then we have $x_3^2 - ny_3^2 = k_1k_2$. \par
  We say that the triple $(x_3, y_3, k_1k_2)$ is the \emph{composition} of the $(x_1,y_1,k_1)$ and $(x_2,y_2,k_2)$. \par
  \textbf{Observation.} If we have any solution with $x, y\neq 0$, we can generate infinitely many solutions (since the composition is strictly larger than the previous solutions).
  \begin{example}{}
    We start with $n = 2$, and let $x = 3, y = 2$. We can find another solution by composing that solution with itself. We find that the composition of $(3, 2, 1)$ with itself is $(17, 12, 1)$.
  \end{example}
  \textbf{Fact.} Sometimes, we can find solutions to Pell's equation from solutions to $x^2 - ny^2 = k$ for some $k\in\Z$.
  \begin{example}{}
    If $k = -1$, then composing with itself yields a solution to Pell's equation.
  \end{example}
\end{document}
