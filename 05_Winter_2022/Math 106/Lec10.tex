\documentclass[class=article, crop=false]{standalone}
\usepackage[margin=1in]{geometry}

\usepackage{amsmath}
\usepackage{amssymb}
\usepackage{amsthm}
\usepackage{comment}
\usepackage{enumitem}
\usepackage{fancyhdr}
\usepackage{hyperref}
\usepackage{mathrsfs}
\usepackage{mathtools}
\usepackage{standalone}
\usepackage{tcolorbox}
\usepackage{tikz}
\usepackage{xcolor}

\usetikzlibrary{decorations.pathreplacing}
\tcbuselibrary{skins}

\DeclareMathOperator{\lcm}{lcm}
\DeclareMathOperator{\proj}{proj}
\DeclareMathOperator{\vspan}{span}
\DeclareMathOperator{\im}{im}
\DeclareMathOperator{\range}{range}
\DeclareMathOperator{\Diff}{Diff}
\DeclareMathOperator{\Int}{Int}
\DeclareMathOperator{\fcn}{fcn}
\DeclareMathOperator{\id}{id}

\newcommand{\N}{\ensuremath{\mathbb{N}}}
\newcommand{\Z}{\ensuremath{\mathbb{Z}}}
\newcommand{\Q}{\ensuremath{\mathbb{Q}}}
\newcommand{\R}{\ensuremath{\mathbb{R}}}
\newcommand{\C}{\ensuremath{\mathbb{C}}}
\newcommand{\F}{\ensuremath{\mathbb{F}}}
\newcommand{\lam}{\ensuremath{\lambda}}
\newcommand{\nab}{\ensuremath{\nabla}}
\newcommand{\eps}{\ensuremath{\varepsilon}}
\newcommand{\es}{\ensuremath{\varnothing}}

\newcommand{\abs}[1]{\ensuremath{\left\lvert #1 \right\rvert}}
\newcommand{\bigpar}[1]{\ensuremath{\left( #1 \right)}}
\newcommand{\norm}[1]{\ensuremath{\lVert #1\rVert}}
\newcommand{\set}[1]{\ensuremath{\left\{#1\right\}}}
\newcommand{\tuple}[1]{\ensuremath{\left\langle #1 \right\rangle}}
\newcommand{\floor}[1]{\ensuremath{\left\lfloor #1 \right\rfloor}}
\newcommand{\ceil}[1]{\ensuremath{\left\lceil #1 \right\rceil}}

\newcommand{\chapternum}{}
\newcommand{\ex}[1]{\noindent\textbf{Exercise \chapternum.{#1}.}}

\newcommand{\dx}[1]{\,\mathrm{d}#1}
\newcommand{\inv}{\ensuremath{^{-1}}}
\newcommand{\sm}{\setminus}
\newcommand{\sse}{\subseteq}
\newcommand{\ceq}{\coloneqq}

\definecolor{problemBackground}{RGB}{212,232,246}
\newenvironment{problem}[1]
  {
    \begin{tcolorbox}[
      boxrule=.5pt,
      titlerule=.5pt,
      sharp corners,
      colback=problemBackground
    ]
    \ifx &#1& \textbf{Problem. }
    \else \textbf{Problem #1.} \fi
  }
  {
    \end{tcolorbox}
  }

\definecolor{exampleBackground}{RGB}{255,249,248}
\definecolor{exampleAccent}{RGB}{158,60,14}
\newenvironment{example}[1]
  {
    \begin{tcolorbox}[
      boxrule=.5pt,
      sharp corners,
      colback=exampleBackground,
      colframe=exampleAccent,
    ]
    \color{exampleAccent}\textbf{Example.} \emph{#1}\color{black}
  }
  {
    \end{tcolorbox}
  }

\definecolor{theoremBackground}{RGB}{234,243,251}
\definecolor{theoremAccent}{RGB}{0,116,183}
\newenvironment{theorem}[1]
  {
    \begin{tcolorbox}[
      boxrule=.5pt,
      titlerule=.5pt,
      sharp corners,
      colback=theoremBackground,
      colframe=theoremAccent
    ]
      \color{theoremAccent}\textbf{Theorem --- }\emph{#1}\\\color{black}
  }
  {
    \end{tcolorbox}
  }

\definecolor{noteBackground}{RGB}{244,249,244}
\definecolor{noteAccent}{RGB}{34,139,34}
\newenvironment{note}[1]
  {
  \begin{tcolorbox}[
    enhanced,
    boxrule=0pt,
    frame hidden,
    sharp corners,
    colback=noteBackground,
    borderline west={3pt}{-1.5pt}{noteAccent}
    ]
    \ifx &#1& \color{noteAccent}\textbf{Note. }\color{black}
    \else \color{noteAccent}\textbf{Note (#1). }\color{black} \fi
    }
    {
  \end{tcolorbox}
  }

\definecolor{definitionBackground}{RGB}{246,246,246}
\newenvironment{definition}[1]
  {
    \begin{tcolorbox}[
      enhanced,
      boxrule=0pt,
      frame hidden,
      sharp corners,
      colback=definitionBackground,
      borderline west={3pt}{-1.5pt}{black}
    ]
    \textbf{Definition. }\emph{#1}\\
  }
  {
    \end{tcolorbox}
  }
\newenvironment{amatrix}[2]{
    \left[
      \begin{array}{*{#1}{c}|*{#2}c}
  }
  {
      \end{array}
    \right]
  }
\date{\the\year-\the\month-\the\day}
\author{Kyle Chui}


\fancyhf{}
\lhead{Kyle Chui}
\rhead{Page \thepage}
\pagestyle{fancy}

\begin{document}
  \section{Lecture 10}
  \subsection{The Euclidean Algorithm}
  This algorithm is used to find the greatest common divisor of two numbers. It was most likely known before Euclid, but it first appears in the Elements.
  \subsubsection{Performing the Algorithm}
  We wish to find $\gcd(a, b)$ where $a, b\in \Z$.
  \begin{itemize}
    \item Construct $a_1 = \max(a, b) - \min(a, b)$ and $b_1 = \min(a, b)$. Note that $\gcd(a, b) = \gcd(a_1, b_1)$.
    \item Repeat this until $a_n = b_n$, at which point $\gcd(a, b) = a_n$.
  \end{itemize}
  \subsubsection{Consequences of the Euclidean Algorithm}
  \begin{itemize}
    \item Given $a, b\in \Z^+$, there exist integers $m, n\in\Z$ such that
    \[
      am + bn = \gcd(a, b).
    \]
    \begin{note}{}
      If you ``work backwards'' through the Euclidean Algorithm, you see that $\gcd(a, b)$ is a linear combination of $a$ and $b$.
    \end{note}
    \item If a prime number $p$ divides $ab$, then $p$ divides a or $p$ divides $b$.
    \begin{proof}
      Suppose $p$ does not divide $a$. We claim that $p$ divides $b$. Since $p$ is prime and does not divide $a$, we have $\gcd(p, a) = 1$. Hence there exist $m, n\in \Z^+$ such that $mp + na = 1$. Multiplying both sides by $b$, we have
      \[
        mpb + nab = 1.
      \]
      Since $p$ divides $ab$, we have $ab = kp$ for some $k\in \Z$. Hence
      \begin{align*}
        mpb + nkp &= b \\
        p(mb + nk) &= b.
      \end{align*}
      Therefore $p$ divides $b$.
    \end{proof}
    \item Each positive integer has a unique factorization into primes.
  \end{itemize}
  \subsection{Greek Number Systems}
  \begin{itemize}
    \item There were different number systems in different parts of the Hellenistic realm.
    \item They are usually decimal (base 10).
    \item They are usually \emph{not} place-valued.
  \end{itemize}
  \begin{example}{Acrophonic System}
    \begin{itemize}
      \item Symbols for 1, 5, 10, 100, 1000, and \emph{sometimes} for 50, 500, etc.
      \item Not place-valued, just add symbols.
      \item Lead to ``Roman numerals'', which were used in Europe till the late Middle Ages (\ttilde 1500).
    \end{itemize}
  \end{example}
  \begin{example}{Alphabetic System (most common)}
    \begin{itemize}
      \item 27 letters of the alphabet were used for 1--9, 10--90, and 100--900.
      \item This was used for numbers up to 999, and it becomes similar to a place-valued system.
      \item The symbol \textbf{M} was used to represent ``multiplication by 10000''.
    \end{itemize}
  \end{example}
\end{document}
