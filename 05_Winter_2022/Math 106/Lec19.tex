\documentclass[class=article, crop=false]{standalone}
\usepackage[margin=1in]{geometry}

\usepackage{amsmath}
\usepackage{amssymb}
\usepackage{amsthm}
\usepackage{comment}
\usepackage{enumitem}
\usepackage{fancyhdr}
\usepackage{hyperref}
\usepackage{mathrsfs}
\usepackage{mathtools}
\usepackage{standalone}
\usepackage{tcolorbox}
\usepackage{tikz}
\usepackage{xcolor}

\usetikzlibrary{decorations.pathreplacing}
\tcbuselibrary{skins}

\DeclareMathOperator{\lcm}{lcm}
\DeclareMathOperator{\proj}{proj}
\DeclareMathOperator{\vspan}{span}
\DeclareMathOperator{\im}{im}
\DeclareMathOperator{\range}{range}
\DeclareMathOperator{\Diff}{Diff}
\DeclareMathOperator{\Int}{Int}
\DeclareMathOperator{\fcn}{fcn}
\DeclareMathOperator{\id}{id}

\newcommand{\N}{\ensuremath{\mathbb{N}}}
\newcommand{\Z}{\ensuremath{\mathbb{Z}}}
\newcommand{\Q}{\ensuremath{\mathbb{Q}}}
\newcommand{\R}{\ensuremath{\mathbb{R}}}
\newcommand{\C}{\ensuremath{\mathbb{C}}}
\newcommand{\F}{\ensuremath{\mathbb{F}}}
\newcommand{\lam}{\ensuremath{\lambda}}
\newcommand{\nab}{\ensuremath{\nabla}}
\newcommand{\eps}{\ensuremath{\varepsilon}}
\newcommand{\es}{\ensuremath{\varnothing}}

\newcommand{\abs}[1]{\ensuremath{\left\lvert #1 \right\rvert}}
\newcommand{\bigpar}[1]{\ensuremath{\left( #1 \right)}}
\newcommand{\norm}[1]{\ensuremath{\lVert #1\rVert}}
\newcommand{\set}[1]{\ensuremath{\left\{#1\right\}}}
\newcommand{\tuple}[1]{\ensuremath{\left\langle #1 \right\rangle}}
\newcommand{\floor}[1]{\ensuremath{\left\lfloor #1 \right\rfloor}}
\newcommand{\ceil}[1]{\ensuremath{\left\lceil #1 \right\rceil}}

\newcommand{\chapternum}{}
\newcommand{\ex}[1]{\noindent\textbf{Exercise \chapternum.{#1}.}}

\newcommand{\dx}[1]{\,\mathrm{d}#1}
\newcommand{\inv}{\ensuremath{^{-1}}}
\newcommand{\sm}{\setminus}
\newcommand{\sse}{\subseteq}
\newcommand{\ceq}{\coloneqq}

\definecolor{problemBackground}{RGB}{212,232,246}
\newenvironment{problem}[1]
  {
    \begin{tcolorbox}[
      boxrule=.5pt,
      titlerule=.5pt,
      sharp corners,
      colback=problemBackground
    ]
    \ifx &#1& \textbf{Problem. }
    \else \textbf{Problem #1.} \fi
  }
  {
    \end{tcolorbox}
  }

\definecolor{exampleBackground}{RGB}{255,249,248}
\definecolor{exampleAccent}{RGB}{158,60,14}
\newenvironment{example}[1]
  {
    \begin{tcolorbox}[
      boxrule=.5pt,
      sharp corners,
      colback=exampleBackground,
      colframe=exampleAccent,
    ]
    \color{exampleAccent}\textbf{Example.} \emph{#1}\color{black}
  }
  {
    \end{tcolorbox}
  }

\definecolor{theoremBackground}{RGB}{234,243,251}
\definecolor{theoremAccent}{RGB}{0,116,183}
\newenvironment{theorem}[1]
  {
    \begin{tcolorbox}[
      boxrule=.5pt,
      titlerule=.5pt,
      sharp corners,
      colback=theoremBackground,
      colframe=theoremAccent
    ]
      \color{theoremAccent}\textbf{Theorem --- }\emph{#1}\\\color{black}
  }
  {
    \end{tcolorbox}
  }

\definecolor{noteBackground}{RGB}{244,249,244}
\definecolor{noteAccent}{RGB}{34,139,34}
\newenvironment{note}[1]
  {
  \begin{tcolorbox}[
    enhanced,
    boxrule=0pt,
    frame hidden,
    sharp corners,
    colback=noteBackground,
    borderline west={3pt}{-1.5pt}{noteAccent}
    ]
    \ifx &#1& \color{noteAccent}\textbf{Note. }\color{black}
    \else \color{noteAccent}\textbf{Note (#1). }\color{black} \fi
    }
    {
  \end{tcolorbox}
  }

\definecolor{definitionBackground}{RGB}{246,246,246}
\newenvironment{definition}[1]
  {
    \begin{tcolorbox}[
      enhanced,
      boxrule=0pt,
      frame hidden,
      sharp corners,
      colback=definitionBackground,
      borderline west={3pt}{-1.5pt}{black}
    ]
    \textbf{Definition. }\emph{#1}\\
  }
  {
    \end{tcolorbox}
  }
\newenvironment{amatrix}[2]{
    \left[
      \begin{array}{*{#1}{c}|*{#2}c}
  }
  {
      \end{array}
    \right]
  }
\date{\the\year-\the\month-\the\day}
\author{Kyle Chui}


\fancyhf{}
\lhead{Kyle Chui}
\rhead{Page \thepage}
\pagestyle{fancy}

\begin{document}
  \section{Lecture 19}
  \begin{note}{}
    If $k > 1$, we can sometimes use it to find solutions.
  \end{note}
  Suppose $x_1^2 - ny_1^2 = k$. Composing with itself, we get a solution $(x_3, y_3)$ to $x^2 - ny^2 = k^2$, i.e.
  \[
    \frac{x_3^2}{k^2} - \frac{ny_3^2}{k^2} = 1,
  \]
  so $\paren{\frac{x_3}{k}, \frac{y_3}{k}}$ is a solution to the Pell equation.
  \begin{note}{}
    This is a \emph{rational} solution to the Pell equation, but not necessarily an \emph{integer} solution.
  \end{note}
  We hope that by composing a rational solution with itself enough times, we can find an integer solution.
  \begin{example}{}
    Consider the equation
    \[
      x^2 - 92y^2 = 1.
    \]
    We take our first guess to be ``small'', so $(x_1, y_1) = (10, 1)$. Hence we have $x_1^2 - 92y_1^2 = 8$. Composing this solution with itself, we have $(10\cdot 10 + 92\cdot 1\cdot 1, 10\cdot 1 + 1\cdot 10, 8^2) = (192, 20, 64)$. Dividing by $8^2$, we get the rational solution $(24, \frac{5}{2}, 1)$. We can then compose this new solution with itself, yielding $(24\cdot 24 + 92\cdot \frac{5}{2}\cdot \frac{5}{2}, 24\cdot \frac{5}{2} + \frac{5}{2}\cdot 24, 1^2) = (1151, 460, 1)$. Therefore we have $(x, y, k) = (1151, 120, 1)$ is an integer solution to
    \[
      x^2 - 92y^2 = 1.
    \]
  \end{example}
  \textbf{Question.} Does this method always work? \par
  No, but it does if we can find $x, y$ such that $x^2 - ny^2 = k$ for $k = \pm 1, \pm 2, \pm 4$. \par
  500 years later, Bhaskara II discovered a new method, called the ``cyclic process'', which always leads to a solution where $k = \pm 1, \pm 2, \pm 4$.
  \begin{note}{}
    A proof that this always works was only given in the 18\tsup{th} century by Lagrange.
  \end{note}
  \subsection{Bhaskara's Method}
  \begin{itemize}
    \item Choose $a, b$ that are co prime, so $\gcd(a, b) = 1$, and $k \geq 1$ such that $a^2 - nb^2 = k$. In other words, the triple $(a, b, k)$ is a solution to the equation.
    \item Choose $m$ and consider the triple $(m, 1, m^2 - n)$.
    \item We can then compose them to get the triple $(am + nb, a + bm, k(m^2 - n))$.
    \item Divide all sides of the equation by $k^2$ to get
    \[
      \paren{\frac{am + nb}{k}}^2 - n\paren{\frac{a + bm}{k}}^2 = \frac{m^2 - n}{k}.
    \]
    If we choose $m$ such that $\frac{m^2 - n}{k}$ is an integer, then also $\frac{am + nb}{k}$ and $\frac{a + bm}{k}$ are integers as well.
  \end{itemize}
  \textbf{Idea.} Choose $m$ in such a way that $\frac{m^2 - n}{k}$ is as small as possible. \\
  \textbf{Fact.} Repeating this process eventually yields a solution with $\frac{m^2 - n}{k} = \pm 1, \pm 2, \pm 4$. 
\end{document}
