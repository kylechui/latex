\documentclass[class=article, crop=false]{standalone}
\usepackage[margin=1in]{geometry}

\usepackage{amsmath}
\usepackage{amssymb}
\usepackage{amsthm}
\usepackage{comment}
\usepackage{enumitem}
\usepackage{fancyhdr}
\usepackage{hyperref}
\usepackage{mathrsfs}
\usepackage{mathtools}
\usepackage{standalone}
\usepackage{tcolorbox}
\usepackage{tikz}
\usepackage{xcolor}

\usetikzlibrary{decorations.pathreplacing}
\tcbuselibrary{skins}

\DeclareMathOperator{\lcm}{lcm}
\DeclareMathOperator{\proj}{proj}
\DeclareMathOperator{\vspan}{span}
\DeclareMathOperator{\im}{im}
\DeclareMathOperator{\range}{range}
\DeclareMathOperator{\Diff}{Diff}
\DeclareMathOperator{\Int}{Int}
\DeclareMathOperator{\fcn}{fcn}
\DeclareMathOperator{\id}{id}

\newcommand{\N}{\ensuremath{\mathbb{N}}}
\newcommand{\Z}{\ensuremath{\mathbb{Z}}}
\newcommand{\Q}{\ensuremath{\mathbb{Q}}}
\newcommand{\R}{\ensuremath{\mathbb{R}}}
\newcommand{\C}{\ensuremath{\mathbb{C}}}
\newcommand{\F}{\ensuremath{\mathbb{F}}}
\newcommand{\lam}{\ensuremath{\lambda}}
\newcommand{\nab}{\ensuremath{\nabla}}
\newcommand{\eps}{\ensuremath{\varepsilon}}
\newcommand{\es}{\ensuremath{\varnothing}}

\newcommand{\abs}[1]{\ensuremath{\left\lvert #1 \right\rvert}}
\newcommand{\bigpar}[1]{\ensuremath{\left( #1 \right)}}
\newcommand{\norm}[1]{\ensuremath{\lVert #1\rVert}}
\newcommand{\set}[1]{\ensuremath{\left\{#1\right\}}}
\newcommand{\tuple}[1]{\ensuremath{\left\langle #1 \right\rangle}}
\newcommand{\floor}[1]{\ensuremath{\left\lfloor #1 \right\rfloor}}
\newcommand{\ceil}[1]{\ensuremath{\left\lceil #1 \right\rceil}}

\newcommand{\chapternum}{}
\newcommand{\ex}[1]{\noindent\textbf{Exercise \chapternum.{#1}.}}

\newcommand{\dx}[1]{\,\mathrm{d}#1}
\newcommand{\inv}{\ensuremath{^{-1}}}
\newcommand{\sm}{\setminus}
\newcommand{\sse}{\subseteq}
\newcommand{\ceq}{\coloneqq}

\definecolor{problemBackground}{RGB}{212,232,246}
\newenvironment{problem}[1]
  {
    \begin{tcolorbox}[
      boxrule=.5pt,
      titlerule=.5pt,
      sharp corners,
      colback=problemBackground
    ]
    \ifx &#1& \textbf{Problem. }
    \else \textbf{Problem #1.} \fi
  }
  {
    \end{tcolorbox}
  }

\definecolor{exampleBackground}{RGB}{255,249,248}
\definecolor{exampleAccent}{RGB}{158,60,14}
\newenvironment{example}[1]
  {
    \begin{tcolorbox}[
      boxrule=.5pt,
      sharp corners,
      colback=exampleBackground,
      colframe=exampleAccent,
    ]
    \color{exampleAccent}\textbf{Example.} \emph{#1}\color{black}
  }
  {
    \end{tcolorbox}
  }

\definecolor{theoremBackground}{RGB}{234,243,251}
\definecolor{theoremAccent}{RGB}{0,116,183}
\newenvironment{theorem}[1]
  {
    \begin{tcolorbox}[
      boxrule=.5pt,
      titlerule=.5pt,
      sharp corners,
      colback=theoremBackground,
      colframe=theoremAccent
    ]
      \color{theoremAccent}\textbf{Theorem --- }\emph{#1}\\\color{black}
  }
  {
    \end{tcolorbox}
  }

\definecolor{noteBackground}{RGB}{244,249,244}
\definecolor{noteAccent}{RGB}{34,139,34}
\newenvironment{note}[1]
  {
  \begin{tcolorbox}[
    enhanced,
    boxrule=0pt,
    frame hidden,
    sharp corners,
    colback=noteBackground,
    borderline west={3pt}{-1.5pt}{noteAccent}
    ]
    \ifx &#1& \color{noteAccent}\textbf{Note. }\color{black}
    \else \color{noteAccent}\textbf{Note (#1). }\color{black} \fi
    }
    {
  \end{tcolorbox}
  }

\definecolor{definitionBackground}{RGB}{246,246,246}
\newenvironment{definition}[1]
  {
    \begin{tcolorbox}[
      enhanced,
      boxrule=0pt,
      frame hidden,
      sharp corners,
      colback=definitionBackground,
      borderline west={3pt}{-1.5pt}{black}
    ]
    \textbf{Definition. }\emph{#1}\\
  }
  {
    \end{tcolorbox}
  }
\newenvironment{amatrix}[2]{
    \left[
      \begin{array}{*{#1}{c}|*{#2}c}
  }
  {
      \end{array}
    \right]
  }
\date{\the\year-\the\month-\the\day}
\author{Kyle Chui}


\fancyhf{}
\lhead{Kyle Chui}
\rhead{Page \thepage}
\pagestyle{fancy}

\begin{document}
  \section{Lecture 9}
  \subsection{Greek Number Theory}
  \textbf{Fact.} The basics of number theory are contained in the Elements, mostly connected to the ``Euclidean algorithm''.
  \begin{itemize}
    \item The only other major advancement in the Greek empire: Diophantos (circa 250 AD, Alexandria), who studied Diophantine equations
    \item Later, there are much more advances in India, China, the Middle East, etc.
  \end{itemize}
  \subsubsection{Polygonal, Prime, Perfect Numbers}
  Polygonal numbers were studied by the Pythagoreans. We start with triangular numbers, which are just equilateral triangles consisting of dots, like $\therefore$. They go $1, 3, 6, 10,\dotsc$. Square numbers are what you think they are, just perfect squares. The pentagonal numbers start $1, 5, 12, 22,\dotsc$.
  \begin{note}{}
    While interesting, the polygonal numbers don't seem to have many uses. They are also given by quadratic equations, i.e.the $n$\tsup{th} pentagonal number is given by
    \[
      P_n = \frac{3n^2 - n}{2}.
    \]
  \end{note}
  \subsubsection{Prime Numbers}
  \begin{definition}{Prime}
    A number is \emph{prime} if it is only divisible by 1 and themselves.
  \end{definition}
  \begin{note}{}
    The Greeks called them ``numbers with no rectangular representations'', where a rectangular representation was a drawing of $n$ dots that wasn't just a line of dots.
  \end{note}
  Euclid proved that there are infinitely many prime numbers, ``Prime numbers are more than any assigned multitude of primes''.
  \begin{proof}
    Suppose towards a contradiction that there are finitely many primes, say $p_1,\dotsc,p_n$. Then observe that $p_1p_2\dotsb p_n + 1$ is not divisible by any $p_i$ for $i = 1,\dotsc,n$, and so must be prime. Hence there must be infinitely many primes.
  \end{proof}
  \subsubsection{Perfect Numbers}
  \begin{definition}{Perfect Numbers}
    A \emph{perfect} number is a number that equals the sum of its divisors (excluding itself).
  \end{definition}
  For example, $6 = 1 + 2 + 3$ and $28 = 1 + 2 + 7 + 14$. \par
  \textbf{Fact.} We know that $k$ is an even perfect number if and only if $k$ takes the form
  \[
    k = 2^{n - 1}(2^n - 1)
  \]
  for $n\in \N$ and $2^n - 1$ is prime (Mersenne prime).
  \paragraph{Open Problems}
  \begin{itemize}
    \item Are there any odd perfect numbers?
    \item Are there infinitely many even perfect numbers/Mersenne primes (51 known)?
  \end{itemize}
\end{document}
