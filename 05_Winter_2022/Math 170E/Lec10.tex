\documentclass[class=article, crop=false]{standalone}
% Import packages
\usepackage[margin=1in]{geometry}

\usepackage{amssymb, amsthm}
\usepackage{comment}
\usepackage{enumitem}
\usepackage{fancyhdr}
\usepackage{hyperref}
\usepackage{import}
\usepackage{mathrsfs, mathtools}
\usepackage{multicol}
\usepackage{pdfpages}
\usepackage{standalone}
\usepackage{transparent}
\usepackage{xcolor}

\usepackage{minted}
\usepackage[many]{tcolorbox}
\tcbuselibrary{minted}

% Declare math operators
\DeclareMathOperator{\lcm}{lcm}
\DeclareMathOperator{\proj}{proj}
\DeclareMathOperator{\vspan}{span}
\DeclareMathOperator{\im}{im}
\DeclareMathOperator{\Diff}{Diff}
\DeclareMathOperator{\Int}{Int}
\DeclareMathOperator{\fcn}{fcn}
\DeclareMathOperator{\id}{id}
\DeclareMathOperator{\rank}{rank}
\DeclareMathOperator{\tr}{tr}
\DeclareMathOperator{\dive}{div}
\DeclareMathOperator{\row}{row}
\DeclareMathOperator{\col}{col}
\DeclareMathOperator{\dom}{dom}
\DeclareMathOperator{\ran}{ran}
\DeclareMathOperator{\Dom}{Dom}
\DeclareMathOperator{\Ran}{Ran}
\DeclareMathOperator{\fl}{fl}
% Macros for letters/variables
\newcommand\ttilde{\kern -.15em\lower .7ex\hbox{\~{}}\kern .04em}
\newcommand{\N}{\ensuremath{\mathbb{N}}}
\newcommand{\Z}{\ensuremath{\mathbb{Z}}}
\newcommand{\Q}{\ensuremath{\mathbb{Q}}}
\newcommand{\R}{\ensuremath{\mathbb{R}}}
\newcommand{\C}{\ensuremath{\mathbb{C}}}
\newcommand{\F}{\ensuremath{\mathbb{F}}}
\newcommand{\M}{\ensuremath{\mathbb{M}}}
\renewcommand{\P}{\ensuremath{\mathbb{P}}}
\newcommand{\lam}{\ensuremath{\lambda}}
\newcommand{\nab}{\ensuremath{\nabla}}
\newcommand{\eps}{\ensuremath{\varepsilon}}
\newcommand{\es}{\ensuremath{\varnothing}}
% Macros for math symbols
\newcommand{\dx}[1]{\,\mathrm{d}#1}
\newcommand{\inv}{\ensuremath{^{-1}}}
\newcommand{\sm}{\setminus}
\newcommand{\sse}{\subseteq}
\newcommand{\ceq}{\coloneqq}
% Macros for pairs of math symbols
\newcommand{\abs}[1]{\ensuremath{\left\lvert #1 \right\rvert}}
\newcommand{\paren}[1]{\ensuremath{\left( #1 \right)}}
\newcommand{\norm}[1]{\ensuremath{\left\lVert #1\right\rVert}}
\newcommand{\set}[1]{\ensuremath{\left\{#1\right\}}}
\newcommand{\tup}[1]{\ensuremath{\left\langle #1 \right\rangle}}
\newcommand{\floor}[1]{\ensuremath{\left\lfloor #1 \right\rfloor}}
\newcommand{\ceil}[1]{\ensuremath{\left\lceil #1 \right\rceil}}
\newcommand{\eclass}[1]{\ensuremath{\left[ #1 \right]}}

\newcommand{\chapternum}{}
\newcommand{\ex}[1]{\noindent\textbf{Exercise \chapternum.{#1}.}}

\newcommand{\tsub}[1]{\textsubscript{#1}}
\newcommand{\tsup}[1]{\textsuperscript{#1}}

\renewcommand{\choose}[2]{\ensuremath{\phantom{}_{#1}C_{#2}}}
\newcommand{\perm}[2]{\ensuremath{\phantom{}_{#1}P_{#2}}}

% Include figures
\newcommand{\incfig}[2][1]{%
    \def\svgwidth{#1\columnwidth}
    \import{./figures/}{#2.pdf_tex}
}

\definecolor{problemBackground}{RGB}{212,232,246}
\newenvironment{problem}[1]
  {
    \begin{tcolorbox}[
      boxrule=.5pt,
      titlerule=.5pt,
      sharp corners,
      colback=problemBackground,
      breakable
    ]
    \ifx &#1& \textbf{Problem. }
    \else \textbf{Problem #1.} \fi
  }
  {
    \end{tcolorbox}
  }
\definecolor{exampleBackground}{RGB}{255,249,248}
\definecolor{exampleAccent}{RGB}{158,60,14}
\newenvironment{example}[1]
  {
    \begin{tcolorbox}[
      boxrule=.5pt,
      sharp corners,
      colback=exampleBackground,
      colframe=exampleAccent,
    ]
    \color{exampleAccent}\textbf{Example.} \emph{#1}\color{black}
  }
  {
    \end{tcolorbox}
  }
\definecolor{theoremBackground}{RGB}{234,243,251}
\definecolor{theoremAccent}{RGB}{0,116,183}
\newenvironment{theorem}[1]
  {
    \begin{tcolorbox}[
      boxrule=.5pt,
      titlerule=.5pt,
      sharp corners,
      colback=theoremBackground,
      colframe=theoremAccent,
      breakable
    ]
      % \color{theoremAccent}\textbf{Theorem --- }\emph{#1}\\\color{black}
    \ifx &#1& \color{theoremAccent}\textbf{Theorem. }\color{black}
    \else \color{theoremAccent}\textbf{Theorem --- }\emph{#1}\\\color{black} \fi
  }
  {
    \end{tcolorbox}
  }
\definecolor{noteBackground}{RGB}{244,249,244}
\definecolor{noteAccent}{RGB}{34,139,34}
\newenvironment{note}[1]
  {
  \begin{tcolorbox}[
    enhanced,
    boxrule=0pt,
    frame hidden,
    sharp corners,
    colback=noteBackground,
    borderline west={3pt}{-1.5pt}{noteAccent},
    breakable
    ]
    \ifx &#1& \color{noteAccent}\textbf{Note. }\color{black}
    \else \color{noteAccent}\textbf{Note (#1). }\color{black} \fi
    }
    {
  \end{tcolorbox}
  }
\definecolor{lemmaBackground}{RGB}{255,247,234}
\definecolor{lemmaAccent}{RGB}{255,153,0}
\newenvironment{lemma}[1]
  {
    \begin{tcolorbox}[
      enhanced,
      boxrule=0pt,
      frame hidden,
      sharp corners,
      colback=lemmaBackground,
      borderline west={3pt}{-1.5pt}{lemmaAccent},
      breakable
    ]
    \ifx &#1& \color{lemmaAccent}\textbf{Lemma. }\color{black}
    \else \color{lemmaAccent}\textbf{Lemma #1. }\color{black} \fi
  }
  {
    \end{tcolorbox}
  }
\definecolor{definitionBackground}{RGB}{246,246,246}
\newenvironment{definition}[1]
  {
    \begin{tcolorbox}[
      enhanced,
      boxrule=0pt,
      frame hidden,
      sharp corners,
      colback=definitionBackground,
      borderline west={3pt}{-1.5pt}{black},
      breakable
    ]
    \textbf{Definition. }\emph{#1}\\
  }
  {
    \end{tcolorbox}
  }

\newenvironment{amatrix}[2]{
    \left[
      \begin{array}{*{#1}{c}|*{#2}c}
  }
  {
      \end{array}
    \right]
  }

\definecolor{codeBackground}{RGB}{253,246,227}
\renewcommand\theFancyVerbLine{\arabic{FancyVerbLine}}
\newtcblisting{cpp}[1][]{
  boxrule=1pt,
  sharp corners,
  colback=codeBackground,
  listing only,
  breakable,
  minted language=cpp,
  minted style=material,
  minted options={
    xleftmargin=15pt,
    baselinestretch=1.2,
    linenos,
  }
}

\author{Kyle Chui}


\fancyhf{}
\lhead{Kyle Chui}
\rhead{Page \thepage}
\pagestyle{fancy}

\begin{document}
  \section{Lecture 10}
  \subsection{Binomial Random Variables}
  \begin{definition}{Binomial Random Variable}
    A \emph{Bernoulli trial} is an experiment that has probability $p\in (0, 1)$ of success and probability $1 - p$ of failure.
    \begin{itemize}
      \item Suppose we run $n \geq 1$ independent, identical Bernoulli trials, and let $X$ be the number of successes.
      \item Then we know that $X$ is a discrete random variable taking values in the set $S = \set{0,1,\dotsc,n}$.
      \item We say that $X$ is a \emph{Binomial random variable} with parameters $n, p$ and write $X\sim \Binomial(n, p)$.
    \end{itemize}
  \end{definition}
  \begin{theorem}{}
    If $X\sim \Binomial(n, p)$ then its PMF is
    \[
      p_X(x) = \binom{n}{x}p^x(1 - p)^{n - x}\quad\text{if}\quad x\in \set{0,1,\dotsc,n}.
    \]
    \begin{proof}
      Recall that $p_X(x) = \P[X = x]$. We know that there are $\binom{n}{x}$ ways to arrange exactly $x$ ``successes'' in a total of $n$ trials. Since each trial has a ``success'' rate of $p$, and a ``failure'' rate of $1 - p$, and they are independent, each arrangement has a $p^x(1 - p)^{n - x}$ probability of occurring. Hence the total probability of any of the arrangements occurring (as they are mutually independent) is
      \[
        p_X(x) = \binom{n}{x}p^x(1 - p)^{n - x}.
      \]
    \end{proof}
  \end{theorem}
  \begin{note}{}
    By the Binomial Theorem we have
    \[
      \sum_{x=0}^{n} p_X(x) = (p + (1 - p))^n = 1.
    \]
  \end{note}
  \begin{theorem}{}
    If $X\sim \Binomial(n, p)$, its MGF is
    \[
      M_X(t) = (1 - p + pe^t)^n.
    \]
    \begin{proof}
      We compute
      \begin{align*}
        M_X(t) &= \E[e^{tx}] \\
               &= \sum_{x=0}^{n}e^{tx}p_X(x) \\
               &= \sum_{x=0}^{n}e^{tx}\binom{n}{x}p^x(1 - p)^{n - x} \\
               &= \sum_{x=0}^{n}\binom{n}{x}(pe^t)^x(1 - p)^{n - x} \\
               &= (1 - p + pe^t)^n.
      \end{align*}
    \end{proof}
  \end{theorem}
  \begin{note}{}
    Recall that $(1 - p + pe^t)$ is the MFG of the Bernoulli random variable.
  \end{note}
  \begin{theorem}{}
    If $X\sim \Binomial(n, p)$ its mean is
    \[
      \E[X] = np.
    \]
    \begin{proof}
      Recall that $M_X(t) = (pe^t + 1 - p)^n$ so $M_X'(t) = n(pe^t + 1 - p)^{n - 1}\cdot pe^t$. Hence
      \[
        \E[X] = M_X'(0) = n(p + 1 - p)^{n - 1}\cdot p = np.
      \]
    \end{proof}
  \end{theorem}
  \begin{note}{}
    Recall that $p$ is the expected value of the Bernoulli random variable.
  \end{note}
  \begin{theorem}{}
    If $X\sim \Binomial(n, p)$ its variance is
    \[
      \var(X) = np(1 - p)
    \]
    \begin{proof}
      Recall that $\var(X) = \E[X^2] - \E[X]^2 = \E[X^2] - (np)^2$. Furthermore, we just showed that
      \[
        M_X'(t) = n(pe^t + 1 - p)^{n - 1}\cdot pe^t,
      \]
      so
      \[
        M_X''(t) = n(n - 1)(pe^t + 1 - p)^{n - 2}\cdot (pe^t)^2 + n(pe^t + 1 - p)^{n - 1}\cdot pe^t.
      \]
      Hence
      \begin{align*}
        \E[X^2] &= M_X''(0) \\
                &= n(n - 1)(p + 1 - p)^{n - 2}\cdot p^2 + n(p + 1 - p)^{n - 1}\cdot p \\
                &= n(n - 1)\cdot p^2 + np.
      \end{align*}
      Therefore
      \begin{align*}
        \var(X) &= n(n - 1)\cdot p^2 + np - (np)^2 \\
                &= n^2p^2 - np^2 + np - n^2p^2 \\
                &= np(1 - p).
      \end{align*}
    \end{proof}
  \end{theorem}
  \begin{note}{}
    Recall that $p(1 - p)$ is the variance of a regular Bernoulli random variable.
  \end{note}
  \subsection{Geometric Random Variables}
  \begin{definition}{Geometric Random Variable}
    Suppose we repeatedly run independent, identical Bernoulli trials with probability $p\in (0, 1)$ of success.
    \begin{itemize}
      \item Let $X$ be the trial on which we first achieve success.
      \item $X$ is a discrete random variable taking values in the set $S = \set{1,2,3,\dotsc}$.
      \item We say that $X$ is a \emph{geometric random variable} with parameter $p$ and write $X\sim \Geometric(p)$.
    \end{itemize}
  \end{definition}
  \begin{note}{}
    Sometimes geometric random variables are defined differently, but this is the definition that we will use for this class.
  \end{note}
  \begin{theorem}{}
    If $X\sim \Geometric(p)$ then its PMF is
    \[
      p_X(x) = (1 - p)^{x - 1}p\quad\text{if}\quad x\in \set{1,2,3,\dotsc}.
    \]
    \begin{proof}
      The probability that $X = x$ is the probability that we have $x - 1$ failures, followed by a success. Since the events are independent, this gives us the desired result.
    \end{proof}
  \end{theorem}
  \begin{note}{}
    For the infinite series version of this, we have
    \begin{align*}
      \sum_{x=1}^{\infty} p_X(x) &= \sum_{x=1}^{\infty} (1 - p)^{x - 1}p \\
                                 &= p \sum_{x=1}^{\infty}(1 - p)^{x - 1} \\
                                 &= p\cdot \frac{1}{1 - (1 - p)} \\
                                 &= 1.
    \end{align*}
  \end{note}
  \begin{theorem}{}
    If $X\sim \Geometric(p)$ then its MGF is 
    \[
      M_X(t) = \frac{e^tp}{1 - (1 - p)e^t}\quad\text{if}\quad t < -\ln(1 - p).
    \]
    \begin{proof}
      We have
      \begin{align*}
        M_X(t) &= \E[e^{tX}] \\
               &= \sum_{x=1}^{\infty} e^{tx}p_X(x) \\
               &= \sum_{x=1}^{\infty} e^{tx}(1 - p)^{x - 1}p \\
               &= e^tp \sum_{x=1}^{\infty} (e^t(1 - p))^{x - 1} \\
               &= e^tp\cdot \frac{1}{1 - (e^t(1 - p))} \\
               &= \frac{e^tp}{1 - (1 - p)e^t},
      \end{align*}
      as desired.
    \end{proof}
  \end{theorem}
  \begin{note}{}
    The condition comes from the denominator, where
    \[
      (1 - p)e^t < 1.
    \]
  \end{note}
  \begin{theorem}{}
    If $X\sim \Geometric(p)$ then its mean is
    \[
      \E[X] = \frac{1}{p}.
    \]
    \begin{proof}
      From the previous theorem, we showed that
      \[
        M_X(t) = \frac{e^tp}{1 - e^t(1 - p)}.
      \]
      Hence
      \begin{align*}
        \ln M_X(t) &= \ln \paren{\frac{e^tp}{1 - e^t(1 - p)}} \\
                   &= \ln (e^tp) - \ln(1 - e^t(1 - p)) \\
                   &= t + \ln p - \ln (1 - e^t(1 - p)).
      \end{align*}
      Thus we have that
      \[
        (\ln M_X)' = 1 - \frac{1}{1 - e^t(1 - p)}\cdot (p - 1)e^t = \frac{1}{1 - e^t(1 - p)}.
      \]
      Therefore
      \[
        \E[X] = (\ln M_X)'(0) = \frac{1}{p}.
      \]
    \end{proof}
  \end{theorem}
  \begin{theorem}{}
    If $X\sim \Geometric(p)$ then its variance is
    \[
      \var(X) = \frac{1 - p}{p^2}.
    \]
    \begin{proof}
      From the last problem we have that
      \[
        (\ln M_X)' = \frac{1}{1 - e^t(1 - p)},
      \]
      so
      \[
        (\ln M_X)'' = -\frac{1}{(1 - e^t(1 - p))^2}\cdot (p - 1)e^t = \frac{e^t(1 - p)}{(1 - e^t(1 - p))^2}.
      \]
      Therefore
      \[
        \var(X) = (\ln M_X)''(0) = \frac{1 - p}{p^2},
      \]
      as desired.
    \end{proof}
  \end{theorem}
\end{document}
