\documentclass[class=article, crop=false]{standalone}
\usepackage[margin=1in]{geometry}

\usepackage{amsmath}
\usepackage{amssymb}
\usepackage{amsthm}
\usepackage{comment}
\usepackage{enumitem}
\usepackage{fancyhdr}
\usepackage{hyperref}
\usepackage{mathrsfs}
\usepackage{mathtools}
\usepackage{standalone}
\usepackage{tcolorbox}
\usepackage{tikz}
\usepackage{xcolor}

\usetikzlibrary{decorations.pathreplacing}
\tcbuselibrary{skins}

\DeclareMathOperator{\lcm}{lcm}
\DeclareMathOperator{\proj}{proj}
\DeclareMathOperator{\vspan}{span}
\DeclareMathOperator{\im}{im}
\DeclareMathOperator{\range}{range}
\DeclareMathOperator{\Diff}{Diff}
\DeclareMathOperator{\Int}{Int}
\DeclareMathOperator{\fcn}{fcn}
\DeclareMathOperator{\id}{id}

\newcommand{\N}{\ensuremath{\mathbb{N}}}
\newcommand{\Z}{\ensuremath{\mathbb{Z}}}
\newcommand{\Q}{\ensuremath{\mathbb{Q}}}
\newcommand{\R}{\ensuremath{\mathbb{R}}}
\newcommand{\C}{\ensuremath{\mathbb{C}}}
\newcommand{\F}{\ensuremath{\mathbb{F}}}
\newcommand{\lam}{\ensuremath{\lambda}}
\newcommand{\nab}{\ensuremath{\nabla}}
\newcommand{\eps}{\ensuremath{\varepsilon}}
\newcommand{\es}{\ensuremath{\varnothing}}

\newcommand{\abs}[1]{\ensuremath{\left\lvert #1 \right\rvert}}
\newcommand{\bigpar}[1]{\ensuremath{\left( #1 \right)}}
\newcommand{\norm}[1]{\ensuremath{\lVert #1\rVert}}
\newcommand{\set}[1]{\ensuremath{\left\{#1\right\}}}
\newcommand{\tuple}[1]{\ensuremath{\left\langle #1 \right\rangle}}
\newcommand{\floor}[1]{\ensuremath{\left\lfloor #1 \right\rfloor}}
\newcommand{\ceil}[1]{\ensuremath{\left\lceil #1 \right\rceil}}

\newcommand{\chapternum}{}
\newcommand{\ex}[1]{\noindent\textbf{Exercise \chapternum.{#1}.}}

\newcommand{\dx}[1]{\,\mathrm{d}#1}
\newcommand{\inv}{\ensuremath{^{-1}}}
\newcommand{\sm}{\setminus}
\newcommand{\sse}{\subseteq}
\newcommand{\ceq}{\coloneqq}

\definecolor{problemBackground}{RGB}{212,232,246}
\newenvironment{problem}[1]
  {
    \begin{tcolorbox}[
      boxrule=.5pt,
      titlerule=.5pt,
      sharp corners,
      colback=problemBackground
    ]
    \ifx &#1& \textbf{Problem. }
    \else \textbf{Problem #1.} \fi
  }
  {
    \end{tcolorbox}
  }

\definecolor{exampleBackground}{RGB}{255,249,248}
\definecolor{exampleAccent}{RGB}{158,60,14}
\newenvironment{example}[1]
  {
    \begin{tcolorbox}[
      boxrule=.5pt,
      sharp corners,
      colback=exampleBackground,
      colframe=exampleAccent,
    ]
    \color{exampleAccent}\textbf{Example.} \emph{#1}\color{black}
  }
  {
    \end{tcolorbox}
  }

\definecolor{theoremBackground}{RGB}{234,243,251}
\definecolor{theoremAccent}{RGB}{0,116,183}
\newenvironment{theorem}[1]
  {
    \begin{tcolorbox}[
      boxrule=.5pt,
      titlerule=.5pt,
      sharp corners,
      colback=theoremBackground,
      colframe=theoremAccent
    ]
      \color{theoremAccent}\textbf{Theorem --- }\emph{#1}\\\color{black}
  }
  {
    \end{tcolorbox}
  }

\definecolor{noteBackground}{RGB}{244,249,244}
\definecolor{noteAccent}{RGB}{34,139,34}
\newenvironment{note}[1]
  {
  \begin{tcolorbox}[
    enhanced,
    boxrule=0pt,
    frame hidden,
    sharp corners,
    colback=noteBackground,
    borderline west={3pt}{-1.5pt}{noteAccent}
    ]
    \ifx &#1& \color{noteAccent}\textbf{Note. }\color{black}
    \else \color{noteAccent}\textbf{Note (#1). }\color{black} \fi
    }
    {
  \end{tcolorbox}
  }

\definecolor{definitionBackground}{RGB}{246,246,246}
\newenvironment{definition}[1]
  {
    \begin{tcolorbox}[
      enhanced,
      boxrule=0pt,
      frame hidden,
      sharp corners,
      colback=definitionBackground,
      borderline west={3pt}{-1.5pt}{black}
    ]
    \textbf{Definition. }\emph{#1}\\
  }
  {
    \end{tcolorbox}
  }
\newenvironment{amatrix}[2]{
    \left[
      \begin{array}{*{#1}{c}|*{#2}c}
  }
  {
      \end{array}
    \right]
  }
\date{\the\year-\the\month-\the\day}
\author{Kyle Chui}


\fancyhf{}
\lhead{Kyle Chui}
\rhead{Page \thepage}
\pagestyle{fancy}

\begin{document}
  \section{Lecture 1}
  The goal of this class is to \emph{quantify randomness}. The main topics for the term are:
  \begin{enumerate}
    \item The fundamentals of probability theory, including conditional probability and enumeration arguments.
    \item Discrete and continuous random variables.
    \item Sequences of i.i.d. random variables, including the Weak Law of Large Numbers and the Central Limit Theorem.
  \end{enumerate}
  \subsection{Properties of Probability}
  Probability theory takes place inside a \emph{probability space} $(\Omega, \mathcal{F}, \P)$.
  \begin{definition}{Probability Space}
    A \emph{probability space} is a triplet $(\Omega, \mathcal{F}, \P)$ satisfying:
    \begin{enumerate}
      \item A non-empty set $\Omega$, called the \emph{sample space}.
      \item A set $\mathcal{F}$ of subsets of $\Omega$ satisfying certain properties:
      \begin{itemize}
        \item Elements of $\mathcal{F}$ are called \emph{events}.
        \item Events $A_1, A_2, \dotsc, A_k$ are called \emph{mutually exclusive} if they are \emph{pairwise disjoint}, i.e. if $i\neq j$ then $A_i\cap A_j = \es$.
        \item Events $A_1, A_2, \dotsc, A_k$ are called \emph{exhaustive} if their union is the sample space, i.e.
        \[
          \bigcup_{j = 1}^k A_j = \Omega.
        \]
        \item For this class you may ignore $\mathcal{F}$ and assume that all subsets of $\Omega$ are events.
      \end{itemize}
      \item A function $\P\colon \mathcal{F}\to [0, 1]$, called a \emph{probability measure}, which satisfies:
      \begin{itemize}
        \item $\P[\Omega] = 1$, or ``the probability that something happens is $1$''.
        \item If $A_1, A_2, \dotsc, A_n$ are mutually exclusive events, then
        \[
          \P\left[\bigcup_{j = 1}^n A_j\right] = \sum_{j=1}^{n} \mathbb{P}[A_j].
        \]
        \item If $A_1, A_2, \dotsc$ are mutually exclusive events, then
        \[
          \P\left[\bigcup_{j = 1}^\infty A_j\right] = \sum_{j=1}^{\infty} \mathbb{P}[A_j].
        \]
      \end{itemize}
    \end{enumerate}
  \end{definition}
  \begin{example}{}
    Suppose I flip two fair coins. Then the sample space can be written as $\Omega = \set{HH, HT, TH, TT}$. The probability measure should be defined as
    \begin{align*}
      P[HH] &= \frac{1}{4} \\ 
      P[HT] &= \frac{1}{4} \\ 
      P[TH] &= \frac{1}{4} \\ 
      P[TT] &= \frac{1}{4}.
    \end{align*}
    The probability of getting exactly one head is hence $\P[\set{HT, TH}] = \mathbb{P}[HT] + \mathbb{P}[TH] = \frac{1}{2}$.
  \end{example}
  \begin{theorem}{}
    $\P[\es] = 0$.
    \begin{proof}
      We know that $\Omega$ and $\es$ are mutually exclusive, since, $\Omega\cap \es = \es$. Thus
      \begin{align*}
        \P[\Omega] &= \P[\Omega\cup \es] \\
                  &= \P[\Omega] + \P[\es],
      \end{align*}
      and so $\P[\es] = 0$.
    \end{proof}
  \end{theorem}
  \begin{theorem}{}
    If $A\sse \Omega$ is an event and $A' = \Omega\sm A$ then
    \[
      \P[A] = 1 - \P[A'].
    \]
    \begin{proof}
      Since we have that $A' = \Omega\sm A$, we know that $A'\cap A = \es$, so they are mutually exclusive. Thus we have
      \begin{align*}
        \P[\Omega] &= \P[A\cup A'] \\
        1 &= \P[A] + \P[A'] \\ 
        \P[A] &= 1 - \P[A'].
      \end{align*}
    \end{proof}
  \end{theorem}
  \begin{theorem}{}
    If $A\sse B$ then
    \[
      \P[B\sm A] = \P[B] - \P[A].
    \]
    \begin{proof}
      We know that $B = A\cup (B\sm A)$ and $A\cap(B\sm A) = \es$. Hence 
      \[
        \P[B] = \P[A\cup (B\sm A)] = \P[A] + \P[B\sm A],
      \]
      and the result follows.
    \end{proof}
  \end{theorem}
  \begin{theorem}{}
    If $A\sse B$ then $\P[A]\leq \P[B]$.
    \begin{proof}
      From the previous theorem we have
      \[
        \P[A]\leq \P[A] + \P[B\sm A] = \P[B].
      \]
    \end{proof}
  \end{theorem}
\end{document}
