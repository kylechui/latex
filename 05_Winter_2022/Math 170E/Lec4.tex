\documentclass[class=article, crop=false]{standalone}
% Import packages
\usepackage[margin=1in]{geometry}

\usepackage{amssymb, amsthm}
\usepackage{comment}
\usepackage{enumitem}
\usepackage{fancyhdr}
\usepackage{hyperref}
\usepackage{import}
\usepackage{mathrsfs, mathtools}
\usepackage{multicol}
\usepackage{pdfpages}
\usepackage{standalone}
\usepackage{transparent}
\usepackage{xcolor}

\usepackage{minted}
\usepackage[many]{tcolorbox}
\tcbuselibrary{minted}

% Declare math operators
\DeclareMathOperator{\lcm}{lcm}
\DeclareMathOperator{\proj}{proj}
\DeclareMathOperator{\vspan}{span}
\DeclareMathOperator{\im}{im}
\DeclareMathOperator{\Diff}{Diff}
\DeclareMathOperator{\Int}{Int}
\DeclareMathOperator{\fcn}{fcn}
\DeclareMathOperator{\id}{id}
\DeclareMathOperator{\rank}{rank}
\DeclareMathOperator{\tr}{tr}
\DeclareMathOperator{\dive}{div}
\DeclareMathOperator{\row}{row}
\DeclareMathOperator{\col}{col}
\DeclareMathOperator{\dom}{dom}
\DeclareMathOperator{\ran}{ran}
\DeclareMathOperator{\Dom}{Dom}
\DeclareMathOperator{\Ran}{Ran}
\DeclareMathOperator{\fl}{fl}
% Macros for letters/variables
\newcommand\ttilde{\kern -.15em\lower .7ex\hbox{\~{}}\kern .04em}
\newcommand{\N}{\ensuremath{\mathbb{N}}}
\newcommand{\Z}{\ensuremath{\mathbb{Z}}}
\newcommand{\Q}{\ensuremath{\mathbb{Q}}}
\newcommand{\R}{\ensuremath{\mathbb{R}}}
\newcommand{\C}{\ensuremath{\mathbb{C}}}
\newcommand{\F}{\ensuremath{\mathbb{F}}}
\newcommand{\M}{\ensuremath{\mathbb{M}}}
\renewcommand{\P}{\ensuremath{\mathbb{P}}}
\newcommand{\lam}{\ensuremath{\lambda}}
\newcommand{\nab}{\ensuremath{\nabla}}
\newcommand{\eps}{\ensuremath{\varepsilon}}
\newcommand{\es}{\ensuremath{\varnothing}}
% Macros for math symbols
\newcommand{\dx}[1]{\,\mathrm{d}#1}
\newcommand{\inv}{\ensuremath{^{-1}}}
\newcommand{\sm}{\setminus}
\newcommand{\sse}{\subseteq}
\newcommand{\ceq}{\coloneqq}
% Macros for pairs of math symbols
\newcommand{\abs}[1]{\ensuremath{\left\lvert #1 \right\rvert}}
\newcommand{\paren}[1]{\ensuremath{\left( #1 \right)}}
\newcommand{\norm}[1]{\ensuremath{\left\lVert #1\right\rVert}}
\newcommand{\set}[1]{\ensuremath{\left\{#1\right\}}}
\newcommand{\tup}[1]{\ensuremath{\left\langle #1 \right\rangle}}
\newcommand{\floor}[1]{\ensuremath{\left\lfloor #1 \right\rfloor}}
\newcommand{\ceil}[1]{\ensuremath{\left\lceil #1 \right\rceil}}
\newcommand{\eclass}[1]{\ensuremath{\left[ #1 \right]}}

\newcommand{\chapternum}{}
\newcommand{\ex}[1]{\noindent\textbf{Exercise \chapternum.{#1}.}}

\newcommand{\tsub}[1]{\textsubscript{#1}}
\newcommand{\tsup}[1]{\textsuperscript{#1}}

\renewcommand{\choose}[2]{\ensuremath{\phantom{}_{#1}C_{#2}}}
\newcommand{\perm}[2]{\ensuremath{\phantom{}_{#1}P_{#2}}}

% Include figures
\newcommand{\incfig}[2][1]{%
    \def\svgwidth{#1\columnwidth}
    \import{./figures/}{#2.pdf_tex}
}

\definecolor{problemBackground}{RGB}{212,232,246}
\newenvironment{problem}[1]
  {
    \begin{tcolorbox}[
      boxrule=.5pt,
      titlerule=.5pt,
      sharp corners,
      colback=problemBackground,
      breakable
    ]
    \ifx &#1& \textbf{Problem. }
    \else \textbf{Problem #1.} \fi
  }
  {
    \end{tcolorbox}
  }
\definecolor{exampleBackground}{RGB}{255,249,248}
\definecolor{exampleAccent}{RGB}{158,60,14}
\newenvironment{example}[1]
  {
    \begin{tcolorbox}[
      boxrule=.5pt,
      sharp corners,
      colback=exampleBackground,
      colframe=exampleAccent,
    ]
    \color{exampleAccent}\textbf{Example.} \emph{#1}\color{black}
  }
  {
    \end{tcolorbox}
  }
\definecolor{theoremBackground}{RGB}{234,243,251}
\definecolor{theoremAccent}{RGB}{0,116,183}
\newenvironment{theorem}[1]
  {
    \begin{tcolorbox}[
      boxrule=.5pt,
      titlerule=.5pt,
      sharp corners,
      colback=theoremBackground,
      colframe=theoremAccent,
      breakable
    ]
      % \color{theoremAccent}\textbf{Theorem --- }\emph{#1}\\\color{black}
    \ifx &#1& \color{theoremAccent}\textbf{Theorem. }\color{black}
    \else \color{theoremAccent}\textbf{Theorem --- }\emph{#1}\\\color{black} \fi
  }
  {
    \end{tcolorbox}
  }
\definecolor{noteBackground}{RGB}{244,249,244}
\definecolor{noteAccent}{RGB}{34,139,34}
\newenvironment{note}[1]
  {
  \begin{tcolorbox}[
    enhanced,
    boxrule=0pt,
    frame hidden,
    sharp corners,
    colback=noteBackground,
    borderline west={3pt}{-1.5pt}{noteAccent},
    breakable
    ]
    \ifx &#1& \color{noteAccent}\textbf{Note. }\color{black}
    \else \color{noteAccent}\textbf{Note (#1). }\color{black} \fi
    }
    {
  \end{tcolorbox}
  }
\definecolor{lemmaBackground}{RGB}{255,247,234}
\definecolor{lemmaAccent}{RGB}{255,153,0}
\newenvironment{lemma}[1]
  {
    \begin{tcolorbox}[
      enhanced,
      boxrule=0pt,
      frame hidden,
      sharp corners,
      colback=lemmaBackground,
      borderline west={3pt}{-1.5pt}{lemmaAccent},
      breakable
    ]
    \ifx &#1& \color{lemmaAccent}\textbf{Lemma. }\color{black}
    \else \color{lemmaAccent}\textbf{Lemma #1. }\color{black} \fi
  }
  {
    \end{tcolorbox}
  }
\definecolor{definitionBackground}{RGB}{246,246,246}
\newenvironment{definition}[1]
  {
    \begin{tcolorbox}[
      enhanced,
      boxrule=0pt,
      frame hidden,
      sharp corners,
      colback=definitionBackground,
      borderline west={3pt}{-1.5pt}{black},
      breakable
    ]
    \textbf{Definition. }\emph{#1}\\
  }
  {
    \end{tcolorbox}
  }

\newenvironment{amatrix}[2]{
    \left[
      \begin{array}{*{#1}{c}|*{#2}c}
  }
  {
      \end{array}
    \right]
  }

\definecolor{codeBackground}{RGB}{253,246,227}
\renewcommand\theFancyVerbLine{\arabic{FancyVerbLine}}
\newtcblisting{cpp}[1][]{
  boxrule=1pt,
  sharp corners,
  colback=codeBackground,
  listing only,
  breakable,
  minted language=cpp,
  minted style=material,
  minted options={
    xleftmargin=15pt,
    baselinestretch=1.2,
    linenos,
  }
}

\author{Kyle Chui}


\fancyhf{}
\lhead{Kyle Chui}
\rhead{Page \thepage}
\pagestyle{fancy}

\begin{document}
  \section{Lecture 4}
  \begin{itemize}
    \item There are $n^r$ \emph{ordered} samples of size $r$ from $n$ objects \emph{with replacement}.
    \item There are $\phantom{}_nP_r$ \emph{ordered} samples of size $r$ from $n$ objects \emph{without replacement}.
    \item There are $\phantom{}_nC_r = \frac{n!}{r!(n - r)!}$ \emph{unordered} samples of size $r$ from $n$ objects \emph{without replacement}.
    \item There are $\phantom{}_{n + r - 1}C_r = \frac{(n + r - 1)!}{r!(n - 1)!}$ \emph{unordered} samples of size $r$ from $n$ objects \emph{with replacement}.
  \end{itemize}
  \subsection{Distinguishable Permutations}
  \begin{itemize}
    \item Suppose we are given $n$ objects, but some of them are identical.
    \item How many \emph{distinguishable} permutations of the $n$ objects are there?
  \end{itemize}
  \begin{theorem}{}
    Suppose you have:
    \begin{itemize}
      \item $n_1$ objects of type $1$,
      \item $n_2$ objects of type $2$,
      \item ...
      \item $n_r$ objects of type $r$.
    \end{itemize}
    Let $n = n_1 + n_2 + \dotsb + n_r$. Then the number of distinguishable permutations is
    \[
      \binom{n}{n_1,n_2,\dotsc,n_r} = \frac{n!}{n_1!n_2!\dotsb n_r!}.
    \]
    \begin{proof}
      We have $n$ locations.
      \begin{itemize}
        \item First choose $n_1$ locations for type $1$ objects: $\choose{n}{n_1}$ choices.
        \item Then choose $n_2$ locations for type $2$ objects: $\choose{n - n_1}{n_2}$ choices.
        \item $\dotsb$
        \item Finally we choose $n_r$ locations for type $r$ objects: $\choose{n - n_1 - \dotsb - n_{r - 1}}{n_r}$ choices.
      \end{itemize}
      Using the multiplication principle to take the product of all of these combinations, we have
      \[
        \binom{n}{n_1,\dotsc,n_r} = \frac{n!}{n_1!n_2!\dotsb n_r!}.
      \]
    \end{proof}
  \end{theorem}
  \begin{note}{}
    An alternate way to think about this theorem is to first consider how many regular permutations of $n$ objects there are ($n!$), and then divide by how many possible times we over count ($n_k!$ for each $1\leq k\leq r$). 
  \end{note}
  \subsection{The Binomial Theorem}
  \begin{theorem}{Binomial Theorem}
    If $n\geq 0$ then
    \[
      (x + y)^n = \sum_{r=0}^{n}\binom{n}{r}x^ry^{n - r},
    \]
    where the \emph{binomial coefficient} is
    \[
      \binom{n}{r} = \choose{n}{r}.
    \]
    \begin{proof}
      If we multiply out $(x + y)^n = \underbrace{(x + y)\dotsb(x + y)}_{n\text{ times}}$ without using the fact that multiplication is commutative, we see that the number of times $x^ry^{n - r}$ appears is equal to how many different ways there are to rearrange $r$ ``$x$'' terms in $n$ total terms.
    \end{proof}
  \end{theorem}
  \begin{theorem}{}
    We have
    \[
      \sum_{r=0}^{n} \binom{n}{r} = 2^n.
    \]
    % \begin{proof}
    %   Observe that
    %   \[
    %     \sum_{r=0}^{n}\binom{n}{r} = \sum_{r=0}^{n}\binom{n}{r}1^r1^{n - r} = (1 + 1)^n = 2^n.
    %   \]
    % \end{proof}
  \end{theorem}
  \begin{theorem}{}
    If $n, r\geq 0$ then
    \[
      (x_1 + x_2 + \dotsb + x_r)^n = \sum_{n_1 + \dotsb + n_r = n}\binom{n}{n_1, n_2, \dotsc, n_r}x_1^{n_1}x_2^{n_2}\dotsb x_r^{n_r}.
    \]
    \begin{proof}
      Similar to the binomial theorem, we have that to get each term we just need to find the number of distinguishable permutations of $n_1$ terms of $x_1$, ..., $n_r$ terms of $x_r$, which is our multinomial coefficient from before.
    \end{proof}
  \end{theorem}
\end{document}
