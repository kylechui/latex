\documentclass[class=article, crop=false]{standalone}
% Import packages
\usepackage[margin=1in]{geometry}

\usepackage{amssymb, amsthm}
\usepackage{comment}
\usepackage{enumitem}
\usepackage{fancyhdr}
\usepackage{hyperref}
\usepackage{import}
\usepackage{mathrsfs, mathtools}
\usepackage{multicol}
\usepackage{pdfpages}
\usepackage{standalone}
\usepackage{transparent}
\usepackage{xcolor}

\usepackage{minted}
\usepackage[many]{tcolorbox}
\tcbuselibrary{minted}

% Declare math operators
\DeclareMathOperator{\lcm}{lcm}
\DeclareMathOperator{\proj}{proj}
\DeclareMathOperator{\vspan}{span}
\DeclareMathOperator{\im}{im}
\DeclareMathOperator{\Diff}{Diff}
\DeclareMathOperator{\Int}{Int}
\DeclareMathOperator{\fcn}{fcn}
\DeclareMathOperator{\id}{id}
\DeclareMathOperator{\rank}{rank}
\DeclareMathOperator{\tr}{tr}
\DeclareMathOperator{\dive}{div}
\DeclareMathOperator{\row}{row}
\DeclareMathOperator{\col}{col}
\DeclareMathOperator{\dom}{dom}
\DeclareMathOperator{\ran}{ran}
\DeclareMathOperator{\Dom}{Dom}
\DeclareMathOperator{\Ran}{Ran}
\DeclareMathOperator{\fl}{fl}
% Macros for letters/variables
\newcommand\ttilde{\kern -.15em\lower .7ex\hbox{\~{}}\kern .04em}
\newcommand{\N}{\ensuremath{\mathbb{N}}}
\newcommand{\Z}{\ensuremath{\mathbb{Z}}}
\newcommand{\Q}{\ensuremath{\mathbb{Q}}}
\newcommand{\R}{\ensuremath{\mathbb{R}}}
\newcommand{\C}{\ensuremath{\mathbb{C}}}
\newcommand{\F}{\ensuremath{\mathbb{F}}}
\newcommand{\M}{\ensuremath{\mathbb{M}}}
\renewcommand{\P}{\ensuremath{\mathbb{P}}}
\newcommand{\lam}{\ensuremath{\lambda}}
\newcommand{\nab}{\ensuremath{\nabla}}
\newcommand{\eps}{\ensuremath{\varepsilon}}
\newcommand{\es}{\ensuremath{\varnothing}}
% Macros for math symbols
\newcommand{\dx}[1]{\,\mathrm{d}#1}
\newcommand{\inv}{\ensuremath{^{-1}}}
\newcommand{\sm}{\setminus}
\newcommand{\sse}{\subseteq}
\newcommand{\ceq}{\coloneqq}
% Macros for pairs of math symbols
\newcommand{\abs}[1]{\ensuremath{\left\lvert #1 \right\rvert}}
\newcommand{\paren}[1]{\ensuremath{\left( #1 \right)}}
\newcommand{\norm}[1]{\ensuremath{\left\lVert #1\right\rVert}}
\newcommand{\set}[1]{\ensuremath{\left\{#1\right\}}}
\newcommand{\tup}[1]{\ensuremath{\left\langle #1 \right\rangle}}
\newcommand{\floor}[1]{\ensuremath{\left\lfloor #1 \right\rfloor}}
\newcommand{\ceil}[1]{\ensuremath{\left\lceil #1 \right\rceil}}
\newcommand{\eclass}[1]{\ensuremath{\left[ #1 \right]}}

\newcommand{\chapternum}{}
\newcommand{\ex}[1]{\noindent\textbf{Exercise \chapternum.{#1}.}}

\newcommand{\tsub}[1]{\textsubscript{#1}}
\newcommand{\tsup}[1]{\textsuperscript{#1}}

\renewcommand{\choose}[2]{\ensuremath{\phantom{}_{#1}C_{#2}}}
\newcommand{\perm}[2]{\ensuremath{\phantom{}_{#1}P_{#2}}}

% Include figures
\newcommand{\incfig}[2][1]{%
    \def\svgwidth{#1\columnwidth}
    \import{./figures/}{#2.pdf_tex}
}

\definecolor{problemBackground}{RGB}{212,232,246}
\newenvironment{problem}[1]
  {
    \begin{tcolorbox}[
      boxrule=.5pt,
      titlerule=.5pt,
      sharp corners,
      colback=problemBackground,
      breakable
    ]
    \ifx &#1& \textbf{Problem. }
    \else \textbf{Problem #1.} \fi
  }
  {
    \end{tcolorbox}
  }
\definecolor{exampleBackground}{RGB}{255,249,248}
\definecolor{exampleAccent}{RGB}{158,60,14}
\newenvironment{example}[1]
  {
    \begin{tcolorbox}[
      boxrule=.5pt,
      sharp corners,
      colback=exampleBackground,
      colframe=exampleAccent,
    ]
    \color{exampleAccent}\textbf{Example.} \emph{#1}\color{black}
  }
  {
    \end{tcolorbox}
  }
\definecolor{theoremBackground}{RGB}{234,243,251}
\definecolor{theoremAccent}{RGB}{0,116,183}
\newenvironment{theorem}[1]
  {
    \begin{tcolorbox}[
      boxrule=.5pt,
      titlerule=.5pt,
      sharp corners,
      colback=theoremBackground,
      colframe=theoremAccent,
      breakable
    ]
      % \color{theoremAccent}\textbf{Theorem --- }\emph{#1}\\\color{black}
    \ifx &#1& \color{theoremAccent}\textbf{Theorem. }\color{black}
    \else \color{theoremAccent}\textbf{Theorem --- }\emph{#1}\\\color{black} \fi
  }
  {
    \end{tcolorbox}
  }
\definecolor{noteBackground}{RGB}{244,249,244}
\definecolor{noteAccent}{RGB}{34,139,34}
\newenvironment{note}[1]
  {
  \begin{tcolorbox}[
    enhanced,
    boxrule=0pt,
    frame hidden,
    sharp corners,
    colback=noteBackground,
    borderline west={3pt}{-1.5pt}{noteAccent},
    breakable
    ]
    \ifx &#1& \color{noteAccent}\textbf{Note. }\color{black}
    \else \color{noteAccent}\textbf{Note (#1). }\color{black} \fi
    }
    {
  \end{tcolorbox}
  }
\definecolor{lemmaBackground}{RGB}{255,247,234}
\definecolor{lemmaAccent}{RGB}{255,153,0}
\newenvironment{lemma}[1]
  {
    \begin{tcolorbox}[
      enhanced,
      boxrule=0pt,
      frame hidden,
      sharp corners,
      colback=lemmaBackground,
      borderline west={3pt}{-1.5pt}{lemmaAccent},
      breakable
    ]
    \ifx &#1& \color{lemmaAccent}\textbf{Lemma. }\color{black}
    \else \color{lemmaAccent}\textbf{Lemma #1. }\color{black} \fi
  }
  {
    \end{tcolorbox}
  }
\definecolor{definitionBackground}{RGB}{246,246,246}
\newenvironment{definition}[1]
  {
    \begin{tcolorbox}[
      enhanced,
      boxrule=0pt,
      frame hidden,
      sharp corners,
      colback=definitionBackground,
      borderline west={3pt}{-1.5pt}{black},
      breakable
    ]
    \textbf{Definition. }\emph{#1}\\
  }
  {
    \end{tcolorbox}
  }

\newenvironment{amatrix}[2]{
    \left[
      \begin{array}{*{#1}{c}|*{#2}c}
  }
  {
      \end{array}
    \right]
  }

\definecolor{codeBackground}{RGB}{253,246,227}
\renewcommand\theFancyVerbLine{\arabic{FancyVerbLine}}
\newtcblisting{cpp}[1][]{
  boxrule=1pt,
  sharp corners,
  colback=codeBackground,
  listing only,
  breakable,
  minted language=cpp,
  minted style=material,
  minted options={
    xleftmargin=15pt,
    baselinestretch=1.2,
    linenos,
  }
}

\author{Kyle Chui}


\fancyhf{}
\lhead{Kyle Chui}
\rhead{Page \thepage}
\pagestyle{fancy}

\begin{document}
  \section{Lecture 14}
  We still use the same notation for variance,
  \[
    \var(X) = \E[(X - \E[X])^2],
  \]
  and the standard deviation is still defined by $\sigma_X^2 = \var(X)$.
  \begin{theorem}{}
    If $X$ is a continuous random variable then
    \[
      \var(X) = \E[X^2] - \E[X]^2.
    \]
    \begin{proof}
      The exact same as for the discrete case!
    \end{proof}
  \end{theorem}
  \begin{definition}{Moment Generating Function}
    If $X$ is a continuous random variable we define its \emph{moment generating function} to be 
    \[
      M_X(t) = \E[e^{tX}],
    \]
    for all $t\in\R$ for which this makes sense.
  \end{definition}
  The properties that we have proved before for discrete variables still apply in this continuous case.
  \subsection{The Exponential Distribution}
  \begin{example}{Customers at a Coffee Shop}
    \begin{itemize}
      \item Customers arrive at a coffee shop according to an approximate Poisson process with rate 1 customer per minute.
      \item Let $X$ be the arrival time (in minutes) of the first customer.
      \item What is the probability that $X \leq \frac{1}{2}$?
    \end{itemize}
    Let $N$ be the number of arrivals in half a minute, so $N\sim \Poisson(\frac{1}{2})$. Then
    \[
      \P[X \leq \frac{1}{2}] = \P[N \geq 1] = 1 - \P[N = 0] = 1 - e^{-\frac{1}{2}},
    \]
    since $p_N(x) = \paren{\frac{1}{2}}^x\cdot \frac{1}{x!}e^{-\frac{1}{2}}$ for $x = 0,1,2,\dotsc$.
  \end{example}
  \begin{definition}{Exponential Distribution}
    Consider an approximate Poisson process with rate $\lam > 0$ per unit time.
    \begin{itemize}
      \item Let $X$ be the time of the first arrival.
      \item We say that $X$ is \emph{exponentially distributed} with mean waiting time $\theta = \frac{1}{\lam}$ and write $X\sim \Exponential(\theta)$.
    \end{itemize}
  \end{definition}
  \begin{note}{}
    Some textbooks/authors use $\lam$ as the parameter instead of $\theta$.
  \end{note}
  \begin{theorem}{}
    If $\theta > 0$ and $X\sim \Exponential(\theta)$ then its PDF is
    \[
      f_X(x) = \frac{1}{\theta}e^{-\frac{x}{\theta}}\quad\text{if}\quad x > 0.
    \]
    \begin{proof}
      Consider an approximate Poisson process with rate $\lam = \frac{1}{\theta}$. Clearly, there are no arrivals before time 0, so $F_X(x) = 0$ if $x < 0$. If $x > 0$, let $N$ be the number of arrivals in time interval $[0, x]$, so $N\sim \Poisson(\lam x)$. As in our example
      \begin{align*}
        F_X(x) &= \P[X \leq x] \\
               &= \P[N \geq 1] \\
               &= 1 - \P[N = 0] \\
               &= 1 - e^{-\lam x}.
      \end{align*}
      So we have
      \[
        F_X(x) = \begin{cases}
          1 - e^{-\lam x} & \text{if } x > 0, \\
          0 & \text{if }x \leq 0.
        \end{cases}
      \]
      Hence
      \[
        f_X(x) = F_X'(x) = \lam e^{-\lam x},
      \]
      and substituting $\lam = \frac{1}{\theta}$ gives us the desired result.
    \end{proof}
  \end{theorem}
  \begin{note}{}
    If we take $x\to\infty$ for the CDF $F_X(x)$, we see that $F_X(x)\to 1$.
  \end{note}
  \begin{theorem}{}
    If $\theta > 0$ and $X\sim \Exponential(\theta)$ then its MGF is
    \[
      M_X(t) = \frac{1}{1 - \theta t}\quad\text{if }\quad t <\frac{1}{\theta}.
    \]
    \begin{proof}
      We compute
      \begin{align*}
        M_X(t) &= \E[e^{tX}] \\
               &= \int_{0}^{\infty}e^{tx}\cdot \frac{1}{\theta}e^{-\frac{x}{\theta}} \,\mathrm dx \\
               &= \frac{1}{\theta} \int_{0}^{\infty}e^{(t - \frac{1}{\theta}) x} \,\mathrm dx \\
               &= \frac{1}{\theta}\cdot -\frac{1}{t - \frac{1}{\theta}} \tag{$t < \frac{1}{\theta}$} \\
               &= \frac{1}{1 - \theta t}.
      \end{align*}
    \end{proof}
  \end{theorem}
  \begin{theorem}{}
    If $\theta > 0$ and $X\sim \Exponential(\theta)$ then
    \begin{align*}
      \E[X] &= \theta, \\
      \var(X) &= \theta^2.
    \end{align*}
    \begin{proof}
      We compute
      \begin{align*}
        \ln M_X(t) &= -\ln(1 - \theta t) \\
        (\ln M_X)'(t) &= \frac{\theta}{1 - \theta t} \\
        (\ln M_X)''(t) &= \frac{\theta^2}{(1 - \theta t)^2}.
      \end{align*}
      Substituting $t = 0$ into the above derivatives yields the desired results.
    \end{proof}
  \end{theorem}
  \subsection{Random Processes}
  \begin{definition}{Random Process}
    A \emph{random process} is a collection of random variables, indexed by a ``time'' parameter. For example:
    \begin{itemize}
      \item Approximate Poisson process
      \item Bernoulli process (repeated flips of an unfair coin)
    \end{itemize}
  \end{definition}
\end{document}
