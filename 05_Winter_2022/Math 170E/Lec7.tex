\documentclass[class=article, crop=false]{standalone}
\usepackage[margin=1in]{geometry}

\usepackage{amsmath}
\usepackage{amssymb}
\usepackage{amsthm}
\usepackage{comment}
\usepackage{enumitem}
\usepackage{fancyhdr}
\usepackage{hyperref}
\usepackage{mathrsfs}
\usepackage{mathtools}
\usepackage{standalone}
\usepackage{tcolorbox}
\usepackage{tikz}
\usepackage{xcolor}

\usetikzlibrary{decorations.pathreplacing}
\tcbuselibrary{skins}

\DeclareMathOperator{\lcm}{lcm}
\DeclareMathOperator{\proj}{proj}
\DeclareMathOperator{\vspan}{span}
\DeclareMathOperator{\im}{im}
\DeclareMathOperator{\range}{range}
\DeclareMathOperator{\Diff}{Diff}
\DeclareMathOperator{\Int}{Int}
\DeclareMathOperator{\fcn}{fcn}
\DeclareMathOperator{\id}{id}

\newcommand{\N}{\ensuremath{\mathbb{N}}}
\newcommand{\Z}{\ensuremath{\mathbb{Z}}}
\newcommand{\Q}{\ensuremath{\mathbb{Q}}}
\newcommand{\R}{\ensuremath{\mathbb{R}}}
\newcommand{\C}{\ensuremath{\mathbb{C}}}
\newcommand{\F}{\ensuremath{\mathbb{F}}}
\newcommand{\lam}{\ensuremath{\lambda}}
\newcommand{\nab}{\ensuremath{\nabla}}
\newcommand{\eps}{\ensuremath{\varepsilon}}
\newcommand{\es}{\ensuremath{\varnothing}}

\newcommand{\abs}[1]{\ensuremath{\left\lvert #1 \right\rvert}}
\newcommand{\bigpar}[1]{\ensuremath{\left( #1 \right)}}
\newcommand{\norm}[1]{\ensuremath{\lVert #1\rVert}}
\newcommand{\set}[1]{\ensuremath{\left\{#1\right\}}}
\newcommand{\tuple}[1]{\ensuremath{\left\langle #1 \right\rangle}}
\newcommand{\floor}[1]{\ensuremath{\left\lfloor #1 \right\rfloor}}
\newcommand{\ceil}[1]{\ensuremath{\left\lceil #1 \right\rceil}}

\newcommand{\chapternum}{}
\newcommand{\ex}[1]{\noindent\textbf{Exercise \chapternum.{#1}.}}

\newcommand{\dx}[1]{\,\mathrm{d}#1}
\newcommand{\inv}{\ensuremath{^{-1}}}
\newcommand{\sm}{\setminus}
\newcommand{\sse}{\subseteq}
\newcommand{\ceq}{\coloneqq}

\definecolor{problemBackground}{RGB}{212,232,246}
\newenvironment{problem}[1]
  {
    \begin{tcolorbox}[
      boxrule=.5pt,
      titlerule=.5pt,
      sharp corners,
      colback=problemBackground
    ]
    \ifx &#1& \textbf{Problem. }
    \else \textbf{Problem #1.} \fi
  }
  {
    \end{tcolorbox}
  }

\definecolor{exampleBackground}{RGB}{255,249,248}
\definecolor{exampleAccent}{RGB}{158,60,14}
\newenvironment{example}[1]
  {
    \begin{tcolorbox}[
      boxrule=.5pt,
      sharp corners,
      colback=exampleBackground,
      colframe=exampleAccent,
    ]
    \color{exampleAccent}\textbf{Example.} \emph{#1}\color{black}
  }
  {
    \end{tcolorbox}
  }

\definecolor{theoremBackground}{RGB}{234,243,251}
\definecolor{theoremAccent}{RGB}{0,116,183}
\newenvironment{theorem}[1]
  {
    \begin{tcolorbox}[
      boxrule=.5pt,
      titlerule=.5pt,
      sharp corners,
      colback=theoremBackground,
      colframe=theoremAccent
    ]
      \color{theoremAccent}\textbf{Theorem --- }\emph{#1}\\\color{black}
  }
  {
    \end{tcolorbox}
  }

\definecolor{noteBackground}{RGB}{244,249,244}
\definecolor{noteAccent}{RGB}{34,139,34}
\newenvironment{note}[1]
  {
  \begin{tcolorbox}[
    enhanced,
    boxrule=0pt,
    frame hidden,
    sharp corners,
    colback=noteBackground,
    borderline west={3pt}{-1.5pt}{noteAccent}
    ]
    \ifx &#1& \color{noteAccent}\textbf{Note. }\color{black}
    \else \color{noteAccent}\textbf{Note (#1). }\color{black} \fi
    }
    {
  \end{tcolorbox}
  }

\definecolor{definitionBackground}{RGB}{246,246,246}
\newenvironment{definition}[1]
  {
    \begin{tcolorbox}[
      enhanced,
      boxrule=0pt,
      frame hidden,
      sharp corners,
      colback=definitionBackground,
      borderline west={3pt}{-1.5pt}{black}
    ]
    \textbf{Definition. }\emph{#1}\\
  }
  {
    \end{tcolorbox}
  }
\newenvironment{amatrix}[2]{
    \left[
      \begin{array}{*{#1}{c}|*{#2}c}
  }
  {
      \end{array}
    \right]
  }
\date{\the\year-\the\month-\the\day}
\author{Kyle Chui}


\fancyhf{}
\lhead{Kyle Chui}
\rhead{Page \thepage}
\pagestyle{fancy}

\begin{document}
  \section{Lecture 7}
  \subsection{Discrete Random Variables}
  \begin{definition}{Random Variable}
    Given a set $S$, a \emph{random variable} is a function $X\colon \Omega\to S$ satisfying certain properties. For the sake of this class, we will assume that all functions $X\colon \Omega\to S$ are random variables. Some notation follows:
    \begin{align*}
      \P[X = x] &= \P{\set{\omega\in\Omega\mid X(\omega) = x}}, \\
      \P[X\in A] &= \P{\set{\omega\in\Omega\mid X(\omega)\in A}}.
    \end{align*}
  \end{definition}
  \begin{definition}{Discrete Random Variable}
    Let $X\colon \Omega\to S$ be a random variable. We say that $X$ is a \emph{discrete random variable} if $S\sse \R$ is a countable set. We define the \emph{probability mass function} (PMF) of $X$ to be the function $p_X\colon S\to [0, 1]$ given by
    \[
      p_X(x) = \P[X = x].
    \]
    We define the \emph{cumulative distribution function} (CDF) of $X$ to be the function $F_X\colon \R\to [0, 1]$ given by
    \[
      F_X(x) = \P[X \leq x].
    \]
    We say that two random variables $X, Y$ are \emph{identically distributed} if they have the same CDF, and we write $X\sim Y$.
  \end{definition}
  \begin{definition}{Uniform Distribution}
    Let $m \geq 1$. We say that a discrete random variable $X$ is \emph{uniformly distributed} on $\set{1, 2, \dotsc, m}$ and write $X\sim\Uniform(\set{1,2,\dotsc,m})$ if it has PMF
    \[
      p_X(x) = \frac{1}{m}\quad\text{if } x\in \set{1,2,\dotsc,m}.
    \]
  \end{definition}
  If $X\sim\Uniform (\set{1,2,\dotsc,m})$, then it has CDF
  \[
    F_X(x) = \begin{cases}
      0 & \text{if } x < 1, \\
      \frac{k}{m} & \text{if } k\leq x < k + 1 \text{ and } k\in \set{1,2,\dotsc,m - 1}, \\
      1 & \text{if }x \geq m.
    \end{cases}
  \]
  If $X$ is a discrete random variable taking values in a countable set $S\sse \R$ and $A\sse \R$ is any set then
  \[
    \P[X\in A] = \sum_{x\in A\cap S} p_X(x).
  \]
  \begin{proof}
    Since $S$ is countable, we know that $A\cap S = \set{a_1,a_2,\dotsc,a_n}$ (or $\set{a_1,a_2,\dotsc}$). Hence
    \begin{align*}
      \P[X\in A] &= \P[X\in \set{a_1,\dotsc,a_n}] \\
                 &= \P\left[\bigcup_{j = 1}^n \set{X = a_j}\right] \\
                 &= \sum_{j=1}^{n} \P[X = a_j] \\
                 &= \sum_{j=1}^{n} p_X(a_j) \\
                 &= \sum_{x\in A\cap S}p_X(x).
    \end{align*}
  \end{proof}
  If $X$ is a discrete random variable taking values in a countable set $S\sse\R$ then
  \[
    F_X(x) = \sum_{\substack{y \leq x \\y\in S}} p_X(y).
  \]
  \begin{proof}
    Observe that
    \begin{align*}
      F_X(x) &= \P[X \leq x] \\
             &= \P[X\in (-\infty, x]] \\
             &= \sum_{y\in (-\infty, x]\cap S}p_X(y) \\
             &= \sum_{\substack{y \leq x \\y\in S}} p_X(y).
    \end{align*}
  \end{proof}
  If $X$ is a discrete random variable and $a < b$ then
  \[
    \P[a < X \leq b] = F_X(b) - F_X(a).
  \]
  \begin{proof}
    Observe that
    \begin{align*}
      \P[a < X \leq b] &= \P[X\in (a, b]] \\
                       &= \sum_{x\in (a, b]\cap S}p_X(x) \\
                       &= \sum_{\substack{x \leq b\\x \in S}}p_X(x) - \sum_{\substack{x \leq a\\x \in S}}p_X(x) \\
                       &= F_X(b) - F_X(a).
    \end{align*}
  \end{proof}
  If $X$ is a discrete random variable taking values in a countable set $S\sse \R$ then
  \[
    \sum_{x\in S}p_X(x) = 1.
  \]
  \begin{proof}
    Observe that
    \begin{align*}
      \sum_{x\in S}p_X(x) &= \sum_{x\in \R\cap S} \\
                          &= \P[X\in R] \\
                          &= \P[\Omega] \\
                          &= 1.
    \end{align*}
  \end{proof}
\end{document}
