\documentclass[class=article, crop=false]{standalone}
\usepackage[margin=1in]{geometry}

\usepackage{amsmath}
\usepackage{amssymb}
\usepackage{amsthm}
\usepackage{comment}
\usepackage{enumitem}
\usepackage{fancyhdr}
\usepackage{hyperref}
\usepackage{mathrsfs}
\usepackage{mathtools}
\usepackage{standalone}
\usepackage{tcolorbox}
\usepackage{tikz}
\usepackage{xcolor}

\usetikzlibrary{decorations.pathreplacing}
\tcbuselibrary{skins}

\DeclareMathOperator{\lcm}{lcm}
\DeclareMathOperator{\proj}{proj}
\DeclareMathOperator{\vspan}{span}
\DeclareMathOperator{\im}{im}
\DeclareMathOperator{\range}{range}
\DeclareMathOperator{\Diff}{Diff}
\DeclareMathOperator{\Int}{Int}
\DeclareMathOperator{\fcn}{fcn}
\DeclareMathOperator{\id}{id}

\newcommand{\N}{\ensuremath{\mathbb{N}}}
\newcommand{\Z}{\ensuremath{\mathbb{Z}}}
\newcommand{\Q}{\ensuremath{\mathbb{Q}}}
\newcommand{\R}{\ensuremath{\mathbb{R}}}
\newcommand{\C}{\ensuremath{\mathbb{C}}}
\newcommand{\F}{\ensuremath{\mathbb{F}}}
\newcommand{\lam}{\ensuremath{\lambda}}
\newcommand{\nab}{\ensuremath{\nabla}}
\newcommand{\eps}{\ensuremath{\varepsilon}}
\newcommand{\es}{\ensuremath{\varnothing}}

\newcommand{\abs}[1]{\ensuremath{\left\lvert #1 \right\rvert}}
\newcommand{\bigpar}[1]{\ensuremath{\left( #1 \right)}}
\newcommand{\norm}[1]{\ensuremath{\lVert #1\rVert}}
\newcommand{\set}[1]{\ensuremath{\left\{#1\right\}}}
\newcommand{\tuple}[1]{\ensuremath{\left\langle #1 \right\rangle}}
\newcommand{\floor}[1]{\ensuremath{\left\lfloor #1 \right\rfloor}}
\newcommand{\ceil}[1]{\ensuremath{\left\lceil #1 \right\rceil}}

\newcommand{\chapternum}{}
\newcommand{\ex}[1]{\noindent\textbf{Exercise \chapternum.{#1}.}}

\newcommand{\dx}[1]{\,\mathrm{d}#1}
\newcommand{\inv}{\ensuremath{^{-1}}}
\newcommand{\sm}{\setminus}
\newcommand{\sse}{\subseteq}
\newcommand{\ceq}{\coloneqq}

\definecolor{problemBackground}{RGB}{212,232,246}
\newenvironment{problem}[1]
  {
    \begin{tcolorbox}[
      boxrule=.5pt,
      titlerule=.5pt,
      sharp corners,
      colback=problemBackground
    ]
    \ifx &#1& \textbf{Problem. }
    \else \textbf{Problem #1.} \fi
  }
  {
    \end{tcolorbox}
  }

\definecolor{exampleBackground}{RGB}{255,249,248}
\definecolor{exampleAccent}{RGB}{158,60,14}
\newenvironment{example}[1]
  {
    \begin{tcolorbox}[
      boxrule=.5pt,
      sharp corners,
      colback=exampleBackground,
      colframe=exampleAccent,
    ]
    \color{exampleAccent}\textbf{Example.} \emph{#1}\color{black}
  }
  {
    \end{tcolorbox}
  }

\definecolor{theoremBackground}{RGB}{234,243,251}
\definecolor{theoremAccent}{RGB}{0,116,183}
\newenvironment{theorem}[1]
  {
    \begin{tcolorbox}[
      boxrule=.5pt,
      titlerule=.5pt,
      sharp corners,
      colback=theoremBackground,
      colframe=theoremAccent
    ]
      \color{theoremAccent}\textbf{Theorem --- }\emph{#1}\\\color{black}
  }
  {
    \end{tcolorbox}
  }

\definecolor{noteBackground}{RGB}{244,249,244}
\definecolor{noteAccent}{RGB}{34,139,34}
\newenvironment{note}[1]
  {
  \begin{tcolorbox}[
    enhanced,
    boxrule=0pt,
    frame hidden,
    sharp corners,
    colback=noteBackground,
    borderline west={3pt}{-1.5pt}{noteAccent}
    ]
    \ifx &#1& \color{noteAccent}\textbf{Note. }\color{black}
    \else \color{noteAccent}\textbf{Note (#1). }\color{black} \fi
    }
    {
  \end{tcolorbox}
  }

\definecolor{definitionBackground}{RGB}{246,246,246}
\newenvironment{definition}[1]
  {
    \begin{tcolorbox}[
      enhanced,
      boxrule=0pt,
      frame hidden,
      sharp corners,
      colback=definitionBackground,
      borderline west={3pt}{-1.5pt}{black}
    ]
    \textbf{Definition. }\emph{#1}\\
  }
  {
    \end{tcolorbox}
  }
\newenvironment{amatrix}[2]{
    \left[
      \begin{array}{*{#1}{c}|*{#2}c}
  }
  {
      \end{array}
    \right]
  }
\date{\the\year-\the\month-\the\day}
\author{Kyle Chui}


\fancyhf{}
\lhead{Kyle Chui}
\rhead{Page \thepage}
\pagestyle{fancy}

\begin{document}
  \section{Lecture 11}
  \subsection{Negative Binomial Distribution}
  \begin{definition}{Negative Binomial Distribution}
    Suppose we repeatedly run independent, identical Bernoulli trials with probability $p\in (0, 1)$ of success.
    \begin{itemize}
      \item Let $r \geq 1$ and let $X$ be the trial on which we first achieve the $r$\tsup{th} success.
      \item $X$ is a discrete random variable taking values in the set $S = \set{r, r + 1, r + 2, \dotsc}$.
      \item We say that $X$ is a \emph{negative binomial random variable} with parameters $r, p$ and write $X\sim \NegativeBinomial(r, p)$.
    \end{itemize}
  \end{definition}
  \begin{note}{}
    If we ask about a negative binomial with parameter $(1, p)$, we see that this should be the same as a Geometric random variable.
  \end{note}
  \begin{theorem}{}
    If $X\sim \NegativeBinomial(r, p)$, then its PMF is
    \[
      p_X(x) = \binom{x - 1}{r - 1}p^r(1 - p)^{x - r}\quad\text{if}\quad x\in \set{r, r + 1, \dotsc}
    \]
    \begin{proof}
      If we want to get our $r$\tsup{th} success on the $x$\tsup{th} trial, then we must have gotten $r - 1$ successes in the last $x - 1$ trials, and a success on the last trial. Notice that the former is similar to a binomial distribution. Hence the PMF should be
      \begin{align*}
        p_X(x) &= \binom{x - 1}{r - 1}p^{r - 1}(1 - p)^{(x - 1) - (r - 1)}\cdot p \\
               &= \binom{x - 1}{r - 1}p^r(1 - p)^{x - r}.
      \end{align*}
    \end{proof}
  \end{theorem}
  \begin{theorem}{}
    If $r \geq 1$ is an integer and $0 < s < 1$ then
    \[
      \paren{\frac{1}{1 - s}}^r = \sum_{x=r}^{\infty}\binom{x - 1}{r - 1}s^{x - r}.
    \]
    \begin{proof}
      Let $g(s) = (1 - s)^{-r}$ be the left hand side of the above. We apply Taylor's theorem:
      \[
        g(s) = \sum_{\ell=0}^{\infty} \frac{1}{\ell!}\cdot \frac{\mathrm{d}^\ell g}{\mathrm{d}s^\ell}(0)s^\ell.
      \]
      Observe that:
      \begin{itemize}
        \item If $\ell = 0$ then $\frac{\mathrm{d}^\ell g}{\mathrm{d}s^\ell}(0) = g(0) = 1$
        \item If $\ell = 1$ then $\frac{\mathrm{d}^\ell g}{\mathrm{d}s^\ell}(0) = g'(0) = r$
        \item If $\ell = 1$ then $\frac{\mathrm{d}^\ell g}{\mathrm{d}s^\ell}(0) = g'(0) = r(r - 1)$
      \end{itemize}
      In general,
      \[
        \frac{\mathrm{d}^\ell g}{\mathrm{d}s^\ell}(0) = r(r + 1)\dotsb(r + \ell - 1) = \frac{(r + \ell - 1)!}{(r - 1)!}.
      \]
      Hence
      \[
        g(s) = \sum_{\ell=0}^{\infty} \frac{1}{\ell!}\cdot \frac{(r + \ell - 1)!}{(r - 1)!}\cdot s^\ell.
      \]
      If we let $x = r + \ell$, we have $\ell = x - r$, so
      \begin{align*}
        g(s) &= \sum_{x=r}^{\infty} \frac{1}{(x - r)!}\cdot \frac{(x - 1)!}{(r - 1)!}\cdot s^{x - r} \\
             &= \sum_{x=r}^{\infty} \binom{x - 1}{r - 1}s^{x - r}.
      \end{align*}
    \end{proof}
  \end{theorem}
  \begin{note}{}
    Recall that if $X\sim \NegativeBinomial(r, p)$ then
    \[
      p_X(x) = \binom{x - 1}{r - 1} p^r(1 - p)^{x - r}.
    \]
    Hence
    \begin{align*}
      \sum_{x=r}^{\infty} p_X(x) &= \sum_{x=r}^{\infty} \binom{x - 1}{r - 1} p^r(1 - p)^{x - r} \\
                                 &= p^r \sum_{x=r}^{\infty} \binom{x - 1}{r - 1} (1 - p)^{x - r} \\
                                 &= p^r\cdot (1 - (1 - p))^{-r} \\
                                 &= 1.
    \end{align*}
  \end{note}
  \begin{theorem}{}
    If $X\sim \NegativeBinomial(r, p)$ then its MGF is
    \[
      M_X(t) = \paren{\frac{e^tp}{1 - (1 - p)e^t}}^r\quad\text{if}\quad t < -\ln(1 - p).
    \]
    \begin{proof}
      We compute that the MGF is
      \begin{align*}
        M_X(t) &= \E[e^{tX}] \\
               &= \sum_{x=r}^{\infty} e^{tx}p_X(x) \\
               &= \sum_{x=r}^{\infty} e^{tx}\binom{x - 1}{r - 1}p^r(1 - p)^{x - r} \\
               &= e^{tr}p^r \sum_{x=r}^{\infty} \binom{x - 1}{r - 1}\paren{e^t(1 - p)}^{x - r} \\
               &= e^{tr}p^r\cdot \frac{1}{(1 - (e^t(1 - p)))^r} \\
               &= \paren{\frac{e^tp}{1 - (1 - p)e^t}}^r.
      \end{align*}
    \end{proof}
  \end{theorem}
  \begin{theorem}{}
    If $X\sim \NegativeBinomial(r, p)$ then
    \begin{align*}
      \E[X] &= \frac{r}{p}, \\
      \var(X) &= \frac{r(1 - p)}{p^2}.
    \end{align*}
    \begin{proof}
      Let
      \[
        h(t) = \ln \paren{\frac{pe^t}{1 - e^t(1 - p)}}.
      \]
      Since $X$ is a negative binomial random variable, its moment generating function is geometric. We know that for geometric random variables, $h'(0) = \frac{1}{p}$ and $h''(0) = \frac{1 - p}{p^2}$. Hence if $X\sim \NegativeBinomial(r, p)$ then
      \begin{align*}
        \ln M_X(t) &= \ln \paren{\frac{pe^t}{1 - e^t(1 - p)}}^r \\
                   &= r\cdot h(t).
      \end{align*}
      Therefore
      \begin{align*}
        \E[X] &= \frac{r}{p}, \\
        \var(X) &= \frac{r(1 - p)}{p^2}.
      \end{align*}
    \end{proof}
  \end{theorem}
  \subsection{Poisson Distribution}
  \begin{definition}{Poisson Random Variable}
    We make the following assumptions about arrivals in a given time interval:
    \begin{itemize}
      \item If the time intervals $(a_1, b_1], (a_2, b_2],\dotsc,(a_n, b_n]$ are disjoin then the number of arrivals in each time interval are independent.
      \item If $h = b - a$ is sufficiently small then the probability of exactly one arrival in the time interval $(a, b]$ is $\lam h$.
      \item If $h = b - a$ then the probability of having more than one arrival in the time interval $(a, b]$ converges to 0 as $h\to 0$.
    \end{itemize}
    An arrival process satisfying these assumptions is called an \emph{approximate Poisson process}. If we take $X$ to be the number of arrivals in a unit of time, then we call $X$ a \emph{Poisson random variable} and write $X\sim \Poisson(\lam)$.
  \end{definition}
  \begin{theorem}{}
    If $X\sim \Poisson(\lam)$ then it has PMF
    \[
      p_X(x) = e^{-\lam} \frac{\lam^x}{x!}\quad\text{if}\quad x\in \set{0,1,2,\dotsc}.
    \]
    \begin{proof}
      Let $X$ be the number of arrivals in a unit interval with width $\frac{1}{n}$. As $n$ gets very large, we can think of the number of arrivals in each interval to be either 0 or 1 (Bernoulli trial with success rate $\frac{\lam}{n}$). Thus the overall setup looks like a Binomial distribution, so
      \[
        X\approx \Binomial\paren{n, \frac{\lam}{n}}.
      \]
      We now just need to take $n\to\infty$. Hence
      \begin{align*}
        p_X(x) &= \lim_{n\to \infty} \binom{n}{x}\paren{\frac{\lam}{n}}^x \paren{1 - \frac{\lam}{n}}^{x - n} \\
               &= \lim_{n\to \infty} \frac{\lam^x}{x!}\cdot \frac{n\cdot (n - 1)\dotsb(n - x + 1)}{n\cdot n\dotsb n}\cdot \paren{1 - \frac{\lam}{n}}^{x - n} \\
               &= \lim_{n\to \infty} \frac{\lam^x}{x!} \paren{1 - \frac{\lam}{n}}^{x - n} \\
               &= \frac{\lam^x}{x!}\cdot e^{-\lam}.
      \end{align*}
    \end{proof}
  \end{theorem}
\end{document}
