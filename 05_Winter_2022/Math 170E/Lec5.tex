\documentclass[class=article, crop=false]{standalone}
\usepackage[margin=1in]{geometry}

\usepackage{amsmath}
\usepackage{amssymb}
\usepackage{amsthm}
\usepackage{comment}
\usepackage{enumitem}
\usepackage{fancyhdr}
\usepackage{hyperref}
\usepackage{mathrsfs}
\usepackage{mathtools}
\usepackage{standalone}
\usepackage{tcolorbox}
\usepackage{tikz}
\usepackage{xcolor}

\usetikzlibrary{decorations.pathreplacing}
\tcbuselibrary{skins}

\DeclareMathOperator{\lcm}{lcm}
\DeclareMathOperator{\proj}{proj}
\DeclareMathOperator{\vspan}{span}
\DeclareMathOperator{\im}{im}
\DeclareMathOperator{\range}{range}
\DeclareMathOperator{\Diff}{Diff}
\DeclareMathOperator{\Int}{Int}
\DeclareMathOperator{\fcn}{fcn}
\DeclareMathOperator{\id}{id}

\newcommand{\N}{\ensuremath{\mathbb{N}}}
\newcommand{\Z}{\ensuremath{\mathbb{Z}}}
\newcommand{\Q}{\ensuremath{\mathbb{Q}}}
\newcommand{\R}{\ensuremath{\mathbb{R}}}
\newcommand{\C}{\ensuremath{\mathbb{C}}}
\newcommand{\F}{\ensuremath{\mathbb{F}}}
\newcommand{\lam}{\ensuremath{\lambda}}
\newcommand{\nab}{\ensuremath{\nabla}}
\newcommand{\eps}{\ensuremath{\varepsilon}}
\newcommand{\es}{\ensuremath{\varnothing}}

\newcommand{\abs}[1]{\ensuremath{\left\lvert #1 \right\rvert}}
\newcommand{\bigpar}[1]{\ensuremath{\left( #1 \right)}}
\newcommand{\norm}[1]{\ensuremath{\lVert #1\rVert}}
\newcommand{\set}[1]{\ensuremath{\left\{#1\right\}}}
\newcommand{\tuple}[1]{\ensuremath{\left\langle #1 \right\rangle}}
\newcommand{\floor}[1]{\ensuremath{\left\lfloor #1 \right\rfloor}}
\newcommand{\ceil}[1]{\ensuremath{\left\lceil #1 \right\rceil}}

\newcommand{\chapternum}{}
\newcommand{\ex}[1]{\noindent\textbf{Exercise \chapternum.{#1}.}}

\newcommand{\dx}[1]{\,\mathrm{d}#1}
\newcommand{\inv}{\ensuremath{^{-1}}}
\newcommand{\sm}{\setminus}
\newcommand{\sse}{\subseteq}
\newcommand{\ceq}{\coloneqq}

\definecolor{problemBackground}{RGB}{212,232,246}
\newenvironment{problem}[1]
  {
    \begin{tcolorbox}[
      boxrule=.5pt,
      titlerule=.5pt,
      sharp corners,
      colback=problemBackground
    ]
    \ifx &#1& \textbf{Problem. }
    \else \textbf{Problem #1.} \fi
  }
  {
    \end{tcolorbox}
  }

\definecolor{exampleBackground}{RGB}{255,249,248}
\definecolor{exampleAccent}{RGB}{158,60,14}
\newenvironment{example}[1]
  {
    \begin{tcolorbox}[
      boxrule=.5pt,
      sharp corners,
      colback=exampleBackground,
      colframe=exampleAccent,
    ]
    \color{exampleAccent}\textbf{Example.} \emph{#1}\color{black}
  }
  {
    \end{tcolorbox}
  }

\definecolor{theoremBackground}{RGB}{234,243,251}
\definecolor{theoremAccent}{RGB}{0,116,183}
\newenvironment{theorem}[1]
  {
    \begin{tcolorbox}[
      boxrule=.5pt,
      titlerule=.5pt,
      sharp corners,
      colback=theoremBackground,
      colframe=theoremAccent
    ]
      \color{theoremAccent}\textbf{Theorem --- }\emph{#1}\\\color{black}
  }
  {
    \end{tcolorbox}
  }

\definecolor{noteBackground}{RGB}{244,249,244}
\definecolor{noteAccent}{RGB}{34,139,34}
\newenvironment{note}[1]
  {
  \begin{tcolorbox}[
    enhanced,
    boxrule=0pt,
    frame hidden,
    sharp corners,
    colback=noteBackground,
    borderline west={3pt}{-1.5pt}{noteAccent}
    ]
    \ifx &#1& \color{noteAccent}\textbf{Note. }\color{black}
    \else \color{noteAccent}\textbf{Note (#1). }\color{black} \fi
    }
    {
  \end{tcolorbox}
  }

\definecolor{definitionBackground}{RGB}{246,246,246}
\newenvironment{definition}[1]
  {
    \begin{tcolorbox}[
      enhanced,
      boxrule=0pt,
      frame hidden,
      sharp corners,
      colback=definitionBackground,
      borderline west={3pt}{-1.5pt}{black}
    ]
    \textbf{Definition. }\emph{#1}\\
  }
  {
    \end{tcolorbox}
  }
\newenvironment{amatrix}[2]{
    \left[
      \begin{array}{*{#1}{c}|*{#2}c}
  }
  {
      \end{array}
    \right]
  }
\date{\the\year-\the\month-\the\day}
\author{Kyle Chui}


\fancyhf{}
\lhead{Kyle Chui}
\rhead{Page \thepage}
\pagestyle{fancy}

\begin{document}
  \section{Lecture 5}
  \subsection{Conditional Probability}
  Suppose that $A, B\sse \Omega$ are events. If we know that the event $B$ occurs, how does this affect the probability that $A$ occurs? \par
  We write $\P[A]$ for the probability of $A$, and $\P[A\mid B]$ for the probability of $A$ \emph{conditioned on} $B$.
  \begin{example}{}
    \begin{itemize}
      \item Suppose that 5\% of UCLA students take a math class and 6\% take a physics class.
      \item Suppose that 80\% of UCLA students that take a math class also take a physics class.
      \item If we know that a randomly chosen student takes a math class, what is the probability they take a physics class?
    \end{itemize}
    Let us define 
    \begin{align*}
      \Omega &= \set{\text{All UCLA students}}, \\
      A &= \set{\text{Students taking a physics class}}, \\
      B &= \set{\text{Students taking a math class}}.
    \end{align*}
    Then we want to find $\P[A\mid B]$. Thus we have
    \begin{align*}
      \P[A\mid B] &= \frac{\P[A\cap B]}{\P[B]} \\
                  &= \frac{0.05\cdot 0.8}{0.05} \\
                  &= 0.8.
    \end{align*}
  \end{example}
  \begin{definition}{Conditional Probability}
    Suppose $A, B$ are events. Then the probability of $A$ \emph{conditioned on} $B$ is given by
    \[
      \P[A\mid B] = \frac{\P[A\cap B]}{\P[B]}.
    \]
  \end{definition}
  \begin{theorem}{}
    If $B\sse \Omega$ is an event so that $\P[B]\neq 0$ then $\P[\cdot \mid B]$ is a probability measure. Precisely:
    \begin{itemize}
      \item $\P[\Omega\mid B] = 1$.
      \begin{proof}
        Observe that
        \[
          \P[\Omega\mid B] = \frac{\P[\Omega\cap B]}{\P[B]} = \frac{\P[B]}{\P[B]} = 1.
        \]
      \end{proof}
      \item If $A_1,A_2,\dotsc,A_n$ are mutually exclusive events then
      \[
        \P\left.\left[\bigcup_{j = 1}^nA_j\right| B\right] = \sum_{j=1}^{n} \P[A_j\mid B].
      \]
      \begin{proof}
        If $A_1,\dotsc,A_n$ are mutually exclusive, so are $A_1\cap B, A_2\cap B, \dotsc, A_n\cap B$. Then
        \begin{align*}
          \P\left.\left[\bigcup_{j = 1}^nA_j\right| B\right] &= \frac{\P\left[\paren{\bigcup_{j = 1}^n A_j}\cap B\right]}{\P[B]} \\
                                                          &= \frac{\P[\bigcup_{j = 1}^n(A_j\cap B)]}{\P[B]} \\
                                                          &= \sum_{j=1}^{n} \frac{\P[A_j\cap B]}{\P[B]} \\
                                                          &= \sum_{j=1}^{n} \P[A_j\mid B].
        \end{align*}
      \end{proof}
      \item If $A_1,A_2,\dotsc,A_n$ are mutually exclusive events then
      \[
        \P\left.\left[\bigcup_{j = 1}^\infty A_j\right| B\right] = \sum_{j=1}^{\infty} \P[A_j\mid B].
      \]
      \begin{proof}
        The proof for this is very similar to the one above.
      \end{proof}
    \end{itemize}
  \end{theorem}
  \begin{theorem}{}
    If $A$ and $B$ are independent events so that $\P[B]\neq 0$ then
    \[
      \P[A\mid B] = \P[A].
    \]
    \begin{proof}
      Observe that
      \[
        \P[A\mid B] = \frac{\P[A\cap B]}{\P[B]} = \frac{\P[A]\cdot \P[B]}{\P[B]} = \P[A].
      \]
    \end{proof}
  \end{theorem}
\end{document}
