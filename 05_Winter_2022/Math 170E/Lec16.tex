\documentclass[class=article, crop=false]{standalone}
% Import packages
\usepackage[margin=1in]{geometry}

\usepackage{amssymb, amsthm}
\usepackage{comment}
\usepackage{enumitem}
\usepackage{fancyhdr}
\usepackage{hyperref}
\usepackage{import}
\usepackage{mathrsfs, mathtools}
\usepackage{multicol}
\usepackage{pdfpages}
\usepackage{standalone}
\usepackage{transparent}
\usepackage{xcolor}

\usepackage{minted}
\usepackage[many]{tcolorbox}
\tcbuselibrary{minted}

% Declare math operators
\DeclareMathOperator{\lcm}{lcm}
\DeclareMathOperator{\proj}{proj}
\DeclareMathOperator{\vspan}{span}
\DeclareMathOperator{\im}{im}
\DeclareMathOperator{\Diff}{Diff}
\DeclareMathOperator{\Int}{Int}
\DeclareMathOperator{\fcn}{fcn}
\DeclareMathOperator{\id}{id}
\DeclareMathOperator{\rank}{rank}
\DeclareMathOperator{\tr}{tr}
\DeclareMathOperator{\dive}{div}
\DeclareMathOperator{\row}{row}
\DeclareMathOperator{\col}{col}
\DeclareMathOperator{\dom}{dom}
\DeclareMathOperator{\ran}{ran}
\DeclareMathOperator{\Dom}{Dom}
\DeclareMathOperator{\Ran}{Ran}
\DeclareMathOperator{\fl}{fl}
% Macros for letters/variables
\newcommand\ttilde{\kern -.15em\lower .7ex\hbox{\~{}}\kern .04em}
\newcommand{\N}{\ensuremath{\mathbb{N}}}
\newcommand{\Z}{\ensuremath{\mathbb{Z}}}
\newcommand{\Q}{\ensuremath{\mathbb{Q}}}
\newcommand{\R}{\ensuremath{\mathbb{R}}}
\newcommand{\C}{\ensuremath{\mathbb{C}}}
\newcommand{\F}{\ensuremath{\mathbb{F}}}
\newcommand{\M}{\ensuremath{\mathbb{M}}}
\renewcommand{\P}{\ensuremath{\mathbb{P}}}
\newcommand{\lam}{\ensuremath{\lambda}}
\newcommand{\nab}{\ensuremath{\nabla}}
\newcommand{\eps}{\ensuremath{\varepsilon}}
\newcommand{\es}{\ensuremath{\varnothing}}
% Macros for math symbols
\newcommand{\dx}[1]{\,\mathrm{d}#1}
\newcommand{\inv}{\ensuremath{^{-1}}}
\newcommand{\sm}{\setminus}
\newcommand{\sse}{\subseteq}
\newcommand{\ceq}{\coloneqq}
% Macros for pairs of math symbols
\newcommand{\abs}[1]{\ensuremath{\left\lvert #1 \right\rvert}}
\newcommand{\paren}[1]{\ensuremath{\left( #1 \right)}}
\newcommand{\norm}[1]{\ensuremath{\left\lVert #1\right\rVert}}
\newcommand{\set}[1]{\ensuremath{\left\{#1\right\}}}
\newcommand{\tup}[1]{\ensuremath{\left\langle #1 \right\rangle}}
\newcommand{\floor}[1]{\ensuremath{\left\lfloor #1 \right\rfloor}}
\newcommand{\ceil}[1]{\ensuremath{\left\lceil #1 \right\rceil}}
\newcommand{\eclass}[1]{\ensuremath{\left[ #1 \right]}}

\newcommand{\chapternum}{}
\newcommand{\ex}[1]{\noindent\textbf{Exercise \chapternum.{#1}.}}

\newcommand{\tsub}[1]{\textsubscript{#1}}
\newcommand{\tsup}[1]{\textsuperscript{#1}}

\renewcommand{\choose}[2]{\ensuremath{\phantom{}_{#1}C_{#2}}}
\newcommand{\perm}[2]{\ensuremath{\phantom{}_{#1}P_{#2}}}

% Include figures
\newcommand{\incfig}[2][1]{%
    \def\svgwidth{#1\columnwidth}
    \import{./figures/}{#2.pdf_tex}
}

\definecolor{problemBackground}{RGB}{212,232,246}
\newenvironment{problem}[1]
  {
    \begin{tcolorbox}[
      boxrule=.5pt,
      titlerule=.5pt,
      sharp corners,
      colback=problemBackground,
      breakable
    ]
    \ifx &#1& \textbf{Problem. }
    \else \textbf{Problem #1.} \fi
  }
  {
    \end{tcolorbox}
  }
\definecolor{exampleBackground}{RGB}{255,249,248}
\definecolor{exampleAccent}{RGB}{158,60,14}
\newenvironment{example}[1]
  {
    \begin{tcolorbox}[
      boxrule=.5pt,
      sharp corners,
      colback=exampleBackground,
      colframe=exampleAccent,
    ]
    \color{exampleAccent}\textbf{Example.} \emph{#1}\color{black}
  }
  {
    \end{tcolorbox}
  }
\definecolor{theoremBackground}{RGB}{234,243,251}
\definecolor{theoremAccent}{RGB}{0,116,183}
\newenvironment{theorem}[1]
  {
    \begin{tcolorbox}[
      boxrule=.5pt,
      titlerule=.5pt,
      sharp corners,
      colback=theoremBackground,
      colframe=theoremAccent,
      breakable
    ]
      % \color{theoremAccent}\textbf{Theorem --- }\emph{#1}\\\color{black}
    \ifx &#1& \color{theoremAccent}\textbf{Theorem. }\color{black}
    \else \color{theoremAccent}\textbf{Theorem --- }\emph{#1}\\\color{black} \fi
  }
  {
    \end{tcolorbox}
  }
\definecolor{noteBackground}{RGB}{244,249,244}
\definecolor{noteAccent}{RGB}{34,139,34}
\newenvironment{note}[1]
  {
  \begin{tcolorbox}[
    enhanced,
    boxrule=0pt,
    frame hidden,
    sharp corners,
    colback=noteBackground,
    borderline west={3pt}{-1.5pt}{noteAccent},
    breakable
    ]
    \ifx &#1& \color{noteAccent}\textbf{Note. }\color{black}
    \else \color{noteAccent}\textbf{Note (#1). }\color{black} \fi
    }
    {
  \end{tcolorbox}
  }
\definecolor{lemmaBackground}{RGB}{255,247,234}
\definecolor{lemmaAccent}{RGB}{255,153,0}
\newenvironment{lemma}[1]
  {
    \begin{tcolorbox}[
      enhanced,
      boxrule=0pt,
      frame hidden,
      sharp corners,
      colback=lemmaBackground,
      borderline west={3pt}{-1.5pt}{lemmaAccent},
      breakable
    ]
    \ifx &#1& \color{lemmaAccent}\textbf{Lemma. }\color{black}
    \else \color{lemmaAccent}\textbf{Lemma #1. }\color{black} \fi
  }
  {
    \end{tcolorbox}
  }
\definecolor{definitionBackground}{RGB}{246,246,246}
\newenvironment{definition}[1]
  {
    \begin{tcolorbox}[
      enhanced,
      boxrule=0pt,
      frame hidden,
      sharp corners,
      colback=definitionBackground,
      borderline west={3pt}{-1.5pt}{black},
      breakable
    ]
    \textbf{Definition. }\emph{#1}\\
  }
  {
    \end{tcolorbox}
  }

\newenvironment{amatrix}[2]{
    \left[
      \begin{array}{*{#1}{c}|*{#2}c}
  }
  {
      \end{array}
    \right]
  }

\definecolor{codeBackground}{RGB}{253,246,227}
\renewcommand\theFancyVerbLine{\arabic{FancyVerbLine}}
\newtcblisting{cpp}[1][]{
  boxrule=1pt,
  sharp corners,
  colback=codeBackground,
  listing only,
  breakable,
  minted language=cpp,
  minted style=material,
  minted options={
    xleftmargin=15pt,
    baselinestretch=1.2,
    linenos,
  }
}

\author{Kyle Chui}


\fancyhf{}
\lhead{Kyle Chui}
\rhead{Page \thepage}
\pagestyle{fancy}

\begin{document}
  \section{Lecture 16}
  \subsection{The Normal Distribution}
  In general with large samples, we observe the same distribution over and over again.
  \begin{definition}{Normal Distribution}
    We say a continuous random variable $X$ is \emph{normally distributed} with mean $\mu$ and variance $\sigma^2$ if it has PDF
    \[
      f_X(x) = \frac{1}{\sqrt{2\pi\sigma^2}}e^{-\frac{(x - \mu)^2}{2\sigma^2}}\quad\text{for}\quad x\in\R.
    \]
    We write $X\sim \mathcal{N}(\mu, \sigma^2)$. If $\mu = 0$ and $\sigma^2 = 1$ we say that $X$ is a \emph{standard normal} random variable.
  \end{definition}
  \begin{theorem}{}
    We have
    \[
      \int_{-\infty}^{\infty}\frac{1}{\sqrt{2\pi\sigma^2}}e^{-\frac{(t - \mu)^2}{2\sigma^2}} \,\mathrm dt = 1.
    \]
    \begin{proof}
      Let $x = \frac{t - \mu}{\sigma}$ so $\mathrm{d}x = \frac{1}{\sigma}\mathrm{d}t$. Then
      \[
        \int_{-\infty}^{\infty}\frac{1}{\sqrt{2\pi\sigma^2}}e^{-\frac{(t - \mu)^2}{2\sigma^2}} \,\mathrm dt = \int_{-\infty}^{\infty} \frac{1}{\sqrt{2\pi}} e^{-\frac{x^2}{2}} \,\mathrm dx.
      \]
    \end{proof}
    Let's call the above $I$. Then we have
    \begin{align*}
      I^2 &= \paren{\int_{-\infty}^{\infty} \frac{1}{\sqrt{2\pi}} e^{-\frac{x^2}{2}} \,\mathrm dx}\paren{\int_{-\infty}^{\infty} \frac{1}{\sqrt{2\pi}} e^{-\frac{y^2}{2}} \,\mathrm dy} \\
          &= \frac{1}{2\pi} \int_{-\infty}^{\infty}\int_{-\infty}^{\infty}e^{-\frac{x^2 + y^2}{2}} \,\mathrm dx \,\mathrm dy \\
          &= \frac{1}{2\pi}\int_{0}^{2\pi}\int_{0}^{\infty} e^{-\frac{1}{2}r^2}r\,\mathrm dr \,\mathrm d\theta \\
          &= 1.
    \end{align*}
    Therefore $I = 1$.
  \end{theorem}
  \begin{theorem}{}
    If $X\sim \mathcal{N}(\mu, \sigma^2)$ then it has MGF
    \[
      M_X(t) = e^{\mu t + \frac{1}{2}\sigma^2t^2}.
    \]
    \begin{proof}
      We compute
      \begin{align*}
        M_X(t) &= \E[e^{tX}] \\
               &= \int_{-\infty}^{\infty} \frac{1}{\sqrt{2\pi\sigma^2}} e^{-\frac{(x - \mu)^2}{2\sigma^2}}e^{tx} \,\mathrm dx \\
               &= \text{a lot of computation} \\
               &= e^{\mu t + \frac{1}{2}\sigma^2t^2}.
      \end{align*}
    \end{proof}
  \end{theorem}
  \begin{theorem}{}
    If $X\sim \mathcal{N}(\mu, \sigma^2)$ then
    \begin{align*}
      \E[X] &= \mu, \\
      \var(X) &= \sigma^2.
    \end{align*}
    \begin{proof}
      As with before, we compute $(\ln M_X)'$ and $(\ln M_X)''$ and evaluate them at 0 to get the desired results.
    \end{proof}
  \end{theorem}
  \begin{theorem}{}
    If
    \[
      \Phi(x) = \int_{-\infty}^{x} \frac{1}{\sqrt{2\pi}}e^{-\frac{t^2}{2}} \,\mathrm dt
    \]
    is the CDF of the standard normal distribution then
    \[
      \Phi(-x) = 1 - \Phi(x).
    \]
  \end{theorem}
  \begin{theorem}{}
    If $X\sim \mathcal{N}(0, 1)$, then $-X\sim \mathcal{N}(0, 1)$.
    \begin{proof}
      Let $Y = -X$. Then
      \begin{align*}
        F_Y(y) &= \P[Y \leq y] \\
               &= \P[-X \leq y] \\
               &= \P[-y \leq X] \\
               &= 1 - \P[X < -y] \\
               &= 1 - \Phi(-y) \\
               &= \Phi(y).
      \end{align*}
    \end{proof}
  \end{theorem}
  \begin{theorem}{}
    If $X\sim \mathcal{N}(\mu, \sigma^2)$ then $Z = \frac{1}{\sigma}(X - \mu)\sim \mathcal{N}(0, 1)$.
    \begin{proof}
      We compute
      \begin{align*}
        F_Z(z) &= \P[Z \leq z] \\
               &= \P[\frac{1}{\sigma}(X - \mu) \leq z] \\
               &= \P[X \leq \mu + \sigma z] \\
               &= \int_{-\infty}^{\mu + \sigma z} \frac{1}{\sqrt{2\pi\sigma^2}} e^{-\frac{1}{2\sigma^2}(x - \mu)} \,\mathrm dx \\
               &= \Phi(z).
      \end{align*}
    \end{proof}
  \end{theorem}
\end{document}
