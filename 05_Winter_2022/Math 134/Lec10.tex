\documentclass[class=article, crop=false]{standalone}
\usepackage[margin=1in]{geometry}

\usepackage{amsmath}
\usepackage{amssymb}
\usepackage{amsthm}
\usepackage{comment}
\usepackage{enumitem}
\usepackage{fancyhdr}
\usepackage{hyperref}
\usepackage{mathrsfs}
\usepackage{mathtools}
\usepackage{standalone}
\usepackage{tcolorbox}
\usepackage{tikz}
\usepackage{xcolor}

\usetikzlibrary{decorations.pathreplacing}
\tcbuselibrary{skins}

\DeclareMathOperator{\lcm}{lcm}
\DeclareMathOperator{\proj}{proj}
\DeclareMathOperator{\vspan}{span}
\DeclareMathOperator{\im}{im}
\DeclareMathOperator{\range}{range}
\DeclareMathOperator{\Diff}{Diff}
\DeclareMathOperator{\Int}{Int}
\DeclareMathOperator{\fcn}{fcn}
\DeclareMathOperator{\id}{id}

\newcommand{\N}{\ensuremath{\mathbb{N}}}
\newcommand{\Z}{\ensuremath{\mathbb{Z}}}
\newcommand{\Q}{\ensuremath{\mathbb{Q}}}
\newcommand{\R}{\ensuremath{\mathbb{R}}}
\newcommand{\C}{\ensuremath{\mathbb{C}}}
\newcommand{\F}{\ensuremath{\mathbb{F}}}
\newcommand{\lam}{\ensuremath{\lambda}}
\newcommand{\nab}{\ensuremath{\nabla}}
\newcommand{\eps}{\ensuremath{\varepsilon}}
\newcommand{\es}{\ensuremath{\varnothing}}

\newcommand{\abs}[1]{\ensuremath{\left\lvert #1 \right\rvert}}
\newcommand{\bigpar}[1]{\ensuremath{\left( #1 \right)}}
\newcommand{\norm}[1]{\ensuremath{\lVert #1\rVert}}
\newcommand{\set}[1]{\ensuremath{\left\{#1\right\}}}
\newcommand{\tuple}[1]{\ensuremath{\left\langle #1 \right\rangle}}
\newcommand{\floor}[1]{\ensuremath{\left\lfloor #1 \right\rfloor}}
\newcommand{\ceil}[1]{\ensuremath{\left\lceil #1 \right\rceil}}

\newcommand{\chapternum}{}
\newcommand{\ex}[1]{\noindent\textbf{Exercise \chapternum.{#1}.}}

\newcommand{\dx}[1]{\,\mathrm{d}#1}
\newcommand{\inv}{\ensuremath{^{-1}}}
\newcommand{\sm}{\setminus}
\newcommand{\sse}{\subseteq}
\newcommand{\ceq}{\coloneqq}

\definecolor{problemBackground}{RGB}{212,232,246}
\newenvironment{problem}[1]
  {
    \begin{tcolorbox}[
      boxrule=.5pt,
      titlerule=.5pt,
      sharp corners,
      colback=problemBackground
    ]
    \ifx &#1& \textbf{Problem. }
    \else \textbf{Problem #1.} \fi
  }
  {
    \end{tcolorbox}
  }

\definecolor{exampleBackground}{RGB}{255,249,248}
\definecolor{exampleAccent}{RGB}{158,60,14}
\newenvironment{example}[1]
  {
    \begin{tcolorbox}[
      boxrule=.5pt,
      sharp corners,
      colback=exampleBackground,
      colframe=exampleAccent,
    ]
    \color{exampleAccent}\textbf{Example.} \emph{#1}\color{black}
  }
  {
    \end{tcolorbox}
  }

\definecolor{theoremBackground}{RGB}{234,243,251}
\definecolor{theoremAccent}{RGB}{0,116,183}
\newenvironment{theorem}[1]
  {
    \begin{tcolorbox}[
      boxrule=.5pt,
      titlerule=.5pt,
      sharp corners,
      colback=theoremBackground,
      colframe=theoremAccent
    ]
      \color{theoremAccent}\textbf{Theorem --- }\emph{#1}\\\color{black}
  }
  {
    \end{tcolorbox}
  }

\definecolor{noteBackground}{RGB}{244,249,244}
\definecolor{noteAccent}{RGB}{34,139,34}
\newenvironment{note}[1]
  {
  \begin{tcolorbox}[
    enhanced,
    boxrule=0pt,
    frame hidden,
    sharp corners,
    colback=noteBackground,
    borderline west={3pt}{-1.5pt}{noteAccent}
    ]
    \ifx &#1& \color{noteAccent}\textbf{Note. }\color{black}
    \else \color{noteAccent}\textbf{Note (#1). }\color{black} \fi
    }
    {
  \end{tcolorbox}
  }

\definecolor{definitionBackground}{RGB}{246,246,246}
\newenvironment{definition}[1]
  {
    \begin{tcolorbox}[
      enhanced,
      boxrule=0pt,
      frame hidden,
      sharp corners,
      colback=definitionBackground,
      borderline west={3pt}{-1.5pt}{black}
    ]
    \textbf{Definition. }\emph{#1}\\
  }
  {
    \end{tcolorbox}
  }
\newenvironment{amatrix}[2]{
    \left[
      \begin{array}{*{#1}{c}|*{#2}c}
  }
  {
      \end{array}
    \right]
  }
\date{\the\year-\the\month-\the\day}
\author{Kyle Chui}


\fancyhf{}
\lhead{Kyle Chui}
\rhead{Page \thepage}
\pagestyle{fancy}

\begin{document}
  \section{Lecture 10}
  \begin{note}{}
    Hysteresis is a concept that appears due to the non-reversibility as the parameter $r$ varies. If we look at the bifurcation diagram's stable branches, we take one path as we increase $r$ past 0, but take a different path as we decrease $r$ back below 0.
  \end{note}
  \subsection{Taylor's Theorem}
  \subsubsection{Single Variable}
  Let $F(t)$ be continuous and $\displaystyle \frac{\mathrm{d}^nF}{\mathrm{d}t^n}$ is continuous for $1 \leq n \leq N + 1$. Then we can write
  \[
    F(t) = \sum_{n=0}^{N} \frac{1}{n!} \frac{\mathrm{d}^nF}{\mathrm{d}t^n}(0)t^n + R_N(t),
  \]
  where $\displaystyle R_N(t) = \underbrace{\frac{1}{(N + 1)!} \frac{\mathrm{d}^{N + 1}F}{\mathrm{d}t^{N + 1}}\paren{\tilde t}t^{n + 1}}_{\text{Lagrange form residue}}$ for some $\tilde t\in (0, t)$.
  \subsubsection{Multi Variable}
  Let $f(x, r)$ be smooth (so $\displaystyle\frac{\partial^n}{\partial x^n} \frac{\partial^m}{\partial r^m}f$ is continuous for all $m, n \geq 1$). Then let $F(t) = f(tx, tr)$ and apply Taylor's Theorem. We have
  \[
    \frac{\mathrm{d}^n}{\mathrm{d}t^n} F(t) = \sum_{j=0}^{n} \binom{n}{j} \frac{\partial^nf}{\partial_x^{n - j}\partial_r^j}(tx, tr)x^{n - j}r^j.
  \]
  When we take $t = 1$, we have
  \begin{align*}
    f(x, t) &= F(1) \\
            &= \sum_{n=0}^{N} \frac{1}{n!} \frac{\mathrm{d}^nF}{\mathrm{d}t^n}(0) + R_N(0) \\
            &= \sum_{n=0}^{N} \sum_{j=0}^{n} \frac{1}{(n - j)!j!} \frac{\partial^nf}{\partial_x^{n - j}\partial r^j}(0, 0)x^{n - j}r^j + R_N(0)
  \end{align*}
  \begin{theorem}{Taylor's Theorem}
    Suppose all partial derivatives of $f(x, r)$ up to order $N + 1$ are continuous. Then,
    \[
      f(x, r) = \sum_{n=0}^{N} \sum_{j=0}^{n} \frac{1}{(n - j)!j!} \frac{\partial^nf}{\partial x^{n - j}\partial r^j}(0, 0) x^{n - j}r^j + R_N(x, r),
    \]
    where the remainder term can be written as
    \[
      R_N(x, r) = \sum_{j=0}^{N + 1} \frac{1}{(N + 1 - j)!j!} \frac{\partial^{N + 1}f}{\partial x^{N + 1 - j}\partial r^j}(tx, tr) x^{N + 1 - j}r^j,
    \]
    for some $0 < t < 1$.
  \end{theorem}
  \begin{example}{Special Case} \par
    Consider the case where $N = 2$ for the Taylor expansion. Plugging that into the formula, we have that the quadratic expansion of $f(x, r)$ at $(0, 0)$ is
    \begin{align*}
      f(x, r) &= f(0, 0) + \frac{\partial f}{\partial x}(0, 0)\cdot x + \frac{\partial f}{\partial r}(0, 0)\cdot r \\
              &+ \frac{\partial f}{\partial x\partial r}(0, 0)\cdot xr + \frac{1}{2}\cdot \frac{\partial^2 f}{\partial x^2}(0, 0)\cdot x^2 + \frac{1}{2}\cdot \frac{\partial^2 f}{\partial r^2}(0, 0)\cdot r^2.
    \end{align*}
  \end{example}
  \subsection{Normal Forms}
  Consider the Taylor expansion for a saddle-node bifurcation about the point $(x^*, r^*)$:
  \begin{align*}
    \dot{x} &= f(x, r) \\
            &= f(x^*, r^*) + \underbrace{\partial x f(x^*, r^*)}_{q_1}(x - x^*) + \underbrace{\partial r f(x^*, r^*)}_{p_1}(r - r^*) \\
            &+ \underbrace{\frac{1}{2}\partial_{xx} f(x^*, r^*)}_{q_2}(x - x^*)^2 + \underbrace{\partial_{xr} f(x^*, r^*)}_{p_2}(x - x^*)(r - r^*) \\
            &+ \underbrace{\frac{1}{2}\partial_{rr} f(x^*, r^*)}_{Q}(r - r^*)^2 + \text{higher order terms}.
  \end{align*}
  \begin{theorem}{}
    Suppose that $f(x^*, r^*) = 0$, $q_1 = 0$, $p_1\neq 0$, $q_2\neq 0$. Then $\dot{x} = f(x, r)$ undergoes a saddle-node bifurcation at $(x^*, r^*)$ and
    \[
      \dot{x} = \frac{\partial f}{\partial r}(x^*, r^*)(r - r^*) + \frac{1}{2}\partial_{xx} f(x^*, r^*)(x - x^*)^2 + \mathcal{O}(\eps^3)
    \]
    for $\abs{r - r^*} < \eps^2$ and $\abs{x - x^*} < \eps$. Moreover, there exists a change of variables
    \[
      (t, x, r)\mapsto (s, y, R)
    \]
    such that
    \begin{align*}
      \dot{x} &= p_1(r - r^*) + q_2(x - x^*)^2 + \text{higher order terms}
    \end{align*}
    takes the form
    \[
      \frac{\mathrm{d}y}{\mathrm{d}s} = R + y^2\tag{Saddle-node bifurcation}
    \]
    near $(0, 0) = (y(x^*), R(r^*))$.
  \end{theorem}
\end{document}
