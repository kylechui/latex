\documentclass[class=article, crop=false]{standalone}
% Import packages
\usepackage[margin=1in]{geometry}

\usepackage{amssymb, amsthm}
\usepackage{comment}
\usepackage{enumitem}
\usepackage{fancyhdr}
\usepackage{hyperref}
\usepackage{import}
\usepackage{mathrsfs, mathtools}
\usepackage{multicol}
\usepackage{pdfpages}
\usepackage{standalone}
\usepackage{transparent}
\usepackage{xcolor}

\usepackage{minted}
\usepackage[many]{tcolorbox}
\tcbuselibrary{minted}

% Declare math operators
\DeclareMathOperator{\lcm}{lcm}
\DeclareMathOperator{\proj}{proj}
\DeclareMathOperator{\vspan}{span}
\DeclareMathOperator{\im}{im}
\DeclareMathOperator{\Diff}{Diff}
\DeclareMathOperator{\Int}{Int}
\DeclareMathOperator{\fcn}{fcn}
\DeclareMathOperator{\id}{id}
\DeclareMathOperator{\rank}{rank}
\DeclareMathOperator{\tr}{tr}
\DeclareMathOperator{\dive}{div}
\DeclareMathOperator{\row}{row}
\DeclareMathOperator{\col}{col}
\DeclareMathOperator{\dom}{dom}
\DeclareMathOperator{\ran}{ran}
\DeclareMathOperator{\Dom}{Dom}
\DeclareMathOperator{\Ran}{Ran}
\DeclareMathOperator{\fl}{fl}
% Macros for letters/variables
\newcommand\ttilde{\kern -.15em\lower .7ex\hbox{\~{}}\kern .04em}
\newcommand{\N}{\ensuremath{\mathbb{N}}}
\newcommand{\Z}{\ensuremath{\mathbb{Z}}}
\newcommand{\Q}{\ensuremath{\mathbb{Q}}}
\newcommand{\R}{\ensuremath{\mathbb{R}}}
\newcommand{\C}{\ensuremath{\mathbb{C}}}
\newcommand{\F}{\ensuremath{\mathbb{F}}}
\newcommand{\M}{\ensuremath{\mathbb{M}}}
\renewcommand{\P}{\ensuremath{\mathbb{P}}}
\newcommand{\lam}{\ensuremath{\lambda}}
\newcommand{\nab}{\ensuremath{\nabla}}
\newcommand{\eps}{\ensuremath{\varepsilon}}
\newcommand{\es}{\ensuremath{\varnothing}}
% Macros for math symbols
\newcommand{\dx}[1]{\,\mathrm{d}#1}
\newcommand{\inv}{\ensuremath{^{-1}}}
\newcommand{\sm}{\setminus}
\newcommand{\sse}{\subseteq}
\newcommand{\ceq}{\coloneqq}
% Macros for pairs of math symbols
\newcommand{\abs}[1]{\ensuremath{\left\lvert #1 \right\rvert}}
\newcommand{\paren}[1]{\ensuremath{\left( #1 \right)}}
\newcommand{\norm}[1]{\ensuremath{\left\lVert #1\right\rVert}}
\newcommand{\set}[1]{\ensuremath{\left\{#1\right\}}}
\newcommand{\tup}[1]{\ensuremath{\left\langle #1 \right\rangle}}
\newcommand{\floor}[1]{\ensuremath{\left\lfloor #1 \right\rfloor}}
\newcommand{\ceil}[1]{\ensuremath{\left\lceil #1 \right\rceil}}
\newcommand{\eclass}[1]{\ensuremath{\left[ #1 \right]}}

\newcommand{\chapternum}{}
\newcommand{\ex}[1]{\noindent\textbf{Exercise \chapternum.{#1}.}}

\newcommand{\tsub}[1]{\textsubscript{#1}}
\newcommand{\tsup}[1]{\textsuperscript{#1}}

\renewcommand{\choose}[2]{\ensuremath{\phantom{}_{#1}C_{#2}}}
\newcommand{\perm}[2]{\ensuremath{\phantom{}_{#1}P_{#2}}}

% Include figures
\newcommand{\incfig}[2][1]{%
    \def\svgwidth{#1\columnwidth}
    \import{./figures/}{#2.pdf_tex}
}

\definecolor{problemBackground}{RGB}{212,232,246}
\newenvironment{problem}[1]
  {
    \begin{tcolorbox}[
      boxrule=.5pt,
      titlerule=.5pt,
      sharp corners,
      colback=problemBackground,
      breakable
    ]
    \ifx &#1& \textbf{Problem. }
    \else \textbf{Problem #1.} \fi
  }
  {
    \end{tcolorbox}
  }
\definecolor{exampleBackground}{RGB}{255,249,248}
\definecolor{exampleAccent}{RGB}{158,60,14}
\newenvironment{example}[1]
  {
    \begin{tcolorbox}[
      boxrule=.5pt,
      sharp corners,
      colback=exampleBackground,
      colframe=exampleAccent,
    ]
    \color{exampleAccent}\textbf{Example.} \emph{#1}\color{black}
  }
  {
    \end{tcolorbox}
  }
\definecolor{theoremBackground}{RGB}{234,243,251}
\definecolor{theoremAccent}{RGB}{0,116,183}
\newenvironment{theorem}[1]
  {
    \begin{tcolorbox}[
      boxrule=.5pt,
      titlerule=.5pt,
      sharp corners,
      colback=theoremBackground,
      colframe=theoremAccent,
      breakable
    ]
      % \color{theoremAccent}\textbf{Theorem --- }\emph{#1}\\\color{black}
    \ifx &#1& \color{theoremAccent}\textbf{Theorem. }\color{black}
    \else \color{theoremAccent}\textbf{Theorem --- }\emph{#1}\\\color{black} \fi
  }
  {
    \end{tcolorbox}
  }
\definecolor{noteBackground}{RGB}{244,249,244}
\definecolor{noteAccent}{RGB}{34,139,34}
\newenvironment{note}[1]
  {
  \begin{tcolorbox}[
    enhanced,
    boxrule=0pt,
    frame hidden,
    sharp corners,
    colback=noteBackground,
    borderline west={3pt}{-1.5pt}{noteAccent},
    breakable
    ]
    \ifx &#1& \color{noteAccent}\textbf{Note. }\color{black}
    \else \color{noteAccent}\textbf{Note (#1). }\color{black} \fi
    }
    {
  \end{tcolorbox}
  }
\definecolor{lemmaBackground}{RGB}{255,247,234}
\definecolor{lemmaAccent}{RGB}{255,153,0}
\newenvironment{lemma}[1]
  {
    \begin{tcolorbox}[
      enhanced,
      boxrule=0pt,
      frame hidden,
      sharp corners,
      colback=lemmaBackground,
      borderline west={3pt}{-1.5pt}{lemmaAccent},
      breakable
    ]
    \ifx &#1& \color{lemmaAccent}\textbf{Lemma. }\color{black}
    \else \color{lemmaAccent}\textbf{Lemma #1. }\color{black} \fi
  }
  {
    \end{tcolorbox}
  }
\definecolor{definitionBackground}{RGB}{246,246,246}
\newenvironment{definition}[1]
  {
    \begin{tcolorbox}[
      enhanced,
      boxrule=0pt,
      frame hidden,
      sharp corners,
      colback=definitionBackground,
      borderline west={3pt}{-1.5pt}{black},
      breakable
    ]
    \textbf{Definition. }\emph{#1}\\
  }
  {
    \end{tcolorbox}
  }

\newenvironment{amatrix}[2]{
    \left[
      \begin{array}{*{#1}{c}|*{#2}c}
  }
  {
      \end{array}
    \right]
  }

\definecolor{codeBackground}{RGB}{253,246,227}
\renewcommand\theFancyVerbLine{\arabic{FancyVerbLine}}
\newtcblisting{cpp}[1][]{
  boxrule=1pt,
  sharp corners,
  colback=codeBackground,
  listing only,
  breakable,
  minted language=cpp,
  minted style=material,
  minted options={
    xleftmargin=15pt,
    baselinestretch=1.2,
    linenos,
  }
}

\author{Kyle Chui}


\fancyhf{}
\lhead{Kyle Chui}
\rhead{Page \thepage}
\pagestyle{fancy}

\begin{document}
  \section{Lecture 3}
  \textbf{Question.} Can we say more about what happens close to fixed points?
  \begin{example}{} Consider the equation given by
    \[
      \dot{x} = x(x + 1)(x - 1)^2,
    \]
    which has stable points at $-1$, $0$, and $1$. \par
    We will use something called the \emph{local method}. We define a new function $\eta(t) = x(t) + 1$, so $x(t) = \eta(t) - 1$. Hence $\dot{x}(t) = \dot{\eta}(t)$. Furthermore,
    \begin{align*}
      x(x + 1)(x - 1)^2 &= (\eta - 1)\eta(\eta - 2)^2 \\
                        &= -4\eta + O(\eta^2), \tag{$\eta\to 0$}
    \end{align*}
    so $\dot{\eta}\approx -4\eta$. Near $x = -1, \eta = x + 1$ and it satisfies $\dot{\eta} = -4\eta$, so $\eta(t)\approx Ce^{-4t}$. We can see that this approaches $0$ as $t\to\infty$, so points around $x = -1$ will approach $-1$.
  \end{example}
  In general, we have the following method: \par
  Assume that $x^*$ is a fixed point of $\dot{x} = f(x)$, i.e. $f(x^*) = 0$. Let $\eta = x - x^*$. Then
  \begin{align*}
    \dot{\eta} &= \dot{x} \\
            &= f(x) \\
            &= f(x^*) + f'(x^*)\eta + \frac{1}{2}f''(x^*)\eta^2 + \dotsb \\
            &= f'(x^*)\eta + O(\eta^2). \tag{$\eta\to 0$}
  \end{align*}
  Hence the equation
  \[
    \dot{\eta} = f'(x^*)\eta
  \]
  is the \emph{linearization} at $x = x^*$. We know that the solution to such a differential equation is
  \[
    \eta(t) = Ce^{f'(x^*)t} = \begin{cases}
      0, & f'(x^*) < 0 \\
      \pm\infty, & f'(x^*) > 0
    \end{cases},
  \]
  as $t\to\infty$. In the first case, the terms near $x^*$ will tend towards $x^*$, and in the latter they will diverge from $x^*$.
  \begin{theorem}{}
    Suppose that $x^*$ is a fixed point of the system $\dot{x} = f(x)$. Then if
    \begin{itemize}
      \item $f'(x^*) < 0$, the fixed point $x^*$ is stable.
      \item $f'(x^*) > 0$, the fixed point $x^*$ is unstable.
    \end{itemize}
  \end{theorem}
  \textbf{Question.} What happens if $f'(x^*) = 0$? Anything can happen/the test is inconclusive. \par
  \begin{itemize}
    \item Consider the equation $\dot{x} = x^3$. We see that $x^* = 0$ is a critical point, and that $f'(x) = 3x^2$, so $f'(x^*) = 0$. Using our usual graphical methods, we can see that points to the left of $x^*$ will approach $x^*$, and so will points to the right, and so $x^* = 0$ is a stable point.
    \item If we use the equation $\dot{x} = -x^3$, we get the direct opposite, that is $x^* = 0$ is unstable despite having the same critical point.
    \item If we look at the behavior of $\dot{x} = x^2$, then we get that the critical point at $x^* = 0$ is half-stable.
    \item If we consider the equation $\dot{x} = 0$, then every number on the real line is a critical point, and so the solutions don't move at all.
  \end{itemize}
  \subsection{Potentials}
  \begin{itemize}
    \item Let $f\colon \R\to \R$ be smooth and consider the system
    \[
      \dot{x} = f(x).
    \]
    \item A function $V\colon \R\to \R$ so that
    \[
      f(x) = -V'(x)
    \]
    is called a \emph{potential} for $f$.
    \item Our system can be written as a \emph{gradient flow}
    \[
      \dot{x} = -V'(x).
    \]
  \end{itemize}
  \begin{note}{}
    Potential functions are \emph{not} unique, since you can always add a constant.
  \end{note}
\end{document}
