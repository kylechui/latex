\documentclass[class=article, crop=false]{standalone}
% Import packages
\usepackage[margin=1in]{geometry}

\usepackage{amssymb, amsthm}
\usepackage{comment}
\usepackage{enumitem}
\usepackage{fancyhdr}
\usepackage{hyperref}
\usepackage{import}
\usepackage{mathrsfs, mathtools}
\usepackage{multicol}
\usepackage{pdfpages}
\usepackage{standalone}
\usepackage{transparent}
\usepackage{xcolor}

\usepackage{minted}
\usepackage[many]{tcolorbox}
\tcbuselibrary{minted}

% Declare math operators
\DeclareMathOperator{\lcm}{lcm}
\DeclareMathOperator{\proj}{proj}
\DeclareMathOperator{\vspan}{span}
\DeclareMathOperator{\im}{im}
\DeclareMathOperator{\Diff}{Diff}
\DeclareMathOperator{\Int}{Int}
\DeclareMathOperator{\fcn}{fcn}
\DeclareMathOperator{\id}{id}
\DeclareMathOperator{\rank}{rank}
\DeclareMathOperator{\tr}{tr}
\DeclareMathOperator{\dive}{div}
\DeclareMathOperator{\row}{row}
\DeclareMathOperator{\col}{col}
\DeclareMathOperator{\dom}{dom}
\DeclareMathOperator{\ran}{ran}
\DeclareMathOperator{\Dom}{Dom}
\DeclareMathOperator{\Ran}{Ran}
\DeclareMathOperator{\fl}{fl}
% Macros for letters/variables
\newcommand\ttilde{\kern -.15em\lower .7ex\hbox{\~{}}\kern .04em}
\newcommand{\N}{\ensuremath{\mathbb{N}}}
\newcommand{\Z}{\ensuremath{\mathbb{Z}}}
\newcommand{\Q}{\ensuremath{\mathbb{Q}}}
\newcommand{\R}{\ensuremath{\mathbb{R}}}
\newcommand{\C}{\ensuremath{\mathbb{C}}}
\newcommand{\F}{\ensuremath{\mathbb{F}}}
\newcommand{\M}{\ensuremath{\mathbb{M}}}
\renewcommand{\P}{\ensuremath{\mathbb{P}}}
\newcommand{\lam}{\ensuremath{\lambda}}
\newcommand{\nab}{\ensuremath{\nabla}}
\newcommand{\eps}{\ensuremath{\varepsilon}}
\newcommand{\es}{\ensuremath{\varnothing}}
% Macros for math symbols
\newcommand{\dx}[1]{\,\mathrm{d}#1}
\newcommand{\inv}{\ensuremath{^{-1}}}
\newcommand{\sm}{\setminus}
\newcommand{\sse}{\subseteq}
\newcommand{\ceq}{\coloneqq}
% Macros for pairs of math symbols
\newcommand{\abs}[1]{\ensuremath{\left\lvert #1 \right\rvert}}
\newcommand{\paren}[1]{\ensuremath{\left( #1 \right)}}
\newcommand{\norm}[1]{\ensuremath{\left\lVert #1\right\rVert}}
\newcommand{\set}[1]{\ensuremath{\left\{#1\right\}}}
\newcommand{\tup}[1]{\ensuremath{\left\langle #1 \right\rangle}}
\newcommand{\floor}[1]{\ensuremath{\left\lfloor #1 \right\rfloor}}
\newcommand{\ceil}[1]{\ensuremath{\left\lceil #1 \right\rceil}}
\newcommand{\eclass}[1]{\ensuremath{\left[ #1 \right]}}

\newcommand{\chapternum}{}
\newcommand{\ex}[1]{\noindent\textbf{Exercise \chapternum.{#1}.}}

\newcommand{\tsub}[1]{\textsubscript{#1}}
\newcommand{\tsup}[1]{\textsuperscript{#1}}

\renewcommand{\choose}[2]{\ensuremath{\phantom{}_{#1}C_{#2}}}
\newcommand{\perm}[2]{\ensuremath{\phantom{}_{#1}P_{#2}}}

% Include figures
\newcommand{\incfig}[2][1]{%
    \def\svgwidth{#1\columnwidth}
    \import{./figures/}{#2.pdf_tex}
}

\definecolor{problemBackground}{RGB}{212,232,246}
\newenvironment{problem}[1]
  {
    \begin{tcolorbox}[
      boxrule=.5pt,
      titlerule=.5pt,
      sharp corners,
      colback=problemBackground,
      breakable
    ]
    \ifx &#1& \textbf{Problem. }
    \else \textbf{Problem #1.} \fi
  }
  {
    \end{tcolorbox}
  }
\definecolor{exampleBackground}{RGB}{255,249,248}
\definecolor{exampleAccent}{RGB}{158,60,14}
\newenvironment{example}[1]
  {
    \begin{tcolorbox}[
      boxrule=.5pt,
      sharp corners,
      colback=exampleBackground,
      colframe=exampleAccent,
    ]
    \color{exampleAccent}\textbf{Example.} \emph{#1}\color{black}
  }
  {
    \end{tcolorbox}
  }
\definecolor{theoremBackground}{RGB}{234,243,251}
\definecolor{theoremAccent}{RGB}{0,116,183}
\newenvironment{theorem}[1]
  {
    \begin{tcolorbox}[
      boxrule=.5pt,
      titlerule=.5pt,
      sharp corners,
      colback=theoremBackground,
      colframe=theoremAccent,
      breakable
    ]
      % \color{theoremAccent}\textbf{Theorem --- }\emph{#1}\\\color{black}
    \ifx &#1& \color{theoremAccent}\textbf{Theorem. }\color{black}
    \else \color{theoremAccent}\textbf{Theorem --- }\emph{#1}\\\color{black} \fi
  }
  {
    \end{tcolorbox}
  }
\definecolor{noteBackground}{RGB}{244,249,244}
\definecolor{noteAccent}{RGB}{34,139,34}
\newenvironment{note}[1]
  {
  \begin{tcolorbox}[
    enhanced,
    boxrule=0pt,
    frame hidden,
    sharp corners,
    colback=noteBackground,
    borderline west={3pt}{-1.5pt}{noteAccent},
    breakable
    ]
    \ifx &#1& \color{noteAccent}\textbf{Note. }\color{black}
    \else \color{noteAccent}\textbf{Note (#1). }\color{black} \fi
    }
    {
  \end{tcolorbox}
  }
\definecolor{lemmaBackground}{RGB}{255,247,234}
\definecolor{lemmaAccent}{RGB}{255,153,0}
\newenvironment{lemma}[1]
  {
    \begin{tcolorbox}[
      enhanced,
      boxrule=0pt,
      frame hidden,
      sharp corners,
      colback=lemmaBackground,
      borderline west={3pt}{-1.5pt}{lemmaAccent},
      breakable
    ]
    \ifx &#1& \color{lemmaAccent}\textbf{Lemma. }\color{black}
    \else \color{lemmaAccent}\textbf{Lemma #1. }\color{black} \fi
  }
  {
    \end{tcolorbox}
  }
\definecolor{definitionBackground}{RGB}{246,246,246}
\newenvironment{definition}[1]
  {
    \begin{tcolorbox}[
      enhanced,
      boxrule=0pt,
      frame hidden,
      sharp corners,
      colback=definitionBackground,
      borderline west={3pt}{-1.5pt}{black},
      breakable
    ]
    \textbf{Definition. }\emph{#1}\\
  }
  {
    \end{tcolorbox}
  }

\newenvironment{amatrix}[2]{
    \left[
      \begin{array}{*{#1}{c}|*{#2}c}
  }
  {
      \end{array}
    \right]
  }

\definecolor{codeBackground}{RGB}{253,246,227}
\renewcommand\theFancyVerbLine{\arabic{FancyVerbLine}}
\newtcblisting{cpp}[1][]{
  boxrule=1pt,
  sharp corners,
  colback=codeBackground,
  listing only,
  breakable,
  minted language=cpp,
  minted style=material,
  minted options={
    xleftmargin=15pt,
    baselinestretch=1.2,
    linenos,
  }
}

\author{Kyle Chui}


\fancyhf{}
\lhead{Kyle Chui}
\rhead{Page \thepage}
\pagestyle{fancy}

\begin{document}
  \section{Lecture 3}
  \subsection{Math Definitions}
  For the purposes of this class, a Hilbert space is a complex vector space with an inner product. We define our inner product to be
  \[
    \braket{\begin{pmatrix}
      \alpha_1 \\
      \alpha_2
    \end{pmatrix}}{\begin{pmatrix}
      \beta_1 \\
      \beta_2
    \end{pmatrix}} = \alpha_1^*\beta_1 + \alpha_2^*\beta_2 = \begin{pmatrix}
      \alpha_1 \\
      \alpha_2
    \end{pmatrix}^\dagger \begin{pmatrix}
      \beta_1 \\
      \beta_2
    \end{pmatrix},
  \]
  where $\alpha_1, \alpha_2, \beta_1, \beta_2\in\C$. We are also going to use Dirac notation:
  \[
    \ket{\psi} = \begin{pmatrix}
      \alpha_1 \\
      \alpha_2
    \end{pmatrix}\quad\text{and}\quad \ket{\varphi} = \begin{pmatrix}
      \beta_1 \\
      \beta_2
    \end{pmatrix}.
  \]
  Furthermore, $\ket{\psi}^\dagger = \bra{\psi}$. We also use the following notations:
  \begin{align*}
    \ket{+} &= \frac{1}{\sqrt{2}}(\ket{0} + \ket{1}) \\
    \ket{-} &= \frac{1}{\sqrt{2}}(\ket{0} - \ket{1}).
  \end{align*}
  Then we have $H\ket{0} = \ket{+}$ and $H\ket{1} = \ket{-}$.
  \begin{note}{}
    Unitary matrices preserve inner products.
  \end{note}
  We also have a notion of an outer product, defined by
  \[
    \begin{pmatrix}
      \alpha_1 \\
      \alpha_2
    \end{pmatrix} \begin{pmatrix}
      \beta_1^* & \beta_2^*
    \end{pmatrix} = \begin{pmatrix}
      \alpha_1 \\
      \alpha_2
    \end{pmatrix} \begin{pmatrix}
      \beta_1 \\
      \beta_2
    \end{pmatrix}^\dagger = \begin{pmatrix}
      \alpha_1\beta_1^* & \alpha_1\beta_2^* \\
      \alpha_2\beta_1^* & \alpha_2\beta_2^*
    \end{pmatrix} = \ket{\psi}\bra{\varphi}.
  \]
  \subsection{Change of Basis}
  Suppose we have two bases: $\set{\ket{0}, \ket{1}}$ and $\set{\ket{\psi}, \ket{\varphi}}$. Suppose we want to change a vector $\ket{v}$ from the former basis to the latter basis. Then we have
  \begin{align*}
    \ket{v} &= \braket{\psi}{v}\cdot \ket{\psi} + \braket{\varphi}{v}\cdot \ket{\varphi} \\
            &= \abs{\braket{\psi}{v}}^2 + \abs{\braket{\varphi}{v}}^2.
  \end{align*}
  \subsection{Multiple Qubits}
  Suppose we have the start state
  \[
    \ket{\psi} = \alpha_{00}\ket{00} + \alpha_{01}\ket{01} + \alpha_{10}\ket{10} + \alpha_{11}\ket{11}.
  \]
  Suppose we measured the first qubit to be zero. Then the new state is
  \[
    \frac{1}{\sqrt{\abs{\alpha_{00}}^2 + \abs{\alpha_{01}}^2}}(\alpha_{00}\ket{00} + \alpha_{01}\ket{01}) = \frac{\ket{0}}{\sqrt{\abs{\alpha_{00}}^2 + \abs{\alpha_{01}}^2}}\otimes (\alpha_{00}\ket{0} + \alpha_{01}\ket{1}).
  \]
  \subsection{Tensor Product}
  \begin{definition}{Tensor Product}
    We define the \emph{tensor product} (for two vectors) to be
    \[
      \begin{pmatrix}
        \alpha_1 \\
        \alpha_2
      \end{pmatrix}\otimes \begin{pmatrix}
        \beta_1^* & \beta_2^*
      \end{pmatrix} = \begin{pmatrix}
        \alpha_1\beta_1^* & \alpha_1\beta_2^* \\
        \alpha_2\beta_1^* & \alpha_2\beta_2^*
      \end{pmatrix},
    \]
    which is the same as the outer product. In general, we have
    \[
      \begin{pmatrix}
        \alpha_{00} & \alpha_{01} \\
        \alpha_{10} & \alpha_{11}
      \end{pmatrix}\otimes B = \begin{pmatrix}
        \alpha_{00}B & \alpha_{01}B \\
        \alpha_{10}B & \alpha_{11}B
      \end{pmatrix}.
    \]
  \end{definition}
  Using this definition, we have
  \begin{align*}
    \ket{00} &= \ket{0}\otimes \ket{0} \\
             &= \begin{pmatrix}
              1 \\
              0
             \end{pmatrix} \otimes \begin{pmatrix}
              1 \\
              0
             \end{pmatrix} \\
             &= \begin{pmatrix}
              1\cdot 1 \\
              1\cdot 0 \\
              0\cdot 1 \\
              0\cdot 0
             \end{pmatrix} \\
             &= \begin{pmatrix}
              1 \\
              0 \\
              0 \\
              0
             \end{pmatrix}.
  \end{align*}
  \begin{note}{}
    If we have $\ket{\psi}$ where $\psi$ is some binary string of length $n$, then we have that $\ket{\psi}$ is a $2^n$-dimensional vector where the $\psi$\tsup{th} element (in decimal, zero-indexed) is a 1 and all the rest are 0.
  \end{note}
  \subsubsection{General Rules}
  \begin{itemize}
    \item The tensor product is \emph{not} commutative, but is associative
    \item $A\otimes (B + C) = A\otimes B + A\otimes C$
    \item $(A\otimes B)(C\otimes D) = (AC)\otimes (BD)$
    \item $(\alpha A)\otimes B = A\otimes (\alpha B) = \alpha(A\otimes B)$
  \end{itemize}
\end{document}
