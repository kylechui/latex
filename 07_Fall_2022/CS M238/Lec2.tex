\documentclass[class=article, crop=false]{standalone}
\usepackage[margin=1in]{geometry}

\usepackage{amsmath}
\usepackage{amssymb}
\usepackage{amsthm}
\usepackage{comment}
\usepackage{enumitem}
\usepackage{fancyhdr}
\usepackage{hyperref}
\usepackage{mathrsfs}
\usepackage{mathtools}
\usepackage{standalone}
\usepackage{tcolorbox}
\usepackage{tikz}
\usepackage{xcolor}

\usetikzlibrary{decorations.pathreplacing}
\tcbuselibrary{skins}

\DeclareMathOperator{\lcm}{lcm}
\DeclareMathOperator{\proj}{proj}
\DeclareMathOperator{\vspan}{span}
\DeclareMathOperator{\im}{im}
\DeclareMathOperator{\range}{range}
\DeclareMathOperator{\Diff}{Diff}
\DeclareMathOperator{\Int}{Int}
\DeclareMathOperator{\fcn}{fcn}
\DeclareMathOperator{\id}{id}

\newcommand{\N}{\ensuremath{\mathbb{N}}}
\newcommand{\Z}{\ensuremath{\mathbb{Z}}}
\newcommand{\Q}{\ensuremath{\mathbb{Q}}}
\newcommand{\R}{\ensuremath{\mathbb{R}}}
\newcommand{\C}{\ensuremath{\mathbb{C}}}
\newcommand{\F}{\ensuremath{\mathbb{F}}}
\newcommand{\lam}{\ensuremath{\lambda}}
\newcommand{\nab}{\ensuremath{\nabla}}
\newcommand{\eps}{\ensuremath{\varepsilon}}
\newcommand{\es}{\ensuremath{\varnothing}}

\newcommand{\abs}[1]{\ensuremath{\left\lvert #1 \right\rvert}}
\newcommand{\bigpar}[1]{\ensuremath{\left( #1 \right)}}
\newcommand{\norm}[1]{\ensuremath{\lVert #1\rVert}}
\newcommand{\set}[1]{\ensuremath{\left\{#1\right\}}}
\newcommand{\tuple}[1]{\ensuremath{\left\langle #1 \right\rangle}}
\newcommand{\floor}[1]{\ensuremath{\left\lfloor #1 \right\rfloor}}
\newcommand{\ceil}[1]{\ensuremath{\left\lceil #1 \right\rceil}}

\newcommand{\chapternum}{}
\newcommand{\ex}[1]{\noindent\textbf{Exercise \chapternum.{#1}.}}

\newcommand{\dx}[1]{\,\mathrm{d}#1}
\newcommand{\inv}{\ensuremath{^{-1}}}
\newcommand{\sm}{\setminus}
\newcommand{\sse}{\subseteq}
\newcommand{\ceq}{\coloneqq}

\definecolor{problemBackground}{RGB}{212,232,246}
\newenvironment{problem}[1]
  {
    \begin{tcolorbox}[
      boxrule=.5pt,
      titlerule=.5pt,
      sharp corners,
      colback=problemBackground
    ]
    \ifx &#1& \textbf{Problem. }
    \else \textbf{Problem #1.} \fi
  }
  {
    \end{tcolorbox}
  }

\definecolor{exampleBackground}{RGB}{255,249,248}
\definecolor{exampleAccent}{RGB}{158,60,14}
\newenvironment{example}[1]
  {
    \begin{tcolorbox}[
      boxrule=.5pt,
      sharp corners,
      colback=exampleBackground,
      colframe=exampleAccent,
    ]
    \color{exampleAccent}\textbf{Example.} \emph{#1}\color{black}
  }
  {
    \end{tcolorbox}
  }

\definecolor{theoremBackground}{RGB}{234,243,251}
\definecolor{theoremAccent}{RGB}{0,116,183}
\newenvironment{theorem}[1]
  {
    \begin{tcolorbox}[
      boxrule=.5pt,
      titlerule=.5pt,
      sharp corners,
      colback=theoremBackground,
      colframe=theoremAccent
    ]
      \color{theoremAccent}\textbf{Theorem --- }\emph{#1}\\\color{black}
  }
  {
    \end{tcolorbox}
  }

\definecolor{noteBackground}{RGB}{244,249,244}
\definecolor{noteAccent}{RGB}{34,139,34}
\newenvironment{note}[1]
  {
  \begin{tcolorbox}[
    enhanced,
    boxrule=0pt,
    frame hidden,
    sharp corners,
    colback=noteBackground,
    borderline west={3pt}{-1.5pt}{noteAccent}
    ]
    \ifx &#1& \color{noteAccent}\textbf{Note. }\color{black}
    \else \color{noteAccent}\textbf{Note (#1). }\color{black} \fi
    }
    {
  \end{tcolorbox}
  }

\definecolor{definitionBackground}{RGB}{246,246,246}
\newenvironment{definition}[1]
  {
    \begin{tcolorbox}[
      enhanced,
      boxrule=0pt,
      frame hidden,
      sharp corners,
      colback=definitionBackground,
      borderline west={3pt}{-1.5pt}{black}
    ]
    \textbf{Definition. }\emph{#1}\\
  }
  {
    \end{tcolorbox}
  }
\newenvironment{amatrix}[2]{
    \left[
      \begin{array}{*{#1}{c}|*{#2}c}
  }
  {
      \end{array}
    \right]
  }
\date{\the\year-\the\month-\the\day}
\author{Kyle Chui}


\fancyhf{}
\lhead{Kyle Chui}
\rhead{Page \thepage}
\pagestyle{fancy}

\begin{document}
  \section{Lecture 2}
  \begin{center}
    \begin{tabular}{c|c|c}
      & Classical Computing & Quantum Computing \\
      \hline
      Software & Boolean Algebra & Linear Algebra \\
      \hline
      Hardware & Classical Mechanics & Quantum Mechanics \\
      & e.g. semiconductors & e.g. superconductors
    \end{tabular}
  \end{center}
  \subsection{The Four Postulates of Quantum Mechanics}
  \subsubsection{State Space Rule}
  The state space is a unit vector with dimension $2^n$, where $n$ is the number of qubits. We use the following abbreviations:
  \[
    \ket{0} = \begin{pmatrix}
      1 \\
      0
    \end{pmatrix}\quad\text{and}\quad \ket{1} = \begin{pmatrix}
      0 \\
      1
    \end{pmatrix}.
  \]
  The state space rule says that we are working with superpositions (or linear combinations).
  \subsubsection{Composition Rule}
  We use the tensor product to compose objects; for each additional qubit the \emph{dimension} of our state space doubles.
  \subsubsection{Step Rule}
  We use unitary matrices to get from one state to the next state. This is because unitary matrices send unit vectors to unit vectors. For example,
  \[
    U\ket{\psi} = \ket{\varphi}.
  \]
  Note that this means that the size of $U$ is $2^n\times 2^n$.
  \subsubsection{Measurement Rule}
  There is a way to get information out of a quantum computer. When we measure a computation performed with $n$ qubits, we get $n$ bits out of the system.
  \subsection{Computations}
  For probabilistic programming, our state is no longer a bit vector, but rather a probability vector. The matrix used to update this state is called a stochastic matrix. We can take the tensor product of two vectors to get the next state.
  \[
    \begin{pmatrix}
      a \\
      b
    \end{pmatrix}\otimes \begin{pmatrix}
      c \\
      d
    \end{pmatrix} = \begin{pmatrix}
      ac \\
      ad \\
      bc \\
      bd
    \end{pmatrix}.
  \]
  \begin{center}
    \begin{tabular}{c|c|c}
      & Probabilistic Computing & Quantum Computing \\
      \hline
      Numbers & Real & Complex \\
      \hline
      State & Vector of Probabilities & Vector of Amplitudes  \\
            & $\sum p_i = 1$ & $\sum \abs{a_i}^2 = 1$ \\
      \hline
      Step & Stochastic Matrix & Unitary Matrix
    \end{tabular}
  \end{center}
  The Hadamard matrix is given by
  \[
    H = \begin{pmatrix}
      \frac{1}{\sqrt{2}} & \frac{1}{\sqrt{2}} \\
      \frac{1}{\sqrt{2}} & -\frac{1}{\sqrt{2}}
    \end{pmatrix}.
  \]
  Observe that
  \begin{align*}
    H\ket{0} &= \frac{1}{\sqrt{2}}\begin{pmatrix}
      1 & 1 \\
      1 & -1
    \end{pmatrix}\begin{pmatrix}
      1 \\
      0
    \end{pmatrix} \\
             &= \begin{pmatrix}
              \frac{1}{\sqrt{2}} \\
              \frac{1}{\sqrt{2}}
             \end{pmatrix} \\
             &= \frac{1}{\sqrt{2}}\ket{0} + \frac{1}{\sqrt{2}}\ket{1}.
  \end{align*}
  We also have
  \begin{align*}
    H\ket{1} &= \frac{1}{\sqrt{2}}\begin{pmatrix}
      1 & 1 \\
      1 & -1
    \end{pmatrix}\begin{pmatrix}
      0 \\
      1
    \end{pmatrix} \\
             &= \begin{pmatrix}
              \frac{1}{\sqrt{2}} \\
              -\frac{1}{\sqrt{2}}
             \end{pmatrix} \\
             &= \frac{1}{\sqrt{2}}\ket{0} - \frac{1}{\sqrt{2}}\ket{1}.
  \end{align*}
  This is analogous to the ``fair flip'' matrix for probabilistic computing. \par
  \subsection{Quantum Computing Trick \#1}
  One of the problems is that many of our operations are non-unitary/irreversible, so how can we use them? With the help of a helper bit $b$, we may define $U_f\colon \set{0, 1}^2\to \set{0, 1}^2$ by
  \[
    U_f(x, b) = (x, b\oplus f(x)),
  \]
  where $\oplus$ is the XOR operator. Via some computation, we find that $(U_f\circ U_f)(x, b) = (x, b)$, so we have successfully made our operation reversible. \par
  This allows us to run the quantum computation for a while. We can then stop it and reverse a few steps to help debug a program.
\end{document}
