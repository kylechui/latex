\documentclass[class=article, crop=false]{standalone}
\usepackage[margin=1in]{geometry}

\usepackage{amsmath}
\usepackage{amssymb}
\usepackage{amsthm}
\usepackage{comment}
\usepackage{enumitem}
\usepackage{fancyhdr}
\usepackage{hyperref}
\usepackage{mathrsfs}
\usepackage{mathtools}
\usepackage{standalone}
\usepackage{tcolorbox}
\usepackage{tikz}
\usepackage{xcolor}

\usetikzlibrary{decorations.pathreplacing}
\tcbuselibrary{skins}

\DeclareMathOperator{\lcm}{lcm}
\DeclareMathOperator{\proj}{proj}
\DeclareMathOperator{\vspan}{span}
\DeclareMathOperator{\im}{im}
\DeclareMathOperator{\range}{range}
\DeclareMathOperator{\Diff}{Diff}
\DeclareMathOperator{\Int}{Int}
\DeclareMathOperator{\fcn}{fcn}
\DeclareMathOperator{\id}{id}

\newcommand{\N}{\ensuremath{\mathbb{N}}}
\newcommand{\Z}{\ensuremath{\mathbb{Z}}}
\newcommand{\Q}{\ensuremath{\mathbb{Q}}}
\newcommand{\R}{\ensuremath{\mathbb{R}}}
\newcommand{\C}{\ensuremath{\mathbb{C}}}
\newcommand{\F}{\ensuremath{\mathbb{F}}}
\newcommand{\lam}{\ensuremath{\lambda}}
\newcommand{\nab}{\ensuremath{\nabla}}
\newcommand{\eps}{\ensuremath{\varepsilon}}
\newcommand{\es}{\ensuremath{\varnothing}}

\newcommand{\abs}[1]{\ensuremath{\left\lvert #1 \right\rvert}}
\newcommand{\bigpar}[1]{\ensuremath{\left( #1 \right)}}
\newcommand{\norm}[1]{\ensuremath{\lVert #1\rVert}}
\newcommand{\set}[1]{\ensuremath{\left\{#1\right\}}}
\newcommand{\tuple}[1]{\ensuremath{\left\langle #1 \right\rangle}}
\newcommand{\floor}[1]{\ensuremath{\left\lfloor #1 \right\rfloor}}
\newcommand{\ceil}[1]{\ensuremath{\left\lceil #1 \right\rceil}}

\newcommand{\chapternum}{}
\newcommand{\ex}[1]{\noindent\textbf{Exercise \chapternum.{#1}.}}

\newcommand{\dx}[1]{\,\mathrm{d}#1}
\newcommand{\inv}{\ensuremath{^{-1}}}
\newcommand{\sm}{\setminus}
\newcommand{\sse}{\subseteq}
\newcommand{\ceq}{\coloneqq}

\definecolor{problemBackground}{RGB}{212,232,246}
\newenvironment{problem}[1]
  {
    \begin{tcolorbox}[
      boxrule=.5pt,
      titlerule=.5pt,
      sharp corners,
      colback=problemBackground
    ]
    \ifx &#1& \textbf{Problem. }
    \else \textbf{Problem #1.} \fi
  }
  {
    \end{tcolorbox}
  }

\definecolor{exampleBackground}{RGB}{255,249,248}
\definecolor{exampleAccent}{RGB}{158,60,14}
\newenvironment{example}[1]
  {
    \begin{tcolorbox}[
      boxrule=.5pt,
      sharp corners,
      colback=exampleBackground,
      colframe=exampleAccent,
    ]
    \color{exampleAccent}\textbf{Example.} \emph{#1}\color{black}
  }
  {
    \end{tcolorbox}
  }

\definecolor{theoremBackground}{RGB}{234,243,251}
\definecolor{theoremAccent}{RGB}{0,116,183}
\newenvironment{theorem}[1]
  {
    \begin{tcolorbox}[
      boxrule=.5pt,
      titlerule=.5pt,
      sharp corners,
      colback=theoremBackground,
      colframe=theoremAccent
    ]
      \color{theoremAccent}\textbf{Theorem --- }\emph{#1}\\\color{black}
  }
  {
    \end{tcolorbox}
  }

\definecolor{noteBackground}{RGB}{244,249,244}
\definecolor{noteAccent}{RGB}{34,139,34}
\newenvironment{note}[1]
  {
  \begin{tcolorbox}[
    enhanced,
    boxrule=0pt,
    frame hidden,
    sharp corners,
    colback=noteBackground,
    borderline west={3pt}{-1.5pt}{noteAccent}
    ]
    \ifx &#1& \color{noteAccent}\textbf{Note. }\color{black}
    \else \color{noteAccent}\textbf{Note (#1). }\color{black} \fi
    }
    {
  \end{tcolorbox}
  }

\definecolor{definitionBackground}{RGB}{246,246,246}
\newenvironment{definition}[1]
  {
    \begin{tcolorbox}[
      enhanced,
      boxrule=0pt,
      frame hidden,
      sharp corners,
      colback=definitionBackground,
      borderline west={3pt}{-1.5pt}{black}
    ]
    \textbf{Definition. }\emph{#1}\\
  }
  {
    \end{tcolorbox}
  }
\newenvironment{amatrix}[2]{
    \left[
      \begin{array}{*{#1}{c}|*{#2}c}
  }
  {
      \end{array}
    \right]
  }
\date{\the\year-\the\month-\the\day}
\author{Kyle Chui}


\fancyhf{}
\lhead{Kyle Chui}
\rhead{Page \thepage}
\pagestyle{fancy}

\begin{document}
  \section{Lecture 2}
  One of the big ideas of data representation is \emph{composition}, e.g. we can take pairs of integers to represent the rationals.
  \subsection{Prefix-Free Encoding}
  \begin{definition}{Prefix-Free Encoding}
    We say that a function $E\colon \Theta\to \set{0, 1}^*$ is \emph{prefix-free} if for all $x,y\in\Theta$ where $x\neq y$, we have $E(x)$ \emph{is not} a prefix of $E(y)$.
  \end{definition}
  \begin{example}{}
    Consider the $NtoB$ function, that converts natural numbers to their binary counterparts. Then we know that $NtoB$ is \emph{not} a prefix-free encoding, since $NtoB(1) = 1$ is a prefix for $NtoB(3) = 11$.
  \end{example}
  \begin{example}{}
    Consider the modified $\overline{ZtoB}$ encoding function (which duplicates each bit and adds 01 to the end). This is a prefix-free function, since the 01 for $x$ must match with the 01 in $y$; then for $x$ to be a prefix for $y$ we must have $x = y$.
  \end{example}
  Prefix-free encodings are quite useful, since the bit representation is unique and unambiguous (we don't need any special characters to demarcate objects).
  \begin{theorem}{}
    Suppose we have a prefix-free encoding
    \[
      E\colon \Theta\to \set{0, 1}^*.
    \]
    We define $\overline{E}\colon \Theta^*\to \set{0, 1}^*$ by
    \[
      \overline{E}((x_1,x_2,\dotsc,x_n)) = E(x_1)\circ E(x_2)\dotsb E(x_n).
    \]
    Then $\overline{E}$ is a valid encoding of $\Theta^*$.
    
  \end{theorem}
  \begin{proof}
    Suppose we have the binary sequence
    \[
      \overline{E}((x_1,x_2,\dotsc,x_n)) = E(x_1)\circ E(x_2)\dotsb E(x_n).
    \]
    Then we define the following decoding algorithm:
    \begin{algorithm}
      \DontPrintSemicolon
      \caption{}
      \While{we still have remaining bits}{
        Keep reading in bits until the sequence matches an encoding.\;
        Once we find it, that is the data representation for our next object; go to step 2.\;
      }
    \end{algorithm} \\
    The above algorithm is a valid decoder, since our encoding is prefix-free and the data representation is unambiguous.
  \end{proof}
  \begin{theorem}{}
    Suppose $E\colon \Theta\to \set{0, 1}^*$ is an encoding. We claim that there exists a prefix-free encoding $pfE\colon \Theta\to \set{0, 1}^*$ defined by the result of:
    \begin{itemize}
      \item Computing $E(a)$
      \item Replacing each 0 with 00, and each 1 with 11
      \item Appending 01 at the end
    \end{itemize}
    Then we have that $pfE$ is a prefix-free encoding.
  \end{theorem}
  \begin{proof}
    Let $x$ have the binary representation $E(x) = b_0b_1\dotsc b_n$. Then $pfE(x) = b_0b_0b_1b_1\dotsc b_nb_n01$. Since the 01 may only occur at the end, we have that $pfE$ is a prefix-free encoding.
  \end{proof}
  We now have ways to:
  \begin{itemize}
    \item Represent integers as binary strings in $\set{0, 1}^*$.
    \item Represent integers using a prefix-free encoding.
    \item Represent lists of integers.
    \item Represent lists of integers using a prefix-free encoding.
    \item Represent lists of lists of integers.
  \end{itemize}
  \subsubsection{Efficiency of Prefix-Free Encodings}
  The length of $pfE(x)$ is $2\cdot \abs{E(x)} + 2$. There exists a different transformation where the new encoding has length
  \[
    \abs{E(x)} + 2\cdot \log_2\abs{E(x)} + 2.
  \]
  Can we represent real numbers as $\set{0, 1}^*$? In other words, is there an injective mapping $E\colon \R\to \set{0, 1}^*$? \par
  \begin{note}{Limitations of Encoding}
    No, since the number of binary strings is countable and the number of real numbers is uncountable.
  \end{note}
  \subsection{Algorithms}
  \begin{definition}{Boolean Circuit}
    Some computations with AND/OR/NOT as basic operations.
  \end{definition}
  \begin{definition}{Directed Acyclic Graph}
    A directed graph that has no cycles.
  \end{definition}
\end{document}
