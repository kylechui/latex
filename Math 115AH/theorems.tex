\documentclass[class=article, crop=false]{standalone}
\usepackage[margin=1in]{geometry}

\usepackage{amsmath}
\usepackage{amssymb}
\usepackage{amsthm}
\usepackage{comment}
\usepackage{enumitem}
\usepackage{fancyhdr}
\usepackage{hyperref}
\usepackage{mathrsfs}
\usepackage{mathtools}
\usepackage{standalone}
\usepackage{tcolorbox}
\usepackage{tikz}
\usepackage{xcolor}

\usetikzlibrary{decorations.pathreplacing}
\tcbuselibrary{skins}

\DeclareMathOperator{\lcm}{lcm}
\DeclareMathOperator{\proj}{proj}
\DeclareMathOperator{\vspan}{span}
\DeclareMathOperator{\im}{im}
\DeclareMathOperator{\range}{range}
\DeclareMathOperator{\Diff}{Diff}
\DeclareMathOperator{\Int}{Int}
\DeclareMathOperator{\fcn}{fcn}
\DeclareMathOperator{\id}{id}

\newcommand{\N}{\ensuremath{\mathbb{N}}}
\newcommand{\Z}{\ensuremath{\mathbb{Z}}}
\newcommand{\Q}{\ensuremath{\mathbb{Q}}}
\newcommand{\R}{\ensuremath{\mathbb{R}}}
\newcommand{\C}{\ensuremath{\mathbb{C}}}
\newcommand{\F}{\ensuremath{\mathbb{F}}}
\newcommand{\lam}{\ensuremath{\lambda}}
\newcommand{\nab}{\ensuremath{\nabla}}
\newcommand{\eps}{\ensuremath{\varepsilon}}
\newcommand{\es}{\ensuremath{\varnothing}}

\newcommand{\abs}[1]{\ensuremath{\left\lvert #1 \right\rvert}}
\newcommand{\bigpar}[1]{\ensuremath{\left( #1 \right)}}
\newcommand{\norm}[1]{\ensuremath{\lVert #1\rVert}}
\newcommand{\set}[1]{\ensuremath{\left\{#1\right\}}}
\newcommand{\tuple}[1]{\ensuremath{\left\langle #1 \right\rangle}}
\newcommand{\floor}[1]{\ensuremath{\left\lfloor #1 \right\rfloor}}
\newcommand{\ceil}[1]{\ensuremath{\left\lceil #1 \right\rceil}}

\newcommand{\chapternum}{}
\newcommand{\ex}[1]{\noindent\textbf{Exercise \chapternum.{#1}.}}

\newcommand{\dx}[1]{\,\mathrm{d}#1}
\newcommand{\inv}{\ensuremath{^{-1}}}
\newcommand{\sm}{\setminus}
\newcommand{\sse}{\subseteq}
\newcommand{\ceq}{\coloneqq}

\definecolor{problemBackground}{RGB}{212,232,246}
\newenvironment{problem}[1]
  {
    \begin{tcolorbox}[
      boxrule=.5pt,
      titlerule=.5pt,
      sharp corners,
      colback=problemBackground
    ]
    \ifx &#1& \textbf{Problem. }
    \else \textbf{Problem #1.} \fi
  }
  {
    \end{tcolorbox}
  }

\definecolor{exampleBackground}{RGB}{255,249,248}
\definecolor{exampleAccent}{RGB}{158,60,14}
\newenvironment{example}[1]
  {
    \begin{tcolorbox}[
      boxrule=.5pt,
      sharp corners,
      colback=exampleBackground,
      colframe=exampleAccent,
    ]
    \color{exampleAccent}\textbf{Example.} \emph{#1}\color{black}
  }
  {
    \end{tcolorbox}
  }

\definecolor{theoremBackground}{RGB}{234,243,251}
\definecolor{theoremAccent}{RGB}{0,116,183}
\newenvironment{theorem}[1]
  {
    \begin{tcolorbox}[
      boxrule=.5pt,
      titlerule=.5pt,
      sharp corners,
      colback=theoremBackground,
      colframe=theoremAccent
    ]
      \color{theoremAccent}\textbf{Theorem --- }\emph{#1}\\\color{black}
  }
  {
    \end{tcolorbox}
  }

\definecolor{noteBackground}{RGB}{244,249,244}
\definecolor{noteAccent}{RGB}{34,139,34}
\newenvironment{note}[1]
  {
  \begin{tcolorbox}[
    enhanced,
    boxrule=0pt,
    frame hidden,
    sharp corners,
    colback=noteBackground,
    borderline west={3pt}{-1.5pt}{noteAccent}
    ]
    \ifx &#1& \color{noteAccent}\textbf{Note. }\color{black}
    \else \color{noteAccent}\textbf{Note (#1). }\color{black} \fi
    }
    {
  \end{tcolorbox}
  }

\definecolor{definitionBackground}{RGB}{246,246,246}
\newenvironment{definition}[1]
  {
    \begin{tcolorbox}[
      enhanced,
      boxrule=0pt,
      frame hidden,
      sharp corners,
      colback=definitionBackground,
      borderline west={3pt}{-1.5pt}{black}
    ]
    \textbf{Definition. }\emph{#1}\\
  }
  {
    \end{tcolorbox}
  }
\newenvironment{amatrix}[2]{
    \left[
      \begin{array}{*{#1}{c}|*{#2}c}
  }
  {
      \end{array}
    \right]
  }
\date{\the\year-\the\month-\the\day}
\author{Kyle Chui}


\fancyhf{}
\lhead{\leftmark}
\rhead{Page \thepage}
\pagestyle{fancy}

\begin{document}
  \section{Vector Spaces}
  \begin{theorem}{Subspace Theorem}
    Let $V$ be a vector space over $F$, $\es\neq W\sse V$ a subset. Then the following are equivalent:
    \begin{enumerate}[label=(\alph*)]
      \item $W$ is a subspace of $V$
      \item $W$ is closed under addition and scalar multiplication from $V$
      \item For all $w_1,w_2\in W$, for all $\alpha\in F$, $\alpha w_1+w_2\in W$
    \end{enumerate}
  \end{theorem}
  \begin{theorem}{Toss In Theorem}
    Let $V$ be a vector space over $F$, $\es\neq S\sse V$ a linearly independent subset. Suppose that $v\in V\sm \vspan(S)$. Then $S\cup \set{v}$ is linearly independent.
  \end{theorem}
  \begin{theorem}{Coordinate Theorem}
    Let $V$ be a finite dimensional vector space over $F$ with basis $\mathcal{B} = \set{v_1,\dotsc,v_n}$ and $v\in V$. Then there exists unique $\alpha_1,\dotsc,\alpha_n\in F$ such that
    \[
      v = \alpha_1v_1 + \dotsb + \alpha_nv_n.
    \]
    We call $\alpha_1,\dotsc,\alpha_n$ the \emph{coordinates} of $v$ relative to the basis $\mathcal{B}$ and call $\alpha_i$ the $i$th coordinate relative to $\mathcal{B}$.
  \end{theorem}
  \begin{theorem}{Important Exercise (General Toss Out Theorem)}
    Let $V$ be a vector space over $F$, with $v_1,\dotsc,v_n\in V$. Then
    \[
      \vspan(v_1,\dotsc,v_n) = \vspan(v_2,\dotsc,v_n)
    \]
    if and only if $v_1\in \vspan(v_2,\dotsc,v_n)$.
  \end{theorem}
  \begin{theorem}{Toss Out Theorem}
    Let $V$ be a vector space over $F$. If $V$ can be spanned by finitely many vectors then $V$ is a finite dimensional vector space over $F$. More precisely, if 
    \[
      V = \vspan(v_1,\dotsc,v_n),
    \]
    then a subset of $\set{v_1,\dotsc,v_n}$ is a basis for $V$.
  \end{theorem}
  \begin{theorem}{Replacement Theorem}
    Let $V$ be a vector space over $F$, with $\set{v_1,\dotsc,v_n}$ a basis for $V$. Suppose that $v\in V$ satisfies
    \[
      v = \alpha_1v_1 + \dotsb + \alpha_nv_n, \tag*{$\alpha_1,\dotsc,\alpha_n\in F, \alpha_i\neq 0$}
    \]
    Then
    \[
      \set{v_1,\dotsc,v_{i-1},v,v_{i+1},\dotsc,v_n}
    \]
    is also a basis for $V$.
  \end{theorem}
  \begin{theorem}{Main Theorem}
    Suppose $V$ is a vector space over $F$ with $V = \vspan(v_1,\dotsc,v_n)$. Then any linearly independent subset of $V$ has at most $n$ elements.
  \end{theorem}
  \begin{theorem}{Extension Theorem}
    Let $V$ be a finite dimensional vector space over $F$, $W\sse V$ a subspace. Then every linearly independent subset $S$ in $W$ is finite and part of a basis for $W$ which is a finite dimensional vector space over $F$.
  \end{theorem}
  \begin{theorem}{Counting Theorem}
    Let $V$ be a vector space over $F$, with $W_1,W_2\sse V$ subspaces. Suppose that both $W_1$ and $W_2$ are finite dimensional vector spaces over $F$. Then
    \begin{enumerate}[label=(\alph*)]
      \item $W_1\cap W_2$ is a finite dimensional vector space over $F$.
      \item $W_1 + W_2$ is a finite dimensional vector space over $F$.
      \item $\dim(W_1) + \dim(W_2) = \dim(W_1 + W_2) + \dim(W_1\cap W_2)$.
    \end{enumerate}
  \end{theorem}
  \section{Linear Transformations}
  \begin{theorem}{Rank-Nullity (Dimension) Theorem}
    Let $V$ and $W$ be vector spaces over a field $F$, and let $V$ be finite dimensional. Let $T\colon V\to W$ be a linear transformation. Then $\im(T)$ and $\ker(T)$ are finite dimensional vector spaces over $F$, and $\dim(V) = \dim(\ker(T)) + \dim(\im(T))$.
  \end{theorem}
  \begin{definition}{'Morphisms}
    Let $V$ and $W$ be vector spaces over the field $F$. Let $T\colon V\to W$ be linear.
    \begin{enumerate}[label=(\alph*)]
      \item We say that $T$ is a monomorphism or nonsingular if $T$ is injective.
      \item We say that $T$ is an epimorphism if $T$ is surjective.
      \item We say that $T$ is an isomorphism if $T$ is bijective and $T\inv\colon W\to V$ is linear.
      \begin{itemize}
        \item We only need to show that $T$ is bijective because that implies that $T\inv$ is linear.
      \end{itemize}
    \end{enumerate}
  \end{definition}
  \begin{theorem}{Monomorphism Theorem}
    Let $T\colon V\to W$ be a linear transformation. Then the following are equivalent.
    \begin{enumerate}[label=(\alph*)]
      \item $T$ is injective.
      \item $T$ takes linearly independent sets in $V$ to linearly independent sets in $W$.
      \item $\ker(T) = \set{0}$.
      \item $\dim(\ker(T)) = 0$.
    \end{enumerate}
  \end{theorem}
  \begin{theorem}{Isomorphism Theorem}
    Suppose that $V$ and $W$ are finite dimensional vector spaces over $F$ with $\dim(V) = \dim(W)$. Let $T\colon V\to W$ be a linear transformation. Then the following are equivalent.
    \begin{enumerate}[label=(\alph*)]
      \item $T$ is an isomorphism.
      \item $T$ is a monomorphism.
      \item $T$ is an epimorphism.
      \item If $\mathcal{B} = \set{v_1, \dotsc, v_n}$ is a basis for $V$, then $\set{Tv_1, \dotsc,Tv_n}$ is a basis for $W$.
      \item There exists a basis $\mathcal{B}$ of $V$ that maps to a basis of $W$.
    \end{enumerate}
  \end{theorem}
  \begin{theorem}{Universal Property of Vector Spaces}
    Let $V$ be a finite dimensional vector space over $F$, and let $\mathcal{B} = \set{v_1, \dotsc,v_n}$ be a basis for $V$. Let $W$ be a vector space over $F$, and let $w_1,\dotsc,w_n\in W$ (not necessarily distinct). Then there exists a unique linear transformation $T$ with $T\colon v_i\mapsto w_i$.
  \end{theorem}
  \begin{theorem}{Classification of Finite Dimensional Vector Spaces}
    Let $V$ and $W$ be finite dimensional vector spaces over the field $F$. Then $V\cong W$ if and only if $\dim(V) = \dim(W)$.
  \end{theorem}
  \section{Linear Transformations and Matrices}
  \begin{theorem}{Matrix Theory Theorem (MTT)}
    Let $V$ and $W$ be finite dimensional vector spaces of dimesnion $n$ and $m$ over $F$ respectively, and let $\mathcal{B}$ and $\mathcal{C}$ be ordered bases for $V$ and $W$. Then the map
    \begin{align*}
      \varphi\colon L(V, W)&\to F^{m\times n} \\
      T &\mapsto [T]_{\mathcal{B},\mathcal{C}}
    \end{align*}
    is an isomorphism. In particular, $\dim(L(V,W)) = mn$.
  \end{theorem}
  \begin{theorem}{12.2}
    Let $V$, $W$, $U$ be finite dimensional vector spaces over $F$ with ordered bases $\mathcal{B}$, $\mathcal{C}$, $\mathcal{D}$ respectively. If $T\colon V\to W$ and $S\colon W\to U$ are linear, then
    \[
      [S\circ T]_{\mathcal{B}, \mathcal{D}} = [S]_{\mathcal{C},\mathcal{D}}\cdot [T]_{\mathcal{B},\mathcal{C}}.
    \]
  \end{theorem}
  \begin{theorem}{Change of Basis Theorem}
    Let $V$ and $W$ be finite dimensional vector spaces over $F$ with ordered bases $\mathcal{B}, \mathcal{B}'$ for $V$ and $\mathcal{C}$, $\mathcal{C}'$ for $W$. Let $T\colon V\to W$ be linear. Then
    \begin{align*}
      [T]_{\mathcal{B},\mathcal{C}} &= [1_W]_{\mathcal{C}',\mathcal{C}}[T]_{\mathcal{B}',\mathcal{C}'}[1_V]_{\mathcal{B},\mathcal{B}'} \\
      &= [1_W]_{\mathcal{C},\mathcal{C}'}\inv[T]_{\mathcal{B}',\mathcal{C}'}[1_V]_{\mathcal{B},\mathcal{B}'} \\
      &= [1_W]_{\mathcal{C}',\mathcal{C}}[T]_{\mathcal{B}',\mathcal{C}'}[1_V]_{\mathcal{B}',\mathcal{B}}\inv
    \end{align*}
  \end{theorem}
  \begin{theorem}{Similar Matrices $\iff$ Same Transformation, Different Bases (self-named)}
    Let $A, B\in \M_n(F)$. Then $A\sim B$ if and only if there exists a vector space over $F$, $\dim(V) = n$, $T\colon V\to V$ linear and ordered bases $\mathcal{B}$, $\mathcal{C}$ for $V$ such that
    \[
      A = [T]_{\mathcal{B}}\quad \text{and} \quad B = [T]_{\mathcal{C}}.
    \]
    In other words, $A\sim B$ if and only if they represent the same linear transformation relative to (possibly) different ordered bases.
  \end{theorem}
  \begin{definition}{Eigenstuffs}
    Let $0\neq V$ be a vector space over $F$, $T\colon V\to V$ a \emph{linear operator} (a linear transformation where the domain = target) and $\lam\in F$. Set
    \[
      S_\lam\ceq T - 1_V\colon V\to V
    \]
    a linear operator, so
    \[
      S_\lam(v) = Tv - \lam v \tag{For all $v\in V$}
    \]
    We say $\lam$ is an \emph{eigenvalue} of $T$ if $S_\lam$ is \emph{not} injective ($\ker(S_\lam)\neq 0$).
    Let
    \[
      E_T(\lam)\ceq \ker(S_\lam) = \set{v\in V\mid Tv - \lam v = 0} = \set{v\in V\mid Tv = \lam v}.
    \]
    If $E_T(\lam)\neq 0$, we call $E_T(\lam)$ an \emph{eigenspace} of $V$ relative to $T$, $\lam$, and any vector $v\in E_T(\lam)$ is called an \emph{eigenvector} of $T$ relative to $\lam$. So if $T\colon V\to V$ is linear, $\lam\in F$ is an eigenvalue of $T$ if and only if there exists some non-zero vector $v\in V$ such that $Tv = \lam v$.
  \end{definition}
  \begin{note}{Eigenspaces are Subspaces}
    Eigenspaces are just kernels, so they are subspaces.
  \end{note}
  \begin{definition}{Characteristic Polynomial}
    Let $A\in \M_n(F)$. Define
    \[
      f_A = \det(tI - A)\in F[t],
    \]
    called the \emph{characteristic polynommial} of $A$.
  \end{definition}
  \textbf{Proposition.} If $A, B\in \M_n(F)$ are similar, then $f_A = f_B$.
  \begin{proof}
    If $A = C\inv BC$, where $C$ is invertible, then 
    \begin{align*}
      f_A &= \det(C\inv (tI - B)C) \\
          &= \det(C\inv)\det(tI-B)\det(C) \\
          &= \det(tI-B) \\
          &= f_B.
    \end{align*}
  \end{proof}
  \begin{theorem}{14.3}
    Let $V$ be a finite dimensional vector space over $F$, and let $T\colon V\to V$ be linear. Then the eigenvalues of $T$ are precisely the roots of $f_T$.
  \end{theorem}
  \begin{theorem}{Cayley-Hamilton Theorem}
    Let $A\in \M_n(F)$. Then $f_A(A) = 0$. In other words, you get $0$ when you plug $A$ into the expansion of the determinant $f_A$. \par
    \textbf{WARNING!!!} You \emph{must} prove this in order to use this!!!
  \end{theorem}
\end{document}

