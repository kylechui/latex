\documentclass[class=article, crop=false]{standalone}
\usepackage[margin=1in]{geometry}

\usepackage{amsmath}
\usepackage{amssymb}
\usepackage{amsthm}
\usepackage{comment}
\usepackage{enumitem}
\usepackage{fancyhdr}
\usepackage{hyperref}
\usepackage{mathrsfs}
\usepackage{mathtools}
\usepackage{standalone}
\usepackage{tcolorbox}
\usepackage{tikz}
\usepackage{xcolor}

\usetikzlibrary{decorations.pathreplacing}
\tcbuselibrary{skins}

\DeclareMathOperator{\lcm}{lcm}
\DeclareMathOperator{\proj}{proj}
\DeclareMathOperator{\vspan}{span}
\DeclareMathOperator{\im}{im}
\DeclareMathOperator{\range}{range}
\DeclareMathOperator{\Diff}{Diff}
\DeclareMathOperator{\Int}{Int}
\DeclareMathOperator{\fcn}{fcn}
\DeclareMathOperator{\id}{id}

\newcommand{\N}{\ensuremath{\mathbb{N}}}
\newcommand{\Z}{\ensuremath{\mathbb{Z}}}
\newcommand{\Q}{\ensuremath{\mathbb{Q}}}
\newcommand{\R}{\ensuremath{\mathbb{R}}}
\newcommand{\C}{\ensuremath{\mathbb{C}}}
\newcommand{\F}{\ensuremath{\mathbb{F}}}
\newcommand{\lam}{\ensuremath{\lambda}}
\newcommand{\nab}{\ensuremath{\nabla}}
\newcommand{\eps}{\ensuremath{\varepsilon}}
\newcommand{\es}{\ensuremath{\varnothing}}

\newcommand{\abs}[1]{\ensuremath{\left\lvert #1 \right\rvert}}
\newcommand{\bigpar}[1]{\ensuremath{\left( #1 \right)}}
\newcommand{\norm}[1]{\ensuremath{\lVert #1\rVert}}
\newcommand{\set}[1]{\ensuremath{\left\{#1\right\}}}
\newcommand{\tuple}[1]{\ensuremath{\left\langle #1 \right\rangle}}
\newcommand{\floor}[1]{\ensuremath{\left\lfloor #1 \right\rfloor}}
\newcommand{\ceil}[1]{\ensuremath{\left\lceil #1 \right\rceil}}

\newcommand{\chapternum}{}
\newcommand{\ex}[1]{\noindent\textbf{Exercise \chapternum.{#1}.}}

\newcommand{\dx}[1]{\,\mathrm{d}#1}
\newcommand{\inv}{\ensuremath{^{-1}}}
\newcommand{\sm}{\setminus}
\newcommand{\sse}{\subseteq}
\newcommand{\ceq}{\coloneqq}

\definecolor{problemBackground}{RGB}{212,232,246}
\newenvironment{problem}[1]
  {
    \begin{tcolorbox}[
      boxrule=.5pt,
      titlerule=.5pt,
      sharp corners,
      colback=problemBackground
    ]
    \ifx &#1& \textbf{Problem. }
    \else \textbf{Problem #1.} \fi
  }
  {
    \end{tcolorbox}
  }

\definecolor{exampleBackground}{RGB}{255,249,248}
\definecolor{exampleAccent}{RGB}{158,60,14}
\newenvironment{example}[1]
  {
    \begin{tcolorbox}[
      boxrule=.5pt,
      sharp corners,
      colback=exampleBackground,
      colframe=exampleAccent,
    ]
    \color{exampleAccent}\textbf{Example.} \emph{#1}\color{black}
  }
  {
    \end{tcolorbox}
  }

\definecolor{theoremBackground}{RGB}{234,243,251}
\definecolor{theoremAccent}{RGB}{0,116,183}
\newenvironment{theorem}[1]
  {
    \begin{tcolorbox}[
      boxrule=.5pt,
      titlerule=.5pt,
      sharp corners,
      colback=theoremBackground,
      colframe=theoremAccent
    ]
      \color{theoremAccent}\textbf{Theorem --- }\emph{#1}\\\color{black}
  }
  {
    \end{tcolorbox}
  }

\definecolor{noteBackground}{RGB}{244,249,244}
\definecolor{noteAccent}{RGB}{34,139,34}
\newenvironment{note}[1]
  {
  \begin{tcolorbox}[
    enhanced,
    boxrule=0pt,
    frame hidden,
    sharp corners,
    colback=noteBackground,
    borderline west={3pt}{-1.5pt}{noteAccent}
    ]
    \ifx &#1& \color{noteAccent}\textbf{Note. }\color{black}
    \else \color{noteAccent}\textbf{Note (#1). }\color{black} \fi
    }
    {
  \end{tcolorbox}
  }

\definecolor{definitionBackground}{RGB}{246,246,246}
\newenvironment{definition}[1]
  {
    \begin{tcolorbox}[
      enhanced,
      boxrule=0pt,
      frame hidden,
      sharp corners,
      colback=definitionBackground,
      borderline west={3pt}{-1.5pt}{black}
    ]
    \textbf{Definition. }\emph{#1}\\
  }
  {
    \end{tcolorbox}
  }
\newenvironment{amatrix}[2]{
    \left[
      \begin{array}{*{#1}{c}|*{#2}c}
  }
  {
      \end{array}
    \right]
  }
\date{\the\year-\the\month-\the\day}
\author{Kyle Chui}


\fancyhf{}
\lhead{Kyle Chui}
\rhead{Page \thepage}
\pagestyle{fancy}

\begin{document}
  \section{Taish\=o Period (1912--1926)}
  \begin{itemize}
    \item 1912--1913 Taish\=o financial and political crisis
    \item $1915$ Twenty-one demands to China
    \item 1918-1922 Japan attempts to control eastern Siberia
    \item $1918$ Rice riots
    \item $1918$ Hara Takashi government
    \item $1921$ Women gain right to attend political meetings (not vote)
    \item $1923$ Great Kant\=o earthquake
    \item $1925$ Universal suffrage law (all males $25$ and older)
    \item $1925$ Peace preservation law
  \end{itemize}
  \subsection{Taish\=o Political Crisis}
  \begin{itemize}
    \item 1912--1913 Overspending leads to a financial crisis and competition between budget priorities of military and domestic spending priorities
    \item Seiy\=ukai overthrows the cabinet
    \item Defeated Prime Minister founds another political party
    \item Presence of two parties in diet made government more democratic, but the position of the military was also strengthened
  \end{itemize}
  \begin{note}{}
    It was very easy for the military to move ex-military members into government, giving them more power.
  \end{note}
  \subsection{Japanese Economy during WWI}
  \begin{itemize}
    \item $1914$ Japan declares war on Germany (seizes German colonies in East Asia and Pacific)
    \item $1915$ Twenty-one demands to China
    \begin{itemize}
      \item Japan obtains additional investment rights in Shandong peninsula
      \item Demands for severely unequal treaty agreements
      \begin{note}{}
        There is severe backlash from the Chinese for this.
      \end{note}
    \end{itemize}
    \item Economic boom (demand for industrial products, less competition from Europe)
    \item Increased agricultural/industrial production
    \begin{note}{}
      A lot of workers work very hard (with little protections) in the mines/textiles, because it's \emph{better} than working on the farms.
    \end{note}
    \item Expensive attempt to halt Bolshevik Russia expansion in Eastern Siberia
    \item $1918$ Economic boom causes inflation, widening inequality (leading to rice riots)
    \begin{note}{}
      Japan's economic expansion is inextricably linked to its foreign exploits.
    \end{note}
  \end{itemize}
  Hara Takashi (also known as Hara Kei) was the Prime Minister from 1918--1921, and was the first ``commoner'' to serve. He was assassinated in $1921$ by a right-wing fanatic.
  \subsection{Post WWI Japan}
  \begin{itemize}
    \item $1919$ Treaty of Versailles---Debate on racial equality clause (all races are equal clause) rejected because many European countries have colonies across the world. It was not politically beneficial to ratify this clause, so it's vetoed by the other countries
    \begin{note}{}
      This is a big loss for Japan, as they were the only non-white global power at the time.
    \end{note}
    \item $1919$ Brutal suppression of Korean independence movement
    \item 1919--1920 Formation of the League of Nations
    \item 1921--1922 Washington Conference---Four Power Treaty (France, Great Britain, Japan, US)
  \end{itemize}
  \subsection{Taish\=o Society}
  \begin{itemize}
    \item Old middle class and new middle class in cities
    \item Emerging urban social groups and categories
    \item Immigrants from colonies ($2500$ to $420000$ Koreans from $1910$ to $1930$)
    \item Rural areas very hierarchical, widening the gap between landlords and tenants
  \end{itemize}
  \subsection{Taish\=o ``Culture''}
  \begin{note}{}
    We use the word ``culture'' here, but it really translates better to the word ``modern''.
  \end{note}
  \begin{itemize}
    \item Urbanisation, modern social classes
    \item High levels of literacy
    \item Magazines (aimed at young intellectuals, women, students, etc.)
    \item Literature, Philosophy, Political radicalism
    \item Feminist Liberation (sexual, financial, political)
    \item Fashion, Literary fiction, Caf\'e culture
    \item Jazz, internationalism, modern art, ``Culture Homes'' (modern homes)
  \end{itemize}
  \begin{note}{}
    The ``Moga'' or ``Modern Girl'' was iconic to the Taish\=o period. She is seen as ``cool'' and many women aspire to be her. The male counterpart was the ``Mobo'' or ``Modern Boy''.
  \end{note}
  \subsection{Women's Suffrage Movement}
  Part of a trend of simultaneous increased oppression and liberalism at the same time.
  \subsection{Great Kant\=o Earthquake}
  \begin{itemize}
    \item September 1\tsup{st}, 1923
    \item Magnitude 7.9
    \item $570000$ dwellings destroyed in Tokyo (3/4 of the city)
    \item Over $140000$ estimated dead
  \end{itemize}
  After the earthquake, there was a massacre of Koreans, and prominent activists were murdered by police. There was a national recovery and international aid.
\end{document}
