\documentclass[class=article, crop=false]{standalone}
\usepackage[margin=1in]{geometry}

\usepackage{amsmath}
\usepackage{amssymb}
\usepackage{amsthm}
\usepackage{comment}
\usepackage{enumitem}
\usepackage{fancyhdr}
\usepackage{hyperref}
\usepackage{mathrsfs}
\usepackage{mathtools}
\usepackage{standalone}
\usepackage{tcolorbox}
\usepackage{tikz}
\usepackage{xcolor}

\usetikzlibrary{decorations.pathreplacing}
\tcbuselibrary{skins}

\DeclareMathOperator{\lcm}{lcm}
\DeclareMathOperator{\proj}{proj}
\DeclareMathOperator{\vspan}{span}
\DeclareMathOperator{\im}{im}
\DeclareMathOperator{\range}{range}
\DeclareMathOperator{\Diff}{Diff}
\DeclareMathOperator{\Int}{Int}
\DeclareMathOperator{\fcn}{fcn}
\DeclareMathOperator{\id}{id}

\newcommand{\N}{\ensuremath{\mathbb{N}}}
\newcommand{\Z}{\ensuremath{\mathbb{Z}}}
\newcommand{\Q}{\ensuremath{\mathbb{Q}}}
\newcommand{\R}{\ensuremath{\mathbb{R}}}
\newcommand{\C}{\ensuremath{\mathbb{C}}}
\newcommand{\F}{\ensuremath{\mathbb{F}}}
\newcommand{\lam}{\ensuremath{\lambda}}
\newcommand{\nab}{\ensuremath{\nabla}}
\newcommand{\eps}{\ensuremath{\varepsilon}}
\newcommand{\es}{\ensuremath{\varnothing}}

\newcommand{\abs}[1]{\ensuremath{\left\lvert #1 \right\rvert}}
\newcommand{\bigpar}[1]{\ensuremath{\left( #1 \right)}}
\newcommand{\norm}[1]{\ensuremath{\lVert #1\rVert}}
\newcommand{\set}[1]{\ensuremath{\left\{#1\right\}}}
\newcommand{\tuple}[1]{\ensuremath{\left\langle #1 \right\rangle}}
\newcommand{\floor}[1]{\ensuremath{\left\lfloor #1 \right\rfloor}}
\newcommand{\ceil}[1]{\ensuremath{\left\lceil #1 \right\rceil}}

\newcommand{\chapternum}{}
\newcommand{\ex}[1]{\noindent\textbf{Exercise \chapternum.{#1}.}}

\newcommand{\dx}[1]{\,\mathrm{d}#1}
\newcommand{\inv}{\ensuremath{^{-1}}}
\newcommand{\sm}{\setminus}
\newcommand{\sse}{\subseteq}
\newcommand{\ceq}{\coloneqq}

\definecolor{problemBackground}{RGB}{212,232,246}
\newenvironment{problem}[1]
  {
    \begin{tcolorbox}[
      boxrule=.5pt,
      titlerule=.5pt,
      sharp corners,
      colback=problemBackground
    ]
    \ifx &#1& \textbf{Problem. }
    \else \textbf{Problem #1.} \fi
  }
  {
    \end{tcolorbox}
  }

\definecolor{exampleBackground}{RGB}{255,249,248}
\definecolor{exampleAccent}{RGB}{158,60,14}
\newenvironment{example}[1]
  {
    \begin{tcolorbox}[
      boxrule=.5pt,
      sharp corners,
      colback=exampleBackground,
      colframe=exampleAccent,
    ]
    \color{exampleAccent}\textbf{Example.} \emph{#1}\color{black}
  }
  {
    \end{tcolorbox}
  }

\definecolor{theoremBackground}{RGB}{234,243,251}
\definecolor{theoremAccent}{RGB}{0,116,183}
\newenvironment{theorem}[1]
  {
    \begin{tcolorbox}[
      boxrule=.5pt,
      titlerule=.5pt,
      sharp corners,
      colback=theoremBackground,
      colframe=theoremAccent
    ]
      \color{theoremAccent}\textbf{Theorem --- }\emph{#1}\\\color{black}
  }
  {
    \end{tcolorbox}
  }

\definecolor{noteBackground}{RGB}{244,249,244}
\definecolor{noteAccent}{RGB}{34,139,34}
\newenvironment{note}[1]
  {
  \begin{tcolorbox}[
    enhanced,
    boxrule=0pt,
    frame hidden,
    sharp corners,
    colback=noteBackground,
    borderline west={3pt}{-1.5pt}{noteAccent}
    ]
    \ifx &#1& \color{noteAccent}\textbf{Note. }\color{black}
    \else \color{noteAccent}\textbf{Note (#1). }\color{black} \fi
    }
    {
  \end{tcolorbox}
  }

\definecolor{definitionBackground}{RGB}{246,246,246}
\newenvironment{definition}[1]
  {
    \begin{tcolorbox}[
      enhanced,
      boxrule=0pt,
      frame hidden,
      sharp corners,
      colback=definitionBackground,
      borderline west={3pt}{-1.5pt}{black}
    ]
    \textbf{Definition. }\emph{#1}\\
  }
  {
    \end{tcolorbox}
  }
\newenvironment{amatrix}[2]{
    \left[
      \begin{array}{*{#1}{c}|*{#2}c}
  }
  {
      \end{array}
    \right]
  }
\date{\the\year-\the\month-\the\day}
\author{Kyle Chui}


\fancyhf{}
\lhead{Kyle Chui}
\rhead{Page \thepage}
\pagestyle{fancy}

\begin{document}
  \section{Early to Mid-Meiji}
  \subsection{Early-Mid Meiji (1868--1895) Key Reforms}
  \begin{itemize}
    \item Abolition of the four-class social system (eventually replaced by nobles and commoners)
    \item Abolition of domains and establishment of prefectures
    \item First national newspapers
    \item National single currency act
    \item First railways
    \item Western dress made mandatory at official functions
    \item National elementary education
    \item National conscripted military
    \item Creation of national land tax systen
    \item Ban on Christianity lifted
    \item First political associations formed
    \item National army puts down Satsuma REbellion
    \item Ryukyu annexed
    \item First politial Parties
    \item European-style cabinet created
    \item Promulgation on of constitution
    \item Imperial rescript on Education
  \end{itemize}
  \subsection{The Three ``Fathers''}
  \subsubsection{Okuma Shigenobu}
  \begin{itemize}
    \item Monetary reform (the tax guy), minted single currency
    \item National Land Tax Reform (Foreign minister, minister of finance)
    \item Founded several of the earliest politcal parties (and founded a university)
  \end{itemize}
  \subsubsection{Yamagata Aritomo}
  \begin{itemize}
    \item Went to Europe to study military systems
    \item Military conscription
    \item War minister, modeled the Japanese military on Britain and other European countries
  \end{itemize}
  \subsubsection{Ito Hirobumi}
  \begin{itemize}
    \item Sent by Choshu domain to England in 1863
    \item Was a member of the Iwakura Mission
    \item Drafted the Meiji Constitution (modeled on Prussian constitution)
    \item Became the first Prime Minister of Japan (held the position four more times)
  \end{itemize}
  \begin{note}{}
    These three ``fathers'' of early-modern Japanese society drew on heavy influences from the West.
  \end{note}
  \subsection{Modern Economy}
  \begin{itemize}
    \item Railway lines
    \item Specialised banks offering long-term credit
    \item Industrialisation
    \item Mining
    \item Zaibatsu and state monopolies
  \end{itemize}
  \begin{note}{}
    This exacerbates the quality of life differences between the poor and the rich.
  \end{note}
  \subsection{National Newspapers}
  \begin{itemize}
    \item First paper established in 1871
    \item Dissemination and control of information
    \item Press laws issued in $1875$ and $1877$ (censorship, police supervision)
  \end{itemize}
  \subsection{Fukuzara Yukichi (1835--1901)}
  He is the main person behind the Meiji restoration, and came from humble beginnings (not from an aristocratic family).
  \begin{itemize}
    \item Influenced by British liberalism
    \item Believed in Social Darwinism
    \item Believed in the three stages of civilisation:
    \begin{itemize}
      \item Primitive: Africa and Australia
      \item Semi-developed: Turkey, China, Japan
      \item Civilised: Europe and the United States
    \end{itemize}
    \item Firm believer that education can change anybody and lift anyone up
  \end{itemize}
  \subsection{Modern Education System}
  \begin{itemize}
    \item $1873$ The ministry of education is established
    \item Students are sent to US and Europe
    \item Foreign instructors brought to Japan
    \item Naional school system
    \item Tokyo Imperial University established
    \item Imperial Rescript on Education
  \end{itemize}
  \begin{note}{}
    The goals of this were to provide all citizens with basic life skills, instill a sense of loyalty to the country, and to shape their moral character.
  \end{note}
  Fukuzawa Yukichi realises that Japan has a government but the people are divided---if there were to be a war with another nation, only some would fight on behalf of Japan and the rest would just watch.
  \subsection{The First Political Parties}
  \begin{itemize}
    \item Jiyut\=o (Liberal Party), started by Itagaki Taisuke
    \item Kaishint\=o (Progressive Party), started by \=Okuma Shigenobu
    \item Teiseit\=o (Imperial Rule Party), started by Fukuchi Gen'ichir\=o
  \end{itemize}
  \subsection{Meiji Constitution (1889)}
  \begin{itemize}
    \item The Emperor is the head of the Empire
    \item ``The conditions necessary for being a Japanese subject shall be determined by law''
    \item The legislature is set up with a bicameral system---House of Peers and House of Representatives (which is elected by property owners, $1.1$\% of people)
  \end{itemize}
  At a governmental level, there were oligarchs (Genr\=o) between the legislature and the Emperor.
  \subsection{Sino-Japanese War (1894--1895)}
  This is the first major test for Japan.
  \begin{itemize}
    \item Fought over control of Korea (with China)
    \item Treaty of Shimonoseki
    \begin{itemize}
      \item They cede claim over Korea, territory $f$ Taiwan, Pescador islands, and part of Liaodong peninsula
      \item Payment of indemnity and opening of seven ports to Japan
      \item Qing dynasty severely weakened
    \end{itemize}
  \end{itemize}
\end{document}
