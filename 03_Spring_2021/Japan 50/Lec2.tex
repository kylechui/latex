\documentclass[class=article, crop=false]{standalone}
% Import packages
\usepackage[margin=1in]{geometry}

\usepackage{amssymb, amsthm}
\usepackage{comment}
\usepackage{enumitem}
\usepackage{fancyhdr}
\usepackage{hyperref}
\usepackage{import}
\usepackage{mathrsfs, mathtools}
\usepackage{multicol}
\usepackage{pdfpages}
\usepackage{standalone}
\usepackage{transparent}
\usepackage{xcolor}

\usepackage{minted}
\usepackage[many]{tcolorbox}
\tcbuselibrary{minted}

% Declare math operators
\DeclareMathOperator{\lcm}{lcm}
\DeclareMathOperator{\proj}{proj}
\DeclareMathOperator{\vspan}{span}
\DeclareMathOperator{\im}{im}
\DeclareMathOperator{\Diff}{Diff}
\DeclareMathOperator{\Int}{Int}
\DeclareMathOperator{\fcn}{fcn}
\DeclareMathOperator{\id}{id}
\DeclareMathOperator{\rank}{rank}
\DeclareMathOperator{\tr}{tr}
\DeclareMathOperator{\dive}{div}
\DeclareMathOperator{\row}{row}
\DeclareMathOperator{\col}{col}
\DeclareMathOperator{\dom}{dom}
\DeclareMathOperator{\ran}{ran}
\DeclareMathOperator{\Dom}{Dom}
\DeclareMathOperator{\Ran}{Ran}
\DeclareMathOperator{\fl}{fl}
% Macros for letters/variables
\newcommand\ttilde{\kern -.15em\lower .7ex\hbox{\~{}}\kern .04em}
\newcommand{\N}{\ensuremath{\mathbb{N}}}
\newcommand{\Z}{\ensuremath{\mathbb{Z}}}
\newcommand{\Q}{\ensuremath{\mathbb{Q}}}
\newcommand{\R}{\ensuremath{\mathbb{R}}}
\newcommand{\C}{\ensuremath{\mathbb{C}}}
\newcommand{\F}{\ensuremath{\mathbb{F}}}
\newcommand{\M}{\ensuremath{\mathbb{M}}}
\renewcommand{\P}{\ensuremath{\mathbb{P}}}
\newcommand{\lam}{\ensuremath{\lambda}}
\newcommand{\nab}{\ensuremath{\nabla}}
\newcommand{\eps}{\ensuremath{\varepsilon}}
\newcommand{\es}{\ensuremath{\varnothing}}
% Macros for math symbols
\newcommand{\dx}[1]{\,\mathrm{d}#1}
\newcommand{\inv}{\ensuremath{^{-1}}}
\newcommand{\sm}{\setminus}
\newcommand{\sse}{\subseteq}
\newcommand{\ceq}{\coloneqq}
% Macros for pairs of math symbols
\newcommand{\abs}[1]{\ensuremath{\left\lvert #1 \right\rvert}}
\newcommand{\paren}[1]{\ensuremath{\left( #1 \right)}}
\newcommand{\norm}[1]{\ensuremath{\left\lVert #1\right\rVert}}
\newcommand{\set}[1]{\ensuremath{\left\{#1\right\}}}
\newcommand{\tup}[1]{\ensuremath{\left\langle #1 \right\rangle}}
\newcommand{\floor}[1]{\ensuremath{\left\lfloor #1 \right\rfloor}}
\newcommand{\ceil}[1]{\ensuremath{\left\lceil #1 \right\rceil}}
\newcommand{\eclass}[1]{\ensuremath{\left[ #1 \right]}}

\newcommand{\chapternum}{}
\newcommand{\ex}[1]{\noindent\textbf{Exercise \chapternum.{#1}.}}

\newcommand{\tsub}[1]{\textsubscript{#1}}
\newcommand{\tsup}[1]{\textsuperscript{#1}}

\renewcommand{\choose}[2]{\ensuremath{\phantom{}_{#1}C_{#2}}}
\newcommand{\perm}[2]{\ensuremath{\phantom{}_{#1}P_{#2}}}

% Include figures
\newcommand{\incfig}[2][1]{%
    \def\svgwidth{#1\columnwidth}
    \import{./figures/}{#2.pdf_tex}
}

\definecolor{problemBackground}{RGB}{212,232,246}
\newenvironment{problem}[1]
  {
    \begin{tcolorbox}[
      boxrule=.5pt,
      titlerule=.5pt,
      sharp corners,
      colback=problemBackground,
      breakable
    ]
    \ifx &#1& \textbf{Problem. }
    \else \textbf{Problem #1.} \fi
  }
  {
    \end{tcolorbox}
  }
\definecolor{exampleBackground}{RGB}{255,249,248}
\definecolor{exampleAccent}{RGB}{158,60,14}
\newenvironment{example}[1]
  {
    \begin{tcolorbox}[
      boxrule=.5pt,
      sharp corners,
      colback=exampleBackground,
      colframe=exampleAccent,
    ]
    \color{exampleAccent}\textbf{Example.} \emph{#1}\color{black}
  }
  {
    \end{tcolorbox}
  }
\definecolor{theoremBackground}{RGB}{234,243,251}
\definecolor{theoremAccent}{RGB}{0,116,183}
\newenvironment{theorem}[1]
  {
    \begin{tcolorbox}[
      boxrule=.5pt,
      titlerule=.5pt,
      sharp corners,
      colback=theoremBackground,
      colframe=theoremAccent,
      breakable
    ]
      % \color{theoremAccent}\textbf{Theorem --- }\emph{#1}\\\color{black}
    \ifx &#1& \color{theoremAccent}\textbf{Theorem. }\color{black}
    \else \color{theoremAccent}\textbf{Theorem --- }\emph{#1}\\\color{black} \fi
  }
  {
    \end{tcolorbox}
  }
\definecolor{noteBackground}{RGB}{244,249,244}
\definecolor{noteAccent}{RGB}{34,139,34}
\newenvironment{note}[1]
  {
  \begin{tcolorbox}[
    enhanced,
    boxrule=0pt,
    frame hidden,
    sharp corners,
    colback=noteBackground,
    borderline west={3pt}{-1.5pt}{noteAccent},
    breakable
    ]
    \ifx &#1& \color{noteAccent}\textbf{Note. }\color{black}
    \else \color{noteAccent}\textbf{Note (#1). }\color{black} \fi
    }
    {
  \end{tcolorbox}
  }
\definecolor{lemmaBackground}{RGB}{255,247,234}
\definecolor{lemmaAccent}{RGB}{255,153,0}
\newenvironment{lemma}[1]
  {
    \begin{tcolorbox}[
      enhanced,
      boxrule=0pt,
      frame hidden,
      sharp corners,
      colback=lemmaBackground,
      borderline west={3pt}{-1.5pt}{lemmaAccent},
      breakable
    ]
    \ifx &#1& \color{lemmaAccent}\textbf{Lemma. }\color{black}
    \else \color{lemmaAccent}\textbf{Lemma #1. }\color{black} \fi
  }
  {
    \end{tcolorbox}
  }
\definecolor{definitionBackground}{RGB}{246,246,246}
\newenvironment{definition}[1]
  {
    \begin{tcolorbox}[
      enhanced,
      boxrule=0pt,
      frame hidden,
      sharp corners,
      colback=definitionBackground,
      borderline west={3pt}{-1.5pt}{black},
      breakable
    ]
    \textbf{Definition. }\emph{#1}\\
  }
  {
    \end{tcolorbox}
  }

\newenvironment{amatrix}[2]{
    \left[
      \begin{array}{*{#1}{c}|*{#2}c}
  }
  {
      \end{array}
    \right]
  }

\definecolor{codeBackground}{RGB}{253,246,227}
\renewcommand\theFancyVerbLine{\arabic{FancyVerbLine}}
\newtcblisting{cpp}[1][]{
  boxrule=1pt,
  sharp corners,
  colback=codeBackground,
  listing only,
  breakable,
  minted language=cpp,
  minted style=material,
  minted options={
    xleftmargin=15pt,
    baselinestretch=1.2,
    linenos,
  }
}

\author{Kyle Chui}


\fancyhf{}
\lhead{Kyle Chui}
\rhead{Page \thepage}
\pagestyle{fancy}

\begin{document}
  Many gifts are given/obtained by the Japanese during this time period, and the gifts originate from far away places. They are gradually traded across continents to get to Japan. Receiving gifts indicated that you were a person with status. \par
  In exchange for military help, the smaller (and more Southern) Korean civilisation of Baekje trades with ``Wa'' (Japan) in order to fend off the large Korean civilisation, Goguryeo. \par
  Early inscriptions of characters are found on swords (even though most of Japan is still illiterate), which indicate Japan's prevalence towards conquest. The scribes who could write often put their own names down onto the swords (the kings didn't know because they couldn't read). \par
  Later on, China splits into two kingdoms and Korea splits into three different kingdoms, so the political situation in Japan gets more and more complicated.
  \subsection{Late Tomb Period (Sixth Century)}
  \begin{itemize}
    \item Frequent contact and conflict with Korean kingdoms.
    \item The emergence of hereditary royal lineages.
    \item Different clans emerge, i.e. \=Otomo, Mononobe, Soga.
    \item Power is shared among various factions.
    \item Gradual monopoly of court by the Soga (possible of Korean origin).
    \item Adoption of Buddhism.
    \item Limited, highly specialised use of writing.
  \end{itemize}
  \subsection{Asuka Period (592-710)}
  \subsubsection{Asuka Facts}
  \begin{itemize}
    \item Asuka is the site of the Yamato court during most of the 7\tsup{th} century.
    \item Main Events:
    \begin{itemize}
      \item 607: Early hints of imperial vision (letter to Sui emperor in China).
      \item 645: Decline of the Soga lineage/assertion of royal power (Taika reforms).
      \item 663: Defeat of Paekchon River.
      \item 672: Jinshin War
    \end{itemize}
    \item First state of imperial-style state.
    \item State bureaucracy becomes fully literate.
    \item Strong influence from the Korean peninsula.
  \end{itemize}
  \subsubsection{Suiko and Prince Sh\=otoku}
  \begin{itemize}
    \item Suiko (r. 592-628) was the first of six female rulers in the 7\tsup{th} to 8\tsup{th} centuries
    \item Prince Sh\=otoku (574-622)
    \begin{itemize}
      \item Known in life as ``Prince of the Upper Palace''.
      \item Assisted Suiko in government.
      \item Hagiographic portrayal as Buddhist, Confucian, and Daoist sage. 
    \end{itemize}
  \end{itemize}
  \subsubsection{Letter to the Sui Emperor (607)}
  The emperor of Japan (Suiko) sends a letter to the emperor of China, who dismisses this and says ``don't ever make me listen to barbarian letters that contain this kind of insolence again''.
  \begin{note}{}
    The imperial vision at this point is very imaginary---the government embellishes things in order to imbue the surroundings with a sense of political authority. The hope is to turn this imaginary vision into some sort of reality.
  \end{note}
  \subsubsection{Asuka Buddhism}
  \begin{itemize}
    \item Buddhism first arrives in Japan in the mid-sixth century.
    \item Temples built around the Asuka capital throughout the seventh century (models of Mount Sumeru).
    \item Early Japanese Buddhism is an eclectic cult that focuses on health and longevity.
  \end{itemize}
  Religion offers the majestifying of political authority via large sculptures/architecture.
  \subsubsection{Decline of the Soga (645)}
  \begin{itemize}
    \item The Soga were the most powerful court lineage.
    \begin{itemize}
      \item Sponsored Buddhism (which came from Korean peninsula).
      \item Married daughters to rulers.
    \end{itemize}
    \item Main branch of the Soga lineage was defeated by a young prince called Naka no Oe (later Emperor Tenchi) and by leader of the Nakatomi linage (who received the name ``Fujiwara'').
    \item Marks the beginning of reforms to centralise power in the ruler.
  \end{itemize}
  \subsubsection{Defeat of Paekchon River (663)}
  \begin{itemize}
    \item Conflict over the Korean peninsula between the Silla and Tang court, and the Paekche, Koguryo, and Yamato.
    \item Paekche and Koguryo are destroyed (they become refugees to Yamato).
    \item Silla unifies the peninsula and turns against Tang.
    \item Silla influence on Yamato.
  \end{itemize}
  \subsubsection{Jinshin War (Civil War)}
  \begin{itemize}
    \item A war between the younger brother and son of previous ruler.
    \item The younger brother wins and becomes the first ruler to adopt the title ``Heavenly Sovereign''.
    \item Emperor Tenmu (r. 672-686).
    \item Development of literate state.
    \item Building of first imperial-style capital, which is then viewed as the center of the world.
  \end{itemize}
  \subsection{What is ``Civilisation''?}
  From the Latin \emph{civilis}, meaning ``relating to a citizen'' (which means ``inhabitant of a city''), ``courteous''. \\
  ``Civilisation''---the condition of being civilised. \\
  ``Civilised''---that which is not barbaric. \\
  ``Civilise''---to educate those in a barbaric or uncivilised state.
  \subsubsection{The Idea of an Imperial State}
  There is a supreme ruler, a universal realm (unified by communications, calendar, currency, taxes), a central capital (where all roads lead), and a universal legal code (governed by the sovereign). \\
  Colours represent the rank of someone, where higher ranked people tended to don various shades of purple.

\end{document}
