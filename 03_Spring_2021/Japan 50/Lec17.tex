\documentclass[class=article, crop=false]{standalone}
\usepackage[margin=1in]{geometry}

\usepackage{amsmath}
\usepackage{amssymb}
\usepackage{amsthm}
\usepackage{comment}
\usepackage{enumitem}
\usepackage{fancyhdr}
\usepackage{hyperref}
\usepackage{mathrsfs}
\usepackage{mathtools}
\usepackage{standalone}
\usepackage{tcolorbox}
\usepackage{tikz}
\usepackage{xcolor}

\usetikzlibrary{decorations.pathreplacing}
\tcbuselibrary{skins}

\DeclareMathOperator{\lcm}{lcm}
\DeclareMathOperator{\proj}{proj}
\DeclareMathOperator{\vspan}{span}
\DeclareMathOperator{\im}{im}
\DeclareMathOperator{\range}{range}
\DeclareMathOperator{\Diff}{Diff}
\DeclareMathOperator{\Int}{Int}
\DeclareMathOperator{\fcn}{fcn}
\DeclareMathOperator{\id}{id}

\newcommand{\N}{\ensuremath{\mathbb{N}}}
\newcommand{\Z}{\ensuremath{\mathbb{Z}}}
\newcommand{\Q}{\ensuremath{\mathbb{Q}}}
\newcommand{\R}{\ensuremath{\mathbb{R}}}
\newcommand{\C}{\ensuremath{\mathbb{C}}}
\newcommand{\F}{\ensuremath{\mathbb{F}}}
\newcommand{\lam}{\ensuremath{\lambda}}
\newcommand{\nab}{\ensuremath{\nabla}}
\newcommand{\eps}{\ensuremath{\varepsilon}}
\newcommand{\es}{\ensuremath{\varnothing}}

\newcommand{\abs}[1]{\ensuremath{\left\lvert #1 \right\rvert}}
\newcommand{\bigpar}[1]{\ensuremath{\left( #1 \right)}}
\newcommand{\norm}[1]{\ensuremath{\lVert #1\rVert}}
\newcommand{\set}[1]{\ensuremath{\left\{#1\right\}}}
\newcommand{\tuple}[1]{\ensuremath{\left\langle #1 \right\rangle}}
\newcommand{\floor}[1]{\ensuremath{\left\lfloor #1 \right\rfloor}}
\newcommand{\ceil}[1]{\ensuremath{\left\lceil #1 \right\rceil}}

\newcommand{\chapternum}{}
\newcommand{\ex}[1]{\noindent\textbf{Exercise \chapternum.{#1}.}}

\newcommand{\dx}[1]{\,\mathrm{d}#1}
\newcommand{\inv}{\ensuremath{^{-1}}}
\newcommand{\sm}{\setminus}
\newcommand{\sse}{\subseteq}
\newcommand{\ceq}{\coloneqq}

\definecolor{problemBackground}{RGB}{212,232,246}
\newenvironment{problem}[1]
  {
    \begin{tcolorbox}[
      boxrule=.5pt,
      titlerule=.5pt,
      sharp corners,
      colback=problemBackground
    ]
    \ifx &#1& \textbf{Problem. }
    \else \textbf{Problem #1.} \fi
  }
  {
    \end{tcolorbox}
  }

\definecolor{exampleBackground}{RGB}{255,249,248}
\definecolor{exampleAccent}{RGB}{158,60,14}
\newenvironment{example}[1]
  {
    \begin{tcolorbox}[
      boxrule=.5pt,
      sharp corners,
      colback=exampleBackground,
      colframe=exampleAccent,
    ]
    \color{exampleAccent}\textbf{Example.} \emph{#1}\color{black}
  }
  {
    \end{tcolorbox}
  }

\definecolor{theoremBackground}{RGB}{234,243,251}
\definecolor{theoremAccent}{RGB}{0,116,183}
\newenvironment{theorem}[1]
  {
    \begin{tcolorbox}[
      boxrule=.5pt,
      titlerule=.5pt,
      sharp corners,
      colback=theoremBackground,
      colframe=theoremAccent
    ]
      \color{theoremAccent}\textbf{Theorem --- }\emph{#1}\\\color{black}
  }
  {
    \end{tcolorbox}
  }

\definecolor{noteBackground}{RGB}{244,249,244}
\definecolor{noteAccent}{RGB}{34,139,34}
\newenvironment{note}[1]
  {
  \begin{tcolorbox}[
    enhanced,
    boxrule=0pt,
    frame hidden,
    sharp corners,
    colback=noteBackground,
    borderline west={3pt}{-1.5pt}{noteAccent}
    ]
    \ifx &#1& \color{noteAccent}\textbf{Note. }\color{black}
    \else \color{noteAccent}\textbf{Note (#1). }\color{black} \fi
    }
    {
  \end{tcolorbox}
  }

\definecolor{definitionBackground}{RGB}{246,246,246}
\newenvironment{definition}[1]
  {
    \begin{tcolorbox}[
      enhanced,
      boxrule=0pt,
      frame hidden,
      sharp corners,
      colback=definitionBackground,
      borderline west={3pt}{-1.5pt}{black}
    ]
    \textbf{Definition. }\emph{#1}\\
  }
  {
    \end{tcolorbox}
  }
\newenvironment{amatrix}[2]{
    \left[
      \begin{array}{*{#1}{c}|*{#2}c}
  }
  {
      \end{array}
    \right]
  }
\date{\the\year-\the\month-\the\day}
\author{Kyle Chui}


\fancyhf{}
\lhead{Kyle Chui}
\rhead{Page \thepage}
\pagestyle{fancy}

\begin{document}
  \section{Heisei Japan (with a bit of Reiwa)}
  Japanese literature is beginning to be exported internationally, along with other kinds of media.
  \subsection{The Heisei Depression}
  \begin{itemize}
    \item $1991$ Beginning of the ``Heisei Depression''
    \item $1997$ Sany\=o Securities and Hokkaid\=o Takushoku bank declare bankruptcy
    \item Land and real estate begins to be overinflated
  \end{itemize}
  \subsubsection{Post-bubble Japanese Economy and Society}
  \begin{itemize}
    \item Decline of the lifetime employment system
    \begin{note}{}
      Women need to work now because there is no longer work stability (demotions and firings).
    \end{note}
    \item Transition from ``salary men'' and ``office ladies'' to ``freeters'' and ``parasite singles''
    \begin{note}{}
      ``Freeters'' refers to ``free time''---describes those who don't really care about their jobs.
    \end{note}
    \item Aging population (due to high longevity, not many children)
    \begin{itemize}
      \item Labour shortage (not specific to Japan)
      \item Need for more immigration (shift from old Japan with stricter guidelines on immigration)
    \end{itemize}
    \begin{note}{}
      Over time, the number of young people in Japan is decreasing faster than the number of old people. Furthermore, the number of foreign workers in Japan has been increasing over time.
    \end{note}
  \end{itemize}
  \subsubsection{Contemporary Culture and National Identity}
  \begin{itemize}
    \item ``Cool Japan''---International export of media (anime, writing, etc.)
    \begin{note}{}
      This idea of a ``cool Japan'' exists mostly outside of Japan.
    \end{note}
    \item Questioning of modern Japanese social, economic, and cultural situation
    \begin{itemize}
      \item What does it mean to be Japanese? Who gets to belong to ``Japan''? Who gets to be fully included?
      \begin{note}{}
        Different social/civil groups routinely face challenges in Japan, i.e. LGBT rights, women, etc.
      \end{note}
    \end{itemize}
  \end{itemize}
  \subsubsection{War Memory and the High School Textbook Controversy}
  \begin{itemize}
    \item Ienaga Sabur\=o wrote a textbook
    \begin{itemize}
      \item ``Greater East Asia War'' (POV of wartime Japanese government)
      \item ``Pacific War'' (POV of Americans)
      \item Fifteen year war, Asia--Pacific war (more neutral)
    \end{itemize}
    \begin{note}{}
      There have been some dedicated to educating the public about the atrocities that the Japanese government has committed, but they have been stifled by other groups.
    \end{note}
  \end{itemize}
  \begin{note}{}
    There has been textbook reforms that paint Japan in better light ($2005$), referring to the Nanjing Massacre as the ``Nanjing Incident''.
  \end{note}
  \subsection{Heisei Japan}
  \subsubsection{Tohoku Earthquake and Tsunami (2011)}
  \begin{itemize}
    \item Giant earthquake caused a tsunami, and was followed by the Fukushima nuclear meltdown
    \item Followed by aftershocks for over a year
    \item Major effect on national consciousness, public discourse, art, literature, views of nuclear power, and visions for the future
    \begin{note}{}
      Japan originally set high goals for a decade from the incident, but then COVID-19 hit.
    \end{note}
  \end{itemize}
\end{document}
