\documentclass[class=article, crop=false]{standalone}
% Import packages
\usepackage[margin=1in]{geometry}

\usepackage{amssymb, amsthm}
\usepackage{comment}
\usepackage{enumitem}
\usepackage{fancyhdr}
\usepackage{hyperref}
\usepackage{import}
\usepackage{mathrsfs, mathtools}
\usepackage{multicol}
\usepackage{pdfpages}
\usepackage{standalone}
\usepackage{transparent}
\usepackage{xcolor}

\usepackage{minted}
\usepackage[many]{tcolorbox}
\tcbuselibrary{minted}

% Declare math operators
\DeclareMathOperator{\lcm}{lcm}
\DeclareMathOperator{\proj}{proj}
\DeclareMathOperator{\vspan}{span}
\DeclareMathOperator{\im}{im}
\DeclareMathOperator{\Diff}{Diff}
\DeclareMathOperator{\Int}{Int}
\DeclareMathOperator{\fcn}{fcn}
\DeclareMathOperator{\id}{id}
\DeclareMathOperator{\rank}{rank}
\DeclareMathOperator{\tr}{tr}
\DeclareMathOperator{\dive}{div}
\DeclareMathOperator{\row}{row}
\DeclareMathOperator{\col}{col}
\DeclareMathOperator{\dom}{dom}
\DeclareMathOperator{\ran}{ran}
\DeclareMathOperator{\Dom}{Dom}
\DeclareMathOperator{\Ran}{Ran}
\DeclareMathOperator{\fl}{fl}
% Macros for letters/variables
\newcommand\ttilde{\kern -.15em\lower .7ex\hbox{\~{}}\kern .04em}
\newcommand{\N}{\ensuremath{\mathbb{N}}}
\newcommand{\Z}{\ensuremath{\mathbb{Z}}}
\newcommand{\Q}{\ensuremath{\mathbb{Q}}}
\newcommand{\R}{\ensuremath{\mathbb{R}}}
\newcommand{\C}{\ensuremath{\mathbb{C}}}
\newcommand{\F}{\ensuremath{\mathbb{F}}}
\newcommand{\M}{\ensuremath{\mathbb{M}}}
\renewcommand{\P}{\ensuremath{\mathbb{P}}}
\newcommand{\lam}{\ensuremath{\lambda}}
\newcommand{\nab}{\ensuremath{\nabla}}
\newcommand{\eps}{\ensuremath{\varepsilon}}
\newcommand{\es}{\ensuremath{\varnothing}}
% Macros for math symbols
\newcommand{\dx}[1]{\,\mathrm{d}#1}
\newcommand{\inv}{\ensuremath{^{-1}}}
\newcommand{\sm}{\setminus}
\newcommand{\sse}{\subseteq}
\newcommand{\ceq}{\coloneqq}
% Macros for pairs of math symbols
\newcommand{\abs}[1]{\ensuremath{\left\lvert #1 \right\rvert}}
\newcommand{\paren}[1]{\ensuremath{\left( #1 \right)}}
\newcommand{\norm}[1]{\ensuremath{\left\lVert #1\right\rVert}}
\newcommand{\set}[1]{\ensuremath{\left\{#1\right\}}}
\newcommand{\tup}[1]{\ensuremath{\left\langle #1 \right\rangle}}
\newcommand{\floor}[1]{\ensuremath{\left\lfloor #1 \right\rfloor}}
\newcommand{\ceil}[1]{\ensuremath{\left\lceil #1 \right\rceil}}
\newcommand{\eclass}[1]{\ensuremath{\left[ #1 \right]}}

\newcommand{\chapternum}{}
\newcommand{\ex}[1]{\noindent\textbf{Exercise \chapternum.{#1}.}}

\newcommand{\tsub}[1]{\textsubscript{#1}}
\newcommand{\tsup}[1]{\textsuperscript{#1}}

\renewcommand{\choose}[2]{\ensuremath{\phantom{}_{#1}C_{#2}}}
\newcommand{\perm}[2]{\ensuremath{\phantom{}_{#1}P_{#2}}}

% Include figures
\newcommand{\incfig}[2][1]{%
    \def\svgwidth{#1\columnwidth}
    \import{./figures/}{#2.pdf_tex}
}

\definecolor{problemBackground}{RGB}{212,232,246}
\newenvironment{problem}[1]
  {
    \begin{tcolorbox}[
      boxrule=.5pt,
      titlerule=.5pt,
      sharp corners,
      colback=problemBackground,
      breakable
    ]
    \ifx &#1& \textbf{Problem. }
    \else \textbf{Problem #1.} \fi
  }
  {
    \end{tcolorbox}
  }
\definecolor{exampleBackground}{RGB}{255,249,248}
\definecolor{exampleAccent}{RGB}{158,60,14}
\newenvironment{example}[1]
  {
    \begin{tcolorbox}[
      boxrule=.5pt,
      sharp corners,
      colback=exampleBackground,
      colframe=exampleAccent,
    ]
    \color{exampleAccent}\textbf{Example.} \emph{#1}\color{black}
  }
  {
    \end{tcolorbox}
  }
\definecolor{theoremBackground}{RGB}{234,243,251}
\definecolor{theoremAccent}{RGB}{0,116,183}
\newenvironment{theorem}[1]
  {
    \begin{tcolorbox}[
      boxrule=.5pt,
      titlerule=.5pt,
      sharp corners,
      colback=theoremBackground,
      colframe=theoremAccent,
      breakable
    ]
      % \color{theoremAccent}\textbf{Theorem --- }\emph{#1}\\\color{black}
    \ifx &#1& \color{theoremAccent}\textbf{Theorem. }\color{black}
    \else \color{theoremAccent}\textbf{Theorem --- }\emph{#1}\\\color{black} \fi
  }
  {
    \end{tcolorbox}
  }
\definecolor{noteBackground}{RGB}{244,249,244}
\definecolor{noteAccent}{RGB}{34,139,34}
\newenvironment{note}[1]
  {
  \begin{tcolorbox}[
    enhanced,
    boxrule=0pt,
    frame hidden,
    sharp corners,
    colback=noteBackground,
    borderline west={3pt}{-1.5pt}{noteAccent},
    breakable
    ]
    \ifx &#1& \color{noteAccent}\textbf{Note. }\color{black}
    \else \color{noteAccent}\textbf{Note (#1). }\color{black} \fi
    }
    {
  \end{tcolorbox}
  }
\definecolor{lemmaBackground}{RGB}{255,247,234}
\definecolor{lemmaAccent}{RGB}{255,153,0}
\newenvironment{lemma}[1]
  {
    \begin{tcolorbox}[
      enhanced,
      boxrule=0pt,
      frame hidden,
      sharp corners,
      colback=lemmaBackground,
      borderline west={3pt}{-1.5pt}{lemmaAccent},
      breakable
    ]
    \ifx &#1& \color{lemmaAccent}\textbf{Lemma. }\color{black}
    \else \color{lemmaAccent}\textbf{Lemma #1. }\color{black} \fi
  }
  {
    \end{tcolorbox}
  }
\definecolor{definitionBackground}{RGB}{246,246,246}
\newenvironment{definition}[1]
  {
    \begin{tcolorbox}[
      enhanced,
      boxrule=0pt,
      frame hidden,
      sharp corners,
      colback=definitionBackground,
      borderline west={3pt}{-1.5pt}{black},
      breakable
    ]
    \textbf{Definition. }\emph{#1}\\
  }
  {
    \end{tcolorbox}
  }

\newenvironment{amatrix}[2]{
    \left[
      \begin{array}{*{#1}{c}|*{#2}c}
  }
  {
      \end{array}
    \right]
  }

\definecolor{codeBackground}{RGB}{253,246,227}
\renewcommand\theFancyVerbLine{\arabic{FancyVerbLine}}
\newtcblisting{cpp}[1][]{
  boxrule=1pt,
  sharp corners,
  colback=codeBackground,
  listing only,
  breakable,
  minted language=cpp,
  minted style=material,
  minted options={
    xleftmargin=15pt,
    baselinestretch=1.2,
    linenos,
  }
}

\author{Kyle Chui}


\fancyhf{}
\lhead{Kyle Chui}
\rhead{Page \thepage}
\pagestyle{fancy}

\begin{document}
  \section{Jesuits In 16th Century Japan}
  \subsection{Japanese Relations With the World}
  \begin{itemize}
    \item Early relations with the Korean Peninsula and China
    \item Spread of Buddhism across Asia and to Japan
    \item Portuguese missionaries in the 16\tsup{th} century
    \item Dutch Trade 16\tsup{th}--19\tsup{th} centuries (Japan highly regulates outside interaction)
    \item Arrival of US ships in the 19\tsup{th} century
    \item Modern era
  \end{itemize}
  \subsection{Early Japanese Relations with Asia}
  \begin{itemize}
    \item Tributary missions to different imperial dynasties in China
    \item Relations with Korean kingdoms
    \item Buddhism arrives in Japan from Korean peninsula
    \item Ideal of imperial rule is transmitted to Japan via Chinese writing and written texts
    \begin{note}{}
      Japanese rulers measure themselves according to this ideal of imperial rule.
    \end{note}
  \end{itemize}
  \subsection{Background to Arrival of Jesuits in Japan}
  \begin{itemize}
    \item $1368$ Establishment of Ming Dynasty
    \item $1392$ Establishment of Choson (Yi) Dynasty
    \item $1401$ Ashikaga Yoshimitsu establishes trade with Ming
    \item 1405--1433 Ming voyages to South-east Asia and India
    \begin{itemize}
      \item $1488$ Portuguese explorers reach the southern tip of Africa
      \item $1498$ Portuguese establish spice trade with India
      \begin{note}{}
        The Portuguese want to go to Africa (for gold) and India (for spices, silk).
      \end{note}
      \item $1513$ Portuguese establish trade with the Ming dynasty
      \item 1519--1522 Earliest known circumnavigation of the world (Magellan)
      \item $1521$ Edict of worms (Charles V declares Luther a heretic)
      \item $1540$ Foundation of Jesuit order
      \begin{note}{}
        Jesuits move East as a reaction to the Protestant Reformation.
      \end{note}
    \end{itemize}
    \item $1543$ Portuguese reach Tanegashima
    \item $1549$ Jesuit Francis Xavier arrives in Japan
    \item $1557$ Portuguese establish Macau
  \end{itemize}
  \subsection{Christian Century}
  \begin{itemize}
    \item $1549$ Jesuit Francis Xavier arrives in Japan
    \item $1557$ Macau given to Portuguese by Ming Emperor in gratitude for subduing pirates
    \item $1570$ Establishment of silk trade from Macau to Nagasaki
    \begin{note}{}
      The Portuguese become middlemen for China-Japan trade.
    \end{note}
    \item $1587$ Hideyoshi's edict on expelling Christian missionaries (not strictly enforced)
    \item $1592$ Franciscans arrive in Japan
    \item $1606$ Christianity declared illegal
    \item $1614$ Campaign to eradicate Christianity (more than $300000$ converts)
    \item $1624$ Spanish expelled from Japan
    \item $1627$ Shimabara rebellion (Portuguese expelled from Japan)
  \end{itemize}
  \begin{note}{}
    The Japanese perceived the Portuguese as barbarians from the South.
  \end{note}
  \subsection{Jesuit Objectives in Japan}
  \begin{itemize}
    \item Jesuit mission was part of Portuguese Empire's overseas activities.
    \item Objectives:
    \begin{itemize}
      \item Offering counsel and military technology (muskets, cannon)
      \item Establish trade
      \item Propagate Christian faith
    \end{itemize}
  \end{itemize}
  \subsection{Famous Jesuit Missionaries}
  \begin{itemize}
    \item Francisco Xavier (1506--1552) arrived in 1549
    \item Cosme de Torres (1510--1570)
    \item Luis Fr\' ois (1532--1597) arrived in 1563
    \item Alessandro Valignano (1539--1606) arrived in 1579
    \item Jo\~ao Rodrigues (1561--1633)
    \begin{note}{}
      A lot of these missionaries wrote many books documenting facts about Japan and its people.
    \end{note}
  \end{itemize}
  \subsubsection{Tensh\=o Embassy of 1582}
  Four young boys were brought to Europe, that had been educated in Portuguese, to show the royalty in Europe that Christianity had spread to all corners of the world. The goal was to bring them back and have these four boys tell the Japanese emperor about all they had seen, and convert all of Japan. However, Hideyoshi had died before they came back, so the second part of the plan failed.
  \subsection{Late 16th Century}
  \begin{itemize}
    \item $1587$ Hideyoshi limits propagation of Christianity by issuing an edict, expelling Christian missionaries. This also prohibits the sale of Japanese people as slaves
    \item $1588$ Hideyoshi edicts prohibiting arms and piracy
    \item $1590$ Hideyoshi responds to a letter from the King of Korea announcing the intent to invade
    \item $1592$ Invasion of Choson (second invasion in 1597) a.k.a. the Imjin war
    \item $1593$ Ming emperor sends large force to aid Choson, but the expense severely weakens the Ming
    \item $1597$ Second Choson invasion
    \item $1598$ Hideyoshi dies
    \item $1600$ Battle of Sekigahara
    \item $1603$ Establishment of Tokugawa shogunate
    \begin{note}{}
      Hideyoshi wanted to conquer---Korea, China, India, but was unsuccessful. These invasions of Korea are costly, and make it easier for Tokugawa to take over after Hideyoshi dies.
    \end{note}
  \end{itemize}
\end{document}
