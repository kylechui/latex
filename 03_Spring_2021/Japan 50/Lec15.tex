\documentclass[class=article, crop=false]{standalone}
\usepackage[margin=1in]{geometry}

\usepackage{amsmath}
\usepackage{amssymb}
\usepackage{amsthm}
\usepackage{comment}
\usepackage{enumitem}
\usepackage{fancyhdr}
\usepackage{hyperref}
\usepackage{mathrsfs}
\usepackage{mathtools}
\usepackage{standalone}
\usepackage{tcolorbox}
\usepackage{tikz}
\usepackage{xcolor}

\usetikzlibrary{decorations.pathreplacing}
\tcbuselibrary{skins}

\DeclareMathOperator{\lcm}{lcm}
\DeclareMathOperator{\proj}{proj}
\DeclareMathOperator{\vspan}{span}
\DeclareMathOperator{\im}{im}
\DeclareMathOperator{\range}{range}
\DeclareMathOperator{\Diff}{Diff}
\DeclareMathOperator{\Int}{Int}
\DeclareMathOperator{\fcn}{fcn}
\DeclareMathOperator{\id}{id}

\newcommand{\N}{\ensuremath{\mathbb{N}}}
\newcommand{\Z}{\ensuremath{\mathbb{Z}}}
\newcommand{\Q}{\ensuremath{\mathbb{Q}}}
\newcommand{\R}{\ensuremath{\mathbb{R}}}
\newcommand{\C}{\ensuremath{\mathbb{C}}}
\newcommand{\F}{\ensuremath{\mathbb{F}}}
\newcommand{\lam}{\ensuremath{\lambda}}
\newcommand{\nab}{\ensuremath{\nabla}}
\newcommand{\eps}{\ensuremath{\varepsilon}}
\newcommand{\es}{\ensuremath{\varnothing}}

\newcommand{\abs}[1]{\ensuremath{\left\lvert #1 \right\rvert}}
\newcommand{\bigpar}[1]{\ensuremath{\left( #1 \right)}}
\newcommand{\norm}[1]{\ensuremath{\lVert #1\rVert}}
\newcommand{\set}[1]{\ensuremath{\left\{#1\right\}}}
\newcommand{\tuple}[1]{\ensuremath{\left\langle #1 \right\rangle}}
\newcommand{\floor}[1]{\ensuremath{\left\lfloor #1 \right\rfloor}}
\newcommand{\ceil}[1]{\ensuremath{\left\lceil #1 \right\rceil}}

\newcommand{\chapternum}{}
\newcommand{\ex}[1]{\noindent\textbf{Exercise \chapternum.{#1}.}}

\newcommand{\dx}[1]{\,\mathrm{d}#1}
\newcommand{\inv}{\ensuremath{^{-1}}}
\newcommand{\sm}{\setminus}
\newcommand{\sse}{\subseteq}
\newcommand{\ceq}{\coloneqq}

\definecolor{problemBackground}{RGB}{212,232,246}
\newenvironment{problem}[1]
  {
    \begin{tcolorbox}[
      boxrule=.5pt,
      titlerule=.5pt,
      sharp corners,
      colback=problemBackground
    ]
    \ifx &#1& \textbf{Problem. }
    \else \textbf{Problem #1.} \fi
  }
  {
    \end{tcolorbox}
  }

\definecolor{exampleBackground}{RGB}{255,249,248}
\definecolor{exampleAccent}{RGB}{158,60,14}
\newenvironment{example}[1]
  {
    \begin{tcolorbox}[
      boxrule=.5pt,
      sharp corners,
      colback=exampleBackground,
      colframe=exampleAccent,
    ]
    \color{exampleAccent}\textbf{Example.} \emph{#1}\color{black}
  }
  {
    \end{tcolorbox}
  }

\definecolor{theoremBackground}{RGB}{234,243,251}
\definecolor{theoremAccent}{RGB}{0,116,183}
\newenvironment{theorem}[1]
  {
    \begin{tcolorbox}[
      boxrule=.5pt,
      titlerule=.5pt,
      sharp corners,
      colback=theoremBackground,
      colframe=theoremAccent
    ]
      \color{theoremAccent}\textbf{Theorem --- }\emph{#1}\\\color{black}
  }
  {
    \end{tcolorbox}
  }

\definecolor{noteBackground}{RGB}{244,249,244}
\definecolor{noteAccent}{RGB}{34,139,34}
\newenvironment{note}[1]
  {
  \begin{tcolorbox}[
    enhanced,
    boxrule=0pt,
    frame hidden,
    sharp corners,
    colback=noteBackground,
    borderline west={3pt}{-1.5pt}{noteAccent}
    ]
    \ifx &#1& \color{noteAccent}\textbf{Note. }\color{black}
    \else \color{noteAccent}\textbf{Note (#1). }\color{black} \fi
    }
    {
  \end{tcolorbox}
  }

\definecolor{definitionBackground}{RGB}{246,246,246}
\newenvironment{definition}[1]
  {
    \begin{tcolorbox}[
      enhanced,
      boxrule=0pt,
      frame hidden,
      sharp corners,
      colback=definitionBackground,
      borderline west={3pt}{-1.5pt}{black}
    ]
    \textbf{Definition. }\emph{#1}\\
  }
  {
    \end{tcolorbox}
  }
\newenvironment{amatrix}[2]{
    \left[
      \begin{array}{*{#1}{c}|*{#2}c}
  }
  {
      \end{array}
    \right]
  }
\date{\the\year-\the\month-\the\day}
\author{Kyle Chui}


\fancyhf{}
\lhead{Kyle Chui}
\rhead{Page \thepage}
\pagestyle{fancy}

\begin{document}
  \section{The Occupation}
  \subsection{The End of the War}
  \begin{itemize}
    \item Daily air raids by US bombers from Nov. $1944$ to Aug. 1945
    \item Bombing of Tokyo (Mar. 9--10, 1945)
    \begin{itemize}
      \item Over 100K dead, 1M homeless, $16$ square miles destroyed
    \end{itemize}
    \item Battle of Okinawa (March--July 1945)
    \begin{itemize}
      \item Over 250K dead (including 150K civilians)
    \end{itemize}
    \item Atomic bombs on August 6\tsup{th} and 9\tsup{th} $1945$ at Hiroshima and Nagasaki, respectively
    \item Japan surrenders on August $15$, 1945
    \item Around 40\% of Japan's urban area is destroyed, with nine million homeless
  \end{itemize}
  \subsection{The Occupation}
  \begin{note}{}
    The goal was to demilitarise Japan and keep Japan from its imperialistic tendencies. It was also supposed to make Japan more democratic.
  \end{note}
  \begin{itemize}
    \item Establishment of the UN in April of 1945
    \item Supreme Commander for the Allied Powers is General Douglas MacArthur
    \item Military Tribunals
    \item $1947$ Constitution
    \begin{itemize}
      \item Emperor symbol of the state
      \item Renunciation of War and armed forces
      \item Rights of people: universal suffrage
    \end{itemize}
    \item Japan--US Security Treaty (1951--1952)
  \end{itemize}
  \subsection{Immediate Aftermath of the War}
  \begin{itemize}
    \item 9M homeless, extreme poverty
    \item Average household spent 68\% of income on food in 1946
    \item Average height/weight of schoolchildren decreased until 1948
    \begin{note}{}
      Malnourishment was a major problem during this period.
    \end{note}
    \item Barter economy, black markets, paralysed rationing system
    \item Prostitution system for 300K occupying troops
  \end{itemize}
  \begin{note}{}
    Many ideas/groups in Japan ``as we know them'' arise during this time period.
  \end{note}
  \subsection{Postwar Government}
  \begin{itemize}
    \item Executive Branch
    \begin{itemize}
      \item Emperor, Prime minister (appointed by Emperor as directed by Diet), and Cabinet (Appointed by Prime minister)
    \end{itemize}
    \item Legislative Branch (Diet)
    \begin{itemize}
      \item House of Councilors (upper house)
      \item House of Representatives (lower house)
    \end{itemize}
    \item Judicial Branch
    \begin{itemize}
      \item Judges appointed by Emperor as directed by Diet
    \end{itemize}
  \end{itemize}
  New political parties are formed that are essentially descendants of the old political parties that were banned in 1942.
  \begin{note}{}
    Preamble of the $1947$ Constitution includes a clause saying that ``all nations deserve to be respected'', a stark contrast to the original Japanese ideals.
  \end{note}
  \subsection{Reforms During Occupation}
  \begin{itemize}
    \item Reintroduction of electoral politics/political parties
    \item Education system restructured to be more like US education, and militaristic/authoritarian content is purged
    \item Language reform (writing system simplified/standardised)
    \item Unsuccessful attempt to decentralise economy
    \begin{note}{}
      Different from the centralised war economy.
    \end{note}
    \item Initial support of labour movements, which are later repressed
  \end{itemize}
  During the Cold War, Japan becomes the US ``bulwark'' against communism in Asia. There is censorship and repression by occupation forces.
  \subsection{Postwar Economic Recovery}
  \begin{itemize}
    \item Korean War (1950--1953)
    \item Shipbuilding, iron and steel production, manufacturing (decline of agriculture)
    \item Annual growth of 10\% by $1955$ (to 1974)
    \item Auto mobile industry, electronics
    \item Life expectancy rises dramatically (from $47$ in $1935$ to $68$ in $1960$)
    \item Environmental problems
    \begin{itemize}
      \item Pollution
      \item Minamata disease
    \end{itemize}
  \end{itemize}
  \subsection{Postwar Society}
  \begin{itemize}
    \item Initial ``New Deal'' progressive reforms abandoned with Cold War
    \item Increased urbanisation
    \item Breaking down of traditional extended family
    \item Rights of women
    \item Cultural nationalism
    \begin{itemize}
      \item Japanese exceptionalism/uniqueness
    \end{itemize}
  \end{itemize}
\end{document}
