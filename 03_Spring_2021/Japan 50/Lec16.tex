\documentclass[class=article, crop=false]{standalone}
% Import packages
\usepackage[margin=1in]{geometry}

\usepackage{amssymb, amsthm}
\usepackage{comment}
\usepackage{enumitem}
\usepackage{fancyhdr}
\usepackage{hyperref}
\usepackage{import}
\usepackage{mathrsfs, mathtools}
\usepackage{multicol}
\usepackage{pdfpages}
\usepackage{standalone}
\usepackage{transparent}
\usepackage{xcolor}

\usepackage{minted}
\usepackage[many]{tcolorbox}
\tcbuselibrary{minted}

% Declare math operators
\DeclareMathOperator{\lcm}{lcm}
\DeclareMathOperator{\proj}{proj}
\DeclareMathOperator{\vspan}{span}
\DeclareMathOperator{\im}{im}
\DeclareMathOperator{\Diff}{Diff}
\DeclareMathOperator{\Int}{Int}
\DeclareMathOperator{\fcn}{fcn}
\DeclareMathOperator{\id}{id}
\DeclareMathOperator{\rank}{rank}
\DeclareMathOperator{\tr}{tr}
\DeclareMathOperator{\dive}{div}
\DeclareMathOperator{\row}{row}
\DeclareMathOperator{\col}{col}
\DeclareMathOperator{\dom}{dom}
\DeclareMathOperator{\ran}{ran}
\DeclareMathOperator{\Dom}{Dom}
\DeclareMathOperator{\Ran}{Ran}
\DeclareMathOperator{\fl}{fl}
% Macros for letters/variables
\newcommand\ttilde{\kern -.15em\lower .7ex\hbox{\~{}}\kern .04em}
\newcommand{\N}{\ensuremath{\mathbb{N}}}
\newcommand{\Z}{\ensuremath{\mathbb{Z}}}
\newcommand{\Q}{\ensuremath{\mathbb{Q}}}
\newcommand{\R}{\ensuremath{\mathbb{R}}}
\newcommand{\C}{\ensuremath{\mathbb{C}}}
\newcommand{\F}{\ensuremath{\mathbb{F}}}
\newcommand{\M}{\ensuremath{\mathbb{M}}}
\renewcommand{\P}{\ensuremath{\mathbb{P}}}
\newcommand{\lam}{\ensuremath{\lambda}}
\newcommand{\nab}{\ensuremath{\nabla}}
\newcommand{\eps}{\ensuremath{\varepsilon}}
\newcommand{\es}{\ensuremath{\varnothing}}
% Macros for math symbols
\newcommand{\dx}[1]{\,\mathrm{d}#1}
\newcommand{\inv}{\ensuremath{^{-1}}}
\newcommand{\sm}{\setminus}
\newcommand{\sse}{\subseteq}
\newcommand{\ceq}{\coloneqq}
% Macros for pairs of math symbols
\newcommand{\abs}[1]{\ensuremath{\left\lvert #1 \right\rvert}}
\newcommand{\paren}[1]{\ensuremath{\left( #1 \right)}}
\newcommand{\norm}[1]{\ensuremath{\left\lVert #1\right\rVert}}
\newcommand{\set}[1]{\ensuremath{\left\{#1\right\}}}
\newcommand{\tup}[1]{\ensuremath{\left\langle #1 \right\rangle}}
\newcommand{\floor}[1]{\ensuremath{\left\lfloor #1 \right\rfloor}}
\newcommand{\ceil}[1]{\ensuremath{\left\lceil #1 \right\rceil}}
\newcommand{\eclass}[1]{\ensuremath{\left[ #1 \right]}}

\newcommand{\chapternum}{}
\newcommand{\ex}[1]{\noindent\textbf{Exercise \chapternum.{#1}.}}

\newcommand{\tsub}[1]{\textsubscript{#1}}
\newcommand{\tsup}[1]{\textsuperscript{#1}}

\renewcommand{\choose}[2]{\ensuremath{\phantom{}_{#1}C_{#2}}}
\newcommand{\perm}[2]{\ensuremath{\phantom{}_{#1}P_{#2}}}

% Include figures
\newcommand{\incfig}[2][1]{%
    \def\svgwidth{#1\columnwidth}
    \import{./figures/}{#2.pdf_tex}
}

\definecolor{problemBackground}{RGB}{212,232,246}
\newenvironment{problem}[1]
  {
    \begin{tcolorbox}[
      boxrule=.5pt,
      titlerule=.5pt,
      sharp corners,
      colback=problemBackground,
      breakable
    ]
    \ifx &#1& \textbf{Problem. }
    \else \textbf{Problem #1.} \fi
  }
  {
    \end{tcolorbox}
  }
\definecolor{exampleBackground}{RGB}{255,249,248}
\definecolor{exampleAccent}{RGB}{158,60,14}
\newenvironment{example}[1]
  {
    \begin{tcolorbox}[
      boxrule=.5pt,
      sharp corners,
      colback=exampleBackground,
      colframe=exampleAccent,
    ]
    \color{exampleAccent}\textbf{Example.} \emph{#1}\color{black}
  }
  {
    \end{tcolorbox}
  }
\definecolor{theoremBackground}{RGB}{234,243,251}
\definecolor{theoremAccent}{RGB}{0,116,183}
\newenvironment{theorem}[1]
  {
    \begin{tcolorbox}[
      boxrule=.5pt,
      titlerule=.5pt,
      sharp corners,
      colback=theoremBackground,
      colframe=theoremAccent,
      breakable
    ]
      % \color{theoremAccent}\textbf{Theorem --- }\emph{#1}\\\color{black}
    \ifx &#1& \color{theoremAccent}\textbf{Theorem. }\color{black}
    \else \color{theoremAccent}\textbf{Theorem --- }\emph{#1}\\\color{black} \fi
  }
  {
    \end{tcolorbox}
  }
\definecolor{noteBackground}{RGB}{244,249,244}
\definecolor{noteAccent}{RGB}{34,139,34}
\newenvironment{note}[1]
  {
  \begin{tcolorbox}[
    enhanced,
    boxrule=0pt,
    frame hidden,
    sharp corners,
    colback=noteBackground,
    borderline west={3pt}{-1.5pt}{noteAccent},
    breakable
    ]
    \ifx &#1& \color{noteAccent}\textbf{Note. }\color{black}
    \else \color{noteAccent}\textbf{Note (#1). }\color{black} \fi
    }
    {
  \end{tcolorbox}
  }
\definecolor{lemmaBackground}{RGB}{255,247,234}
\definecolor{lemmaAccent}{RGB}{255,153,0}
\newenvironment{lemma}[1]
  {
    \begin{tcolorbox}[
      enhanced,
      boxrule=0pt,
      frame hidden,
      sharp corners,
      colback=lemmaBackground,
      borderline west={3pt}{-1.5pt}{lemmaAccent},
      breakable
    ]
    \ifx &#1& \color{lemmaAccent}\textbf{Lemma. }\color{black}
    \else \color{lemmaAccent}\textbf{Lemma #1. }\color{black} \fi
  }
  {
    \end{tcolorbox}
  }
\definecolor{definitionBackground}{RGB}{246,246,246}
\newenvironment{definition}[1]
  {
    \begin{tcolorbox}[
      enhanced,
      boxrule=0pt,
      frame hidden,
      sharp corners,
      colback=definitionBackground,
      borderline west={3pt}{-1.5pt}{black},
      breakable
    ]
    \textbf{Definition. }\emph{#1}\\
  }
  {
    \end{tcolorbox}
  }

\newenvironment{amatrix}[2]{
    \left[
      \begin{array}{*{#1}{c}|*{#2}c}
  }
  {
      \end{array}
    \right]
  }

\definecolor{codeBackground}{RGB}{253,246,227}
\renewcommand\theFancyVerbLine{\arabic{FancyVerbLine}}
\newtcblisting{cpp}[1][]{
  boxrule=1pt,
  sharp corners,
  colback=codeBackground,
  listing only,
  breakable,
  minted language=cpp,
  minted style=material,
  minted options={
    xleftmargin=15pt,
    baselinestretch=1.2,
    linenos,
  }
}

\author{Kyle Chui}


\fancyhf{}
\lhead{Kyle Chui}
\rhead{Page \thepage}
\pagestyle{fancy}

\begin{document}
  \section{Postwar Japan (1960--1989)}
  \subsection{1960 Protests against US--Japan Security Treaty}
  \begin{itemize}
    \item $1960$ Treaty of Mutual Cooperation and Security between Japan and US
    \begin{itemize}
      \item US forces stationed in Japan
      \item Ten year term
      \item Opposition to the treaty within the Diet
      \item Up to $16$ million people took to the streets in protest
    \end{itemize}
  \end{itemize}
  \subsection{Postwar Politics}
  \begin{itemize}
    \item Political Parties
    \begin{itemize}
      \item Liberal Democratic Party (LDP) (wins every election during this time period)
      \item Japanese Socialist Party
      \item Democratic Socialist Party
      \item Japanese Communist Party
      \item Clean Government Part
    \end{itemize}
    \item Ikeda Hayato (PM from 1960--1964)
    \begin{itemize}
      \item Repaired US--Japan relations
      \item Doubled the size of Japan's economy in $7$ years (promised 10)
      \item Set up universal health insurance and pension plan in 1961
      \item Presided over the Tokyo Olympics in 1964
    \end{itemize}
  \end{itemize}
  \subsection{Postwar Economy}
  \begin{itemize}
    \item Ministry of International Trade and Industry oversees foreign trade
    \item Prioritised industries: shipbuilding, steel, cars, electronics, robotics, biotechnology
    \item Enterprise groups of affiliated industries
    \item Steel production was 89\% of the US by 1974
    \item Japan's annual car production surpasses $10$M units in the 1980s
    \item Lie expectancy goes from $47$ in $1935$ to $68$ in $1960$ to $78$ in $1990$
    \begin{note}{}
      We see the rise of a nation-wide ``middle class'' across Japan, defined by making electronics/technology.
    \end{note}
  \end{itemize}
  \subsection{Postwar Social Transformation}
  \begin{itemize}
    \item Higher education boom (95\% beyond 9\tsup{th} grade by 1990)
    \item Growth of cities and suburbs (greater Tokyo area's population increases)
    \begin{note}{}
      The government builds a lot of infrastructure.
    \end{note}
    \item Decline of three-generation family
    \begin{note}{}
      Women transition to managing the home/children, because there are no longer grandparents to watch over the home.
    \end{note}
    \item Postwar company culture (lifetime employment)
    \item ``New middle class''
    \begin{itemize}
      \item ``Three treasures''---In the 1950s, they were the Monochrome TV, washing machine and refrigerator, whereas in the 1960s they were the Colour TV, the air conditioner and the car
    \end{itemize}
    \item Divisions
    \begin{itemize}
      \item 540K Koreans live in Japan as resident aliens
      \item Bra cumin screened out of job applications
      \item Growth of new religions
    \end{itemize}
  \end{itemize}
  \subsection{Postwar Culture}
  \begin{itemize}
    \item Television becomes increasingly popular (especially after the $1964$ Tokyo Olympics, the first Olympics in Asia)
    \item Growth of the publishing industry (also due to increase in education)
  \end{itemize}
  \begin{note}{}
    The $1964$ Olympics was the first Olympics to be telecast internationally.
  \end{note}
  \subsection{Japanese Mass Media Exports}
  \begin{itemize}
    \item Hello Kitty, Astro boy, Doraemon
  \end{itemize}
  \subsection{1980s Japan---The Age of Affluence}
  \begin{itemize}
    \item Japan's car production surpasses 10M units per year, second only to the US
    \item $1979$ Sony's first Walkman
    \item $1987$ Per capita income surpasses US
    \item Travel abroad takes off
    \item ``Japan-bashing'' in US an Europe
    \item Japan becomes trendy/cool
    \item Japan becomes a popular area of study in North America and Europe
  \end{itemize}
\end{document}
