\documentclass[class=article, crop=false]{standalone}
% Import packages
\usepackage[margin=1in]{geometry}

\usepackage{amssymb, amsthm}
\usepackage{comment}
\usepackage{enumitem}
\usepackage{fancyhdr}
\usepackage{hyperref}
\usepackage{import}
\usepackage{mathrsfs, mathtools}
\usepackage{multicol}
\usepackage{pdfpages}
\usepackage{standalone}
\usepackage{transparent}
\usepackage{xcolor}

\usepackage{minted}
\usepackage[many]{tcolorbox}
\tcbuselibrary{minted}

% Declare math operators
\DeclareMathOperator{\lcm}{lcm}
\DeclareMathOperator{\proj}{proj}
\DeclareMathOperator{\vspan}{span}
\DeclareMathOperator{\im}{im}
\DeclareMathOperator{\Diff}{Diff}
\DeclareMathOperator{\Int}{Int}
\DeclareMathOperator{\fcn}{fcn}
\DeclareMathOperator{\id}{id}
\DeclareMathOperator{\rank}{rank}
\DeclareMathOperator{\tr}{tr}
\DeclareMathOperator{\dive}{div}
\DeclareMathOperator{\row}{row}
\DeclareMathOperator{\col}{col}
\DeclareMathOperator{\dom}{dom}
\DeclareMathOperator{\ran}{ran}
\DeclareMathOperator{\Dom}{Dom}
\DeclareMathOperator{\Ran}{Ran}
\DeclareMathOperator{\fl}{fl}
% Macros for letters/variables
\newcommand\ttilde{\kern -.15em\lower .7ex\hbox{\~{}}\kern .04em}
\newcommand{\N}{\ensuremath{\mathbb{N}}}
\newcommand{\Z}{\ensuremath{\mathbb{Z}}}
\newcommand{\Q}{\ensuremath{\mathbb{Q}}}
\newcommand{\R}{\ensuremath{\mathbb{R}}}
\newcommand{\C}{\ensuremath{\mathbb{C}}}
\newcommand{\F}{\ensuremath{\mathbb{F}}}
\newcommand{\M}{\ensuremath{\mathbb{M}}}
\renewcommand{\P}{\ensuremath{\mathbb{P}}}
\newcommand{\lam}{\ensuremath{\lambda}}
\newcommand{\nab}{\ensuremath{\nabla}}
\newcommand{\eps}{\ensuremath{\varepsilon}}
\newcommand{\es}{\ensuremath{\varnothing}}
% Macros for math symbols
\newcommand{\dx}[1]{\,\mathrm{d}#1}
\newcommand{\inv}{\ensuremath{^{-1}}}
\newcommand{\sm}{\setminus}
\newcommand{\sse}{\subseteq}
\newcommand{\ceq}{\coloneqq}
% Macros for pairs of math symbols
\newcommand{\abs}[1]{\ensuremath{\left\lvert #1 \right\rvert}}
\newcommand{\paren}[1]{\ensuremath{\left( #1 \right)}}
\newcommand{\norm}[1]{\ensuremath{\left\lVert #1\right\rVert}}
\newcommand{\set}[1]{\ensuremath{\left\{#1\right\}}}
\newcommand{\tup}[1]{\ensuremath{\left\langle #1 \right\rangle}}
\newcommand{\floor}[1]{\ensuremath{\left\lfloor #1 \right\rfloor}}
\newcommand{\ceil}[1]{\ensuremath{\left\lceil #1 \right\rceil}}
\newcommand{\eclass}[1]{\ensuremath{\left[ #1 \right]}}

\newcommand{\chapternum}{}
\newcommand{\ex}[1]{\noindent\textbf{Exercise \chapternum.{#1}.}}

\newcommand{\tsub}[1]{\textsubscript{#1}}
\newcommand{\tsup}[1]{\textsuperscript{#1}}

\renewcommand{\choose}[2]{\ensuremath{\phantom{}_{#1}C_{#2}}}
\newcommand{\perm}[2]{\ensuremath{\phantom{}_{#1}P_{#2}}}

% Include figures
\newcommand{\incfig}[2][1]{%
    \def\svgwidth{#1\columnwidth}
    \import{./figures/}{#2.pdf_tex}
}

\definecolor{problemBackground}{RGB}{212,232,246}
\newenvironment{problem}[1]
  {
    \begin{tcolorbox}[
      boxrule=.5pt,
      titlerule=.5pt,
      sharp corners,
      colback=problemBackground,
      breakable
    ]
    \ifx &#1& \textbf{Problem. }
    \else \textbf{Problem #1.} \fi
  }
  {
    \end{tcolorbox}
  }
\definecolor{exampleBackground}{RGB}{255,249,248}
\definecolor{exampleAccent}{RGB}{158,60,14}
\newenvironment{example}[1]
  {
    \begin{tcolorbox}[
      boxrule=.5pt,
      sharp corners,
      colback=exampleBackground,
      colframe=exampleAccent,
    ]
    \color{exampleAccent}\textbf{Example.} \emph{#1}\color{black}
  }
  {
    \end{tcolorbox}
  }
\definecolor{theoremBackground}{RGB}{234,243,251}
\definecolor{theoremAccent}{RGB}{0,116,183}
\newenvironment{theorem}[1]
  {
    \begin{tcolorbox}[
      boxrule=.5pt,
      titlerule=.5pt,
      sharp corners,
      colback=theoremBackground,
      colframe=theoremAccent,
      breakable
    ]
      % \color{theoremAccent}\textbf{Theorem --- }\emph{#1}\\\color{black}
    \ifx &#1& \color{theoremAccent}\textbf{Theorem. }\color{black}
    \else \color{theoremAccent}\textbf{Theorem --- }\emph{#1}\\\color{black} \fi
  }
  {
    \end{tcolorbox}
  }
\definecolor{noteBackground}{RGB}{244,249,244}
\definecolor{noteAccent}{RGB}{34,139,34}
\newenvironment{note}[1]
  {
  \begin{tcolorbox}[
    enhanced,
    boxrule=0pt,
    frame hidden,
    sharp corners,
    colback=noteBackground,
    borderline west={3pt}{-1.5pt}{noteAccent},
    breakable
    ]
    \ifx &#1& \color{noteAccent}\textbf{Note. }\color{black}
    \else \color{noteAccent}\textbf{Note (#1). }\color{black} \fi
    }
    {
  \end{tcolorbox}
  }
\definecolor{lemmaBackground}{RGB}{255,247,234}
\definecolor{lemmaAccent}{RGB}{255,153,0}
\newenvironment{lemma}[1]
  {
    \begin{tcolorbox}[
      enhanced,
      boxrule=0pt,
      frame hidden,
      sharp corners,
      colback=lemmaBackground,
      borderline west={3pt}{-1.5pt}{lemmaAccent},
      breakable
    ]
    \ifx &#1& \color{lemmaAccent}\textbf{Lemma. }\color{black}
    \else \color{lemmaAccent}\textbf{Lemma #1. }\color{black} \fi
  }
  {
    \end{tcolorbox}
  }
\definecolor{definitionBackground}{RGB}{246,246,246}
\newenvironment{definition}[1]
  {
    \begin{tcolorbox}[
      enhanced,
      boxrule=0pt,
      frame hidden,
      sharp corners,
      colback=definitionBackground,
      borderline west={3pt}{-1.5pt}{black},
      breakable
    ]
    \textbf{Definition. }\emph{#1}\\
  }
  {
    \end{tcolorbox}
  }

\newenvironment{amatrix}[2]{
    \left[
      \begin{array}{*{#1}{c}|*{#2}c}
  }
  {
      \end{array}
    \right]
  }

\definecolor{codeBackground}{RGB}{253,246,227}
\renewcommand\theFancyVerbLine{\arabic{FancyVerbLine}}
\newtcblisting{cpp}[1][]{
  boxrule=1pt,
  sharp corners,
  colback=codeBackground,
  listing only,
  breakable,
  minted language=cpp,
  minted style=material,
  minted options={
    xleftmargin=15pt,
    baselinestretch=1.2,
    linenos,
  }
}

\author{Kyle Chui}


\fancyhf{}
\lhead{Kyle Chui}
\rhead{Page \thepage}
\pagestyle{fancy}

\begin{document}
  \section{Mid to Late Meiji}
  \subsection{Late Meiji (1895--1912)}
  \begin{itemize}
    \item 1894--1895 Sino-Japanese War
    \item $1898$ Spanish-American War
    \item $1900$ Boxer Rebellion in China
    \item $1902$ British-Japanese alliance
    \item 1904--1905 Russo-Japanese War
    \item $1906$ Establishment of South Manchurian Railway
    \item $1910$ Japan-Korea Annexation Treaty
    \item $1911$ Japan regains control over customs duties (end of unequal treaties)
    \item $1912$ Republic of China is established
  \end{itemize}
  \begin{note}{}
    Before this time, the main distinctions in Japanese society were between classes---now the main distinction is between the Japanese and foreigners. There is more of a sense of loyalty to one's country.
  \end{note}
  \subsection{National Language and Nation-State}
  There are four key elements to make a modern nation:
  \begin{enumerate}
    \item Land (territorial sovereignty)
    \item Law (a national legal system)
    \item Race (familial relation between citizens)
    \item Unity (of politics, history, religion, language, and education)
  \end{enumerate}
  \begin{note}{}
    Japan was one of the only non-white civilisations that was going through imperialism.
  \end{note}
  \subsection{Boxer Rebellion}
  \begin{itemize}
    \item Unrest in North-east china after the Sino-Japanese War
    \item ``Boxers'' were martial religious societies
    \item Anti-imperialist, anti-foreign, anti-Christian popular uprising
    \item Put down by an alliance of the ``civilised countries'' (Japan, US, Germany, etc.)
  \end{itemize}
  \subsection{Russo-Japanese War (1904--1905)}
  \begin{itemize}
    \item Fought due to tensions over Korea and Manchuria
    \begin{note}{}
      Unprecedented casualties due to new weaponry.
    \end{note}
    \item Portsmouth Treaty (1905)---Allowed Japan rights over Korea, expansion into Manchuria, southern half of Sakhalin 
    \item $1910$ Japan-Korea annexation treaty (full annexation)
    \item End to unequal treaties (restoration of tariff autonomy) and extraterritoriality
  \end{itemize}
  \subsection{Late Meiji Politics}
  \begin{itemize}
    \item Decrease in the power of the Oligarchs (Genr\=o)
    \item New political parties
    \item New generation of Government bureaucrats
    \item Increased connections between politics and business
  \end{itemize}
  \subsection{Economic and Social Developments}
  \begin{itemize}
    \item Development of Industry
    \begin{itemize}
      \item Steel, textiles
      \item Increased imports, urbanisation, and labour productivity
    \end{itemize}
    \item Exploitation of workers (leads to labour unions) 
    \item Social unrest---Politically radical movements
    \item Emigration policies---there are now many Japanese outside of Japan
  \end{itemize}
  \subsection{Socialist Movements}
  \begin{itemize}
    \item Founding of the Social Democratic Party
    \item ``Commoner News''---A socialist paper reporting on about labour unrest
    \item ``Women of the World''---Feminist newspaper
  \end{itemize}
  \subsection{Riots in Tokyo}
  \begin{itemize}
    \item September 5--7, $1905$ (Hibiya Park)
    \begin{itemize}
      \item Against peace ending Russo-Japanese War
      \item For ``constitutional government''
      \item $19$ killed, 2/3 of police boxes, $15$ street cars destroyed, $311$ arrested
    \end{itemize}
    \item Match 15--18, September 5--8, $1906$ (Hibiya Park) 
    \begin{itemize}
      \item Against street car fare increase, ``unconstitutional'' behaviour of government
      \item Street cars damages, police boxes destroyed, scores of injured, arrests
    \end{itemize}
    \item February $10$, $1913$ (Outside Diet)
    \begin{itemize}
      \item For constitutional government
      \item A lot of chaos and violence
    \end{itemize}
  \end{itemize}
  \subsection{Individuality}
  \begin{itemize}
    \item Individual human beings/individual nations
    \item Legal individual (the nation is the sum of its citizens)
    \item Political individual (representation)
    \item Spiritual individual (notion of inner self)
    \item Medical/Psychological individual
    \item The individual self as a fundamental literary topic (self-expression)
  \end{itemize}
\end{document}
