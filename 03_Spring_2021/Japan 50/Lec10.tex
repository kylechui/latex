\documentclass[class=article, crop=false]{standalone}
% Import packages
\usepackage[margin=1in]{geometry}

\usepackage{amssymb, amsthm}
\usepackage{comment}
\usepackage{enumitem}
\usepackage{fancyhdr}
\usepackage{hyperref}
\usepackage{import}
\usepackage{mathrsfs, mathtools}
\usepackage{multicol}
\usepackage{pdfpages}
\usepackage{standalone}
\usepackage{transparent}
\usepackage{xcolor}

\usepackage{minted}
\usepackage[many]{tcolorbox}
\tcbuselibrary{minted}

% Declare math operators
\DeclareMathOperator{\lcm}{lcm}
\DeclareMathOperator{\proj}{proj}
\DeclareMathOperator{\vspan}{span}
\DeclareMathOperator{\im}{im}
\DeclareMathOperator{\Diff}{Diff}
\DeclareMathOperator{\Int}{Int}
\DeclareMathOperator{\fcn}{fcn}
\DeclareMathOperator{\id}{id}
\DeclareMathOperator{\rank}{rank}
\DeclareMathOperator{\tr}{tr}
\DeclareMathOperator{\dive}{div}
\DeclareMathOperator{\row}{row}
\DeclareMathOperator{\col}{col}
\DeclareMathOperator{\dom}{dom}
\DeclareMathOperator{\ran}{ran}
\DeclareMathOperator{\Dom}{Dom}
\DeclareMathOperator{\Ran}{Ran}
\DeclareMathOperator{\fl}{fl}
% Macros for letters/variables
\newcommand\ttilde{\kern -.15em\lower .7ex\hbox{\~{}}\kern .04em}
\newcommand{\N}{\ensuremath{\mathbb{N}}}
\newcommand{\Z}{\ensuremath{\mathbb{Z}}}
\newcommand{\Q}{\ensuremath{\mathbb{Q}}}
\newcommand{\R}{\ensuremath{\mathbb{R}}}
\newcommand{\C}{\ensuremath{\mathbb{C}}}
\newcommand{\F}{\ensuremath{\mathbb{F}}}
\newcommand{\M}{\ensuremath{\mathbb{M}}}
\renewcommand{\P}{\ensuremath{\mathbb{P}}}
\newcommand{\lam}{\ensuremath{\lambda}}
\newcommand{\nab}{\ensuremath{\nabla}}
\newcommand{\eps}{\ensuremath{\varepsilon}}
\newcommand{\es}{\ensuremath{\varnothing}}
% Macros for math symbols
\newcommand{\dx}[1]{\,\mathrm{d}#1}
\newcommand{\inv}{\ensuremath{^{-1}}}
\newcommand{\sm}{\setminus}
\newcommand{\sse}{\subseteq}
\newcommand{\ceq}{\coloneqq}
% Macros for pairs of math symbols
\newcommand{\abs}[1]{\ensuremath{\left\lvert #1 \right\rvert}}
\newcommand{\paren}[1]{\ensuremath{\left( #1 \right)}}
\newcommand{\norm}[1]{\ensuremath{\left\lVert #1\right\rVert}}
\newcommand{\set}[1]{\ensuremath{\left\{#1\right\}}}
\newcommand{\tup}[1]{\ensuremath{\left\langle #1 \right\rangle}}
\newcommand{\floor}[1]{\ensuremath{\left\lfloor #1 \right\rfloor}}
\newcommand{\ceil}[1]{\ensuremath{\left\lceil #1 \right\rceil}}
\newcommand{\eclass}[1]{\ensuremath{\left[ #1 \right]}}

\newcommand{\chapternum}{}
\newcommand{\ex}[1]{\noindent\textbf{Exercise \chapternum.{#1}.}}

\newcommand{\tsub}[1]{\textsubscript{#1}}
\newcommand{\tsup}[1]{\textsuperscript{#1}}

\renewcommand{\choose}[2]{\ensuremath{\phantom{}_{#1}C_{#2}}}
\newcommand{\perm}[2]{\ensuremath{\phantom{}_{#1}P_{#2}}}

% Include figures
\newcommand{\incfig}[2][1]{%
    \def\svgwidth{#1\columnwidth}
    \import{./figures/}{#2.pdf_tex}
}

\definecolor{problemBackground}{RGB}{212,232,246}
\newenvironment{problem}[1]
  {
    \begin{tcolorbox}[
      boxrule=.5pt,
      titlerule=.5pt,
      sharp corners,
      colback=problemBackground,
      breakable
    ]
    \ifx &#1& \textbf{Problem. }
    \else \textbf{Problem #1.} \fi
  }
  {
    \end{tcolorbox}
  }
\definecolor{exampleBackground}{RGB}{255,249,248}
\definecolor{exampleAccent}{RGB}{158,60,14}
\newenvironment{example}[1]
  {
    \begin{tcolorbox}[
      boxrule=.5pt,
      sharp corners,
      colback=exampleBackground,
      colframe=exampleAccent,
    ]
    \color{exampleAccent}\textbf{Example.} \emph{#1}\color{black}
  }
  {
    \end{tcolorbox}
  }
\definecolor{theoremBackground}{RGB}{234,243,251}
\definecolor{theoremAccent}{RGB}{0,116,183}
\newenvironment{theorem}[1]
  {
    \begin{tcolorbox}[
      boxrule=.5pt,
      titlerule=.5pt,
      sharp corners,
      colback=theoremBackground,
      colframe=theoremAccent,
      breakable
    ]
      % \color{theoremAccent}\textbf{Theorem --- }\emph{#1}\\\color{black}
    \ifx &#1& \color{theoremAccent}\textbf{Theorem. }\color{black}
    \else \color{theoremAccent}\textbf{Theorem --- }\emph{#1}\\\color{black} \fi
  }
  {
    \end{tcolorbox}
  }
\definecolor{noteBackground}{RGB}{244,249,244}
\definecolor{noteAccent}{RGB}{34,139,34}
\newenvironment{note}[1]
  {
  \begin{tcolorbox}[
    enhanced,
    boxrule=0pt,
    frame hidden,
    sharp corners,
    colback=noteBackground,
    borderline west={3pt}{-1.5pt}{noteAccent},
    breakable
    ]
    \ifx &#1& \color{noteAccent}\textbf{Note. }\color{black}
    \else \color{noteAccent}\textbf{Note (#1). }\color{black} \fi
    }
    {
  \end{tcolorbox}
  }
\definecolor{lemmaBackground}{RGB}{255,247,234}
\definecolor{lemmaAccent}{RGB}{255,153,0}
\newenvironment{lemma}[1]
  {
    \begin{tcolorbox}[
      enhanced,
      boxrule=0pt,
      frame hidden,
      sharp corners,
      colback=lemmaBackground,
      borderline west={3pt}{-1.5pt}{lemmaAccent},
      breakable
    ]
    \ifx &#1& \color{lemmaAccent}\textbf{Lemma. }\color{black}
    \else \color{lemmaAccent}\textbf{Lemma #1. }\color{black} \fi
  }
  {
    \end{tcolorbox}
  }
\definecolor{definitionBackground}{RGB}{246,246,246}
\newenvironment{definition}[1]
  {
    \begin{tcolorbox}[
      enhanced,
      boxrule=0pt,
      frame hidden,
      sharp corners,
      colback=definitionBackground,
      borderline west={3pt}{-1.5pt}{black},
      breakable
    ]
    \textbf{Definition. }\emph{#1}\\
  }
  {
    \end{tcolorbox}
  }

\newenvironment{amatrix}[2]{
    \left[
      \begin{array}{*{#1}{c}|*{#2}c}
  }
  {
      \end{array}
    \right]
  }

\definecolor{codeBackground}{RGB}{253,246,227}
\renewcommand\theFancyVerbLine{\arabic{FancyVerbLine}}
\newtcblisting{cpp}[1][]{
  boxrule=1pt,
  sharp corners,
  colback=codeBackground,
  listing only,
  breakable,
  minted language=cpp,
  minted style=material,
  minted options={
    xleftmargin=15pt,
    baselinestretch=1.2,
    linenos,
  }
}

\author{Kyle Chui}


\fancyhf{}
\lhead{Kyle Chui}
\rhead{Page \thepage}
\pagestyle{fancy}

\begin{document}
  \section{The Edo Period}
  \subsection{Late Tokugawa Ideology}
  \begin{itemize}
    \item Merchant ideas of utility and economic relations (neo-Confucian concepts re-interpreted in economic terms)
    \item Dutch learning and Western science (the Dutch are the only foreigners that are allowed in Japan)
    \item ``Study of Ancient Meaning''
    \begin{itemize}
      \item A challenge to neo-Confucian Tokugawa orthodoxy
    \end{itemize}
    \item Nativist learning. Shint\=o revivalists
    \item Sonn\=o J\=oi (Revere the sovereign and expel the barbarians)
  \end{itemize}
  \begin{note}{}
    The state doesn't really control the ideology any more---there are rich merchants, academies, etc.
  \end{note}
  The three realms are Heaven, Earth, and people.
  \subsubsection{Dutch Learning}
  \begin{itemize}
    \item Two main areas---medicine (botany, chemistry, physics, etc.) and astronomy (cartography, geography)
    \item Sugita Genpaku's report on an autopsy (1771)
    \item Science and sensationalism
  \end{itemize}
  \subsection{The Economic Decline of the Tokugawa Shogunate}
  \begin{itemize}
    \item Financial crises brought on by famines
    \item Kansei Reforms (1789--1791)
    \begin{itemize}
      \item Lowering of rice prices, rent control, restrictions on merchant guilds, freezing of foreign policy, censorship
    \end{itemize}
    \item Tokugawa shogunate budget deficit due to currency devaluations
    \item Samurai high status at odds with impoverished circumstances
    \item Famine, peasant uprisings, urban riots (1830--1840s)
    \item Further shogunate and domain reforms in 1840--1850s
    \begin{itemize}
      \item Major educational and economic reforms (try to make everyone literate)
      \item Ch\=osh\=u and Satsuma more successful than the shogunate (these later on start the Meiji restoration)
    \end{itemize}
  \end{itemize}
  \subsection{Japan Opens to Trade}
  \begin{itemize}
    \item $1842$ End of the first Opium war (between Great Britain and Qing)
    \item 1846--1848 Mexican-American War (US acquires California and New Mexico)
    \item $1848$ The California gold rush
    \item 1853--1854 Convention of Kanagawa (Treaty with United States)
    \item $1858$ Commercial Treaty signed with the United States
    \item Opening of ports causes inflation and more economic hardship
  \end{itemize}
  \subsubsection{Commodore Perry (1794--1858)}
  \begin{itemize}
    \item Fought in the war of 1812
    \item Second in command in the Mexican-American war
    \item Forces Japanese ports to open (via gunboat diplomacy)
  \end{itemize}
  \subsubsection{End of the Tokugawa Shogunate}
  \begin{itemize}
    \item $1858$ Shogunal succession dispute
    \item 1860s Widespread unrest, and there's a movement to expel the ``barbarian'' Westerners
    \item 1864--1868 Wars between shogunate and Ch\=osh\=u/Satsuma
    \item 1868--1869 Boshin War
    \item $1868$ Meiji Restoration
  \end{itemize}
  \subsection{Early Meiji}
  \begin{note}{}
    Early Meiji is a lot like Edo Period.
  \end{note}
  \begin{itemize}
    \item $1868$ Charter Oath
    \item $1869$ Emperor moves to Tokyo (new name for Edo)
    \item $1869$ Hokkaido Development Commission established (about expansion)
    \item 1871--1876 A series of reforms:
    \begin{itemize}
      \item Abolition of Han and establishment of prefectures
      \item Creation of national land tax system
      \item National conscripted military
      \item Abolition of four-class military (replaced with nobles and commoners)
      \item Western dress made mandatory at official functions
      \item Ban on Christianity lifted
      \item First national newspapers and railways
    \end{itemize}
    \item $1872$ Ryukyu established as a domain (annexed in 1879)
    \item $1873$ Iwakura Mission
    \begin{note}{}
      Mission to learn how the other governments/large powers worked.
    \end{note}
  \end{itemize}
  \subsubsection{Key Figures}
  \begin{itemize}
    \item \=Okubo Toshimichi (Satsuma)
    \begin{itemize}
      \item Leader of the Early Meiji Movement
      \item Enacted tax reforms
    \end{itemize}
    \item Kido K\=oin (Ch\=osh\=u)
    \begin{itemize}
      \item Drafted the Five-charter oath
      \item Oversaw emperor Meiji's education
    \end{itemize}
    \item Saig\=o Takamori (Satsuma)
    \begin{itemize}
      \item Opponent of modernisation
      \item Proposed invasion of Korea
      \item Rebelled against Meiji government and was defeated
    \end{itemize}
  \end{itemize}
  \subsubsection{Iwakura Mission}
  Objectives:
  \begin{itemize}
    \item Renegotiate unequal treaties
    \item Learn as much as possible about how other governments function
  \end{itemize}
\end{document}
