\documentclass[class=article, crop=false]{standalone}
\usepackage[margin=1in]{geometry}

\usepackage{amsmath}
\usepackage{amssymb}
\usepackage{amsthm}
\usepackage{comment}
\usepackage{enumitem}
\usepackage{fancyhdr}
\usepackage{hyperref}
\usepackage{mathrsfs}
\usepackage{mathtools}
\usepackage{standalone}
\usepackage{tcolorbox}
\usepackage{tikz}
\usepackage{xcolor}

\usetikzlibrary{decorations.pathreplacing}
\tcbuselibrary{skins}

\DeclareMathOperator{\lcm}{lcm}
\DeclareMathOperator{\proj}{proj}
\DeclareMathOperator{\vspan}{span}
\DeclareMathOperator{\im}{im}
\DeclareMathOperator{\range}{range}
\DeclareMathOperator{\Diff}{Diff}
\DeclareMathOperator{\Int}{Int}
\DeclareMathOperator{\fcn}{fcn}
\DeclareMathOperator{\id}{id}

\newcommand{\N}{\ensuremath{\mathbb{N}}}
\newcommand{\Z}{\ensuremath{\mathbb{Z}}}
\newcommand{\Q}{\ensuremath{\mathbb{Q}}}
\newcommand{\R}{\ensuremath{\mathbb{R}}}
\newcommand{\C}{\ensuremath{\mathbb{C}}}
\newcommand{\F}{\ensuremath{\mathbb{F}}}
\newcommand{\lam}{\ensuremath{\lambda}}
\newcommand{\nab}{\ensuremath{\nabla}}
\newcommand{\eps}{\ensuremath{\varepsilon}}
\newcommand{\es}{\ensuremath{\varnothing}}

\newcommand{\abs}[1]{\ensuremath{\left\lvert #1 \right\rvert}}
\newcommand{\bigpar}[1]{\ensuremath{\left( #1 \right)}}
\newcommand{\norm}[1]{\ensuremath{\lVert #1\rVert}}
\newcommand{\set}[1]{\ensuremath{\left\{#1\right\}}}
\newcommand{\tuple}[1]{\ensuremath{\left\langle #1 \right\rangle}}
\newcommand{\floor}[1]{\ensuremath{\left\lfloor #1 \right\rfloor}}
\newcommand{\ceil}[1]{\ensuremath{\left\lceil #1 \right\rceil}}

\newcommand{\chapternum}{}
\newcommand{\ex}[1]{\noindent\textbf{Exercise \chapternum.{#1}.}}

\newcommand{\dx}[1]{\,\mathrm{d}#1}
\newcommand{\inv}{\ensuremath{^{-1}}}
\newcommand{\sm}{\setminus}
\newcommand{\sse}{\subseteq}
\newcommand{\ceq}{\coloneqq}

\definecolor{problemBackground}{RGB}{212,232,246}
\newenvironment{problem}[1]
  {
    \begin{tcolorbox}[
      boxrule=.5pt,
      titlerule=.5pt,
      sharp corners,
      colback=problemBackground
    ]
    \ifx &#1& \textbf{Problem. }
    \else \textbf{Problem #1.} \fi
  }
  {
    \end{tcolorbox}
  }

\definecolor{exampleBackground}{RGB}{255,249,248}
\definecolor{exampleAccent}{RGB}{158,60,14}
\newenvironment{example}[1]
  {
    \begin{tcolorbox}[
      boxrule=.5pt,
      sharp corners,
      colback=exampleBackground,
      colframe=exampleAccent,
    ]
    \color{exampleAccent}\textbf{Example.} \emph{#1}\color{black}
  }
  {
    \end{tcolorbox}
  }

\definecolor{theoremBackground}{RGB}{234,243,251}
\definecolor{theoremAccent}{RGB}{0,116,183}
\newenvironment{theorem}[1]
  {
    \begin{tcolorbox}[
      boxrule=.5pt,
      titlerule=.5pt,
      sharp corners,
      colback=theoremBackground,
      colframe=theoremAccent
    ]
      \color{theoremAccent}\textbf{Theorem --- }\emph{#1}\\\color{black}
  }
  {
    \end{tcolorbox}
  }

\definecolor{noteBackground}{RGB}{244,249,244}
\definecolor{noteAccent}{RGB}{34,139,34}
\newenvironment{note}[1]
  {
  \begin{tcolorbox}[
    enhanced,
    boxrule=0pt,
    frame hidden,
    sharp corners,
    colback=noteBackground,
    borderline west={3pt}{-1.5pt}{noteAccent}
    ]
    \ifx &#1& \color{noteAccent}\textbf{Note. }\color{black}
    \else \color{noteAccent}\textbf{Note (#1). }\color{black} \fi
    }
    {
  \end{tcolorbox}
  }

\definecolor{definitionBackground}{RGB}{246,246,246}
\newenvironment{definition}[1]
  {
    \begin{tcolorbox}[
      enhanced,
      boxrule=0pt,
      frame hidden,
      sharp corners,
      colback=definitionBackground,
      borderline west={3pt}{-1.5pt}{black}
    ]
    \textbf{Definition. }\emph{#1}\\
  }
  {
    \end{tcolorbox}
  }
\newenvironment{amatrix}[2]{
    \left[
      \begin{array}{*{#1}{c}|*{#2}c}
  }
  {
      \end{array}
    \right]
  }
\date{\the\year-\the\month-\the\day}
\author{Kyle Chui}


\fancyhf{}
\lhead{Kyle Chui}
\rhead{Page \thepage}
\pagestyle{fancy}

\begin{document}
  \section{Lecture 10}
  \subsection{Searching for Life in our Solar System}
  \subsubsection{Mass Extinctions}
  There are various dips in the total species diversity in the fossil record. The most recent was $65$ million years ago, when the dinosaurs went extinct. The evidence of this is that all around the world, there is a thin layer of sediment that contains iridium. Below that layer, dinosaur fossils can be found. Above it, no dinosaur fossils are found.
  \begin{note}{}
    Iridium is rare on Earth, but not that rare for asteroids.
  \end{note}
  \begin{note}{}
    Asteroids are darker than charcoal---visible light is horrible for finding them. Instead, we use infrared cameras to find thermal radiation. We can use this to find dangerous ``city killers'' and deflect them.
  \end{note}
  The larger the object, the less frequently the impact occurs. The 6\tsup{th} extinction is already underway---the Anthropocene. We have already killed off many species via climate change.
  \subsubsection{The Building Blocks of Life}
  \begin{itemize}
    \item Fundamentally requires $25$ of the $92$ natural elements
    \item 96\% of biomass is hydrogen, oxygen, carbon, and nitrogen
    \item Life needs basic organic molecules to begin
    \item Observations show organic material covers the solar system
  \end{itemize}
  \begin{note}{}
    Energy decreases with the square of the distance (think Gaussian surface).
  \end{note}
  Life also requires a liquid, which plays roles such as being a solvent, transporting chemicals, and other important jobs.
  \subsubsection{Why Water?}
  \begin{enumerate}
    \item Broad and high liquid range
    \item Density of ice is less than density of water
    \item Water is a polar molecule and so has Hydrogen bonds
  \end{enumerate}
  \begin{note}{}
    The presence of water is usually the bottleneck for searching for life.
  \end{note}
  \subsubsection{Calculating the Surface Temperature of a Planet}
  Planets radiate \emph{thermal radiation}, given by
  \[
    L = (4\pi R^2)(\sigma) T^4.
  \]
  Furthermore, the more distant a planet is from the Sun, the less radiation it receives. Thus we have
  \[
    T^4 = \frac{L}{4\pi R^2\cdot \sigma},
  \]
  where $R$ is the planetary radius.
  \begin{note}{}
    Comets are extremely unlikely to contain life, because it is too cold for liquid water.
  \end{note}
  \subsection{A Tour of the Solar System}
  \begin{itemize}
    \item Probably no life on the Moon/Mercury, because it's too cold for liquid water.
    \item There could be life high in the Venusian atmosphere, where there are liquids. Venus has no tectonic activity, but is geologically active (volcanoes). Venus might have used to be habitable.
    \item Most of Earth's CO\tsub{2} is in (carbonate) rocks.
    \item Evidence that there used to be running water on Mars.
    \item Jupiter has a strong magnetic field generated by its convection layer of metallic Hydrogen. Both Jupiter and Saturn have massive vertical winds.
    \item Uranus and Neptune could have deep ``oceans'', but there's no way to explore this.
    \item The icy moons and dwarf planets in the Solar system could possibly contain life (Europa could have a subsurface ocean via tidal heating). Ceres has bright spots that could be water-rich.
  \end{itemize}
  \subsubsection{Spacecraft Exploration}
  \begin{itemize}
    \item Flyby---Cost-effective and practical
    \item Orbiter---More data from prolonged observations (can minimise cost with gravitational slingshots and aero braking)
    \item Landers/Probes---Direct measurements
    \item Sample return---Benefits of a lander/probe as well as large scale Earth facilities
    \item Tethered capture---For outer solar system? Avoids brief flyby
  \end{itemize}
\end{document}
