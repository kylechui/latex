\documentclass[class=article, crop=false]{standalone}
% Import packages
\usepackage[margin=1in]{geometry}

\usepackage{amssymb, amsthm}
\usepackage{comment}
\usepackage{enumitem}
\usepackage{fancyhdr}
\usepackage{hyperref}
\usepackage{import}
\usepackage{mathrsfs, mathtools}
\usepackage{multicol}
\usepackage{pdfpages}
\usepackage{standalone}
\usepackage{transparent}
\usepackage{xcolor}

\usepackage{minted}
\usepackage[many]{tcolorbox}
\tcbuselibrary{minted}

% Declare math operators
\DeclareMathOperator{\lcm}{lcm}
\DeclareMathOperator{\proj}{proj}
\DeclareMathOperator{\vspan}{span}
\DeclareMathOperator{\im}{im}
\DeclareMathOperator{\Diff}{Diff}
\DeclareMathOperator{\Int}{Int}
\DeclareMathOperator{\fcn}{fcn}
\DeclareMathOperator{\id}{id}
\DeclareMathOperator{\rank}{rank}
\DeclareMathOperator{\tr}{tr}
\DeclareMathOperator{\dive}{div}
\DeclareMathOperator{\row}{row}
\DeclareMathOperator{\col}{col}
\DeclareMathOperator{\dom}{dom}
\DeclareMathOperator{\ran}{ran}
\DeclareMathOperator{\Dom}{Dom}
\DeclareMathOperator{\Ran}{Ran}
\DeclareMathOperator{\fl}{fl}
% Macros for letters/variables
\newcommand\ttilde{\kern -.15em\lower .7ex\hbox{\~{}}\kern .04em}
\newcommand{\N}{\ensuremath{\mathbb{N}}}
\newcommand{\Z}{\ensuremath{\mathbb{Z}}}
\newcommand{\Q}{\ensuremath{\mathbb{Q}}}
\newcommand{\R}{\ensuremath{\mathbb{R}}}
\newcommand{\C}{\ensuremath{\mathbb{C}}}
\newcommand{\F}{\ensuremath{\mathbb{F}}}
\newcommand{\M}{\ensuremath{\mathbb{M}}}
\renewcommand{\P}{\ensuremath{\mathbb{P}}}
\newcommand{\lam}{\ensuremath{\lambda}}
\newcommand{\nab}{\ensuremath{\nabla}}
\newcommand{\eps}{\ensuremath{\varepsilon}}
\newcommand{\es}{\ensuremath{\varnothing}}
% Macros for math symbols
\newcommand{\dx}[1]{\,\mathrm{d}#1}
\newcommand{\inv}{\ensuremath{^{-1}}}
\newcommand{\sm}{\setminus}
\newcommand{\sse}{\subseteq}
\newcommand{\ceq}{\coloneqq}
% Macros for pairs of math symbols
\newcommand{\abs}[1]{\ensuremath{\left\lvert #1 \right\rvert}}
\newcommand{\paren}[1]{\ensuremath{\left( #1 \right)}}
\newcommand{\norm}[1]{\ensuremath{\left\lVert #1\right\rVert}}
\newcommand{\set}[1]{\ensuremath{\left\{#1\right\}}}
\newcommand{\tup}[1]{\ensuremath{\left\langle #1 \right\rangle}}
\newcommand{\floor}[1]{\ensuremath{\left\lfloor #1 \right\rfloor}}
\newcommand{\ceil}[1]{\ensuremath{\left\lceil #1 \right\rceil}}
\newcommand{\eclass}[1]{\ensuremath{\left[ #1 \right]}}

\newcommand{\chapternum}{}
\newcommand{\ex}[1]{\noindent\textbf{Exercise \chapternum.{#1}.}}

\newcommand{\tsub}[1]{\textsubscript{#1}}
\newcommand{\tsup}[1]{\textsuperscript{#1}}

\renewcommand{\choose}[2]{\ensuremath{\phantom{}_{#1}C_{#2}}}
\newcommand{\perm}[2]{\ensuremath{\phantom{}_{#1}P_{#2}}}

% Include figures
\newcommand{\incfig}[2][1]{%
    \def\svgwidth{#1\columnwidth}
    \import{./figures/}{#2.pdf_tex}
}

\definecolor{problemBackground}{RGB}{212,232,246}
\newenvironment{problem}[1]
  {
    \begin{tcolorbox}[
      boxrule=.5pt,
      titlerule=.5pt,
      sharp corners,
      colback=problemBackground,
      breakable
    ]
    \ifx &#1& \textbf{Problem. }
    \else \textbf{Problem #1.} \fi
  }
  {
    \end{tcolorbox}
  }
\definecolor{exampleBackground}{RGB}{255,249,248}
\definecolor{exampleAccent}{RGB}{158,60,14}
\newenvironment{example}[1]
  {
    \begin{tcolorbox}[
      boxrule=.5pt,
      sharp corners,
      colback=exampleBackground,
      colframe=exampleAccent,
    ]
    \color{exampleAccent}\textbf{Example.} \emph{#1}\color{black}
  }
  {
    \end{tcolorbox}
  }
\definecolor{theoremBackground}{RGB}{234,243,251}
\definecolor{theoremAccent}{RGB}{0,116,183}
\newenvironment{theorem}[1]
  {
    \begin{tcolorbox}[
      boxrule=.5pt,
      titlerule=.5pt,
      sharp corners,
      colback=theoremBackground,
      colframe=theoremAccent,
      breakable
    ]
      % \color{theoremAccent}\textbf{Theorem --- }\emph{#1}\\\color{black}
    \ifx &#1& \color{theoremAccent}\textbf{Theorem. }\color{black}
    \else \color{theoremAccent}\textbf{Theorem --- }\emph{#1}\\\color{black} \fi
  }
  {
    \end{tcolorbox}
  }
\definecolor{noteBackground}{RGB}{244,249,244}
\definecolor{noteAccent}{RGB}{34,139,34}
\newenvironment{note}[1]
  {
  \begin{tcolorbox}[
    enhanced,
    boxrule=0pt,
    frame hidden,
    sharp corners,
    colback=noteBackground,
    borderline west={3pt}{-1.5pt}{noteAccent},
    breakable
    ]
    \ifx &#1& \color{noteAccent}\textbf{Note. }\color{black}
    \else \color{noteAccent}\textbf{Note (#1). }\color{black} \fi
    }
    {
  \end{tcolorbox}
  }
\definecolor{lemmaBackground}{RGB}{255,247,234}
\definecolor{lemmaAccent}{RGB}{255,153,0}
\newenvironment{lemma}[1]
  {
    \begin{tcolorbox}[
      enhanced,
      boxrule=0pt,
      frame hidden,
      sharp corners,
      colback=lemmaBackground,
      borderline west={3pt}{-1.5pt}{lemmaAccent},
      breakable
    ]
    \ifx &#1& \color{lemmaAccent}\textbf{Lemma. }\color{black}
    \else \color{lemmaAccent}\textbf{Lemma #1. }\color{black} \fi
  }
  {
    \end{tcolorbox}
  }
\definecolor{definitionBackground}{RGB}{246,246,246}
\newenvironment{definition}[1]
  {
    \begin{tcolorbox}[
      enhanced,
      boxrule=0pt,
      frame hidden,
      sharp corners,
      colback=definitionBackground,
      borderline west={3pt}{-1.5pt}{black},
      breakable
    ]
    \textbf{Definition. }\emph{#1}\\
  }
  {
    \end{tcolorbox}
  }

\newenvironment{amatrix}[2]{
    \left[
      \begin{array}{*{#1}{c}|*{#2}c}
  }
  {
      \end{array}
    \right]
  }

\definecolor{codeBackground}{RGB}{253,246,227}
\renewcommand\theFancyVerbLine{\arabic{FancyVerbLine}}
\newtcblisting{cpp}[1][]{
  boxrule=1pt,
  sharp corners,
  colback=codeBackground,
  listing only,
  breakable,
  minted language=cpp,
  minted style=material,
  minted options={
    xleftmargin=15pt,
    baselinestretch=1.2,
    linenos,
  }
}

\author{Kyle Chui}


\fancyhf{}
\lhead{Kyle Chui}
\rhead{Page \thepage}
\pagestyle{fancy}

\begin{document}
  \section{Lecture 12}
  \subsection{Magnetic Induction in Circuits}
  Take $\vec{B}$ uniform, increasing with $B = \alpha\cdot t$ into the page. We choose the area vector $\vec{A}$ to be out of the page, resulting in
  \[
    \Phi = \int \vec{B}\cdot \mathrm{d}\vec{A} = \vec{B}\cdot \vec{A} = -BA < 0,
  \]
  and
  \[
    \frac{\mathrm{d\Phi}}{\mathrm{d}t} = -A\cdot \frac{\mathrm{d}B}{\mathrm{d}t} = -A\alpha < 0.
  \]
  Then $\eps = -\frac{\mathrm{d\Phi}}{\mathrm{d}t} = A\cdot \alpha > 0$. We know that $\eps$ is defined by
  \[
    \eps = \oint \vec{E}\cdot \mathrm{d}\vec{\ell} > 0,
  \]
  and the direction of the loop is set by the direction of $\vec{A}$. By the right hand rule, this direction must be counter-clockwise. Thus having a positive $\eps$ means $\vec{E}$ is in the same direction as $\mathrm{d}\vec{\ell}$, so $\vec{E}$ is also counter-clockwise.
  \begin{note}{}
    Just like motional EMF: charges will accumulate almost instantaneously since $\vec{E} = 0$ inside the conductor.
  \end{note}
  Let's make this a circuit by attaching a resistor:
% \begin{figure}[ht]
%     \centering
%     \incfig{lecture12d2}
%     \caption{Lecture12D2}
%     \label{fig:lecture12d2}
% \end{figure}
  If we look at the direction of this current, it will induce a magnetic field $B'$ inside of the loop, that is pointing out of the page. This opposes the $B$ that is going into the page.
  \subsection{Concept of Self-inductance L, Solenoid Example}
  We know that the magnetic field inside a solenoid is
  \[
    B_\text{in} = \mu \frac{N}{\ell}I = \mu nI, \tag{$n\ceq \frac{N}{\ell}$}
  \]
  where $N$ is the number of turns, $\ell$ is the length of the solenoid, and $I$ is the current. The flux inside the solenoid is given by
  \[
    \Phi_B = B_\text{in}\cdot A\cdot N = (\mu nI)AN.
  \]
  We differentiate this to get the electromotive force (and assume the current is the only one that changes), 
  \[
    \eps = -\frac{\mathrm{d}\Phi_B}{\mathrm{d}t} = -\mu nAN\cdot \frac{\mathrm{d}I}{\mathrm{d}t}.
  \]
  We define the self-inductance $L$ by $L \ceq \mu nAN$, which simplifies the above into
  \[
    \boxed{\eps = -L\cdot \frac{\mathrm{d}I}{\mathrm{d}t}.}
  \]
  For other geometries:
  \begin{itemize}
    \item $B = (\text{constant of geometry})\cdot I$
    \item $\Phi_B = (\text{another constant of geometry})\cdot I$
    \item $\eps = -\frac{\mathrm{d}\Phi_B}{\mathrm{d}t} = -(\text{some constant})\cdot \frac{\mathrm{d}I}{\mathrm{d}t} = -L\cdot \frac{\mathrm{d}I}{\mathrm{d}t}$
  \end{itemize}
  \subsection{Magnetic Field Energy}
  Consider a voltage $V$ sending a current $I$ through an inductor $L$. To find the power dispersed across the inductor, we use
  \[
    P = \abs{I\eps} = IL\cdot \frac{\mathrm{d}I}{\mathrm{d}t}\tag{into/out of the inductor}
  \]
  Suppose $V$ starts at zero, then increases. The back $\eps$ opposes the increase of current---power needs to be supplied. We use the equation
  \[
    \text{Energy} = \int_{0}^{t}\text{Power}\cdot \mathrm{d}t' \tag{$t'$ is dummy variable}
  \]
  to find the energy stored in the inductor ($U$). Thus we have
  \begin{align*}
    U &= \int_{0}^{t}P \dx t' \\
      &= \int_{0}^{t}IL\cdot \paren{\frac{\mathrm{d}I}{\mathrm{d}t'}}\mathrm{d}t' \\
      &= \int_{0}^{I}I'L \dx I' \tag{Change dummy variable}\\
      &= \frac{1}{2}I^2L.
  \end{align*}
  \begin{note}{}
    This is similar to the equation for the energy of a capacitor, which was $U = \frac{1}{2}CV^2$.
  \end{note}
  \begin{note}{}
    Just as how we think of energy stored in a capacitor as being stored in the electric field between the plates, we may think of energy stored in an inductor as being stored in the magnetic field generated by the inductor.
  \end{note}
  \subsection{Magnetic Field Energy Density---Use Solenoid Calculation}
  We use the three equations we derived earlier:
  \begin{align*}
    \eps &= -L\cdot \frac{\mathrm{d}I}{\mathrm{d}t} \\
    U &= \frac{1}{2}LI^2 \\
    B_\text{in} &= \mu nI
  \end{align*}
  Plugging in the definition of $L$, we have
  \begin{align*}
    U &= \frac{1}{2}\mu n NAI^2 \\
      &= \frac{1}{2}\mu nNA\cdot \frac{B^2}{\mu^2 n^2} \\
      &= \frac{1}{2}NA\cdot \frac{B^2}{\mu n} \\
      &= \frac{\ell AB^2}{2\mu}. \tag{$n = \frac{N}{\ell}$}
  \end{align*}
  Since the volume of the solenoid is $A\cdot \ell$, we have
  \[
    \frac{U}{\text{volume}} = \frac{B^2}{2\mu}.
  \]
  \begin{note}{}
    This is analogous to the capacitor's energy density, given by
    \[
      \frac{U}{\text{volume}} = \frac{\eps}{2}E^2.
    \]
  \end{note}
\end{document}
