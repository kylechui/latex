\documentclass[class=article, crop=false]{standalone}
% Import packages
\usepackage[margin=1in]{geometry}

\usepackage{amssymb, amsthm}
\usepackage{comment}
\usepackage{enumitem}
\usepackage{fancyhdr}
\usepackage{hyperref}
\usepackage{import}
\usepackage{mathrsfs, mathtools}
\usepackage{multicol}
\usepackage{pdfpages}
\usepackage{standalone}
\usepackage{transparent}
\usepackage{xcolor}

\usepackage{minted}
\usepackage[many]{tcolorbox}
\tcbuselibrary{minted}

% Declare math operators
\DeclareMathOperator{\lcm}{lcm}
\DeclareMathOperator{\proj}{proj}
\DeclareMathOperator{\vspan}{span}
\DeclareMathOperator{\im}{im}
\DeclareMathOperator{\Diff}{Diff}
\DeclareMathOperator{\Int}{Int}
\DeclareMathOperator{\fcn}{fcn}
\DeclareMathOperator{\id}{id}
\DeclareMathOperator{\rank}{rank}
\DeclareMathOperator{\tr}{tr}
\DeclareMathOperator{\dive}{div}
\DeclareMathOperator{\row}{row}
\DeclareMathOperator{\col}{col}
\DeclareMathOperator{\dom}{dom}
\DeclareMathOperator{\ran}{ran}
\DeclareMathOperator{\Dom}{Dom}
\DeclareMathOperator{\Ran}{Ran}
\DeclareMathOperator{\fl}{fl}
% Macros for letters/variables
\newcommand\ttilde{\kern -.15em\lower .7ex\hbox{\~{}}\kern .04em}
\newcommand{\N}{\ensuremath{\mathbb{N}}}
\newcommand{\Z}{\ensuremath{\mathbb{Z}}}
\newcommand{\Q}{\ensuremath{\mathbb{Q}}}
\newcommand{\R}{\ensuremath{\mathbb{R}}}
\newcommand{\C}{\ensuremath{\mathbb{C}}}
\newcommand{\F}{\ensuremath{\mathbb{F}}}
\newcommand{\M}{\ensuremath{\mathbb{M}}}
\renewcommand{\P}{\ensuremath{\mathbb{P}}}
\newcommand{\lam}{\ensuremath{\lambda}}
\newcommand{\nab}{\ensuremath{\nabla}}
\newcommand{\eps}{\ensuremath{\varepsilon}}
\newcommand{\es}{\ensuremath{\varnothing}}
% Macros for math symbols
\newcommand{\dx}[1]{\,\mathrm{d}#1}
\newcommand{\inv}{\ensuremath{^{-1}}}
\newcommand{\sm}{\setminus}
\newcommand{\sse}{\subseteq}
\newcommand{\ceq}{\coloneqq}
% Macros for pairs of math symbols
\newcommand{\abs}[1]{\ensuremath{\left\lvert #1 \right\rvert}}
\newcommand{\paren}[1]{\ensuremath{\left( #1 \right)}}
\newcommand{\norm}[1]{\ensuremath{\left\lVert #1\right\rVert}}
\newcommand{\set}[1]{\ensuremath{\left\{#1\right\}}}
\newcommand{\tup}[1]{\ensuremath{\left\langle #1 \right\rangle}}
\newcommand{\floor}[1]{\ensuremath{\left\lfloor #1 \right\rfloor}}
\newcommand{\ceil}[1]{\ensuremath{\left\lceil #1 \right\rceil}}
\newcommand{\eclass}[1]{\ensuremath{\left[ #1 \right]}}

\newcommand{\chapternum}{}
\newcommand{\ex}[1]{\noindent\textbf{Exercise \chapternum.{#1}.}}

\newcommand{\tsub}[1]{\textsubscript{#1}}
\newcommand{\tsup}[1]{\textsuperscript{#1}}

\renewcommand{\choose}[2]{\ensuremath{\phantom{}_{#1}C_{#2}}}
\newcommand{\perm}[2]{\ensuremath{\phantom{}_{#1}P_{#2}}}

% Include figures
\newcommand{\incfig}[2][1]{%
    \def\svgwidth{#1\columnwidth}
    \import{./figures/}{#2.pdf_tex}
}

\definecolor{problemBackground}{RGB}{212,232,246}
\newenvironment{problem}[1]
  {
    \begin{tcolorbox}[
      boxrule=.5pt,
      titlerule=.5pt,
      sharp corners,
      colback=problemBackground,
      breakable
    ]
    \ifx &#1& \textbf{Problem. }
    \else \textbf{Problem #1.} \fi
  }
  {
    \end{tcolorbox}
  }
\definecolor{exampleBackground}{RGB}{255,249,248}
\definecolor{exampleAccent}{RGB}{158,60,14}
\newenvironment{example}[1]
  {
    \begin{tcolorbox}[
      boxrule=.5pt,
      sharp corners,
      colback=exampleBackground,
      colframe=exampleAccent,
    ]
    \color{exampleAccent}\textbf{Example.} \emph{#1}\color{black}
  }
  {
    \end{tcolorbox}
  }
\definecolor{theoremBackground}{RGB}{234,243,251}
\definecolor{theoremAccent}{RGB}{0,116,183}
\newenvironment{theorem}[1]
  {
    \begin{tcolorbox}[
      boxrule=.5pt,
      titlerule=.5pt,
      sharp corners,
      colback=theoremBackground,
      colframe=theoremAccent,
      breakable
    ]
      % \color{theoremAccent}\textbf{Theorem --- }\emph{#1}\\\color{black}
    \ifx &#1& \color{theoremAccent}\textbf{Theorem. }\color{black}
    \else \color{theoremAccent}\textbf{Theorem --- }\emph{#1}\\\color{black} \fi
  }
  {
    \end{tcolorbox}
  }
\definecolor{noteBackground}{RGB}{244,249,244}
\definecolor{noteAccent}{RGB}{34,139,34}
\newenvironment{note}[1]
  {
  \begin{tcolorbox}[
    enhanced,
    boxrule=0pt,
    frame hidden,
    sharp corners,
    colback=noteBackground,
    borderline west={3pt}{-1.5pt}{noteAccent},
    breakable
    ]
    \ifx &#1& \color{noteAccent}\textbf{Note. }\color{black}
    \else \color{noteAccent}\textbf{Note (#1). }\color{black} \fi
    }
    {
  \end{tcolorbox}
  }
\definecolor{lemmaBackground}{RGB}{255,247,234}
\definecolor{lemmaAccent}{RGB}{255,153,0}
\newenvironment{lemma}[1]
  {
    \begin{tcolorbox}[
      enhanced,
      boxrule=0pt,
      frame hidden,
      sharp corners,
      colback=lemmaBackground,
      borderline west={3pt}{-1.5pt}{lemmaAccent},
      breakable
    ]
    \ifx &#1& \color{lemmaAccent}\textbf{Lemma. }\color{black}
    \else \color{lemmaAccent}\textbf{Lemma #1. }\color{black} \fi
  }
  {
    \end{tcolorbox}
  }
\definecolor{definitionBackground}{RGB}{246,246,246}
\newenvironment{definition}[1]
  {
    \begin{tcolorbox}[
      enhanced,
      boxrule=0pt,
      frame hidden,
      sharp corners,
      colback=definitionBackground,
      borderline west={3pt}{-1.5pt}{black},
      breakable
    ]
    \textbf{Definition. }\emph{#1}\\
  }
  {
    \end{tcolorbox}
  }

\newenvironment{amatrix}[2]{
    \left[
      \begin{array}{*{#1}{c}|*{#2}c}
  }
  {
      \end{array}
    \right]
  }

\definecolor{codeBackground}{RGB}{253,246,227}
\renewcommand\theFancyVerbLine{\arabic{FancyVerbLine}}
\newtcblisting{cpp}[1][]{
  boxrule=1pt,
  sharp corners,
  colback=codeBackground,
  listing only,
  breakable,
  minted language=cpp,
  minted style=material,
  minted options={
    xleftmargin=15pt,
    baselinestretch=1.2,
    linenos,
  }
}

\author{Kyle Chui}


\fancyhf{}
\lhead{Kyle Chui}
\rhead{Page \thepage}
\pagestyle{fancy}

\begin{document}
  \section{Lecture 8}
  \subsection{Magnetism in Matter---Atoms}
  For a series of loops, we know that the magnetic moment is given by
  \[
    \vec{\mu} = NIA\cdot \hat{n},
  \]
  where $N$ is the number of loops, $I$ is the current, and $A$ is the area enclosed by the loop. Furthermore, the torque on the loop(s) in an external $\vec{B}$ field is given by $\vec{\tau}=\vec{\mu}\times \vec{B}$. This is analogous to a dipole in an electric field, where we had
  \[
    \vec{p} = q\cdot \vec{\ell}\quad \text{and} \quad \vec{\tau} = \vec{p}\times \vec{E}.
  \]
  \textbf{Question.} What happens if dipoles go to minimum potential energy $U$? \par
  If we call the \emph{electric field generated by the dipole} $\vec{E}'$, then we see that it tends to point in the opposite direction to $\vec{E}_0$, the original electric field. Then we have that the net electric field is given by $\vec{E} = \vec{E}_0 + \vec{E}'$. \par
  In contrast, the magnetic field created by the solenoid is usually in the \emph{same} direction as $\vec{B}_0$, so the magnetic field usually gets \emph{amplified}.
  \subsection{Describing Magnetic Materials}
  The textbook goes over the ``Bohr magneton'' for a single atom. We have the equations 
  \[
    \vec{M} = \frac{\vec{\mu}_\text{total}}{V}\quad \text{and} \quad \vec{B} = \vec{B}_0 + \mu_0\vec{M}.
  \]
  However, the above equations are usually not that useful, so we ignore them. More interestingly, we have
  \begin{align*}
    \mu_0\to \mu &= K_m\cdot \mu_0 \\
    \oint B\cdot \mathrm{d}\ell &= \mu I \\
    B\cdot 2\pi r &= \mu I.
  \end{align*}
  The $K_m$ constant referenced above is called ``permeability''. We replace all of the $\mu_0$'s in the previous equations with just $\mu$.
  \begin{note}{}
    This is analogous to the dielectric constant for capacitors.
  \end{note}
  \begin{example}{}
    The permeability of iron is approximately $1000$, so when you fill a solenoid with iron, you increase its magnetic field output by a factor of $1000$. This is similar to filling the space between the plates of a capacitor with a dielectric, which increased the capacitor's capacitance.
  \end{example}
  \begin{note}{}
    Most materials have a permeability $K_m$ that is close to $1$.
  \end{note}
  We define a new constant called ``susceptibility'' given by $\chi_m \ceq K_m - 1$. This makes it such that have a ``susceptibility'' close to zero are the items that are not very magnetic. \par
  If atomic loops perfectly align, then the currents inside the magnet perfectly cancel out, leaving only a net current along the boundary of the magnet. This generates the same effect as a solenoid, which just has a current looping around and around its boundary. 
  \subsection{Three Types of Magnetic Materials}
  \begin{enumerate}
    \item Iron-like: Ferromagnetic materials have $K_m \gg 1$ (much greater than one).
    \item Paramagnetic materials have $K_m = 1 + \underbrace{\chi_m}_{\mathclap{\text{around }10^{-4}}}$.
    \item Diamagnetic materials have $K_m = 1 - \underbrace{\abs{\chi_m}}_{\mathclap{\text{around }10^{-5}}}$.
  \end{enumerate}
  Ferromagnetic materials are useful:
  \begin{itemize}
    \item They have big $K_m$ due to atoms aligning better than thermally expected (they are usually in disarray).
  \end{itemize}
  In the following diagram (hysteresis curve), the $x$-axis represents the applied magnetic field, and the $y$-axis the induced magnetic field. If you apply a strong magnetic field to a ferromagnetic material, it will also create its own magnetic field, amplifying the effects of the original magnetic field. What's interesting here is that when you remove the external magnetic field, the ferromagnetic field will settle into being a permanent magnet. If you apply a negative magnetic field to the object, the same happens, but in the other direction.
  \begin{center}\resizebox{12cm}{!}{\import{./figures}{Lecture8D1.pdf_tex}}\end{center}
  The black curve above traces the induced field as you apply an external $\vec{B}$, and the blue curves show how the induced magnetic field changes with the external field after the initial application.
\end{document}
