\documentclass[class=article, crop=false]{standalone}
% Import packages
\usepackage[margin=1in]{geometry}

\usepackage{amssymb, amsthm}
\usepackage{comment}
\usepackage{enumitem}
\usepackage{fancyhdr}
\usepackage{hyperref}
\usepackage{import}
\usepackage{mathrsfs, mathtools}
\usepackage{multicol}
\usepackage{pdfpages}
\usepackage{standalone}
\usepackage{transparent}
\usepackage{xcolor}

\usepackage{minted}
\usepackage[many]{tcolorbox}
\tcbuselibrary{minted}

% Declare math operators
\DeclareMathOperator{\lcm}{lcm}
\DeclareMathOperator{\proj}{proj}
\DeclareMathOperator{\vspan}{span}
\DeclareMathOperator{\im}{im}
\DeclareMathOperator{\Diff}{Diff}
\DeclareMathOperator{\Int}{Int}
\DeclareMathOperator{\fcn}{fcn}
\DeclareMathOperator{\id}{id}
\DeclareMathOperator{\rank}{rank}
\DeclareMathOperator{\tr}{tr}
\DeclareMathOperator{\dive}{div}
\DeclareMathOperator{\row}{row}
\DeclareMathOperator{\col}{col}
\DeclareMathOperator{\dom}{dom}
\DeclareMathOperator{\ran}{ran}
\DeclareMathOperator{\Dom}{Dom}
\DeclareMathOperator{\Ran}{Ran}
\DeclareMathOperator{\fl}{fl}
% Macros for letters/variables
\newcommand\ttilde{\kern -.15em\lower .7ex\hbox{\~{}}\kern .04em}
\newcommand{\N}{\ensuremath{\mathbb{N}}}
\newcommand{\Z}{\ensuremath{\mathbb{Z}}}
\newcommand{\Q}{\ensuremath{\mathbb{Q}}}
\newcommand{\R}{\ensuremath{\mathbb{R}}}
\newcommand{\C}{\ensuremath{\mathbb{C}}}
\newcommand{\F}{\ensuremath{\mathbb{F}}}
\newcommand{\M}{\ensuremath{\mathbb{M}}}
\renewcommand{\P}{\ensuremath{\mathbb{P}}}
\newcommand{\lam}{\ensuremath{\lambda}}
\newcommand{\nab}{\ensuremath{\nabla}}
\newcommand{\eps}{\ensuremath{\varepsilon}}
\newcommand{\es}{\ensuremath{\varnothing}}
% Macros for math symbols
\newcommand{\dx}[1]{\,\mathrm{d}#1}
\newcommand{\inv}{\ensuremath{^{-1}}}
\newcommand{\sm}{\setminus}
\newcommand{\sse}{\subseteq}
\newcommand{\ceq}{\coloneqq}
% Macros for pairs of math symbols
\newcommand{\abs}[1]{\ensuremath{\left\lvert #1 \right\rvert}}
\newcommand{\paren}[1]{\ensuremath{\left( #1 \right)}}
\newcommand{\norm}[1]{\ensuremath{\left\lVert #1\right\rVert}}
\newcommand{\set}[1]{\ensuremath{\left\{#1\right\}}}
\newcommand{\tup}[1]{\ensuremath{\left\langle #1 \right\rangle}}
\newcommand{\floor}[1]{\ensuremath{\left\lfloor #1 \right\rfloor}}
\newcommand{\ceil}[1]{\ensuremath{\left\lceil #1 \right\rceil}}
\newcommand{\eclass}[1]{\ensuremath{\left[ #1 \right]}}

\newcommand{\chapternum}{}
\newcommand{\ex}[1]{\noindent\textbf{Exercise \chapternum.{#1}.}}

\newcommand{\tsub}[1]{\textsubscript{#1}}
\newcommand{\tsup}[1]{\textsuperscript{#1}}

\renewcommand{\choose}[2]{\ensuremath{\phantom{}_{#1}C_{#2}}}
\newcommand{\perm}[2]{\ensuremath{\phantom{}_{#1}P_{#2}}}

% Include figures
\newcommand{\incfig}[2][1]{%
    \def\svgwidth{#1\columnwidth}
    \import{./figures/}{#2.pdf_tex}
}

\definecolor{problemBackground}{RGB}{212,232,246}
\newenvironment{problem}[1]
  {
    \begin{tcolorbox}[
      boxrule=.5pt,
      titlerule=.5pt,
      sharp corners,
      colback=problemBackground,
      breakable
    ]
    \ifx &#1& \textbf{Problem. }
    \else \textbf{Problem #1.} \fi
  }
  {
    \end{tcolorbox}
  }
\definecolor{exampleBackground}{RGB}{255,249,248}
\definecolor{exampleAccent}{RGB}{158,60,14}
\newenvironment{example}[1]
  {
    \begin{tcolorbox}[
      boxrule=.5pt,
      sharp corners,
      colback=exampleBackground,
      colframe=exampleAccent,
    ]
    \color{exampleAccent}\textbf{Example.} \emph{#1}\color{black}
  }
  {
    \end{tcolorbox}
  }
\definecolor{theoremBackground}{RGB}{234,243,251}
\definecolor{theoremAccent}{RGB}{0,116,183}
\newenvironment{theorem}[1]
  {
    \begin{tcolorbox}[
      boxrule=.5pt,
      titlerule=.5pt,
      sharp corners,
      colback=theoremBackground,
      colframe=theoremAccent,
      breakable
    ]
      % \color{theoremAccent}\textbf{Theorem --- }\emph{#1}\\\color{black}
    \ifx &#1& \color{theoremAccent}\textbf{Theorem. }\color{black}
    \else \color{theoremAccent}\textbf{Theorem --- }\emph{#1}\\\color{black} \fi
  }
  {
    \end{tcolorbox}
  }
\definecolor{noteBackground}{RGB}{244,249,244}
\definecolor{noteAccent}{RGB}{34,139,34}
\newenvironment{note}[1]
  {
  \begin{tcolorbox}[
    enhanced,
    boxrule=0pt,
    frame hidden,
    sharp corners,
    colback=noteBackground,
    borderline west={3pt}{-1.5pt}{noteAccent},
    breakable
    ]
    \ifx &#1& \color{noteAccent}\textbf{Note. }\color{black}
    \else \color{noteAccent}\textbf{Note (#1). }\color{black} \fi
    }
    {
  \end{tcolorbox}
  }
\definecolor{lemmaBackground}{RGB}{255,247,234}
\definecolor{lemmaAccent}{RGB}{255,153,0}
\newenvironment{lemma}[1]
  {
    \begin{tcolorbox}[
      enhanced,
      boxrule=0pt,
      frame hidden,
      sharp corners,
      colback=lemmaBackground,
      borderline west={3pt}{-1.5pt}{lemmaAccent},
      breakable
    ]
    \ifx &#1& \color{lemmaAccent}\textbf{Lemma. }\color{black}
    \else \color{lemmaAccent}\textbf{Lemma #1. }\color{black} \fi
  }
  {
    \end{tcolorbox}
  }
\definecolor{definitionBackground}{RGB}{246,246,246}
\newenvironment{definition}[1]
  {
    \begin{tcolorbox}[
      enhanced,
      boxrule=0pt,
      frame hidden,
      sharp corners,
      colback=definitionBackground,
      borderline west={3pt}{-1.5pt}{black},
      breakable
    ]
    \textbf{Definition. }\emph{#1}\\
  }
  {
    \end{tcolorbox}
  }

\newenvironment{amatrix}[2]{
    \left[
      \begin{array}{*{#1}{c}|*{#2}c}
  }
  {
      \end{array}
    \right]
  }

\definecolor{codeBackground}{RGB}{253,246,227}
\renewcommand\theFancyVerbLine{\arabic{FancyVerbLine}}
\newtcblisting{cpp}[1][]{
  boxrule=1pt,
  sharp corners,
  colback=codeBackground,
  listing only,
  breakable,
  minted language=cpp,
  minted style=material,
  minted options={
    xleftmargin=15pt,
    baselinestretch=1.2,
    linenos,
  }
}

\author{Kyle Chui}


\fancyhf{}
\lhead{Kyle Chui}
\rhead{Page \thepage}
\pagestyle{fancy}

\begin{document}
  \section{Lecture 22}
  Consider a square wave front moving at the speed of light in the positive $x$ direction. Suppose that $\vec{E}$ is parallel to $\hat{\jmath}$ and $\vec{B}$ is parallel to $\hat{k}$, and that $\vec{E} = \vec{B} = 0$ when $x > ct$ (you are ahead of the wave front). \par
  These wave fronts are consistent with Maxwell's equations, and any linear combination of wave fronts is also a valid wave front, so you can model any wave front using these square pulses.
  \begin{theorem}{Wave Equation}
    For electromagnetic waves, we have
    \[
      \frac{\partial^2\mathbf{B}}{\partial t^2} - c^2\nab^2\mathbf{B} = 0.
    \]
  \end{theorem}
  \begin{note}{}
    If you plug in $\vec{B} = B(\vec{x} - \vec{v}t)$ into the above equation, you see that it works out when $\abs{\vec{v}} = c$.
  \end{note}
  \subsection{Energy and Momentum in Electromagnetic Waves}
  Suppose we have the sine waves:
  \begin{align*}
    \vec{E} &= E_\text{max}\sin(kx - \omega t)\hat{\jmath} \\
    \vec{B} &= B_\text{max}\sin(kx - \omega t)\hat{k}
  \end{align*}
  Let $E_\text{max} = 100$ V/m. Then
  \[
    B_\text{max} = \frac{E_\text{max}}{c} = \frac{100}{3\cdot 10^8}\ \mathrm{T} = 3.33\cdot 10^{-7}\ \mathrm{T}.
  \]
  Then the Energy in an $\vec{E}$-field is given by
  \[
    \frac{U_E}{\text{vol}} = \frac{\text{energy}}{\text{volume}} = \frac{\eps_0}{2}E^2 = \frac{8.85\cdot 10^{-12}}{2}(100)^2 = 4.4\cdot 10^{-8}\ \frac{\mathrm{J}}{\mathrm{m}^3}.
  \]
  Similarly, the energy in the $\vec{B}$-field is given by
  \[
    \frac{U_B}{\text{vol}} = \frac{B^2}{2\mu_0} = \frac{\paren{\frac{E}{c}}}{2\mu_0} = \frac{E^2}{2\mu_0\cdot \frac{1}{\mu_0\eps_0}} = \frac{\eps_0}{2}E^2.
  \]
  Thus we see that the energy stored in both waves is the same. If we add up the waves and average them, we get
  \begin{align*}
    \paren{\frac{U_{EM}}{\text{vol}}}_\text{avg} &= \frac{U_E+U_B}{\text{vol}} \\
                                                 &= \frac{1}{2} \paren{\frac{\eps_0}{2}E_\text{max}^2 + \frac{\eps_0}{2}E_\text{max}^2} \\
                                                 &= \frac{\eps_0}{2}E_\text{max}^2 \\
                                                 &= 4.4\cdot 10^{-8}\ \frac{\mathrm{J}}{\mathrm{m}^3}.
  \end{align*}
  To find the intensity (or power per unit area), we have
  \begin{align*}
    \frac{P}{A} &= \frac{U}{A\cdot t} \\
                &= \frac{U_{EM}\cdot V}{A\cdot t} \\
                &= \frac{U_{EM}\cdot A\cdot ct}{A\cdot t} \\
                &= U_{EM}\cdot c.
  \end{align*}
  \subsection{Electromagnetic Waves in Matter}
  When we have a medium, then we use $\eps$ and $\mu$ instead of $\eps_0$ and $\mu_0$, respectively (to take into account the constants). We have
  \[
    v = \frac{1}{\sqrt{\mu\eps}} = \frac{1}{\sqrt{K_m\mu_0K_e\eps_0}} = \frac{c}{\sqrt{K_mK_e}}.
  \]
  \begin{definition}{Refractive Index}
    We define the refractive index by
    \[
      n = \sqrt{K_mK_e},
    \]
    which is always greater than $1$.
  \end{definition}
\end{document}
