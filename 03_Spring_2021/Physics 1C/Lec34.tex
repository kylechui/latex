\documentclass[class=article, crop=false]{standalone}
\usepackage[margin=1in]{geometry}

\usepackage{amsmath}
\usepackage{amssymb}
\usepackage{amsthm}
\usepackage{comment}
\usepackage{enumitem}
\usepackage{fancyhdr}
\usepackage{hyperref}
\usepackage{mathrsfs}
\usepackage{mathtools}
\usepackage{standalone}
\usepackage{tcolorbox}
\usepackage{tikz}
\usepackage{xcolor}

\usetikzlibrary{decorations.pathreplacing}
\tcbuselibrary{skins}

\DeclareMathOperator{\lcm}{lcm}
\DeclareMathOperator{\proj}{proj}
\DeclareMathOperator{\vspan}{span}
\DeclareMathOperator{\im}{im}
\DeclareMathOperator{\range}{range}
\DeclareMathOperator{\Diff}{Diff}
\DeclareMathOperator{\Int}{Int}
\DeclareMathOperator{\fcn}{fcn}
\DeclareMathOperator{\id}{id}

\newcommand{\N}{\ensuremath{\mathbb{N}}}
\newcommand{\Z}{\ensuremath{\mathbb{Z}}}
\newcommand{\Q}{\ensuremath{\mathbb{Q}}}
\newcommand{\R}{\ensuremath{\mathbb{R}}}
\newcommand{\C}{\ensuremath{\mathbb{C}}}
\newcommand{\F}{\ensuremath{\mathbb{F}}}
\newcommand{\lam}{\ensuremath{\lambda}}
\newcommand{\nab}{\ensuremath{\nabla}}
\newcommand{\eps}{\ensuremath{\varepsilon}}
\newcommand{\es}{\ensuremath{\varnothing}}

\newcommand{\abs}[1]{\ensuremath{\left\lvert #1 \right\rvert}}
\newcommand{\bigpar}[1]{\ensuremath{\left( #1 \right)}}
\newcommand{\norm}[1]{\ensuremath{\lVert #1\rVert}}
\newcommand{\set}[1]{\ensuremath{\left\{#1\right\}}}
\newcommand{\tuple}[1]{\ensuremath{\left\langle #1 \right\rangle}}
\newcommand{\floor}[1]{\ensuremath{\left\lfloor #1 \right\rfloor}}
\newcommand{\ceil}[1]{\ensuremath{\left\lceil #1 \right\rceil}}

\newcommand{\chapternum}{}
\newcommand{\ex}[1]{\noindent\textbf{Exercise \chapternum.{#1}.}}

\newcommand{\dx}[1]{\,\mathrm{d}#1}
\newcommand{\inv}{\ensuremath{^{-1}}}
\newcommand{\sm}{\setminus}
\newcommand{\sse}{\subseteq}
\newcommand{\ceq}{\coloneqq}

\definecolor{problemBackground}{RGB}{212,232,246}
\newenvironment{problem}[1]
  {
    \begin{tcolorbox}[
      boxrule=.5pt,
      titlerule=.5pt,
      sharp corners,
      colback=problemBackground
    ]
    \ifx &#1& \textbf{Problem. }
    \else \textbf{Problem #1.} \fi
  }
  {
    \end{tcolorbox}
  }

\definecolor{exampleBackground}{RGB}{255,249,248}
\definecolor{exampleAccent}{RGB}{158,60,14}
\newenvironment{example}[1]
  {
    \begin{tcolorbox}[
      boxrule=.5pt,
      sharp corners,
      colback=exampleBackground,
      colframe=exampleAccent,
    ]
    \color{exampleAccent}\textbf{Example.} \emph{#1}\color{black}
  }
  {
    \end{tcolorbox}
  }

\definecolor{theoremBackground}{RGB}{234,243,251}
\definecolor{theoremAccent}{RGB}{0,116,183}
\newenvironment{theorem}[1]
  {
    \begin{tcolorbox}[
      boxrule=.5pt,
      titlerule=.5pt,
      sharp corners,
      colback=theoremBackground,
      colframe=theoremAccent
    ]
      \color{theoremAccent}\textbf{Theorem --- }\emph{#1}\\\color{black}
  }
  {
    \end{tcolorbox}
  }

\definecolor{noteBackground}{RGB}{244,249,244}
\definecolor{noteAccent}{RGB}{34,139,34}
\newenvironment{note}[1]
  {
  \begin{tcolorbox}[
    enhanced,
    boxrule=0pt,
    frame hidden,
    sharp corners,
    colback=noteBackground,
    borderline west={3pt}{-1.5pt}{noteAccent}
    ]
    \ifx &#1& \color{noteAccent}\textbf{Note. }\color{black}
    \else \color{noteAccent}\textbf{Note (#1). }\color{black} \fi
    }
    {
  \end{tcolorbox}
  }

\definecolor{definitionBackground}{RGB}{246,246,246}
\newenvironment{definition}[1]
  {
    \begin{tcolorbox}[
      enhanced,
      boxrule=0pt,
      frame hidden,
      sharp corners,
      colback=definitionBackground,
      borderline west={3pt}{-1.5pt}{black}
    ]
    \textbf{Definition. }\emph{#1}\\
  }
  {
    \end{tcolorbox}
  }
\newenvironment{amatrix}[2]{
    \left[
      \begin{array}{*{#1}{c}|*{#2}c}
  }
  {
      \end{array}
    \right]
  }
\date{\the\year-\the\month-\the\day}
\author{Kyle Chui}


\fancyhf{}
\lhead{Kyle Chui}
\rhead{Page \thepage}
\pagestyle{fancy}

\begin{document}
  \section{Lecture 34}
  \subsection{Light Clock Time Dilation}
  From last lecture, we found that the time passage for a moving object can be given by
  \[
    \Delta t = \frac{c}{\sqrt{c^2 - v^2}}\cdot \Delta\tau.
  \]
  Letting $\beta\ceq \frac{v}{c}$, we can simplify the above to
  \[
    \Delta t = \frac{1}{\sqrt{1 - \beta^2}}\cdot \Delta\tau.
  \]
  \subsubsection{The Gamma Factor}
  Let us also define Lorentz factor $\gamma$, by $\gamma \ceq \frac{1}{\sqrt{1 - \beta^2}}$ so $\Delta t = \gamma\cdot \Delta \tau$. We may write
  \[
    \gamma = \frac{1}{\sqrt{1 - \beta^2}} = \frac{1}{\sqrt{1 - \paren{\frac{v}{c}}^2}}.
  \]
  Notice that $\gamma\geq 1$, and is only equal to one when $v = 0$.
  \begin{example}{}
    Consider a train moving at $100$ m/s. Then the Lorentz factor is
    \begin{align*}
      \gamma &= \paren{1 - \paren{\frac{v}{c}}^2}^{-\frac{1}{2}} \\
             &= \paren{1 - \paren{\frac{100}{3\cdot 10^8}}^2}^{-\frac{1}{2}} \\
             &\approx 1 - \paren{-\frac{1}{2}}\paren{\frac{10^2}{3\cdot 10^8}}^2 \\
             &= 1 + 5.6\cdot 10^{-14}.
    \end{align*}
    Note that above we used the approximation that $(1 - \eps)^\alpha\approx 1 - \alpha\eps$.
  \end{example}
  \subsection{Length Contraction}
  Let the distance from the Earth to the Neptune be $L_0$. Suppose there is also a spaceship travelling from Earth to Neptune at speed $v$, so from an observer's perspective on Earth it takes time $t = \frac{L_0}{v}$ to get there. However, from the perspective of someone in the spaceship, we have $L = v\tau$, where $t = \gamma\tau$. Thus we have
  \[
    L = v\cdot \frac{t}{\gamma} = \frac{L_0}{\gamma} < L_0.
  \]
  We see that from the mover's perspective, they are travelling a shorter distance. Furthermore, the spaceship's length is contracted by a factor of $\gamma$ along the direction of motion.
  \subsection{Lorentz Transformation Introduction}
  Using the same notation as we did for Galilean relativity, the Lorentz transform is as follows:
  \begin{align*}
    x' &= \gamma(x - vt) & x &= \gamma(x' + vt') \\
    y' &= y & y &= y' \\
    z' &= z & z &= z' \\
    t' &= \gamma\paren{t - \frac{vx}{c^2}} & t &= \gamma\paren{t' + \frac{vx'}{c^2}}
  \end{align*}
  \begin{note}{}
    There is no change to the variables perpendicular to the direction of motion.
  \end{note}
\end{document}
