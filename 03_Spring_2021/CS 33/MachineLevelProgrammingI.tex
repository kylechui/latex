\documentclass[class=article, crop=false]{standalone}
% Import packages
\usepackage[margin=1in]{geometry}

\usepackage{amssymb, amsthm}
\usepackage{comment}
\usepackage{enumitem}
\usepackage{fancyhdr}
\usepackage{hyperref}
\usepackage{import}
\usepackage{mathrsfs, mathtools}
\usepackage{multicol}
\usepackage{pdfpages}
\usepackage{standalone}
\usepackage{transparent}
\usepackage{xcolor}

\usepackage{minted}
\usepackage[many]{tcolorbox}
\tcbuselibrary{minted}

% Declare math operators
\DeclareMathOperator{\lcm}{lcm}
\DeclareMathOperator{\proj}{proj}
\DeclareMathOperator{\vspan}{span}
\DeclareMathOperator{\im}{im}
\DeclareMathOperator{\Diff}{Diff}
\DeclareMathOperator{\Int}{Int}
\DeclareMathOperator{\fcn}{fcn}
\DeclareMathOperator{\id}{id}
\DeclareMathOperator{\rank}{rank}
\DeclareMathOperator{\tr}{tr}
\DeclareMathOperator{\dive}{div}
\DeclareMathOperator{\row}{row}
\DeclareMathOperator{\col}{col}
\DeclareMathOperator{\dom}{dom}
\DeclareMathOperator{\ran}{ran}
\DeclareMathOperator{\Dom}{Dom}
\DeclareMathOperator{\Ran}{Ran}
\DeclareMathOperator{\fl}{fl}
% Macros for letters/variables
\newcommand\ttilde{\kern -.15em\lower .7ex\hbox{\~{}}\kern .04em}
\newcommand{\N}{\ensuremath{\mathbb{N}}}
\newcommand{\Z}{\ensuremath{\mathbb{Z}}}
\newcommand{\Q}{\ensuremath{\mathbb{Q}}}
\newcommand{\R}{\ensuremath{\mathbb{R}}}
\newcommand{\C}{\ensuremath{\mathbb{C}}}
\newcommand{\F}{\ensuremath{\mathbb{F}}}
\newcommand{\M}{\ensuremath{\mathbb{M}}}
\renewcommand{\P}{\ensuremath{\mathbb{P}}}
\newcommand{\lam}{\ensuremath{\lambda}}
\newcommand{\nab}{\ensuremath{\nabla}}
\newcommand{\eps}{\ensuremath{\varepsilon}}
\newcommand{\es}{\ensuremath{\varnothing}}
% Macros for math symbols
\newcommand{\dx}[1]{\,\mathrm{d}#1}
\newcommand{\inv}{\ensuremath{^{-1}}}
\newcommand{\sm}{\setminus}
\newcommand{\sse}{\subseteq}
\newcommand{\ceq}{\coloneqq}
% Macros for pairs of math symbols
\newcommand{\abs}[1]{\ensuremath{\left\lvert #1 \right\rvert}}
\newcommand{\paren}[1]{\ensuremath{\left( #1 \right)}}
\newcommand{\norm}[1]{\ensuremath{\left\lVert #1\right\rVert}}
\newcommand{\set}[1]{\ensuremath{\left\{#1\right\}}}
\newcommand{\tup}[1]{\ensuremath{\left\langle #1 \right\rangle}}
\newcommand{\floor}[1]{\ensuremath{\left\lfloor #1 \right\rfloor}}
\newcommand{\ceil}[1]{\ensuremath{\left\lceil #1 \right\rceil}}
\newcommand{\eclass}[1]{\ensuremath{\left[ #1 \right]}}

\newcommand{\chapternum}{}
\newcommand{\ex}[1]{\noindent\textbf{Exercise \chapternum.{#1}.}}

\newcommand{\tsub}[1]{\textsubscript{#1}}
\newcommand{\tsup}[1]{\textsuperscript{#1}}

\renewcommand{\choose}[2]{\ensuremath{\phantom{}_{#1}C_{#2}}}
\newcommand{\perm}[2]{\ensuremath{\phantom{}_{#1}P_{#2}}}

% Include figures
\newcommand{\incfig}[2][1]{%
    \def\svgwidth{#1\columnwidth}
    \import{./figures/}{#2.pdf_tex}
}

\definecolor{problemBackground}{RGB}{212,232,246}
\newenvironment{problem}[1]
  {
    \begin{tcolorbox}[
      boxrule=.5pt,
      titlerule=.5pt,
      sharp corners,
      colback=problemBackground,
      breakable
    ]
    \ifx &#1& \textbf{Problem. }
    \else \textbf{Problem #1.} \fi
  }
  {
    \end{tcolorbox}
  }
\definecolor{exampleBackground}{RGB}{255,249,248}
\definecolor{exampleAccent}{RGB}{158,60,14}
\newenvironment{example}[1]
  {
    \begin{tcolorbox}[
      boxrule=.5pt,
      sharp corners,
      colback=exampleBackground,
      colframe=exampleAccent,
    ]
    \color{exampleAccent}\textbf{Example.} \emph{#1}\color{black}
  }
  {
    \end{tcolorbox}
  }
\definecolor{theoremBackground}{RGB}{234,243,251}
\definecolor{theoremAccent}{RGB}{0,116,183}
\newenvironment{theorem}[1]
  {
    \begin{tcolorbox}[
      boxrule=.5pt,
      titlerule=.5pt,
      sharp corners,
      colback=theoremBackground,
      colframe=theoremAccent,
      breakable
    ]
      % \color{theoremAccent}\textbf{Theorem --- }\emph{#1}\\\color{black}
    \ifx &#1& \color{theoremAccent}\textbf{Theorem. }\color{black}
    \else \color{theoremAccent}\textbf{Theorem --- }\emph{#1}\\\color{black} \fi
  }
  {
    \end{tcolorbox}
  }
\definecolor{noteBackground}{RGB}{244,249,244}
\definecolor{noteAccent}{RGB}{34,139,34}
\newenvironment{note}[1]
  {
  \begin{tcolorbox}[
    enhanced,
    boxrule=0pt,
    frame hidden,
    sharp corners,
    colback=noteBackground,
    borderline west={3pt}{-1.5pt}{noteAccent},
    breakable
    ]
    \ifx &#1& \color{noteAccent}\textbf{Note. }\color{black}
    \else \color{noteAccent}\textbf{Note (#1). }\color{black} \fi
    }
    {
  \end{tcolorbox}
  }
\definecolor{lemmaBackground}{RGB}{255,247,234}
\definecolor{lemmaAccent}{RGB}{255,153,0}
\newenvironment{lemma}[1]
  {
    \begin{tcolorbox}[
      enhanced,
      boxrule=0pt,
      frame hidden,
      sharp corners,
      colback=lemmaBackground,
      borderline west={3pt}{-1.5pt}{lemmaAccent},
      breakable
    ]
    \ifx &#1& \color{lemmaAccent}\textbf{Lemma. }\color{black}
    \else \color{lemmaAccent}\textbf{Lemma #1. }\color{black} \fi
  }
  {
    \end{tcolorbox}
  }
\definecolor{definitionBackground}{RGB}{246,246,246}
\newenvironment{definition}[1]
  {
    \begin{tcolorbox}[
      enhanced,
      boxrule=0pt,
      frame hidden,
      sharp corners,
      colback=definitionBackground,
      borderline west={3pt}{-1.5pt}{black},
      breakable
    ]
    \textbf{Definition. }\emph{#1}\\
  }
  {
    \end{tcolorbox}
  }

\newenvironment{amatrix}[2]{
    \left[
      \begin{array}{*{#1}{c}|*{#2}c}
  }
  {
      \end{array}
    \right]
  }

\definecolor{codeBackground}{RGB}{253,246,227}
\renewcommand\theFancyVerbLine{\arabic{FancyVerbLine}}
\newtcblisting{cpp}[1][]{
  boxrule=1pt,
  sharp corners,
  colback=codeBackground,
  listing only,
  breakable,
  minted language=cpp,
  minted style=material,
  minted options={
    xleftmargin=15pt,
    baselinestretch=1.2,
    linenos,
  }
}

\author{Kyle Chui}


\fancyhf{}
\lhead{Kyle Chui}
\rhead{Page \thepage}
\pagestyle{fancy}

\begin{document}
  \section{Machine-Level Programming I: Basics}
  \subsection{History of Intel processors and architectures}
  \begin{itemize}
    \item Dominate laptop/desktop/server market
    \item Evolutionary design
    \begin{itemize}
      \item Backwards compatible (up until $8086$, introduced in 1978).
      \item More features were added as time went on.
    \end{itemize}
    \item Complex instruction set computer (CISC)
    \begin{itemize}
      \item Many different instructions with many different formats.
      \item Hard to match the performance of Reduced Instruction Set Computers (RISC).
      \item Intel has performance, but not great at power efficiency.
    \end{itemize}
  \end{itemize}
  \subsection{C, assembly, machine code}
  \begin{definition}{Instruction Set Architecture (ISA)}
    An \emph{Instruction Set Architecture} is the parts of a processor design that one needs to understand or write assembly/machine code, i.e. instruction set specification, registers.
  \end{definition}
  \begin{definition}{Microarchitecture}
    The \emph{microarchitecture} is the implementation of the architecture, i.e. cache sizes and core frequency.
  \end{definition}
  \begin{definition}{Code Forms}
    There are two code forms---\emph{machine code}, which are the byte-level programs that a processor executes, and \emph{assembly code}, which is a text representation of machine code.
  \end{definition}
  Some examples of ISAs include x86, IA32, x86-64, and ARM.
  \subsubsection{Assembly/Machine Code View}
  CPU:
  \begin{itemize}
    \item PC: Program Counter
    \begin{itemize}
      \item Keeps track of which instruction will be run next, and is called the ``RIP'' for x86-64 architectures.
    \end{itemize}
    \item The Register file is used for heavily used program data (because it is very fast).
    \item Condition codes are special registers that store status information about the most recent arithmetic or logical operation, and are used for conditional branching.
  \end{itemize}
  Memory:
  \begin{itemize}
    \item We think of memory as a large array where each element is a byte.
    \item It holds code and user data, and uses a stack to support procedure calls.
  \end{itemize}
  \begin{center}\resizebox{9cm}{!}{\import{./Figures}{AssemblyMachineCodeView.pdf_tex}}\end{center}
  \subsubsection{Turning C into Object Code}
  We first code in human-readable files, such as \texttt{.c} files, and then compile them into an executable program (which is a binary and not human-readable).
  \begin{enumerate}
    \item We start with a C program, which is written in text.
    \item The code gets compiled into an Asm program, which is still in text.
    \item The Asm program goes through an assembler, which turns it into an object program (a binary).
    \item The object program finally goes through a linker, which turns it into an executable program (also a binary).
  \end{enumerate}
  \subsubsection{Assembly Characteristics: Data Types}
  \begin{itemize}
    \item ``Integer'' data of $1$, $2$, $4$, or $8$ bytes (includes data values and addresses).
    \item Floating point data of $4$, $8$, or $10$ bytes.
    \item Code: Byte sequences encoding series of instructions.
    \item No aggregate types such as arrays or structures (just contiguously allocated bytes in memory).
  \end{itemize}
  \subsubsection{Assembly Characteristics: Operations}
  \begin{itemize}
    \item Perform arithmetic functions on register or memory data.
    \item Transfer data between memory and register.
    \item Transfer control (changing the instruction pointer of the program counter).
  \end{itemize}
  \subsubsection{Object Code}
  Assembler:
  \begin{itemize}
    \item Translates \texttt{.s} into \texttt{.o}.
    \item Binary encoding of each instruction.
    \item Nearly-complete image of executable code.
    \item Missing linkages between code in different files.
  \end{itemize}
  Linker:
  \begin{itemize}
    \item Resolves references between files.
    \item Combines with static run-time libraries (e.g. code for \texttt{malloc}, \texttt{printf}).
    \item Some libraries are \emph{dynamically linked} (but linking occurs when program begins execution).
  \end{itemize}
  \subsection{Analysing Object Code}
  In the following examples, we are analysing the function \texttt{sumstore}.
  \subsubsection{Using the Disassembler}
  The command here would be \texttt{objdump -d sumstore}.
  \begin{itemize}
    \item It analyses the bit pattern for the series of instructions, and then products an approximate rendition of the assembly code. 
    \item Can be run on either a \texttt{.out} (complete executable) or \texttt{.o} file. 
  \end{itemize}
  \subsubsection{Alternate Disassembly Using gdb}
  The commands here would be:
  \begin{verbatim}
    gdb sum
    disassemble sumstore
  \end{verbatim}
  This will produce a list of locations and actions done (in assembly code), but not the actual bits inside each address. To see the first $14$ bytes of \texttt{sumstore}, you can call \texttt{x/14xb sumstore}.
  \begin{note}{}
    Anything that can be interpreted as executable code can be disassembled, but reverse-engineering for many programs is illegal.
  \end{note}
  \subsection{x86-64 Integer Registers}
  Of the $16$ registers listed below for a 64-bit system, the first eight share space with older 32-bit registers. You can reference the first four bytes of the first eight 64-bit registers by replacing the ``r'' with an ``e'' (i.e. \texttt{\%rax} $\to$ \texttt{\%eax}). For the other eight 64-bit registers, you append a \texttt{d} at the end (i.e. \texttt{\%r13} $\to$ \texttt{\%r13d}).
  \begin{center}\begin{tabular}{cccccccc}
    \texttt{\%rax} & \texttt{\%rbx} & \texttt{\%rcx} & \texttt{\%rdx} & \texttt{\%rsi} & \texttt{\%rdi} & \texttt{\%rsp} & \texttt{\%rbp} \\
    \texttt{\%r8} & \texttt{\%r9} & \texttt{\%r10} & \texttt{\%r11} & \texttt{\%r12} & \texttt{\%r13} & \texttt{\%r14} & \texttt{\%r15}
  \end{tabular}\end{center}
  \begin{note}{}
    The \texttt{\%rsp} register is the stack pointer register, and thus is the only one (for the purposes of this class) that cannot be used as a general purpose register.
  \end{note}
  \subsection{Some History: IA32 Registers}
  In older IA32 machines, the \texttt{\%ebp} register was also reserved to point at the top of the stack, in addition to the \texttt{\%esp} register being reserved as the stack pointer register. Furthermore the 32-bit registers could be decomposed even further by removing the \texttt{e} from the beginning of the register name (i.e. \texttt{\%edi} $\to$ \texttt{\%di}). Finally, the first four 16-bit registers (\texttt{\%ax}, \texttt{\%cx}, \texttt{\%dx}, \texttt{\%bx}) could all be split into ``high'' and ``low'' registers, i.e. \texttt{\%cx} could be split into \texttt{\%ch} and \texttt{\%cl}.
  \subsection{Moving Data}
  To move data with assembly code, we have the \texttt{movq source, dest} command.
  \begin{note}{}
    The \texttt{q} in \texttt{movq} stands for ``quad word'', or a 64-bit quantity. There is also \texttt{movl} for ``long word'', or a 32-bit quantity, etc.
  \end{note}
  \subsubsection{Operand Types}
  The \texttt{movq} command takes two parameters for its source and destination, which can either be registers, immediates, or memory locations.
  \begin{itemize}
    \item Immediates are constant integer data, just like constants in C, although they are prefixed with a \texttt{\$}, like \texttt{\$0x400}.
    \begin{itemize}
      \item They are encoded with $1$, $2$, or $4$ bytes.
    \end{itemize}
    \item Registers are one of the $16$ integer registers mentioned earlier.
    \begin{itemize}
      \item We already know that \texttt{\%rsp} is reserved for special use, but some of the other registers also have special use cases for particular instructions.
    \end{itemize}
    \item Memory is $8$ consecutive bytes of memory at the address given by the register, i.e. \texttt{(\%rax)}. There are also various other ``address modes''.
  \end{itemize}
  \subsection{\texttt{movq} Operand Combinations}
  In the following diagram, the first column indicates the source, the second the destination, the third an example of assembly code, and the final an example of C code.
  \begin{note}{}
    Where there are no parentheses around a register, i.e. \texttt{\%rax}, we are referring to writing \emph{directly} to the register. Where there are parentheses around the register, we dereference it and write to where the register points in memory. The only exception to this is the \texttt{leaq} command, which does address computation arithmetic when there are parentheses present.
  \end{note}
  \[
    \texttt{movq} \begin{cases}
      \text{Immediate} & \begin{cases}\begin{aligned}
        \text{Register}&&& \texttt{movq \$0x4, \%rax} && \texttt{temp = 0x4;} \\
        \text{Memory} &&& \texttt{movq \$-147, (\%rax)} && \texttt{*p = -147;}
      \end{aligned}\end{cases} \\[15pt]
        \text{Register} & \begin{cases}\begin{aligned}
          \text{Register}&&& \texttt{movq \%rax, \%rdx} && \texttt{temp2 = temp1;} \\
          \text{Memory}&&& \texttt{movq \%rax, (\%rdx)} && \texttt{*p = temp;}
        \end{aligned}\end{cases} \\[15pt]
        \text{Memory} & \begin{aligned} 
          \hphantom{\begin{cases}\end{cases}}\text{Register} &&& \texttt{movq (\%rax), \%rdx} && \texttt{temp = *p;}
        \end{aligned}
    \end{cases}
  \]
  \begin{note}{}
    You \emph{cannot} do memory-memory transfer with a single instruction.
  \end{note}
  \subsection{Memory Addressing Modes}
  The first two methods below are simple, and the final the most general form of memory addressing.
  \begin{itemize}
    \item Normal: The register \texttt{R} specifies the memory address, and \texttt{(R) = Mem[Reg[R]]} dereferences the pointer. \\
    For example, we have \texttt{movq (\%rcx), \%rax}.
    \item Displacement: The register \texttt{R} specifies the start of the memory region, and the constant displacement \texttt{D} specifies the offset. The syntax is \texttt{D(R) = Mem[Reg[R] + D]}. \\
    For example, we have \texttt{movq 8(\%rbp), \%rdx}.
    \item Most General Form: \texttt{D(Rb, Ri, S)} which translates to \texttt{Mem[Reg[Rb] + S*Reg[Ri] + D]}.
    \begin{itemize}
      \item \texttt{D} is a constant ``displacement'' of $1$, $2$, or $4$ bytes.
      \item \texttt{Rb} is the base register (any of the $16$ integer registers).
      \item \texttt{Ri} is the index register, which can be any register \emph{but} \texttt{\%rsp}.
      \item \texttt{S} is the scale, which is $1$, $2$, $4$, or $8$.
    \end{itemize}
  \end{itemize}
  \section{Arithmetic and Logical Operations}
  \subsection{Address Computation Instruction}
  The \texttt{leaq src, dst} command takes in two parameters:
  \begin{itemize}
    \item \texttt{src} is an address mode expression that we just covered.
    \item \texttt{dst} is the address denoted by that expression.
  \end{itemize}
  \begin{note}{}
    You are essentially computing the address referred to by \texttt{src}, and then setting \texttt{dst} equal to that address. You are \emph{not} dereferencing anything when using \texttt{leaq}.
  \end{note}
  This has several use cases, such as computing addresses without a memory reference (i.e. translation of \texttt{p = \&x[i];}) or computing arithmetic expressions of the form \texttt{x + k*y}, where \texttt{k} is a $1$, $2$, $4$, or $8$ byte value.
  \begin{note}{}
    If the command does not specify the size, i.e. just a \texttt{mov} command, you can always look at the registers that it is operating on.
  \end{note}
  \begin{center}\begin{tabular}{c|c|c|c|c}
    Register & \texttt{\%al} & \texttt{\%ax} & \texttt{\%eax} & \texttt{\%rax} \\
    \hline
    Syntax Example & \texttt{movb} & \texttt{movw} & \texttt{movl} & \texttt{movq} \\
    \hline
    Space & $8$ bits & $16$ bits & $32$ bits & $64$ bits
  \end{tabular}\end{center}
  You can actually have multiple extensions after a command, i.e. \texttt{movzbl}. The first letter after the \texttt{mov} is either a \texttt{z} or a \texttt{s}, indicating whether we are zero extending or sign extending. The other two letters indicate what size we are moving from and to. For instance, \texttt{movzbl} means we are moving from a $8$ bit register to a $32$ bit quantity, and filling in the gaps with zeroes. Sign extension will duplicate the MSB instead of padding the data with zeroes.
  \subsection{Some Arithmetic Operations}
  \begin{itemize}
    \item Two Operand Instructions
    \begin{center}\begin{tabular}{lll}
      \texttt{addq} & Dest = Dest + Src \\
      \texttt{subq} & Dest = Dest - Src \\
      \texttt{imulq} & Dest = Dest * Src \\
      \texttt{salq} & Dest = Dest $\ll$ Src & Also called \texttt{shlq}\\
      \texttt{sarq} & Dest = Dest $\gg$ Src & Arithmetic Shift\\
      \texttt{shrq} & Dest = Dest $\gg$ Src & Logical Shift\\
      \texttt{xorq} & Dest = Dest \^{} Src \\
      \texttt{andq} & Dest = Dest \& Src \\
      \texttt{orq} & Dest = Dest $\vert$ Src
    \end{tabular}\end{center}
    \item One Operand Instructions
    \begin{center}\begin{tabular}{ll}
      \texttt{incq} & Dest = Dest + 1 \\
      \texttt{decq} & Dest = Dest $-$ 1 \\
      \texttt{negq} & Dest = $-$Dest \\
      \texttt{notq} & Dest = \tilde Dest
    \end{tabular}\end{center}
  \end{itemize}
  \begin{note}{}
    There is no distinction between signed and unsigned int because these are all bit-level manipulations.
  \end{note}
\end{document}
