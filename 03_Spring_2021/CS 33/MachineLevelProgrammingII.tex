\documentclass[class=article, crop=false]{standalone}
% Import packages
\usepackage[margin=1in]{geometry}

\usepackage{amssymb, amsthm}
\usepackage{comment}
\usepackage{enumitem}
\usepackage{fancyhdr}
\usepackage{hyperref}
\usepackage{import}
\usepackage{mathrsfs, mathtools}
\usepackage{multicol}
\usepackage{pdfpages}
\usepackage{standalone}
\usepackage{transparent}
\usepackage{xcolor}

\usepackage{minted}
\usepackage[many]{tcolorbox}
\tcbuselibrary{minted}

% Declare math operators
\DeclareMathOperator{\lcm}{lcm}
\DeclareMathOperator{\proj}{proj}
\DeclareMathOperator{\vspan}{span}
\DeclareMathOperator{\im}{im}
\DeclareMathOperator{\Diff}{Diff}
\DeclareMathOperator{\Int}{Int}
\DeclareMathOperator{\fcn}{fcn}
\DeclareMathOperator{\id}{id}
\DeclareMathOperator{\rank}{rank}
\DeclareMathOperator{\tr}{tr}
\DeclareMathOperator{\dive}{div}
\DeclareMathOperator{\row}{row}
\DeclareMathOperator{\col}{col}
\DeclareMathOperator{\dom}{dom}
\DeclareMathOperator{\ran}{ran}
\DeclareMathOperator{\Dom}{Dom}
\DeclareMathOperator{\Ran}{Ran}
\DeclareMathOperator{\fl}{fl}
% Macros for letters/variables
\newcommand\ttilde{\kern -.15em\lower .7ex\hbox{\~{}}\kern .04em}
\newcommand{\N}{\ensuremath{\mathbb{N}}}
\newcommand{\Z}{\ensuremath{\mathbb{Z}}}
\newcommand{\Q}{\ensuremath{\mathbb{Q}}}
\newcommand{\R}{\ensuremath{\mathbb{R}}}
\newcommand{\C}{\ensuremath{\mathbb{C}}}
\newcommand{\F}{\ensuremath{\mathbb{F}}}
\newcommand{\M}{\ensuremath{\mathbb{M}}}
\renewcommand{\P}{\ensuremath{\mathbb{P}}}
\newcommand{\lam}{\ensuremath{\lambda}}
\newcommand{\nab}{\ensuremath{\nabla}}
\newcommand{\eps}{\ensuremath{\varepsilon}}
\newcommand{\es}{\ensuremath{\varnothing}}
% Macros for math symbols
\newcommand{\dx}[1]{\,\mathrm{d}#1}
\newcommand{\inv}{\ensuremath{^{-1}}}
\newcommand{\sm}{\setminus}
\newcommand{\sse}{\subseteq}
\newcommand{\ceq}{\coloneqq}
% Macros for pairs of math symbols
\newcommand{\abs}[1]{\ensuremath{\left\lvert #1 \right\rvert}}
\newcommand{\paren}[1]{\ensuremath{\left( #1 \right)}}
\newcommand{\norm}[1]{\ensuremath{\left\lVert #1\right\rVert}}
\newcommand{\set}[1]{\ensuremath{\left\{#1\right\}}}
\newcommand{\tup}[1]{\ensuremath{\left\langle #1 \right\rangle}}
\newcommand{\floor}[1]{\ensuremath{\left\lfloor #1 \right\rfloor}}
\newcommand{\ceil}[1]{\ensuremath{\left\lceil #1 \right\rceil}}
\newcommand{\eclass}[1]{\ensuremath{\left[ #1 \right]}}

\newcommand{\chapternum}{}
\newcommand{\ex}[1]{\noindent\textbf{Exercise \chapternum.{#1}.}}

\newcommand{\tsub}[1]{\textsubscript{#1}}
\newcommand{\tsup}[1]{\textsuperscript{#1}}

\renewcommand{\choose}[2]{\ensuremath{\phantom{}_{#1}C_{#2}}}
\newcommand{\perm}[2]{\ensuremath{\phantom{}_{#1}P_{#2}}}

% Include figures
\newcommand{\incfig}[2][1]{%
    \def\svgwidth{#1\columnwidth}
    \import{./figures/}{#2.pdf_tex}
}

\definecolor{problemBackground}{RGB}{212,232,246}
\newenvironment{problem}[1]
  {
    \begin{tcolorbox}[
      boxrule=.5pt,
      titlerule=.5pt,
      sharp corners,
      colback=problemBackground,
      breakable
    ]
    \ifx &#1& \textbf{Problem. }
    \else \textbf{Problem #1.} \fi
  }
  {
    \end{tcolorbox}
  }
\definecolor{exampleBackground}{RGB}{255,249,248}
\definecolor{exampleAccent}{RGB}{158,60,14}
\newenvironment{example}[1]
  {
    \begin{tcolorbox}[
      boxrule=.5pt,
      sharp corners,
      colback=exampleBackground,
      colframe=exampleAccent,
    ]
    \color{exampleAccent}\textbf{Example.} \emph{#1}\color{black}
  }
  {
    \end{tcolorbox}
  }
\definecolor{theoremBackground}{RGB}{234,243,251}
\definecolor{theoremAccent}{RGB}{0,116,183}
\newenvironment{theorem}[1]
  {
    \begin{tcolorbox}[
      boxrule=.5pt,
      titlerule=.5pt,
      sharp corners,
      colback=theoremBackground,
      colframe=theoremAccent,
      breakable
    ]
      % \color{theoremAccent}\textbf{Theorem --- }\emph{#1}\\\color{black}
    \ifx &#1& \color{theoremAccent}\textbf{Theorem. }\color{black}
    \else \color{theoremAccent}\textbf{Theorem --- }\emph{#1}\\\color{black} \fi
  }
  {
    \end{tcolorbox}
  }
\definecolor{noteBackground}{RGB}{244,249,244}
\definecolor{noteAccent}{RGB}{34,139,34}
\newenvironment{note}[1]
  {
  \begin{tcolorbox}[
    enhanced,
    boxrule=0pt,
    frame hidden,
    sharp corners,
    colback=noteBackground,
    borderline west={3pt}{-1.5pt}{noteAccent},
    breakable
    ]
    \ifx &#1& \color{noteAccent}\textbf{Note. }\color{black}
    \else \color{noteAccent}\textbf{Note (#1). }\color{black} \fi
    }
    {
  \end{tcolorbox}
  }
\definecolor{lemmaBackground}{RGB}{255,247,234}
\definecolor{lemmaAccent}{RGB}{255,153,0}
\newenvironment{lemma}[1]
  {
    \begin{tcolorbox}[
      enhanced,
      boxrule=0pt,
      frame hidden,
      sharp corners,
      colback=lemmaBackground,
      borderline west={3pt}{-1.5pt}{lemmaAccent},
      breakable
    ]
    \ifx &#1& \color{lemmaAccent}\textbf{Lemma. }\color{black}
    \else \color{lemmaAccent}\textbf{Lemma #1. }\color{black} \fi
  }
  {
    \end{tcolorbox}
  }
\definecolor{definitionBackground}{RGB}{246,246,246}
\newenvironment{definition}[1]
  {
    \begin{tcolorbox}[
      enhanced,
      boxrule=0pt,
      frame hidden,
      sharp corners,
      colback=definitionBackground,
      borderline west={3pt}{-1.5pt}{black},
      breakable
    ]
    \textbf{Definition. }\emph{#1}\\
  }
  {
    \end{tcolorbox}
  }

\newenvironment{amatrix}[2]{
    \left[
      \begin{array}{*{#1}{c}|*{#2}c}
  }
  {
      \end{array}
    \right]
  }

\definecolor{codeBackground}{RGB}{253,246,227}
\renewcommand\theFancyVerbLine{\arabic{FancyVerbLine}}
\newtcblisting{cpp}[1][]{
  boxrule=1pt,
  sharp corners,
  colback=codeBackground,
  listing only,
  breakable,
  minted language=cpp,
  minted style=material,
  minted options={
    xleftmargin=15pt,
    baselinestretch=1.2,
    linenos,
  }
}

\author{Kyle Chui}


\fancyhf{}
\lhead{Kyle Chui}
\rhead{Page \thepage}
\pagestyle{fancy}

\begin{document}
  \section{Machine-Level Programming II: Control}
  \subsection{Control: Condition Codes}
  \subsubsection{Processor State (x86-64, Partial)}
  The information about the currently executing program is stored in the $16$ integer registers.
  \begin{itemize}
    \item Temporary data can be stored in most of the registers.
    \item The location of the runtime stack is stored in \texttt{\%rsp}.
    \item The location of the current code control point (what command is currently being executed) is stored inside the \texttt{\%rip} register.
    \item The status of the recent tests (condition codes)
    \begin{itemize}
      \item CF---Carry Flag
      \item ZF---Zero Flag
      \item SF---Sign Flag
      \item OF---Overflow Flag
    \end{itemize}
    These four condition codes allow us to create and make more complex comparison operators.
  \end{itemize}
  \subsection{Implicit Setting of Condition Codes}
  Condition codes can be implicitly set by arithmetic operations (you can think of this as a side effect of the operations).
  \begin{itemize}
    \item CF is set if there is a carry out from the most significant bit (unsigned overflow).
    \item ZF is set if the result of the operation is zero.
    \item SF is set if the result is less than zero (as a signed integer).
    \item OF is set if there is signed overflow in the MSB.
  \end{itemize}
  The condition codes are not set by the \texttt{leaq} instruction.
  \subsection{Explicit Setting of Condition Codes}
  \begin{itemize}
    \item Explicit setting by compare instruction
    \item Explicit setting by test instruction
  \end{itemize}
  Neither of the above write to a destination. The zero flag is set when \texttt{a\&b == 0}, and the signed flag is set when \texttt{a\&b < 0}.
  \newpage
  \subsection{Reading Condition Codes}
  The \texttt{setX} sets the lowest-order byte of the destination to a zero $r$ one based on combinations of the other condition codes, and does not change the other $7$ bytes. 
  \begin{center}\begin{tabular}{c|c|c}
    SetX & Condition & Description \\
    \hline
    \texttt{sete} & \texttt{ZF} & Equal/Zero \\
    \texttt{setne} & \texttt{\tilde ZF} & Not Equal/Not Zero \\
    \texttt{sets} & \texttt{SF} & Negative \\
    \texttt{setns} & \texttt{\tilde SF} & Non-negative \\
    \texttt{setg} & \texttt{\tilde (SF\^{}OF)\&\tilde ZF} & Greater (Signed) \\
    \texttt{setge} & \texttt{\tilde (SF\^{}OF)} & Greater or Equal (Signed) \\
    \texttt{setl} & \texttt{(SF\^{}OF)} & Less (Signed) \\
    \texttt{setle} & \texttt{(SF\^{}OF) | ZF} & Less or Equal (Signed) \\
    \texttt{seta} & \texttt{\tilde CF\&\tilde ZF} & Above (unsigned) \\
    \texttt{setb} & \texttt{CF} & Below (unsigned) \\
  \end{tabular}\end{center}
  \begin{note}{}
    There are different \texttt{setX} commands for comparing signed vs unsigned integers.
  \end{note}
  You can access to lowest-order byte in an x86-64 register:
  \begin{itemize}
    \item For the first four registers, remove the \texttt{r} and the \texttt{x}, and append an \texttt{l}, i.e. \texttt{\%rax} $\to$ \texttt{\%al}.
    \item For the next four registers, remove the \texttt{r} and append an \texttt{l}, i.e. \texttt{\%rsp} $\to$ \texttt{\%spl}.
    \item For the final eight registers, just append a \texttt{b}, i.e. \texttt{\%r14} $\to$ \texttt{\%r14b}.
  \end{itemize}
  Because the \texttt{setX} commands only ever modify the lowest-order bit, these commands usually need to be combined with other commands. For instance, the \texttt{movzbl} command will move from one register to another, but if the first is smaller, pad the remaining bits with zeroes. For instance, \texttt{movzbl \%al, \%eax} will pad the other bits of \texttt{\%rax} with zeroes.
\end{document}
