\documentclass[class=article, crop=false]{standalone}
\usepackage[margin=1in]{geometry}

\usepackage{amsmath}
\usepackage{amssymb}
\usepackage{amsthm}
\usepackage{comment}
\usepackage{enumitem}
\usepackage{fancyhdr}
\usepackage{hyperref}
\usepackage{mathrsfs}
\usepackage{mathtools}
\usepackage{standalone}
\usepackage{tcolorbox}
\usepackage{tikz}
\usepackage{xcolor}

\usetikzlibrary{decorations.pathreplacing}
\tcbuselibrary{skins}

\DeclareMathOperator{\lcm}{lcm}
\DeclareMathOperator{\proj}{proj}
\DeclareMathOperator{\vspan}{span}
\DeclareMathOperator{\im}{im}
\DeclareMathOperator{\range}{range}
\DeclareMathOperator{\Diff}{Diff}
\DeclareMathOperator{\Int}{Int}
\DeclareMathOperator{\fcn}{fcn}
\DeclareMathOperator{\id}{id}

\newcommand{\N}{\ensuremath{\mathbb{N}}}
\newcommand{\Z}{\ensuremath{\mathbb{Z}}}
\newcommand{\Q}{\ensuremath{\mathbb{Q}}}
\newcommand{\R}{\ensuremath{\mathbb{R}}}
\newcommand{\C}{\ensuremath{\mathbb{C}}}
\newcommand{\F}{\ensuremath{\mathbb{F}}}
\newcommand{\lam}{\ensuremath{\lambda}}
\newcommand{\nab}{\ensuremath{\nabla}}
\newcommand{\eps}{\ensuremath{\varepsilon}}
\newcommand{\es}{\ensuremath{\varnothing}}

\newcommand{\abs}[1]{\ensuremath{\left\lvert #1 \right\rvert}}
\newcommand{\bigpar}[1]{\ensuremath{\left( #1 \right)}}
\newcommand{\norm}[1]{\ensuremath{\lVert #1\rVert}}
\newcommand{\set}[1]{\ensuremath{\left\{#1\right\}}}
\newcommand{\tuple}[1]{\ensuremath{\left\langle #1 \right\rangle}}
\newcommand{\floor}[1]{\ensuremath{\left\lfloor #1 \right\rfloor}}
\newcommand{\ceil}[1]{\ensuremath{\left\lceil #1 \right\rceil}}

\newcommand{\chapternum}{}
\newcommand{\ex}[1]{\noindent\textbf{Exercise \chapternum.{#1}.}}

\newcommand{\dx}[1]{\,\mathrm{d}#1}
\newcommand{\inv}{\ensuremath{^{-1}}}
\newcommand{\sm}{\setminus}
\newcommand{\sse}{\subseteq}
\newcommand{\ceq}{\coloneqq}

\definecolor{problemBackground}{RGB}{212,232,246}
\newenvironment{problem}[1]
  {
    \begin{tcolorbox}[
      boxrule=.5pt,
      titlerule=.5pt,
      sharp corners,
      colback=problemBackground
    ]
    \ifx &#1& \textbf{Problem. }
    \else \textbf{Problem #1.} \fi
  }
  {
    \end{tcolorbox}
  }

\definecolor{exampleBackground}{RGB}{255,249,248}
\definecolor{exampleAccent}{RGB}{158,60,14}
\newenvironment{example}[1]
  {
    \begin{tcolorbox}[
      boxrule=.5pt,
      sharp corners,
      colback=exampleBackground,
      colframe=exampleAccent,
    ]
    \color{exampleAccent}\textbf{Example.} \emph{#1}\color{black}
  }
  {
    \end{tcolorbox}
  }

\definecolor{theoremBackground}{RGB}{234,243,251}
\definecolor{theoremAccent}{RGB}{0,116,183}
\newenvironment{theorem}[1]
  {
    \begin{tcolorbox}[
      boxrule=.5pt,
      titlerule=.5pt,
      sharp corners,
      colback=theoremBackground,
      colframe=theoremAccent
    ]
      \color{theoremAccent}\textbf{Theorem --- }\emph{#1}\\\color{black}
  }
  {
    \end{tcolorbox}
  }

\definecolor{noteBackground}{RGB}{244,249,244}
\definecolor{noteAccent}{RGB}{34,139,34}
\newenvironment{note}[1]
  {
  \begin{tcolorbox}[
    enhanced,
    boxrule=0pt,
    frame hidden,
    sharp corners,
    colback=noteBackground,
    borderline west={3pt}{-1.5pt}{noteAccent}
    ]
    \ifx &#1& \color{noteAccent}\textbf{Note. }\color{black}
    \else \color{noteAccent}\textbf{Note (#1). }\color{black} \fi
    }
    {
  \end{tcolorbox}
  }

\definecolor{definitionBackground}{RGB}{246,246,246}
\newenvironment{definition}[1]
  {
    \begin{tcolorbox}[
      enhanced,
      boxrule=0pt,
      frame hidden,
      sharp corners,
      colback=definitionBackground,
      borderline west={3pt}{-1.5pt}{black}
    ]
    \textbf{Definition. }\emph{#1}\\
  }
  {
    \end{tcolorbox}
  }
\newenvironment{amatrix}[2]{
    \left[
      \begin{array}{*{#1}{c}|*{#2}c}
  }
  {
      \end{array}
    \right]
  }
\date{\the\year-\the\month-\the\day}
\author{Kyle Chui}


\fancyhf{}
\lhead{Kyle Chui}
\rhead{Page \thepage}
\pagestyle{fancy}

\begin{document}
  \section{Lecture 23}
  We now generalize the Gaussian quadrature from the interval $[-1, 1]$ to $[a, b]$. In other words, we want to solve for
  \[
    \int_{a}^{b}f(x) \,\mathrm dx = \int_{-1}^{1}g(t) \,\mathrm dt.
  \]
  Observe that the linear map $t = \frac{x - \frac{b + a}{2}}{\frac{b - a}{2}} = \frac{2x - b - a}{b - a}$ suffices. Solving for $x$ yields $x = \frac{b - a}{2}t + \frac{b + a}{2}$. Thus we may rewrite our integral as
  \[
    \int_{a}^{b}f(x) \,\mathrm dx = \int_{-1}^{1}\frac{b - a}{2}f\paren{\frac{b - a}{2}t + \frac{b + a}{2}} \,\mathrm dt.
  \]
  \subsection{Direct Methods for Solving Systems of Linear Equations}
  Consider the following system:
  \begin{align*}
    x_{11}x_1 + a_{12}x_2 + \dotsb + a_{1n}x_n &= b_1 \\
    x_{21}x_1 + a_{22}x_2 + \dotsb + a_{2n}x_n &= b_2 \\
    x_{31}x_1 + a_{32}x_2 + \dotsb + a_{3n}x_n &= b_3 \\
                                               &\vdots \\
    x_{n1}x_1 + a_{n2}x_2 + \dotsb + a_{nn}x_n &= b_n
  \end{align*}
  We can express it as a matrix of the form
  \[
    A = \begin{bmatrix}
      a_{11} & a_{12} & \dotsb & a_{1n} \\
      a_{21} & a_{22} & \dotsb & a_{2n} \\
      \vdots & \vdots & \ddots & \vdots \\
      a_{n_1} & a_{n_2} & \dotsb & a_{nn}
    \end{bmatrix}
  \]
  We have a few methods to yield the solutions:
  \begin{itemize}
    \item Direct methods: Give exact solutions in a finite number of steps
    \item Iterative methods: Start with an initial guess $x^0$ and produce approximations $x^1,\dotsc,x^n$ which will converge to $x^*$
  \end{itemize}
  \subsubsection{Linear Systems of Equations and Gaussian Elimination}
  The idea is we transform our equation $Ax = b$ into the equation $\tilde Ax = \tilde b$, which has the same solutions but is easier to solve. We have a few \emph{elementary row operations} that we can perform:
  \begin{itemize}
    \item $\lam E_i\to E_i$ for $\lam\neq 0$
    \item $E_i + \lam E_j\to E_i$
    \item $E_j \leftrightarrow E_i$
  \end{itemize}
  Once we get our system of equations into an upper-triangular matrix, we can use backwards-substitution to solve for $x_j$ for $j = 1,\dotsc,n$.
\end{document}
