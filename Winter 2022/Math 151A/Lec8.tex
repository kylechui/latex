\documentclass[class=article, crop=false]{standalone}
\usepackage[margin=1in]{geometry}

\usepackage{amsmath}
\usepackage{amssymb}
\usepackage{amsthm}
\usepackage{comment}
\usepackage{enumitem}
\usepackage{fancyhdr}
\usepackage{hyperref}
\usepackage{mathrsfs}
\usepackage{mathtools}
\usepackage{standalone}
\usepackage{tcolorbox}
\usepackage{tikz}
\usepackage{xcolor}

\usetikzlibrary{decorations.pathreplacing}
\tcbuselibrary{skins}

\DeclareMathOperator{\lcm}{lcm}
\DeclareMathOperator{\proj}{proj}
\DeclareMathOperator{\vspan}{span}
\DeclareMathOperator{\im}{im}
\DeclareMathOperator{\range}{range}
\DeclareMathOperator{\Diff}{Diff}
\DeclareMathOperator{\Int}{Int}
\DeclareMathOperator{\fcn}{fcn}
\DeclareMathOperator{\id}{id}

\newcommand{\N}{\ensuremath{\mathbb{N}}}
\newcommand{\Z}{\ensuremath{\mathbb{Z}}}
\newcommand{\Q}{\ensuremath{\mathbb{Q}}}
\newcommand{\R}{\ensuremath{\mathbb{R}}}
\newcommand{\C}{\ensuremath{\mathbb{C}}}
\newcommand{\F}{\ensuremath{\mathbb{F}}}
\newcommand{\lam}{\ensuremath{\lambda}}
\newcommand{\nab}{\ensuremath{\nabla}}
\newcommand{\eps}{\ensuremath{\varepsilon}}
\newcommand{\es}{\ensuremath{\varnothing}}

\newcommand{\abs}[1]{\ensuremath{\left\lvert #1 \right\rvert}}
\newcommand{\bigpar}[1]{\ensuremath{\left( #1 \right)}}
\newcommand{\norm}[1]{\ensuremath{\lVert #1\rVert}}
\newcommand{\set}[1]{\ensuremath{\left\{#1\right\}}}
\newcommand{\tuple}[1]{\ensuremath{\left\langle #1 \right\rangle}}
\newcommand{\floor}[1]{\ensuremath{\left\lfloor #1 \right\rfloor}}
\newcommand{\ceil}[1]{\ensuremath{\left\lceil #1 \right\rceil}}

\newcommand{\chapternum}{}
\newcommand{\ex}[1]{\noindent\textbf{Exercise \chapternum.{#1}.}}

\newcommand{\dx}[1]{\,\mathrm{d}#1}
\newcommand{\inv}{\ensuremath{^{-1}}}
\newcommand{\sm}{\setminus}
\newcommand{\sse}{\subseteq}
\newcommand{\ceq}{\coloneqq}

\definecolor{problemBackground}{RGB}{212,232,246}
\newenvironment{problem}[1]
  {
    \begin{tcolorbox}[
      boxrule=.5pt,
      titlerule=.5pt,
      sharp corners,
      colback=problemBackground
    ]
    \ifx &#1& \textbf{Problem. }
    \else \textbf{Problem #1.} \fi
  }
  {
    \end{tcolorbox}
  }

\definecolor{exampleBackground}{RGB}{255,249,248}
\definecolor{exampleAccent}{RGB}{158,60,14}
\newenvironment{example}[1]
  {
    \begin{tcolorbox}[
      boxrule=.5pt,
      sharp corners,
      colback=exampleBackground,
      colframe=exampleAccent,
    ]
    \color{exampleAccent}\textbf{Example.} \emph{#1}\color{black}
  }
  {
    \end{tcolorbox}
  }

\definecolor{theoremBackground}{RGB}{234,243,251}
\definecolor{theoremAccent}{RGB}{0,116,183}
\newenvironment{theorem}[1]
  {
    \begin{tcolorbox}[
      boxrule=.5pt,
      titlerule=.5pt,
      sharp corners,
      colback=theoremBackground,
      colframe=theoremAccent
    ]
      \color{theoremAccent}\textbf{Theorem --- }\emph{#1}\\\color{black}
  }
  {
    \end{tcolorbox}
  }

\definecolor{noteBackground}{RGB}{244,249,244}
\definecolor{noteAccent}{RGB}{34,139,34}
\newenvironment{note}[1]
  {
  \begin{tcolorbox}[
    enhanced,
    boxrule=0pt,
    frame hidden,
    sharp corners,
    colback=noteBackground,
    borderline west={3pt}{-1.5pt}{noteAccent}
    ]
    \ifx &#1& \color{noteAccent}\textbf{Note. }\color{black}
    \else \color{noteAccent}\textbf{Note (#1). }\color{black} \fi
    }
    {
  \end{tcolorbox}
  }

\definecolor{definitionBackground}{RGB}{246,246,246}
\newenvironment{definition}[1]
  {
    \begin{tcolorbox}[
      enhanced,
      boxrule=0pt,
      frame hidden,
      sharp corners,
      colback=definitionBackground,
      borderline west={3pt}{-1.5pt}{black}
    ]
    \textbf{Definition. }\emph{#1}\\
  }
  {
    \end{tcolorbox}
  }
\newenvironment{amatrix}[2]{
    \left[
      \begin{array}{*{#1}{c}|*{#2}c}
  }
  {
      \end{array}
    \right]
  }
\date{\the\year-\the\month-\the\day}
\author{Kyle Chui}


\fancyhf{}
\lhead{Kyle Chui}
\rhead{Page \thepage}
\pagestyle{fancy}

\begin{document}
  \section{Lecture 8}
  \subsection{Newton's Method}
  \begin{itemize}
    \item Choose an initial approximation $p_0$ of $p$.
    \item Set $p_n$ to be the $x$-intercept of the tangent line passing through $(p_{n - 1}, f(p_{n - 1}))$. If we wish to solve this equation we have:
    \begin{align*}
      y - f(p_{n - 1}) &= f'(p_{n - 1})\cdot (x - p_{n - 1}) \\
      x &= p_{n - 1} - \frac{f(p_{n - 1})}{f'(p_{n - 1})}.
    \end{align*}
    We can only use this for $f'(p_{n - 1})\neq 0$, since if it were then the tangent line would have no $x$-intercept. We can treat this method as a form of a fixed point method, with $g(x) = x - \frac{f(x)}{f'(x)}$.
  \end{itemize}
  \begin{note}{}
    Newton's method converges faster than the regular fixed point method when $p_0\approx p$.
  \end{note}
  \begin{theorem}{Convergence of Newton's Method}
    Let $f(x)\in C^2[a, b]$. If $p\in (a, b)$ such that $f(p) = 0$ and $f'(p)\neq 0$, then there exists a $\delta > 0$ such that the Newton's method generates a sequence $\set{p_n}_{n = 1}^\infty$ which converges to $p$ for any initial approximation $p_0\in [p - \delta, p + \delta]$.
  \end{theorem}
  \begin{proof}
    Consider Newton's method to be a fixed-point method with
    \[
      g(x) = x - \frac{f(x)}{f'(x)}.
    \]
    We wish to show:
    \begin{enumerate}[label=(\roman*)]
      \item $g(x)$ exists on $(p - \delta, p + \delta)$.
      \item $g(x)$ exists on $(p - \delta, p + \delta)$ and there exists $0 < k < 1$ such that $\abs{g'(x) \leq k}$.
      \item $g(x)\in [p - \delta, p + \delta]$ for $x\in [p - \delta, p + \delta]$.
    \end{enumerate}
    Since $f(x)\in C^2[a, b]$, we know that $f$, $f'$, and $f''$ are continuous on $[a, b]$. Since $f'(p)\neq 0$ and $f'$ continuous, there exists $\delta_1 > 0$ such that $f'(x)\neq 0$ for all $x\in [p - \delta_1, p + \delta_1]$. Hence $g(x)$ is continuous on $[p - \delta_1, p + \delta_1]$. \\
    Observe that
    \[
      g'(x) = 1 - \frac{(f'(x))^2 - f(x)f''(x)}{(f'(x))^2} = \frac{f(x)\cdot f''(x)}{(f'(x))^2}.
    \]
    Since $f$, $f'$, and $f''$ are continuous on $[p - \delta_1, p + \delta_1]$ and $f'(x)\neq 0$ on $[p - \delta_1, p + \delta_1]$, we have that $g'(x)$ is continuous on $[p - \delta_1, p + \delta_1]$. Hence
    \[
      g'(p) = \frac{f(p)f''(p)}{(f'(p))^2} = 0,
    \]
    so there must be some small interval $0 < \delta < \delta_1$ such that $\abs{g'(x)} \leq k$ where $0 < k < 1$. \par
    We must finally show that $g(x)\in [p - \delta, p + \delta]$ when $x\in [p - \delta, p + \delta]$. Let $x\in [p - \delta, p + \delta]$, so
    \begin{align*}
      \abs{g(x) - p} &= \abs{g(x) - g(p)} \\
                     &= \abs{g'(\xi)\cdot (x - p)} \tag{MVT} \\
                     &= \abs{g'(\xi)} \abs{x - p} \\
                     &< \abs{x - p} \\
                     &\leq \delta.
    \end{align*}
    Thus $g(x)\in (p - \delta, p + \delta)$. \\
    Therefore there exists some $\delta > 0$ such that the above three statements hold, and so by the convergence of a fixed point iteration we have that any $p_0\in [p - \delta, p + \delta]$ generates a sequence that converges to $p$.
  \end{proof}
  \begin{note}{}
    This theorem only tells us the \emph{existence} of such a $\delta$, and doesn't tell us anything about $\delta$. In practice, we usually run a few bisection iterations to get closer to the actual root (since it will converge towards the root), and then switch over to Newton's method.
  \end{note}
\end{document}
