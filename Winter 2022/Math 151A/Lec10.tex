\documentclass[class=article, crop=false]{standalone}
% Import packages
\usepackage[margin=1in]{geometry}

\usepackage{amssymb, amsthm}
\usepackage{comment}
\usepackage{enumitem}
\usepackage{fancyhdr}
\usepackage{hyperref}
\usepackage{import}
\usepackage{mathrsfs, mathtools}
\usepackage{multicol}
\usepackage{pdfpages}
\usepackage{standalone}
\usepackage{transparent}
\usepackage{xcolor}

\usepackage{minted}
\usepackage[many]{tcolorbox}
\tcbuselibrary{minted}

% Declare math operators
\DeclareMathOperator{\lcm}{lcm}
\DeclareMathOperator{\proj}{proj}
\DeclareMathOperator{\vspan}{span}
\DeclareMathOperator{\im}{im}
\DeclareMathOperator{\Diff}{Diff}
\DeclareMathOperator{\Int}{Int}
\DeclareMathOperator{\fcn}{fcn}
\DeclareMathOperator{\id}{id}
\DeclareMathOperator{\rank}{rank}
\DeclareMathOperator{\tr}{tr}
\DeclareMathOperator{\dive}{div}
\DeclareMathOperator{\row}{row}
\DeclareMathOperator{\col}{col}
\DeclareMathOperator{\dom}{dom}
\DeclareMathOperator{\ran}{ran}
\DeclareMathOperator{\Dom}{Dom}
\DeclareMathOperator{\Ran}{Ran}
\DeclareMathOperator{\fl}{fl}
% Macros for letters/variables
\newcommand\ttilde{\kern -.15em\lower .7ex\hbox{\~{}}\kern .04em}
\newcommand{\N}{\ensuremath{\mathbb{N}}}
\newcommand{\Z}{\ensuremath{\mathbb{Z}}}
\newcommand{\Q}{\ensuremath{\mathbb{Q}}}
\newcommand{\R}{\ensuremath{\mathbb{R}}}
\newcommand{\C}{\ensuremath{\mathbb{C}}}
\newcommand{\F}{\ensuremath{\mathbb{F}}}
\newcommand{\M}{\ensuremath{\mathbb{M}}}
\renewcommand{\P}{\ensuremath{\mathbb{P}}}
\newcommand{\lam}{\ensuremath{\lambda}}
\newcommand{\nab}{\ensuremath{\nabla}}
\newcommand{\eps}{\ensuremath{\varepsilon}}
\newcommand{\es}{\ensuremath{\varnothing}}
% Macros for math symbols
\newcommand{\dx}[1]{\,\mathrm{d}#1}
\newcommand{\inv}{\ensuremath{^{-1}}}
\newcommand{\sm}{\setminus}
\newcommand{\sse}{\subseteq}
\newcommand{\ceq}{\coloneqq}
% Macros for pairs of math symbols
\newcommand{\abs}[1]{\ensuremath{\left\lvert #1 \right\rvert}}
\newcommand{\paren}[1]{\ensuremath{\left( #1 \right)}}
\newcommand{\norm}[1]{\ensuremath{\left\lVert #1\right\rVert}}
\newcommand{\set}[1]{\ensuremath{\left\{#1\right\}}}
\newcommand{\tup}[1]{\ensuremath{\left\langle #1 \right\rangle}}
\newcommand{\floor}[1]{\ensuremath{\left\lfloor #1 \right\rfloor}}
\newcommand{\ceil}[1]{\ensuremath{\left\lceil #1 \right\rceil}}
\newcommand{\eclass}[1]{\ensuremath{\left[ #1 \right]}}

\newcommand{\chapternum}{}
\newcommand{\ex}[1]{\noindent\textbf{Exercise \chapternum.{#1}.}}

\newcommand{\tsub}[1]{\textsubscript{#1}}
\newcommand{\tsup}[1]{\textsuperscript{#1}}

\renewcommand{\choose}[2]{\ensuremath{\phantom{}_{#1}C_{#2}}}
\newcommand{\perm}[2]{\ensuremath{\phantom{}_{#1}P_{#2}}}

% Include figures
\newcommand{\incfig}[2][1]{%
    \def\svgwidth{#1\columnwidth}
    \import{./figures/}{#2.pdf_tex}
}

\definecolor{problemBackground}{RGB}{212,232,246}
\newenvironment{problem}[1]
  {
    \begin{tcolorbox}[
      boxrule=.5pt,
      titlerule=.5pt,
      sharp corners,
      colback=problemBackground,
      breakable
    ]
    \ifx &#1& \textbf{Problem. }
    \else \textbf{Problem #1.} \fi
  }
  {
    \end{tcolorbox}
  }
\definecolor{exampleBackground}{RGB}{255,249,248}
\definecolor{exampleAccent}{RGB}{158,60,14}
\newenvironment{example}[1]
  {
    \begin{tcolorbox}[
      boxrule=.5pt,
      sharp corners,
      colback=exampleBackground,
      colframe=exampleAccent,
    ]
    \color{exampleAccent}\textbf{Example.} \emph{#1}\color{black}
  }
  {
    \end{tcolorbox}
  }
\definecolor{theoremBackground}{RGB}{234,243,251}
\definecolor{theoremAccent}{RGB}{0,116,183}
\newenvironment{theorem}[1]
  {
    \begin{tcolorbox}[
      boxrule=.5pt,
      titlerule=.5pt,
      sharp corners,
      colback=theoremBackground,
      colframe=theoremAccent,
      breakable
    ]
      % \color{theoremAccent}\textbf{Theorem --- }\emph{#1}\\\color{black}
    \ifx &#1& \color{theoremAccent}\textbf{Theorem. }\color{black}
    \else \color{theoremAccent}\textbf{Theorem --- }\emph{#1}\\\color{black} \fi
  }
  {
    \end{tcolorbox}
  }
\definecolor{noteBackground}{RGB}{244,249,244}
\definecolor{noteAccent}{RGB}{34,139,34}
\newenvironment{note}[1]
  {
  \begin{tcolorbox}[
    enhanced,
    boxrule=0pt,
    frame hidden,
    sharp corners,
    colback=noteBackground,
    borderline west={3pt}{-1.5pt}{noteAccent},
    breakable
    ]
    \ifx &#1& \color{noteAccent}\textbf{Note. }\color{black}
    \else \color{noteAccent}\textbf{Note (#1). }\color{black} \fi
    }
    {
  \end{tcolorbox}
  }
\definecolor{lemmaBackground}{RGB}{255,247,234}
\definecolor{lemmaAccent}{RGB}{255,153,0}
\newenvironment{lemma}[1]
  {
    \begin{tcolorbox}[
      enhanced,
      boxrule=0pt,
      frame hidden,
      sharp corners,
      colback=lemmaBackground,
      borderline west={3pt}{-1.5pt}{lemmaAccent},
      breakable
    ]
    \ifx &#1& \color{lemmaAccent}\textbf{Lemma. }\color{black}
    \else \color{lemmaAccent}\textbf{Lemma #1. }\color{black} \fi
  }
  {
    \end{tcolorbox}
  }
\definecolor{definitionBackground}{RGB}{246,246,246}
\newenvironment{definition}[1]
  {
    \begin{tcolorbox}[
      enhanced,
      boxrule=0pt,
      frame hidden,
      sharp corners,
      colback=definitionBackground,
      borderline west={3pt}{-1.5pt}{black},
      breakable
    ]
    \textbf{Definition. }\emph{#1}\\
  }
  {
    \end{tcolorbox}
  }

\newenvironment{amatrix}[2]{
    \left[
      \begin{array}{*{#1}{c}|*{#2}c}
  }
  {
      \end{array}
    \right]
  }

\definecolor{codeBackground}{RGB}{253,246,227}
\renewcommand\theFancyVerbLine{\arabic{FancyVerbLine}}
\newtcblisting{cpp}[1][]{
  boxrule=1pt,
  sharp corners,
  colback=codeBackground,
  listing only,
  breakable,
  minted language=cpp,
  minted style=material,
  minted options={
    xleftmargin=15pt,
    baselinestretch=1.2,
    linenos,
  }
}

\author{Kyle Chui}


\fancyhf{}
\lhead{Kyle Chui}
\rhead{Page \thepage}
\pagestyle{fancy}

\begin{document}
  \section{Lecture 10}
  For Newton's Method to work, we assume that $f(p) = 0$ and $f'(p)\neq 0$. Let us now consider the case where $f'(p) = 0$. We have
  \[
    g(p) = p - \frac{f(p)}{f'(p)},
  \]
  which is undefined because $f'(p) = 0$. If we try to compute $g'(p)$, we get
  \[
    g'(p) = \frac{f(p)f''(p)}{(f'(p))^2},
  \]
  which is also undefined. Hence in this case Newton's Method converges \emph{slower} than quadratically (specifically linearly).
  \subsection{Multiple Roots (Multiplicity)}
  \begin{definition}{Multiplicity}
    We say that $p$ is a root of $f(x)$ with \emph{multiplicity $m$} if $f(x) = (x - p)^mq(x)$ where $\lim\limits_{x\to p} q(x)\neq 0$.
  \end{definition}
  \begin{example}{}
    Consider the function $f(x) = e^x - x - 1$. We see that it has a root at $p = 0$. We write
    \begin{align*}
      f(x) &= 1 + x + \frac{x^2}{2!} + \dotsb - x - 1 \\
           &= \frac{1}{2}x^2 + \frac{1}{6}x^3 + \dotsb \\
           &= (x - 0)^2\underbrace{\paren{\frac{1}{2} + \frac{1}{6}x + \dotsb}}_{q(x)}.
    \end{align*}
    Hence we have
    \[
      \lim_{x\to 0} q(x) = \frac{1}{2}\neq 0.
    \]
    Therefore $f(x)$ has a root at 0 with multiplicity 2.
  \end{example}
  In general, if the multiplicity of a root is larger than 1, then it converges slower. What this means for Newton's method is that:
  \begin{itemize}
     \item If $f(p) = 0$ and $f'(p)\neq 0$ then the root is a single root, and so the convergence is at least quadratic.
     \item If $f(p) = 0$ and $f'(p) = 0$ then the root has multiplicity at least two, and so the convergence is at best linear.
  \end{itemize}
  \begin{theorem}{}
    The function $f(x)\in C^m[a, b]$ has a zero of multiplicity $m$ at $p$ if and only if $f(p) = f'(p) = \dotsb = f^{(m - 1)}(p) = 0$ and $f^{(m)}(p)\neq 0$.
  \end{theorem}
  \begin{theorem}{}
    If $p$ is a root of $f(x)$ with multiplicity $m$ then $p$ is a single root of
    \[
      \mu(x) = \frac{f(x)}{f'(x)}.
    \]
    \begin{proof}
      If $p$ is a root of $f(x)$ with multiplicity $m$ then we may write
      \[
        f(x) = (x - p)^m\cdot q(x),
      \]
      where $\lim\limits_{x\to p} q(x)\neq 0$. Hence
      \begin{align*}
        \mu(x) &= \frac{(x - p)^m\cdot q(x)}{(x - p)^m\cdot q'(x) + m(x - p)^{m - 1}q(x)} \\
               &= (x - p)\cdot \frac{q(x)}{(x - p)q'(x) + mq(x)}.
      \end{align*}
      Since $\lim\limits_{x\to p} q(x)\neq 0$, we have
      \[
        \lim_{x\to p} \frac{q(x)}{(x - p)q'(x) + mq(x)} = \frac{1}{m} \neq 0.
      \]
      Thus by definition $p$ is a single root of $\mu(x)$.
    \end{proof}
  \end{theorem}
  \subsection{Modified Newton's Method}
  If $p$ is a root of $f(x)$ with multiplicity $m \geq 2$, then we apply Newton's Method to
  \[
    \mu(x)\ceq \frac{f(x)}{f'(x)}.
  \]
  The new method will converge at least quadratically.
  \begin{note}{}
    The drawback of the Modified Newton's Method is that we now require the \emph{second} derivative of $f$, rather than just the first.
  \end{note}
\end{document}
