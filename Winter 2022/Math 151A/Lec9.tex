\documentclass[class=article, crop=false]{standalone}
\usepackage[margin=1in]{geometry}

\usepackage{amsmath}
\usepackage{amssymb}
\usepackage{amsthm}
\usepackage{comment}
\usepackage{enumitem}
\usepackage{fancyhdr}
\usepackage{hyperref}
\usepackage{mathrsfs}
\usepackage{mathtools}
\usepackage{standalone}
\usepackage{tcolorbox}
\usepackage{tikz}
\usepackage{xcolor}

\usetikzlibrary{decorations.pathreplacing}
\tcbuselibrary{skins}

\DeclareMathOperator{\lcm}{lcm}
\DeclareMathOperator{\proj}{proj}
\DeclareMathOperator{\vspan}{span}
\DeclareMathOperator{\im}{im}
\DeclareMathOperator{\range}{range}
\DeclareMathOperator{\Diff}{Diff}
\DeclareMathOperator{\Int}{Int}
\DeclareMathOperator{\fcn}{fcn}
\DeclareMathOperator{\id}{id}

\newcommand{\N}{\ensuremath{\mathbb{N}}}
\newcommand{\Z}{\ensuremath{\mathbb{Z}}}
\newcommand{\Q}{\ensuremath{\mathbb{Q}}}
\newcommand{\R}{\ensuremath{\mathbb{R}}}
\newcommand{\C}{\ensuremath{\mathbb{C}}}
\newcommand{\F}{\ensuremath{\mathbb{F}}}
\newcommand{\lam}{\ensuremath{\lambda}}
\newcommand{\nab}{\ensuremath{\nabla}}
\newcommand{\eps}{\ensuremath{\varepsilon}}
\newcommand{\es}{\ensuremath{\varnothing}}

\newcommand{\abs}[1]{\ensuremath{\left\lvert #1 \right\rvert}}
\newcommand{\bigpar}[1]{\ensuremath{\left( #1 \right)}}
\newcommand{\norm}[1]{\ensuremath{\lVert #1\rVert}}
\newcommand{\set}[1]{\ensuremath{\left\{#1\right\}}}
\newcommand{\tuple}[1]{\ensuremath{\left\langle #1 \right\rangle}}
\newcommand{\floor}[1]{\ensuremath{\left\lfloor #1 \right\rfloor}}
\newcommand{\ceil}[1]{\ensuremath{\left\lceil #1 \right\rceil}}

\newcommand{\chapternum}{}
\newcommand{\ex}[1]{\noindent\textbf{Exercise \chapternum.{#1}.}}

\newcommand{\dx}[1]{\,\mathrm{d}#1}
\newcommand{\inv}{\ensuremath{^{-1}}}
\newcommand{\sm}{\setminus}
\newcommand{\sse}{\subseteq}
\newcommand{\ceq}{\coloneqq}

\definecolor{problemBackground}{RGB}{212,232,246}
\newenvironment{problem}[1]
  {
    \begin{tcolorbox}[
      boxrule=.5pt,
      titlerule=.5pt,
      sharp corners,
      colback=problemBackground
    ]
    \ifx &#1& \textbf{Problem. }
    \else \textbf{Problem #1.} \fi
  }
  {
    \end{tcolorbox}
  }

\definecolor{exampleBackground}{RGB}{255,249,248}
\definecolor{exampleAccent}{RGB}{158,60,14}
\newenvironment{example}[1]
  {
    \begin{tcolorbox}[
      boxrule=.5pt,
      sharp corners,
      colback=exampleBackground,
      colframe=exampleAccent,
    ]
    \color{exampleAccent}\textbf{Example.} \emph{#1}\color{black}
  }
  {
    \end{tcolorbox}
  }

\definecolor{theoremBackground}{RGB}{234,243,251}
\definecolor{theoremAccent}{RGB}{0,116,183}
\newenvironment{theorem}[1]
  {
    \begin{tcolorbox}[
      boxrule=.5pt,
      titlerule=.5pt,
      sharp corners,
      colback=theoremBackground,
      colframe=theoremAccent
    ]
      \color{theoremAccent}\textbf{Theorem --- }\emph{#1}\\\color{black}
  }
  {
    \end{tcolorbox}
  }

\definecolor{noteBackground}{RGB}{244,249,244}
\definecolor{noteAccent}{RGB}{34,139,34}
\newenvironment{note}[1]
  {
  \begin{tcolorbox}[
    enhanced,
    boxrule=0pt,
    frame hidden,
    sharp corners,
    colback=noteBackground,
    borderline west={3pt}{-1.5pt}{noteAccent}
    ]
    \ifx &#1& \color{noteAccent}\textbf{Note. }\color{black}
    \else \color{noteAccent}\textbf{Note (#1). }\color{black} \fi
    }
    {
  \end{tcolorbox}
  }

\definecolor{definitionBackground}{RGB}{246,246,246}
\newenvironment{definition}[1]
  {
    \begin{tcolorbox}[
      enhanced,
      boxrule=0pt,
      frame hidden,
      sharp corners,
      colback=definitionBackground,
      borderline west={3pt}{-1.5pt}{black}
    ]
    \textbf{Definition. }\emph{#1}\\
  }
  {
    \end{tcolorbox}
  }
\newenvironment{amatrix}[2]{
    \left[
      \begin{array}{*{#1}{c}|*{#2}c}
  }
  {
      \end{array}
    \right]
  }
\date{\the\year-\the\month-\the\day}
\author{Kyle Chui}


\fancyhf{}
\lhead{Kyle Chui}
\rhead{Page \thepage}
\pagestyle{fancy}

\begin{document}
  \section{Lecture 9}
  \subsection{Secant Method (extension of Newton's Method)}
  From last lecture, we have
  \[
    p_n = p_{n - 1} - \frac{f(p_{n - 1})}{f'(p_{n - 1})},
  \]
  where $p_0$ is given and $n \geq 1$. However:
  \begin{itemize}
    \item $f'(x)$ might not be available
    \item $f'(x)$ might be expensive to compute
    \item $f'(p_{n - 1})\neq 0$
  \end{itemize}
  Instead, we can try approximating the tangent line using secant lines, giving us
  \begin{align*}
    f'(p_{n - 1}) &= \lim_{x\to p_{n - 1}} \frac{f(x) - f(p_{n - 1})}{x - p_{n - 1}} \\
    f'(p_{n - 1}) &\approx \frac{f(p_{n - 2}) - f(p_{n - 1})}{p_{n - 2} - p_{n - 1}}.
  \end{align*}
  Hence for the Secant Method, we replace $f'(p_{n - 1})$ with $\displaystyle\frac{f(p_{n - 2}) - f(p_{n - 1})}{p_{n - 2} - p_{n - 1}}$, yielding
  \begin{align*}
    p_n &= p_{n - 1} - \frac{f(p_{n - 1})}{f'(p_{n - 1})} \\
        &= p_{n - 1} - f(p_{n - 1})\cdot \frac{p_{n - 2} - p_{n - 1}}{f(p_{n - 2}) - f(p_{n - 1})} \\
        &= p_{n - 1} - f(p_{n - 1})\cdot \frac{p_{n - 1} - p_{n - 2}}{f(p_{n - 1}) - f(p_{n - 2})}.
  \end{align*}
  \begin{note}{}
    In order to run this algorithm, we have to have \emph{two} previous iterations, i.e. we need both $p_{n - 1}$ and $p_{n - 2}$ to run the iteration, instead of just $p_{n - 1}$ like we need for Newton's method.
  \end{note}
  \subsection{Convergence of Some Iterative Methods}
  \begin{theorem}{}
    Let $g\in C^1[a, b]$ such that $g(x)\in [a, b]$ for all $x\in [a, b]$. Then there exists $0 < k < 1$ such that $\abs{g'(x)} < k$ for all $x\in (a, b)$. If $g'(p)\neq 0$, then for any number $p_0\in [a, b]$ with $p_0\neq p$, the sequence
    \[
      p_n = g(p_{n - 1})\tag{$n\geq 1$}
    \]
    converges linearly to the unique fixed point in $[a, b]$.
    \begin{proof}
      We know that the unique fixed point $p$ exists and the sequence $\set{p_n}$ converges, as we have shown before. Observe that if we Taylor expand at $p$, then we get
      \[
        p_n = g(p_{n - 1}) = g(p) + g'(\xi_{n - 1})(p - p_{n - 1}),
      \]
      so
      \[
        p_n - p = g'(\xi_{n - 1})(p - p_{n - 1}).
      \]
      Hence we have
      \begin{align*}
        \lim_{n\to \infty} \frac{\abs{p_n - p}}{\abs{p_{n - 1} - p}^1} &= \lim_{n\to \infty} \frac{\abs{g'(\xi_{n - 1})(p - p_{n - 1})}}{\abs{p_{n - 1} - p}} \\
                                                                       &= \lim_{n\to \infty} \abs{g'(\xi_{n - 1})}.
      \intertext{Since $g\in C^1$, we know that $g'$ is continuous on $[a, b]$. Thus}
                                                                       &= \abs{g'\paren{\lim_{n\to \infty} \xi_{n - 1}}} \\
                                                                       &= \abs{g'(p)} \\
                                                                       &> 0.
      \end{align*}
      Furthermore, since $g'$ is continuous on $[a, b]$, by Extreme Value Theorem we know that it is bounded, so
      \[
        \lim_{n\to \infty} \frac{\abs{p_n - p}}{p_{n - 1} - p} = \lam,
      \]
      where $0 < \lam < \infty$.
    \end{proof}
  \end{theorem}
  \begin{theorem}{}
    Let $\alpha\in\Z$, and $g\in C^\alpha[a, b]$ such that $g(x)\in [a, b]$ for all $x\in [a, b]$. Then there exists $0 < k < 1$ such that $\abs{g'(x)} < k$ for all $x\in (a, b)$. If $g'(p)\neq 0$, then for any number $p_0\in [a, b]$ with $p_0\neq p$, the sequence
    \[
      p_n = g(p_{n - 1})\tag{$n\geq 1$}
    \]
    converges with order $\alpha$ to the unique fixed point in $[a, b]$.
    \begin{proof}
      Just like the last proof, we Taylor expand:
      \begin{align*}
        g(p_{n - 1}) &= g(p) + g'(p)(p - p_{n - 1}) + \dotsb + \frac{g^{(\alpha - 1)}(p)}{(\alpha - 1)!}(p - p_{n - 1})^{\alpha - 1} + \frac{g^{(\alpha)}(\xi_{n - 1})}{\alpha!}(p - p_{n - 1})^\alpha \\
                     &= g(p) + \frac{g^{(\alpha)}(\xi_{n - 1})}{\alpha!}(p - p_{n - 1})^\alpha \\
                     &= p + \frac{g^{(\alpha)}(\xi_{n - 1})}{\alpha!}(p - p_{n - 1})^\alpha.
      \end{align*}
      Hence we have
      \begin{align*}
        \lim_{n\to \infty} \frac{\abs{p_n - p}}{(p - p_{n - 1})^\alpha} &= \lim_{n\to \infty} \frac{g^{(\alpha)}(\xi_{n - 1})}{\alpha!} \\
                                                                        &= \frac{g^{(\alpha)}(p)}{\alpha!}.
      \end{align*}
    \end{proof}
  \end{theorem}
  Thus from the above two theorems we have the following:
  \begin{itemize}
    \item If $g'(p)\neq 0$, then it converges linearly.
    \item If $g'(p) = 0$ but $g''(p)\neq 0$, then it converges quadratically.
    \item If $g'(p) = 0$ and $g''(p) = 0$ but $g'''(p)\neq 0$, then it converges with third order.
  \end{itemize}
  If we apply this to Newton's method, we see that $f(p) = 0$ and $f'(p)\neq 0$. Furthermore, for $g(x)\ceq x - \frac{f(x)}{f'(x)}$, we have $g(p) = p$ and $g'(p) = 0$. Hence we conclude that Newton's method is \emph{at least} quadratic.
  
\end{document}
