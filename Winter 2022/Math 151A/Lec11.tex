\documentclass[class=article, crop=false]{standalone}
% Import packages
\usepackage[margin=1in]{geometry}

\usepackage{amssymb, amsthm}
\usepackage{comment}
\usepackage{enumitem}
\usepackage{fancyhdr}
\usepackage{hyperref}
\usepackage{import}
\usepackage{mathrsfs, mathtools}
\usepackage{multicol}
\usepackage{pdfpages}
\usepackage{standalone}
\usepackage{transparent}
\usepackage{xcolor}

\usepackage{minted}
\usepackage[many]{tcolorbox}
\tcbuselibrary{minted}

% Declare math operators
\DeclareMathOperator{\lcm}{lcm}
\DeclareMathOperator{\proj}{proj}
\DeclareMathOperator{\vspan}{span}
\DeclareMathOperator{\im}{im}
\DeclareMathOperator{\Diff}{Diff}
\DeclareMathOperator{\Int}{Int}
\DeclareMathOperator{\fcn}{fcn}
\DeclareMathOperator{\id}{id}
\DeclareMathOperator{\rank}{rank}
\DeclareMathOperator{\tr}{tr}
\DeclareMathOperator{\dive}{div}
\DeclareMathOperator{\row}{row}
\DeclareMathOperator{\col}{col}
\DeclareMathOperator{\dom}{dom}
\DeclareMathOperator{\ran}{ran}
\DeclareMathOperator{\Dom}{Dom}
\DeclareMathOperator{\Ran}{Ran}
\DeclareMathOperator{\fl}{fl}
% Macros for letters/variables
\newcommand\ttilde{\kern -.15em\lower .7ex\hbox{\~{}}\kern .04em}
\newcommand{\N}{\ensuremath{\mathbb{N}}}
\newcommand{\Z}{\ensuremath{\mathbb{Z}}}
\newcommand{\Q}{\ensuremath{\mathbb{Q}}}
\newcommand{\R}{\ensuremath{\mathbb{R}}}
\newcommand{\C}{\ensuremath{\mathbb{C}}}
\newcommand{\F}{\ensuremath{\mathbb{F}}}
\newcommand{\M}{\ensuremath{\mathbb{M}}}
\renewcommand{\P}{\ensuremath{\mathbb{P}}}
\newcommand{\lam}{\ensuremath{\lambda}}
\newcommand{\nab}{\ensuremath{\nabla}}
\newcommand{\eps}{\ensuremath{\varepsilon}}
\newcommand{\es}{\ensuremath{\varnothing}}
% Macros for math symbols
\newcommand{\dx}[1]{\,\mathrm{d}#1}
\newcommand{\inv}{\ensuremath{^{-1}}}
\newcommand{\sm}{\setminus}
\newcommand{\sse}{\subseteq}
\newcommand{\ceq}{\coloneqq}
% Macros for pairs of math symbols
\newcommand{\abs}[1]{\ensuremath{\left\lvert #1 \right\rvert}}
\newcommand{\paren}[1]{\ensuremath{\left( #1 \right)}}
\newcommand{\norm}[1]{\ensuremath{\left\lVert #1\right\rVert}}
\newcommand{\set}[1]{\ensuremath{\left\{#1\right\}}}
\newcommand{\tup}[1]{\ensuremath{\left\langle #1 \right\rangle}}
\newcommand{\floor}[1]{\ensuremath{\left\lfloor #1 \right\rfloor}}
\newcommand{\ceil}[1]{\ensuremath{\left\lceil #1 \right\rceil}}
\newcommand{\eclass}[1]{\ensuremath{\left[ #1 \right]}}

\newcommand{\chapternum}{}
\newcommand{\ex}[1]{\noindent\textbf{Exercise \chapternum.{#1}.}}

\newcommand{\tsub}[1]{\textsubscript{#1}}
\newcommand{\tsup}[1]{\textsuperscript{#1}}

\renewcommand{\choose}[2]{\ensuremath{\phantom{}_{#1}C_{#2}}}
\newcommand{\perm}[2]{\ensuremath{\phantom{}_{#1}P_{#2}}}

% Include figures
\newcommand{\incfig}[2][1]{%
    \def\svgwidth{#1\columnwidth}
    \import{./figures/}{#2.pdf_tex}
}

\definecolor{problemBackground}{RGB}{212,232,246}
\newenvironment{problem}[1]
  {
    \begin{tcolorbox}[
      boxrule=.5pt,
      titlerule=.5pt,
      sharp corners,
      colback=problemBackground,
      breakable
    ]
    \ifx &#1& \textbf{Problem. }
    \else \textbf{Problem #1.} \fi
  }
  {
    \end{tcolorbox}
  }
\definecolor{exampleBackground}{RGB}{255,249,248}
\definecolor{exampleAccent}{RGB}{158,60,14}
\newenvironment{example}[1]
  {
    \begin{tcolorbox}[
      boxrule=.5pt,
      sharp corners,
      colback=exampleBackground,
      colframe=exampleAccent,
    ]
    \color{exampleAccent}\textbf{Example.} \emph{#1}\color{black}
  }
  {
    \end{tcolorbox}
  }
\definecolor{theoremBackground}{RGB}{234,243,251}
\definecolor{theoremAccent}{RGB}{0,116,183}
\newenvironment{theorem}[1]
  {
    \begin{tcolorbox}[
      boxrule=.5pt,
      titlerule=.5pt,
      sharp corners,
      colback=theoremBackground,
      colframe=theoremAccent,
      breakable
    ]
      % \color{theoremAccent}\textbf{Theorem --- }\emph{#1}\\\color{black}
    \ifx &#1& \color{theoremAccent}\textbf{Theorem. }\color{black}
    \else \color{theoremAccent}\textbf{Theorem --- }\emph{#1}\\\color{black} \fi
  }
  {
    \end{tcolorbox}
  }
\definecolor{noteBackground}{RGB}{244,249,244}
\definecolor{noteAccent}{RGB}{34,139,34}
\newenvironment{note}[1]
  {
  \begin{tcolorbox}[
    enhanced,
    boxrule=0pt,
    frame hidden,
    sharp corners,
    colback=noteBackground,
    borderline west={3pt}{-1.5pt}{noteAccent},
    breakable
    ]
    \ifx &#1& \color{noteAccent}\textbf{Note. }\color{black}
    \else \color{noteAccent}\textbf{Note (#1). }\color{black} \fi
    }
    {
  \end{tcolorbox}
  }
\definecolor{lemmaBackground}{RGB}{255,247,234}
\definecolor{lemmaAccent}{RGB}{255,153,0}
\newenvironment{lemma}[1]
  {
    \begin{tcolorbox}[
      enhanced,
      boxrule=0pt,
      frame hidden,
      sharp corners,
      colback=lemmaBackground,
      borderline west={3pt}{-1.5pt}{lemmaAccent},
      breakable
    ]
    \ifx &#1& \color{lemmaAccent}\textbf{Lemma. }\color{black}
    \else \color{lemmaAccent}\textbf{Lemma #1. }\color{black} \fi
  }
  {
    \end{tcolorbox}
  }
\definecolor{definitionBackground}{RGB}{246,246,246}
\newenvironment{definition}[1]
  {
    \begin{tcolorbox}[
      enhanced,
      boxrule=0pt,
      frame hidden,
      sharp corners,
      colback=definitionBackground,
      borderline west={3pt}{-1.5pt}{black},
      breakable
    ]
    \textbf{Definition. }\emph{#1}\\
  }
  {
    \end{tcolorbox}
  }

\newenvironment{amatrix}[2]{
    \left[
      \begin{array}{*{#1}{c}|*{#2}c}
  }
  {
      \end{array}
    \right]
  }

\definecolor{codeBackground}{RGB}{253,246,227}
\renewcommand\theFancyVerbLine{\arabic{FancyVerbLine}}
\newtcblisting{cpp}[1][]{
  boxrule=1pt,
  sharp corners,
  colback=codeBackground,
  listing only,
  breakable,
  minted language=cpp,
  minted style=material,
  minted options={
    xleftmargin=15pt,
    baselinestretch=1.2,
    linenos,
  }
}

\author{Kyle Chui}


\fancyhf{}
\lhead{Kyle Chui}
\rhead{Page \thepage}
\pagestyle{fancy}

\begin{document}
  \section{Lecture 11}
  \subsection{Lagrange Interpolation Polynomial}
  The problem is that we are given $n + 1$ points of the form $(x_i, y_i)$ for $i = 0,1,\dotsc,n$ ($x_i$ are distinct). We wish to find a polynomial $P_n(x)$ of degree at most ``$n$'' such that
  \[
    P_n(x_i) = y_i
  \]
  for all $i = 0,1,\dotsc,n$.
  \begin{itemize}
    \item If we have a $y^*$ corresponding to some $x^*\neq x_i$, we can approximate it with $y^*\approx P_n(x^*)$.
    \item Given $f(x)$ and distinct $x_i$ ($i = 0,1,\dotsc,n$). We can find a polynomial function $P_n(x)$ of degree at most ``$n$'' such that $P_n(x_i) = f(x_i)$.
    \item We are approximating our function $f$ with a polynomial, and this extends to derivatives and integrals as well.
    \item General form of a polynomial $P_n(x)$ of degree at most ``$n$'':
    \[
      P_n(x) = a_nx^n + a_{n - 1}x^{n - 1} + \dotsb + a_1x + a_0.
    \]
    Hence it suffices to just find the coefficients such that the polynomial value equals the original function value.
  \end{itemize}
  If we try to solve for the coefficients by plugging in the $n + 1$ points, we essentially solve the matrix:
  \[
  \begin{bmatrix}
    1 & x_0 & \dotsb & x_0^2 \\
    1 & x_1 & \dotsb & x_1^n \\
    \vdots & \vdots & \ddots & \vdots \\
    1 & x_n & \dotsb & x_n^n
  \end{bmatrix}
  \begin{bmatrix}
    a_0 \\
    a_1 \\
    \vdots \\
    a_n
  \end{bmatrix} = 
  \begin{bmatrix}
    y_0 \\
    y_1 \\
    \vdots \\
    y_n
  \end{bmatrix}.
  \]
  \begin{note}{}
    If the $x_i$ are distinct, we know that the matrix is non-singular (i.e. there exists a unique solution for the $a_i$).
  \end{note}
  \paragraph{Drawbacks}
  \begin{itemize}
    \item Solving the linear system might be computationally expensive.
    \item The matrix is ``ill-conditioned'', which means that it's ``sensitive to inaccuracies'' (i.e. storage or computation errors).
  \end{itemize}
  \subsection{Lagrange Polynomials}
  We are given $(x_i, y_i)$, where the $x_i$ are distinct.
  \begin{itemize}
    \item Construct base functions (cardinal functions) \par
    Given $x_0,x_1,\dotsc,x_n$ (distinct), construct $\ell_0(x), \ell_1(x),\dotsc,\ell_n(x)$ of degree ``$n$'' polynomials such that
    \[
      \ell_i(x_j) = \delta_{ij} = \begin{cases}1 & \text{if }i = j \\ 0 &\text{otherwise}\end{cases}.
    \]
    Then we have
    \[
      \ell_i(x) = \frac{(x - x_0)(x - x_1)\dotsb(x - x_{i - 1})(x - x_{i + 1})\dotsb(x_i - x_n)}{(x_i - x_0)(x_i - x_1)\dotsb(x_i - x_{i - 1})(x_i - x_{i + 1})\dotsb(x_i - x_n)}.
    \]
    In other words,
    \[
      \ell_i(x) = \prod_{\substack{j = 0\\j\neq i}}^n \frac{x - x_j}{x_i - x_j}.
    \]
  \end{itemize}
  We can then construct $\P_n(x)$ via
  \[
    P_n(x) = \sum_{i=0}^{n}\ell_i(x)\cdot y_i.
  \]
\end{document}
