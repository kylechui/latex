\documentclass[class=article, crop=false]{standalone}
\usepackage[margin=1in]{geometry}

\usepackage{amsmath}
\usepackage{amssymb}
\usepackage{amsthm}
\usepackage{comment}
\usepackage{enumitem}
\usepackage{fancyhdr}
\usepackage{hyperref}
\usepackage{mathrsfs}
\usepackage{mathtools}
\usepackage{standalone}
\usepackage{tcolorbox}
\usepackage{tikz}
\usepackage{xcolor}

\usetikzlibrary{decorations.pathreplacing}
\tcbuselibrary{skins}

\DeclareMathOperator{\lcm}{lcm}
\DeclareMathOperator{\proj}{proj}
\DeclareMathOperator{\vspan}{span}
\DeclareMathOperator{\im}{im}
\DeclareMathOperator{\range}{range}
\DeclareMathOperator{\Diff}{Diff}
\DeclareMathOperator{\Int}{Int}
\DeclareMathOperator{\fcn}{fcn}
\DeclareMathOperator{\id}{id}

\newcommand{\N}{\ensuremath{\mathbb{N}}}
\newcommand{\Z}{\ensuremath{\mathbb{Z}}}
\newcommand{\Q}{\ensuremath{\mathbb{Q}}}
\newcommand{\R}{\ensuremath{\mathbb{R}}}
\newcommand{\C}{\ensuremath{\mathbb{C}}}
\newcommand{\F}{\ensuremath{\mathbb{F}}}
\newcommand{\lam}{\ensuremath{\lambda}}
\newcommand{\nab}{\ensuremath{\nabla}}
\newcommand{\eps}{\ensuremath{\varepsilon}}
\newcommand{\es}{\ensuremath{\varnothing}}

\newcommand{\abs}[1]{\ensuremath{\left\lvert #1 \right\rvert}}
\newcommand{\bigpar}[1]{\ensuremath{\left( #1 \right)}}
\newcommand{\norm}[1]{\ensuremath{\lVert #1\rVert}}
\newcommand{\set}[1]{\ensuremath{\left\{#1\right\}}}
\newcommand{\tuple}[1]{\ensuremath{\left\langle #1 \right\rangle}}
\newcommand{\floor}[1]{\ensuremath{\left\lfloor #1 \right\rfloor}}
\newcommand{\ceil}[1]{\ensuremath{\left\lceil #1 \right\rceil}}

\newcommand{\chapternum}{}
\newcommand{\ex}[1]{\noindent\textbf{Exercise \chapternum.{#1}.}}

\newcommand{\dx}[1]{\,\mathrm{d}#1}
\newcommand{\inv}{\ensuremath{^{-1}}}
\newcommand{\sm}{\setminus}
\newcommand{\sse}{\subseteq}
\newcommand{\ceq}{\coloneqq}

\definecolor{problemBackground}{RGB}{212,232,246}
\newenvironment{problem}[1]
  {
    \begin{tcolorbox}[
      boxrule=.5pt,
      titlerule=.5pt,
      sharp corners,
      colback=problemBackground
    ]
    \ifx &#1& \textbf{Problem. }
    \else \textbf{Problem #1.} \fi
  }
  {
    \end{tcolorbox}
  }

\definecolor{exampleBackground}{RGB}{255,249,248}
\definecolor{exampleAccent}{RGB}{158,60,14}
\newenvironment{example}[1]
  {
    \begin{tcolorbox}[
      boxrule=.5pt,
      sharp corners,
      colback=exampleBackground,
      colframe=exampleAccent,
    ]
    \color{exampleAccent}\textbf{Example.} \emph{#1}\color{black}
  }
  {
    \end{tcolorbox}
  }

\definecolor{theoremBackground}{RGB}{234,243,251}
\definecolor{theoremAccent}{RGB}{0,116,183}
\newenvironment{theorem}[1]
  {
    \begin{tcolorbox}[
      boxrule=.5pt,
      titlerule=.5pt,
      sharp corners,
      colback=theoremBackground,
      colframe=theoremAccent
    ]
      \color{theoremAccent}\textbf{Theorem --- }\emph{#1}\\\color{black}
  }
  {
    \end{tcolorbox}
  }

\definecolor{noteBackground}{RGB}{244,249,244}
\definecolor{noteAccent}{RGB}{34,139,34}
\newenvironment{note}[1]
  {
  \begin{tcolorbox}[
    enhanced,
    boxrule=0pt,
    frame hidden,
    sharp corners,
    colback=noteBackground,
    borderline west={3pt}{-1.5pt}{noteAccent}
    ]
    \ifx &#1& \color{noteAccent}\textbf{Note. }\color{black}
    \else \color{noteAccent}\textbf{Note (#1). }\color{black} \fi
    }
    {
  \end{tcolorbox}
  }

\definecolor{definitionBackground}{RGB}{246,246,246}
\newenvironment{definition}[1]
  {
    \begin{tcolorbox}[
      enhanced,
      boxrule=0pt,
      frame hidden,
      sharp corners,
      colback=definitionBackground,
      borderline west={3pt}{-1.5pt}{black}
    ]
    \textbf{Definition. }\emph{#1}\\
  }
  {
    \end{tcolorbox}
  }
\newenvironment{amatrix}[2]{
    \left[
      \begin{array}{*{#1}{c}|*{#2}c}
  }
  {
      \end{array}
    \right]
  }
\date{\the\year-\the\month-\the\day}
\author{Kyle Chui}


\fancyhf{}
\lhead{Kyle Chui}
\rhead{Page \thepage}
\pagestyle{fancy}

\begin{document}
  \section{Lecture 19}
  For the second case, if we have that
  \[
    M = N_1(h) + k_1h^2 + k_2h^4 + k_3h^6 + \dotsb,
  \]
  then we can use the same method shown in the previous lecture to derive that
  \[
    N_k(h) = \frac{4^{k - 1}N_{k - 1}(\frac{h}{2}) - N_{k - 1}(h)}{4^{k - 1} - 1}.
  \]
  The error term is hence given by
  \[
    M = N_k(h) + \mathcal{O}(h^{2k}).
  \]
  Since the forward difference and backwards difference of $f'(x^*)$, as well as the central difference of $f''(x^*)$ are of the form in the first case, we may apply it to extrapolate. Similarly, the central difference for $f'(x^*)$ takes the form given in the second case.
  \subsection{Numerical Integration}
  Given a function $f(x)$ on $[a, b]$, we want to find an approximation for $\int_{a}^{b}f(x) \,\mathrm dx$. \par
  \paragraph{Numerical Quadrature}
  We know that
  \[
    \int_{a}^{b}f(x) \,\mathrm dx = \sum_{i=0}^{n} a_if(x_i) + \text{Error}.
  \]
  \begin{itemize}
    \item Case 1: $x_0,x_1,\dotsc,x_n\in [a, b]$ are given, and we need to find $a_i$. \par
    We have the points $(x_0, f(x_0)), (x_1, f(x_1)),\dotsc,(x_n, f(x_n))$. We then interpolate at the aforementioned points to get an approximation for $f(x)$, yielding $f(x) = P(x) + \text{Error}$. Hence
    \[
      \int_{a}^{b}f(x) \,\mathrm dx = \int_{a}^{b}P(x) \,\mathrm dx + \int_{a}^{b}\text{Error} \,\mathrm dx.
    \]
    Using our Lagrange interpolation we find that our coefficients take the form
    \[
      a_i = \int_{a}^{b}\ell_i(x) \,\mathrm dx.
    \]
    \item Case 2: Find $x_0,\dotsc,x_n\in [a, b]$ and $a_0,\dotsc,a_n\in\R$.
  \end{itemize}
  \subsubsection{Trapezoidal Rule}
  \begin{theorem}{Weighted Mean Value for Integrals}
    If we have a continuous function $f\in C[a, b]$, and $\int_{a}^{b}g(x) \,\mathrm dx$ exists and $g(x)$ does not change sign on $[a, b]$, then there exists $c\in (a, b)$ such that
    \[
      \int_{a}^{b}f(x)g(x) \,\mathrm dx = f(c)\int_{a}^{b}g(x) \,\mathrm dx.
    \]
  \end{theorem}
  We know that the error term for Trapezoidal rule is given by
  \begin{align*}
    \text{Error}^* &= \int_{x_0}^{x_1} \frac{f''(\xi)}{2}(x - x_0)(x - x_1) \,\mathrm dx \\
                   &= \frac{f''(\xi_c)}{2} \int_{x_0}^{x_1}(x - x_0)(x - x_1) \,\mathrm dx \tag*{$\xi_c\in (x_0, x_1)$} \\
                   &= \frac{f''(\xi_c)}{2} \int_{0}^{h}z(z - h) \,\mathrm dz \\
                   &= \frac{f''(\xi_c)}{2}\cdot -\frac{1}{6}h^3 \\
                   &= -\frac{1}{12}f''(\xi_c)h^3
  \end{align*}
  Hence we may write
  \[
    \int_{a}^{b}f(x) \,\mathrm dx = \frac{h}{2}(f(x_0) + f(x_1)) - \frac{h^3}{12}f''(\xi_c) \tag*{$\xi_c\in (x_0, x_1)$}
  \]
\end{document}
