\documentclass[class=article, crop=false]{standalone}
% Import packages
\usepackage[margin=1in]{geometry}

\usepackage{amssymb, amsthm}
\usepackage{comment}
\usepackage{enumitem}
\usepackage{fancyhdr}
\usepackage{hyperref}
\usepackage{import}
\usepackage{mathrsfs, mathtools}
\usepackage{multicol}
\usepackage{pdfpages}
\usepackage{standalone}
\usepackage{transparent}
\usepackage{xcolor}

\usepackage{minted}
\usepackage[many]{tcolorbox}
\tcbuselibrary{minted}

% Declare math operators
\DeclareMathOperator{\lcm}{lcm}
\DeclareMathOperator{\proj}{proj}
\DeclareMathOperator{\vspan}{span}
\DeclareMathOperator{\im}{im}
\DeclareMathOperator{\Diff}{Diff}
\DeclareMathOperator{\Int}{Int}
\DeclareMathOperator{\fcn}{fcn}
\DeclareMathOperator{\id}{id}
\DeclareMathOperator{\rank}{rank}
\DeclareMathOperator{\tr}{tr}
\DeclareMathOperator{\dive}{div}
\DeclareMathOperator{\row}{row}
\DeclareMathOperator{\col}{col}
\DeclareMathOperator{\dom}{dom}
\DeclareMathOperator{\ran}{ran}
\DeclareMathOperator{\Dom}{Dom}
\DeclareMathOperator{\Ran}{Ran}
\DeclareMathOperator{\fl}{fl}
% Macros for letters/variables
\newcommand\ttilde{\kern -.15em\lower .7ex\hbox{\~{}}\kern .04em}
\newcommand{\N}{\ensuremath{\mathbb{N}}}
\newcommand{\Z}{\ensuremath{\mathbb{Z}}}
\newcommand{\Q}{\ensuremath{\mathbb{Q}}}
\newcommand{\R}{\ensuremath{\mathbb{R}}}
\newcommand{\C}{\ensuremath{\mathbb{C}}}
\newcommand{\F}{\ensuremath{\mathbb{F}}}
\newcommand{\M}{\ensuremath{\mathbb{M}}}
\renewcommand{\P}{\ensuremath{\mathbb{P}}}
\newcommand{\lam}{\ensuremath{\lambda}}
\newcommand{\nab}{\ensuremath{\nabla}}
\newcommand{\eps}{\ensuremath{\varepsilon}}
\newcommand{\es}{\ensuremath{\varnothing}}
% Macros for math symbols
\newcommand{\dx}[1]{\,\mathrm{d}#1}
\newcommand{\inv}{\ensuremath{^{-1}}}
\newcommand{\sm}{\setminus}
\newcommand{\sse}{\subseteq}
\newcommand{\ceq}{\coloneqq}
% Macros for pairs of math symbols
\newcommand{\abs}[1]{\ensuremath{\left\lvert #1 \right\rvert}}
\newcommand{\paren}[1]{\ensuremath{\left( #1 \right)}}
\newcommand{\norm}[1]{\ensuremath{\left\lVert #1\right\rVert}}
\newcommand{\set}[1]{\ensuremath{\left\{#1\right\}}}
\newcommand{\tup}[1]{\ensuremath{\left\langle #1 \right\rangle}}
\newcommand{\floor}[1]{\ensuremath{\left\lfloor #1 \right\rfloor}}
\newcommand{\ceil}[1]{\ensuremath{\left\lceil #1 \right\rceil}}
\newcommand{\eclass}[1]{\ensuremath{\left[ #1 \right]}}

\newcommand{\chapternum}{}
\newcommand{\ex}[1]{\noindent\textbf{Exercise \chapternum.{#1}.}}

\newcommand{\tsub}[1]{\textsubscript{#1}}
\newcommand{\tsup}[1]{\textsuperscript{#1}}

\renewcommand{\choose}[2]{\ensuremath{\phantom{}_{#1}C_{#2}}}
\newcommand{\perm}[2]{\ensuremath{\phantom{}_{#1}P_{#2}}}

% Include figures
\newcommand{\incfig}[2][1]{%
    \def\svgwidth{#1\columnwidth}
    \import{./figures/}{#2.pdf_tex}
}

\definecolor{problemBackground}{RGB}{212,232,246}
\newenvironment{problem}[1]
  {
    \begin{tcolorbox}[
      boxrule=.5pt,
      titlerule=.5pt,
      sharp corners,
      colback=problemBackground,
      breakable
    ]
    \ifx &#1& \textbf{Problem. }
    \else \textbf{Problem #1.} \fi
  }
  {
    \end{tcolorbox}
  }
\definecolor{exampleBackground}{RGB}{255,249,248}
\definecolor{exampleAccent}{RGB}{158,60,14}
\newenvironment{example}[1]
  {
    \begin{tcolorbox}[
      boxrule=.5pt,
      sharp corners,
      colback=exampleBackground,
      colframe=exampleAccent,
    ]
    \color{exampleAccent}\textbf{Example.} \emph{#1}\color{black}
  }
  {
    \end{tcolorbox}
  }
\definecolor{theoremBackground}{RGB}{234,243,251}
\definecolor{theoremAccent}{RGB}{0,116,183}
\newenvironment{theorem}[1]
  {
    \begin{tcolorbox}[
      boxrule=.5pt,
      titlerule=.5pt,
      sharp corners,
      colback=theoremBackground,
      colframe=theoremAccent,
      breakable
    ]
      % \color{theoremAccent}\textbf{Theorem --- }\emph{#1}\\\color{black}
    \ifx &#1& \color{theoremAccent}\textbf{Theorem. }\color{black}
    \else \color{theoremAccent}\textbf{Theorem --- }\emph{#1}\\\color{black} \fi
  }
  {
    \end{tcolorbox}
  }
\definecolor{noteBackground}{RGB}{244,249,244}
\definecolor{noteAccent}{RGB}{34,139,34}
\newenvironment{note}[1]
  {
  \begin{tcolorbox}[
    enhanced,
    boxrule=0pt,
    frame hidden,
    sharp corners,
    colback=noteBackground,
    borderline west={3pt}{-1.5pt}{noteAccent},
    breakable
    ]
    \ifx &#1& \color{noteAccent}\textbf{Note. }\color{black}
    \else \color{noteAccent}\textbf{Note (#1). }\color{black} \fi
    }
    {
  \end{tcolorbox}
  }
\definecolor{lemmaBackground}{RGB}{255,247,234}
\definecolor{lemmaAccent}{RGB}{255,153,0}
\newenvironment{lemma}[1]
  {
    \begin{tcolorbox}[
      enhanced,
      boxrule=0pt,
      frame hidden,
      sharp corners,
      colback=lemmaBackground,
      borderline west={3pt}{-1.5pt}{lemmaAccent},
      breakable
    ]
    \ifx &#1& \color{lemmaAccent}\textbf{Lemma. }\color{black}
    \else \color{lemmaAccent}\textbf{Lemma #1. }\color{black} \fi
  }
  {
    \end{tcolorbox}
  }
\definecolor{definitionBackground}{RGB}{246,246,246}
\newenvironment{definition}[1]
  {
    \begin{tcolorbox}[
      enhanced,
      boxrule=0pt,
      frame hidden,
      sharp corners,
      colback=definitionBackground,
      borderline west={3pt}{-1.5pt}{black},
      breakable
    ]
    \textbf{Definition. }\emph{#1}\\
  }
  {
    \end{tcolorbox}
  }

\newenvironment{amatrix}[2]{
    \left[
      \begin{array}{*{#1}{c}|*{#2}c}
  }
  {
      \end{array}
    \right]
  }

\definecolor{codeBackground}{RGB}{253,246,227}
\renewcommand\theFancyVerbLine{\arabic{FancyVerbLine}}
\newtcblisting{cpp}[1][]{
  boxrule=1pt,
  sharp corners,
  colback=codeBackground,
  listing only,
  breakable,
  minted language=cpp,
  minted style=material,
  minted options={
    xleftmargin=15pt,
    baselinestretch=1.2,
    linenos,
  }
}

\author{Kyle Chui}


\fancyhf{}
\lhead{Kyle Chui}
\rhead{Page \thepage}
\pagestyle{fancy}

\begin{document}
  \section{Lecture 1}
  The goal of this class is to solve \emph{mathematical} problems with the help of \emph{computers}.
  \subsection{Chapter Summaries}
  \begin{enumerate}
    \item Computations in computers
    \begin{itemize}
      \item How to store (real) numbers in the computer
      \begin{note}{}
        If the number has finitely many digits, then it is simple. What about numbers with infinitely many digits, i.e. $\frac{1}{3}$? We have to truncate or round, and store an approximation with finitely many digits.
      \end{note}
      \item How to perform computations
      \begin{note}{}
        From regular math, we know that $\frac{1}{3} + \frac{1}{3} = \frac{2}{3}$. However, in the computer, due to errors, we have $\frac{1}{3}\oplus \frac{1}{3} = ? \oplus ? = ??$.
      \end{note}
      \item Errors
    \end{itemize}
    \item 
    \begin{itemize}
      \item Find roots of $f(x) = 0$ using bisection, Newton's method, ...
      \item Convergence
      \item Convergence order (how fast it converges)
    \end{itemize}
    \item Polynomial interpolation
    \begin{itemize}
      \item Approximate a function $f(x)$ by a polynomial $P(x)$, where $f(x_i) = P(x_i)$ for finitely many $x_i$
      \item \emph{Accuracy} of the polynomial approximations
    \end{itemize}
    \item Numerical differentiations and numerical integrations
    \begin{itemize}
      \item Using the approximations from chapter $3$, we can approximate using
      \[
        f'(x^*)\approx P'(x^*) = \sum_{i=0}^{k}f(x_k)c_k,
      \]
      and
      \[
        \int_{a}^{b}f(x) \,\mathrm dx\approx \int_{a}^{b}P(x) \,\mathrm dx = \sum_{i=1}^{k} f(\overline{x_k})\overline{c_k}.
      \]
      \item Error analysis
    \end{itemize}
    \item[6.7.] Solving linear systems of equations
    \begin{itemize}
      \item Direct methods: Gaussian elimination (computationally expensive)
      \item Iterative methods: (faster and cheaper)
      \item Solution stability
    \end{itemize}
  \end{enumerate}
  \subsection{Round-off errors and computer arithmetics}
  There are three kinds of errors:
  \begin{itemize}
    \item Modeling Error: Occurs when we convert a problem from the real world into the mathematical world.
    \item Method Error: Occurs when we try to solve the mathematical problem numerically.
    \item Round-off error: Occurs when the computer gives an incorrect result with the correct algorithm (comes from storage and computation).
  \end{itemize}
  \subsubsection{Storage}
  Infinite digit real numbers are \emph{stored} as finite digit numbers, using the normalized decimal form of real numbers.
  \begin{definition}{Normalized decimal form of a real number}
    For any $y\in\R$, we may write
    \[
      y = \pm 0.d_1d_2d_3\dotsc d_kd_{k + 1}\dotsc\cdot 10^n,
    \]
    where $0 < d_1\leq 9$, $0\leq d_i\leq 9$, $n$ are integers. For the particular case where $y = 0$, we write $y = 0.0\cdot 10^0$.
  \end{definition}
  \begin{definition}{Normalized machine numbers (Floating-point form)}
    Any machine number $y$ can be written as
    \[
      y = \pm 0.d_1d_2\dotsb d_k\cdot 10^n,
    \]
    where $0 < d_1\leq 9$, $0\leq d_i\leq 9$, $n$ are integers.
  \end{definition}
  We can think of the storage process as mapping normalized real numbers to normalized machine numbers. We do this via rounding or truncating. \par
  Consider some $y\in\R\sm \set{0}$.
  \begin{itemize}
    \item Truncating ($k$-digit truncation of $y = \pm 0.d_1d_2d_3\dotsc d_kd_{k + 1}d_{k + 2}\dotsc\cdot 10^n$)
    Simply omit the digits from $d_{k + 1}$ and onwards, in other words
    \[
      \fl(\pm 0.d_1d_2d_3\dotsc d_kd_{k + 1}d_{k + 2}\dotsc\cdot 10^n) = \pm 0.d_1d_2\dotsc d_k\cdot 10^n.
    \]
    Thus we have $y\approx \fl(y)$.
    \item Rounding ($k$-digit rounding of $y = \pm 0.d_1d_2d_3\dotsc d_kd_{k + 1}d_{k + 2}\dotsc\cdot 10^n$) \par
    If $d_{k + 1} < 5$, then we drop $d_{k + 1}d_{k + 2}\dotsc$ (same with truncating) \par
    If $d_{k + 1} \geq 5$, then add $1$ to $d_k$ and drop $d_{k + 1}d_{k + 2}\dotsc$
    \[
      \fl(\pm 0.d_1d_2d_3\dotsc d_kd_{k + 1}d_{k + 2}\dotsc\cdot 10^n) = \pm \delta_1\delta_2\dotsc \delta_k\cdot 10^m.
    \]
  \end{itemize}
\end{document}
