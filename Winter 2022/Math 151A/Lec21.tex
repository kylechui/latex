\documentclass[class=article, crop=false]{standalone}
% Import packages
\usepackage[margin=1in]{geometry}

\usepackage{amssymb, amsthm}
\usepackage{comment}
\usepackage{enumitem}
\usepackage{fancyhdr}
\usepackage{hyperref}
\usepackage{import}
\usepackage{mathrsfs, mathtools}
\usepackage{multicol}
\usepackage{pdfpages}
\usepackage{standalone}
\usepackage{transparent}
\usepackage{xcolor}

\usepackage{minted}
\usepackage[many]{tcolorbox}
\tcbuselibrary{minted}

% Declare math operators
\DeclareMathOperator{\lcm}{lcm}
\DeclareMathOperator{\proj}{proj}
\DeclareMathOperator{\vspan}{span}
\DeclareMathOperator{\im}{im}
\DeclareMathOperator{\Diff}{Diff}
\DeclareMathOperator{\Int}{Int}
\DeclareMathOperator{\fcn}{fcn}
\DeclareMathOperator{\id}{id}
\DeclareMathOperator{\rank}{rank}
\DeclareMathOperator{\tr}{tr}
\DeclareMathOperator{\dive}{div}
\DeclareMathOperator{\row}{row}
\DeclareMathOperator{\col}{col}
\DeclareMathOperator{\dom}{dom}
\DeclareMathOperator{\ran}{ran}
\DeclareMathOperator{\Dom}{Dom}
\DeclareMathOperator{\Ran}{Ran}
\DeclareMathOperator{\fl}{fl}
% Macros for letters/variables
\newcommand\ttilde{\kern -.15em\lower .7ex\hbox{\~{}}\kern .04em}
\newcommand{\N}{\ensuremath{\mathbb{N}}}
\newcommand{\Z}{\ensuremath{\mathbb{Z}}}
\newcommand{\Q}{\ensuremath{\mathbb{Q}}}
\newcommand{\R}{\ensuremath{\mathbb{R}}}
\newcommand{\C}{\ensuremath{\mathbb{C}}}
\newcommand{\F}{\ensuremath{\mathbb{F}}}
\newcommand{\M}{\ensuremath{\mathbb{M}}}
\renewcommand{\P}{\ensuremath{\mathbb{P}}}
\newcommand{\lam}{\ensuremath{\lambda}}
\newcommand{\nab}{\ensuremath{\nabla}}
\newcommand{\eps}{\ensuremath{\varepsilon}}
\newcommand{\es}{\ensuremath{\varnothing}}
% Macros for math symbols
\newcommand{\dx}[1]{\,\mathrm{d}#1}
\newcommand{\inv}{\ensuremath{^{-1}}}
\newcommand{\sm}{\setminus}
\newcommand{\sse}{\subseteq}
\newcommand{\ceq}{\coloneqq}
% Macros for pairs of math symbols
\newcommand{\abs}[1]{\ensuremath{\left\lvert #1 \right\rvert}}
\newcommand{\paren}[1]{\ensuremath{\left( #1 \right)}}
\newcommand{\norm}[1]{\ensuremath{\left\lVert #1\right\rVert}}
\newcommand{\set}[1]{\ensuremath{\left\{#1\right\}}}
\newcommand{\tup}[1]{\ensuremath{\left\langle #1 \right\rangle}}
\newcommand{\floor}[1]{\ensuremath{\left\lfloor #1 \right\rfloor}}
\newcommand{\ceil}[1]{\ensuremath{\left\lceil #1 \right\rceil}}
\newcommand{\eclass}[1]{\ensuremath{\left[ #1 \right]}}

\newcommand{\chapternum}{}
\newcommand{\ex}[1]{\noindent\textbf{Exercise \chapternum.{#1}.}}

\newcommand{\tsub}[1]{\textsubscript{#1}}
\newcommand{\tsup}[1]{\textsuperscript{#1}}

\renewcommand{\choose}[2]{\ensuremath{\phantom{}_{#1}C_{#2}}}
\newcommand{\perm}[2]{\ensuremath{\phantom{}_{#1}P_{#2}}}

% Include figures
\newcommand{\incfig}[2][1]{%
    \def\svgwidth{#1\columnwidth}
    \import{./figures/}{#2.pdf_tex}
}

\definecolor{problemBackground}{RGB}{212,232,246}
\newenvironment{problem}[1]
  {
    \begin{tcolorbox}[
      boxrule=.5pt,
      titlerule=.5pt,
      sharp corners,
      colback=problemBackground,
      breakable
    ]
    \ifx &#1& \textbf{Problem. }
    \else \textbf{Problem #1.} \fi
  }
  {
    \end{tcolorbox}
  }
\definecolor{exampleBackground}{RGB}{255,249,248}
\definecolor{exampleAccent}{RGB}{158,60,14}
\newenvironment{example}[1]
  {
    \begin{tcolorbox}[
      boxrule=.5pt,
      sharp corners,
      colback=exampleBackground,
      colframe=exampleAccent,
    ]
    \color{exampleAccent}\textbf{Example.} \emph{#1}\color{black}
  }
  {
    \end{tcolorbox}
  }
\definecolor{theoremBackground}{RGB}{234,243,251}
\definecolor{theoremAccent}{RGB}{0,116,183}
\newenvironment{theorem}[1]
  {
    \begin{tcolorbox}[
      boxrule=.5pt,
      titlerule=.5pt,
      sharp corners,
      colback=theoremBackground,
      colframe=theoremAccent,
      breakable
    ]
      % \color{theoremAccent}\textbf{Theorem --- }\emph{#1}\\\color{black}
    \ifx &#1& \color{theoremAccent}\textbf{Theorem. }\color{black}
    \else \color{theoremAccent}\textbf{Theorem --- }\emph{#1}\\\color{black} \fi
  }
  {
    \end{tcolorbox}
  }
\definecolor{noteBackground}{RGB}{244,249,244}
\definecolor{noteAccent}{RGB}{34,139,34}
\newenvironment{note}[1]
  {
  \begin{tcolorbox}[
    enhanced,
    boxrule=0pt,
    frame hidden,
    sharp corners,
    colback=noteBackground,
    borderline west={3pt}{-1.5pt}{noteAccent},
    breakable
    ]
    \ifx &#1& \color{noteAccent}\textbf{Note. }\color{black}
    \else \color{noteAccent}\textbf{Note (#1). }\color{black} \fi
    }
    {
  \end{tcolorbox}
  }
\definecolor{lemmaBackground}{RGB}{255,247,234}
\definecolor{lemmaAccent}{RGB}{255,153,0}
\newenvironment{lemma}[1]
  {
    \begin{tcolorbox}[
      enhanced,
      boxrule=0pt,
      frame hidden,
      sharp corners,
      colback=lemmaBackground,
      borderline west={3pt}{-1.5pt}{lemmaAccent},
      breakable
    ]
    \ifx &#1& \color{lemmaAccent}\textbf{Lemma. }\color{black}
    \else \color{lemmaAccent}\textbf{Lemma #1. }\color{black} \fi
  }
  {
    \end{tcolorbox}
  }
\definecolor{definitionBackground}{RGB}{246,246,246}
\newenvironment{definition}[1]
  {
    \begin{tcolorbox}[
      enhanced,
      boxrule=0pt,
      frame hidden,
      sharp corners,
      colback=definitionBackground,
      borderline west={3pt}{-1.5pt}{black},
      breakable
    ]
    \textbf{Definition. }\emph{#1}\\
  }
  {
    \end{tcolorbox}
  }

\newenvironment{amatrix}[2]{
    \left[
      \begin{array}{*{#1}{c}|*{#2}c}
  }
  {
      \end{array}
    \right]
  }

\definecolor{codeBackground}{RGB}{253,246,227}
\renewcommand\theFancyVerbLine{\arabic{FancyVerbLine}}
\newtcblisting{cpp}[1][]{
  boxrule=1pt,
  sharp corners,
  colback=codeBackground,
  listing only,
  breakable,
  minted language=cpp,
  minted style=material,
  minted options={
    xleftmargin=15pt,
    baselinestretch=1.2,
    linenos,
  }
}

\author{Kyle Chui}


\fancyhf{}
\lhead{Kyle Chui}
\rhead{Page \thepage}
\pagestyle{fancy}

\begin{document}
  \section{Lecture 21}
  \begin{note}{}
    We can increase our accuracy by increasing the number of quadrature points, but this leads to complicated quadrature rules.
  \end{note}
  \subsection{Composed Quadrature Rules}
  \textbf{Idea.} We can cut up our interval into smaller sub-intervals, and use simpler quadrature rules to approximate the area.
  \subsubsection{Composed Trapezoidal Rule}
  We divide $[a, b]$ into ``$n$'' sub-intervals, so $a = x_0, b = x_n$. We can then sum the areas yielded by Trapezoidal rule over the intervals $[x_i, x_{i + 1}]$, for $i\in [0, n)$. The length of each interval is $h = \frac{b - a}{n}$, and $x_j = a + jh$. Hence we have
  \begin{align*}
    \int_{a}^{b}f(x) \,\mathrm dx &= \int_{x_0}^{x_1}f(x) \,\mathrm dx + \dotsb + \int_{x_{n - 1}}^{x_n}f(x) \,\mathrm dx \\
                                  &= \sum_{j=1}^{n} \paren{\frac{h}{2}(f(x_j) + f(x_{j - 1})) - \frac{h^3}{12}f''(\xi_j)} \\
                                  &= \frac{h}{2}(f(x_0) + 2f(x_1) + \dotsb + 2f(x_{n - 1}) + f(x_n)) - \frac{h^3}{12}\sum_{j=1}^{n}f''(\xi_j) \\
                                  &= \frac{h}{2}(f(x_0) + 2f(x_1) + \dotsb + 2f(x_{n - 1}) + f(x_n)) - \frac{h^3}{12}\cdot nf''(\eta) \\
                                  &= \underbrace{\frac{h}{2}(f(x_0) + 2f(x_1) + \dotsb + 2f(x_{n - 1}) + f(x_n))}_{C_T(f)} - \frac{h^2(b - a)}{12}f''(\eta).
  \end{align*}
  \subsubsection{Composed Simpson's Rule}
  Again, we divide $[a, b]$ into $n$ sub-intervals, where $n$ is an even integer. We have
  \begin{align*}
    \int_{a}^{b}f(x) \,\mathrm dx &= \sum_{j=1}^{\frac{n}{2}} \int_{x_{2j - 2}}^{x_{2j}}f(x) \,\mathrm dx \\
                                  &= \sum_{j=1}^{\frac{n}{2}} \frac{h}{3}(f(x_{2j - 2}) + f(4x_{2j - 1}) + f(x_{2j})) - \frac{h^5}{90}f^{(4)}(\xi_j) \\
                                  &= \dotsb \\
                                  &= \frac{h}{3}\paren{f(x_0) + 2\sum_{j=1}^{\frac{n}{2} - 1}f(x_{2j}) + 4\sum_{j=1}^{\frac{n}{2}} f(x_{2j - 1}) + f(x_n)} - \frac{h^5}{90}\sum_{j=1}^{\frac{n}{2}}f^{(4)}(\xi_j) \\
                                  &= \underbrace{\frac{h}{3}\paren{f(x_0) + 2\sum_{j=1}^{\frac{n}{2} - 1}f(x_{2j}) + 4\sum_{j=1}^{\frac{n}{2}} f(x_{2j - 1}) + f(x_n)}}_{C_S(f)} - \frac{h^4(b - a)}{180}f^{(4)}(\eta).
  \end{align*}
  \subsubsection{Composed Midpoint Rule}
  Again, we need to divide $[a, b]$ into $n$ sub-intervals, where $n$ is an even integer. Hence
  \begin{align*}
    \int_{a}^{b}f(x) \,\mathrm dx &= \sum_{j=1}^{\frac{n}{2}} \int_{x_{2j - 2}}^{x_{2j}}f(x) \,\mathrm dx \\
                                  &= \sum_{j=1}^{\frac{n}{2}} 2hf(x_{2j - 1}) + \sum_{j=1}^{\frac{n}{2}} \frac{h^3}{3}f''(\xi_j) \\
                                  &= 2h \sum_{j=1}^{\frac{n}{2}}f(x_{2j - 1}) + \frac{h^2(b - a)}{6}f''(\eta).
  \end{align*}
\end{document}
