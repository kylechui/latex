\documentclass[class=article, crop=false]{standalone}
\usepackage[margin=1in]{geometry}

\usepackage{amsmath}
\usepackage{amssymb}
\usepackage{amsthm}
\usepackage{comment}
\usepackage{enumitem}
\usepackage{fancyhdr}
\usepackage{hyperref}
\usepackage{mathrsfs}
\usepackage{mathtools}
\usepackage{standalone}
\usepackage{tcolorbox}
\usepackage{tikz}
\usepackage{xcolor}

\usetikzlibrary{decorations.pathreplacing}
\tcbuselibrary{skins}

\DeclareMathOperator{\lcm}{lcm}
\DeclareMathOperator{\proj}{proj}
\DeclareMathOperator{\vspan}{span}
\DeclareMathOperator{\im}{im}
\DeclareMathOperator{\range}{range}
\DeclareMathOperator{\Diff}{Diff}
\DeclareMathOperator{\Int}{Int}
\DeclareMathOperator{\fcn}{fcn}
\DeclareMathOperator{\id}{id}

\newcommand{\N}{\ensuremath{\mathbb{N}}}
\newcommand{\Z}{\ensuremath{\mathbb{Z}}}
\newcommand{\Q}{\ensuremath{\mathbb{Q}}}
\newcommand{\R}{\ensuremath{\mathbb{R}}}
\newcommand{\C}{\ensuremath{\mathbb{C}}}
\newcommand{\F}{\ensuremath{\mathbb{F}}}
\newcommand{\lam}{\ensuremath{\lambda}}
\newcommand{\nab}{\ensuremath{\nabla}}
\newcommand{\eps}{\ensuremath{\varepsilon}}
\newcommand{\es}{\ensuremath{\varnothing}}

\newcommand{\abs}[1]{\ensuremath{\left\lvert #1 \right\rvert}}
\newcommand{\bigpar}[1]{\ensuremath{\left( #1 \right)}}
\newcommand{\norm}[1]{\ensuremath{\lVert #1\rVert}}
\newcommand{\set}[1]{\ensuremath{\left\{#1\right\}}}
\newcommand{\tuple}[1]{\ensuremath{\left\langle #1 \right\rangle}}
\newcommand{\floor}[1]{\ensuremath{\left\lfloor #1 \right\rfloor}}
\newcommand{\ceil}[1]{\ensuremath{\left\lceil #1 \right\rceil}}

\newcommand{\chapternum}{}
\newcommand{\ex}[1]{\noindent\textbf{Exercise \chapternum.{#1}.}}

\newcommand{\dx}[1]{\,\mathrm{d}#1}
\newcommand{\inv}{\ensuremath{^{-1}}}
\newcommand{\sm}{\setminus}
\newcommand{\sse}{\subseteq}
\newcommand{\ceq}{\coloneqq}

\definecolor{problemBackground}{RGB}{212,232,246}
\newenvironment{problem}[1]
  {
    \begin{tcolorbox}[
      boxrule=.5pt,
      titlerule=.5pt,
      sharp corners,
      colback=problemBackground
    ]
    \ifx &#1& \textbf{Problem. }
    \else \textbf{Problem #1.} \fi
  }
  {
    \end{tcolorbox}
  }

\definecolor{exampleBackground}{RGB}{255,249,248}
\definecolor{exampleAccent}{RGB}{158,60,14}
\newenvironment{example}[1]
  {
    \begin{tcolorbox}[
      boxrule=.5pt,
      sharp corners,
      colback=exampleBackground,
      colframe=exampleAccent,
    ]
    \color{exampleAccent}\textbf{Example.} \emph{#1}\color{black}
  }
  {
    \end{tcolorbox}
  }

\definecolor{theoremBackground}{RGB}{234,243,251}
\definecolor{theoremAccent}{RGB}{0,116,183}
\newenvironment{theorem}[1]
  {
    \begin{tcolorbox}[
      boxrule=.5pt,
      titlerule=.5pt,
      sharp corners,
      colback=theoremBackground,
      colframe=theoremAccent
    ]
      \color{theoremAccent}\textbf{Theorem --- }\emph{#1}\\\color{black}
  }
  {
    \end{tcolorbox}
  }

\definecolor{noteBackground}{RGB}{244,249,244}
\definecolor{noteAccent}{RGB}{34,139,34}
\newenvironment{note}[1]
  {
  \begin{tcolorbox}[
    enhanced,
    boxrule=0pt,
    frame hidden,
    sharp corners,
    colback=noteBackground,
    borderline west={3pt}{-1.5pt}{noteAccent}
    ]
    \ifx &#1& \color{noteAccent}\textbf{Note. }\color{black}
    \else \color{noteAccent}\textbf{Note (#1). }\color{black} \fi
    }
    {
  \end{tcolorbox}
  }

\definecolor{definitionBackground}{RGB}{246,246,246}
\newenvironment{definition}[1]
  {
    \begin{tcolorbox}[
      enhanced,
      boxrule=0pt,
      frame hidden,
      sharp corners,
      colback=definitionBackground,
      borderline west={3pt}{-1.5pt}{black}
    ]
    \textbf{Definition. }\emph{#1}\\
  }
  {
    \end{tcolorbox}
  }
\newenvironment{amatrix}[2]{
    \left[
      \begin{array}{*{#1}{c}|*{#2}c}
  }
  {
      \end{array}
    \right]
  }
\date{\the\year-\the\month-\the\day}
\author{Kyle Chui}


\fancyhf{}
\lhead{Kyle Chui}
\rhead{Page \thepage}
\pagestyle{fancy}

\begin{document}
  \section{Lecture 17}
  \subsection{Ancient Indian Mathematics}
  They had a decimal place-value system, and the Hindu-Arabic numerals are the direct predecessors to our current numeral system. The oldest Indian Mathematical text that we know of is the ``Sulba Sutras'', circa 8\tsup{th} century BCE.
  \begin{itemize}
    \item It was a Hinduism-related text (consisting of several books written by different authors) that describes geometry related to fire altar constructions
    \item Described the Pythagorean Theorem
    \item Lists many Pythagorean triples
    \item Not unlikely that Pythagoras learned his theorem from this text
    \item Contains geometric solutions to linear and quadratic equations
    \item Contains an approximation of $\sqrt{2}$, correct to 5 decimals
  \end{itemize}
  Other notions:
  \begin{itemize}
    \item They worked with absurdly large numbers, up to 10\tsup{421}
    \item They were not scared of working with infinities, both geometric and numerical
  \end{itemize}
  \subsubsection{The Golden Age}
  The period of 500--1200 CE is called the Golden Age of Indian Mathematics.
  \begin{itemize}
    \item Many geometric and astronomical advancements
    \begin{itemize}
      \item Precise calculations of sine and cosine for many angles
      \item Used this to establish that the Sun is 400 times further away from the Earth than the Moon
      \item An astronomer used approximations of $\pi$ to estimate the circumference of the Earth to within 70 miles of its actual value
    \end{itemize}
    \item Brahmagupta  (598--668 CE)
    \begin{itemize}
      \item Born in Bhinmal, northwestern India
      \item The number zero is usually attributed to him
      \item He was the first to give rules for calculations with zero as a number, and with negative numbers (``debts'')
      \item He abstracted the notion of numbers, and perhaps is the first to allow negative solutions to the quadratic
      \item He established ``Brahmagupta's Formula'' for the area of a cyclic quadrilateral
      \item He was probably the first to study ``rational triangles''
      \item He used the initials of names of colors for unknown values
    \end{itemize}
    \item Bhaskara II (1114--1185 CE)
    \begin{itemize}
      \item Born in Bijapur, southwestern India
      \item Famous astronomer and mathematician
      \item Led a cosmic observatory at Ujjain, one of the main mathematical centers of ancient India
      \item Nicknamed ``greatest mathematician of medieval India''
      \item Established solutions to quadratic, cubic, quartic equations
      \item First general method for finding integer solutions to $x^2 - ny^2 = 1$, the so-called ``Pell's equation''
      \item Solved general quadratic equations with more than one unknown, including negative and irrational solutions
      \item The first to give the meaning ``infinity'' to $\frac{1}{0}$, by observing that $\lim\limits_{n\to 0^+} \frac{1}{n}\to\infty$
      \item Had preliminary concepts of differential and integral calculus
      \item Calculated derivatives and approximations of many functions
      \item Knew about ``Rolle's Theorem''
      \item Knew that at a point where a function reaches an extremum, the derivative is zero
      \item Worked with the concept of ``infinitesimals''
      \item Because of the above, you can think of him as the world's first ``analyst''
    \end{itemize}
  \end{itemize}
\end{document}
