\documentclass[class=article, crop=false]{standalone}
\usepackage[margin=1in]{geometry}

\usepackage{amsmath}
\usepackage{amssymb}
\usepackage{amsthm}
\usepackage{comment}
\usepackage{enumitem}
\usepackage{fancyhdr}
\usepackage{hyperref}
\usepackage{mathrsfs}
\usepackage{mathtools}
\usepackage{standalone}
\usepackage{tcolorbox}
\usepackage{tikz}
\usepackage{xcolor}

\usetikzlibrary{decorations.pathreplacing}
\tcbuselibrary{skins}

\DeclareMathOperator{\lcm}{lcm}
\DeclareMathOperator{\proj}{proj}
\DeclareMathOperator{\vspan}{span}
\DeclareMathOperator{\im}{im}
\DeclareMathOperator{\range}{range}
\DeclareMathOperator{\Diff}{Diff}
\DeclareMathOperator{\Int}{Int}
\DeclareMathOperator{\fcn}{fcn}
\DeclareMathOperator{\id}{id}

\newcommand{\N}{\ensuremath{\mathbb{N}}}
\newcommand{\Z}{\ensuremath{\mathbb{Z}}}
\newcommand{\Q}{\ensuremath{\mathbb{Q}}}
\newcommand{\R}{\ensuremath{\mathbb{R}}}
\newcommand{\C}{\ensuremath{\mathbb{C}}}
\newcommand{\F}{\ensuremath{\mathbb{F}}}
\newcommand{\lam}{\ensuremath{\lambda}}
\newcommand{\nab}{\ensuremath{\nabla}}
\newcommand{\eps}{\ensuremath{\varepsilon}}
\newcommand{\es}{\ensuremath{\varnothing}}

\newcommand{\abs}[1]{\ensuremath{\left\lvert #1 \right\rvert}}
\newcommand{\bigpar}[1]{\ensuremath{\left( #1 \right)}}
\newcommand{\norm}[1]{\ensuremath{\lVert #1\rVert}}
\newcommand{\set}[1]{\ensuremath{\left\{#1\right\}}}
\newcommand{\tuple}[1]{\ensuremath{\left\langle #1 \right\rangle}}
\newcommand{\floor}[1]{\ensuremath{\left\lfloor #1 \right\rfloor}}
\newcommand{\ceil}[1]{\ensuremath{\left\lceil #1 \right\rceil}}

\newcommand{\chapternum}{}
\newcommand{\ex}[1]{\noindent\textbf{Exercise \chapternum.{#1}.}}

\newcommand{\dx}[1]{\,\mathrm{d}#1}
\newcommand{\inv}{\ensuremath{^{-1}}}
\newcommand{\sm}{\setminus}
\newcommand{\sse}{\subseteq}
\newcommand{\ceq}{\coloneqq}

\definecolor{problemBackground}{RGB}{212,232,246}
\newenvironment{problem}[1]
  {
    \begin{tcolorbox}[
      boxrule=.5pt,
      titlerule=.5pt,
      sharp corners,
      colback=problemBackground
    ]
    \ifx &#1& \textbf{Problem. }
    \else \textbf{Problem #1.} \fi
  }
  {
    \end{tcolorbox}
  }

\definecolor{exampleBackground}{RGB}{255,249,248}
\definecolor{exampleAccent}{RGB}{158,60,14}
\newenvironment{example}[1]
  {
    \begin{tcolorbox}[
      boxrule=.5pt,
      sharp corners,
      colback=exampleBackground,
      colframe=exampleAccent,
    ]
    \color{exampleAccent}\textbf{Example.} \emph{#1}\color{black}
  }
  {
    \end{tcolorbox}
  }

\definecolor{theoremBackground}{RGB}{234,243,251}
\definecolor{theoremAccent}{RGB}{0,116,183}
\newenvironment{theorem}[1]
  {
    \begin{tcolorbox}[
      boxrule=.5pt,
      titlerule=.5pt,
      sharp corners,
      colback=theoremBackground,
      colframe=theoremAccent
    ]
      \color{theoremAccent}\textbf{Theorem --- }\emph{#1}\\\color{black}
  }
  {
    \end{tcolorbox}
  }

\definecolor{noteBackground}{RGB}{244,249,244}
\definecolor{noteAccent}{RGB}{34,139,34}
\newenvironment{note}[1]
  {
  \begin{tcolorbox}[
    enhanced,
    boxrule=0pt,
    frame hidden,
    sharp corners,
    colback=noteBackground,
    borderline west={3pt}{-1.5pt}{noteAccent}
    ]
    \ifx &#1& \color{noteAccent}\textbf{Note. }\color{black}
    \else \color{noteAccent}\textbf{Note (#1). }\color{black} \fi
    }
    {
  \end{tcolorbox}
  }

\definecolor{definitionBackground}{RGB}{246,246,246}
\newenvironment{definition}[1]
  {
    \begin{tcolorbox}[
      enhanced,
      boxrule=0pt,
      frame hidden,
      sharp corners,
      colback=definitionBackground,
      borderline west={3pt}{-1.5pt}{black}
    ]
    \textbf{Definition. }\emph{#1}\\
  }
  {
    \end{tcolorbox}
  }
\newenvironment{amatrix}[2]{
    \left[
      \begin{array}{*{#1}{c}|*{#2}c}
  }
  {
      \end{array}
    \right]
  }
\date{\the\year-\the\month-\the\day}
\author{Kyle Chui}


\fancyhf{}
\lhead{Kyle Chui}
\rhead{Page \thepage}
\pagestyle{fancy}

\begin{document}
  \section{Lecture 20}
  \subsection{Conditional Expectation}
  \begin{definition}{}
    Let $X, Y$ be a pair of discrete random variables taking values in sets $S_X, S_Y\sse\R$.
    \begin{itemize}
      \item If $y\in S_Y$, we define the \emph{random variable} $X\mid y$ with PMF
      \begin{align*}
        p_{X\mid Y}(x\mid y) &= \P[X = x\mid Y = y]\quad\text{for}\quad x\in S_X \\
                             &= \frac{p_{X, Y}(x, y)}{p_Y(y)}
      \end{align*}
      \item If $x\in S_X$, we define the \emph{random variable} $Y\mid x$ with PMF
      \begin{align*}
        p_{Y\mid X}(y\mid x) &= \P[Y = y\mid X = x]\quad\text{for}\quad y\in S_Y \\
                             &= \frac{p_{X, Y}(x, y)}{p_X(x)}
      \end{align*}
    \end{itemize}
  \end{definition}
  \begin{theorem}{}
    Let $X, Y$ be discrete random variables taking values in sets $S_X, S_Y\sse\R$.
    \begin{itemize}
      \item Given $y\in S_Y$ we have
      \[
        \sum_{x\in S_X}p_{X\mid Y}(x\mid y) = 1.
      \]
      \item Given $x\in S_X$ we have
      \[
        \sum_{y\in S_Y}p_{Y\mid X}(y\mid x) = 1.
      \]
    \end{itemize}
    \begin{proof}
      We shall just prove the first case, computing that
      \begin{align*}
        \sum_{x\in S_X}p_{X\mid Y}(x\mid y) &= \sum_{x\in S_X} \frac{p_{X, Y}(x, y)}{p_Y(y)} \\
                                            &= \frac{1}{p_Y(y)}\sum_{x\in S_X}p_{X, Y}(x, y) \\
                                            &= \frac{1}{p_Y(y)}\cdot p_Y(y) \\
                                            &= 1.
      \end{align*}
    \end{proof}
  \end{theorem}
  \begin{definition}{Conditional Expectation}
    Let $X, Y$ be discrete random variables taking values in sets $S_X, S_Y\sse\R$.
    \begin{itemize}
      \item Define the function $g\colon S_X\to \R$ by 
      \[
        g(x) = \E[Y\mid x].
      \]
      \item We define the \emph{conditional expectation} of $Y$ conditioned on $X$ to be the \emph{random variable}
      \[
        \E[Y\mid X] = g(X).
      \]
      \item We can define $\E[X\mid Y]$ in a similar fashion.
    \end{itemize}
  \end{definition}
  \begin{theorem}{Law of Iterated Expectation}
    Let $X, Y$ be discrete random variables. Then
    \[
      \E[\E[Y\mid X]] = \E[Y].
    \]
    \begin{proof}
      Let $g(x) = \E[Y\mid x]$ so $\E[Y\mid X] = g(X)$. Then
      \begin{align*}
        \E[\E[Y\mid X]] &= \E[g(X)] \\
                        &= \sum_{x\in S_X}g(x)p_X(x) \\
                        &= \sum_{x\in S_X} \paren{\sum_{y\in S_Y}y\cdot \frac{p_{X, Y}(x, y)}{p_X(x)}\cdot p_X(x)} \\
                        &= \sum_{y\in S_Y}\sum_{x\in S_X}y p_{X, Y}(x, y) \\
                        &= \sum_{y\in S_Y}y\cdot p_Y(y) \\
                        &= \E[Y].
      \end{align*}
    \end{proof}
  \end{theorem}
  \begin{definition}{Conditional Variance}
    Let $X, Y$ be discrete random variables taking values in sets $S_X, S_Y\sse\R$.
    \begin{itemize}
      \item Define the function $h\colon S_X\to \R$ by
      \[
        h(x) = \var(Y\mid x).
      \]
      \item We define the \emph{conditional variance} of $Y$ conditioned on $X$ to be the \emph{random variable}
      \[
        \var(Y\mid X) = h(X).
      \]
      \item We can define $\var(X\mid Y)$ in a similar fashion.
    \end{itemize}
  \end{definition}
  \begin{theorem}{Law of Total Variance}
    Let $X, Y$ be discrete random variables. Then
    \[
      \E[\var(Y\mid X)] + \var(\E[Y\mid X]) = \var(Y).
    \]
    \begin{proof}
      Let $g(x) = \E[Y\mid x]$ and $h(x) = \var(Y\mid x)$. We can think of these functions by the following:
      \begin{align*}
        g(x) &= \sum_{y\in S_Y} y\cdot p_{Y\mid X}(y\mid x) \\
        h(x) &= \sum_{y\in S_Y} y^2\cdot p_{Y\mid X}(y\mid x) - g(x)^2.
      \end{align*}
      Hence we have
      \begin{align*}
        \E[\var(Y\mid X)] &= \E[h(X)] \\
                          &= \sum_{x\in S_X}h(x)p_X(x) \\
                          &= \sum_{x\in S_X}\sum_{y\in S_Y}y^2\cdot p_{Y\mid X}(y\mid x)p_X(x) - \sum_{x\in S_X}g(x)^2p_X(x) \\
                          &= \sum_{x\in S_X}\sum_{y\in S_Y}y^2p_{X, Y}(x, y) - \sum_{x\in S_X}g(x)^2p_X(x) \\
                          &= \E[Y^2] - \E[g(X)^2].
      \end{align*}
      Furthermore, we have
      \begin{align*}
        \var(\E[Y\mid X]) &= \var(g(X)) \\
                          &= \E[g(X)^2] - \E[g(X)]^2 \\
                          &= \E[g(X)^2] - \E[\E[Y\mid X]]^2 \\
                          &= \E[g(X)^2] - \E[Y]^2.
      \end{align*}
      Summing these two expressions, we get $\E[Y^2] - \E[Y]^2 = \var(Y)$.
    \end{proof}
  \end{theorem}
\end{document}
