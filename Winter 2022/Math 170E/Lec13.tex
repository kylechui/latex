\documentclass[class=article, crop=false]{standalone}
% Import packages
\usepackage[margin=1in]{geometry}

\usepackage{amssymb, amsthm}
\usepackage{comment}
\usepackage{enumitem}
\usepackage{fancyhdr}
\usepackage{hyperref}
\usepackage{import}
\usepackage{mathrsfs, mathtools}
\usepackage{multicol}
\usepackage{pdfpages}
\usepackage{standalone}
\usepackage{transparent}
\usepackage{xcolor}

\usepackage{minted}
\usepackage[many]{tcolorbox}
\tcbuselibrary{minted}

% Declare math operators
\DeclareMathOperator{\lcm}{lcm}
\DeclareMathOperator{\proj}{proj}
\DeclareMathOperator{\vspan}{span}
\DeclareMathOperator{\im}{im}
\DeclareMathOperator{\Diff}{Diff}
\DeclareMathOperator{\Int}{Int}
\DeclareMathOperator{\fcn}{fcn}
\DeclareMathOperator{\id}{id}
\DeclareMathOperator{\rank}{rank}
\DeclareMathOperator{\tr}{tr}
\DeclareMathOperator{\dive}{div}
\DeclareMathOperator{\row}{row}
\DeclareMathOperator{\col}{col}
\DeclareMathOperator{\dom}{dom}
\DeclareMathOperator{\ran}{ran}
\DeclareMathOperator{\Dom}{Dom}
\DeclareMathOperator{\Ran}{Ran}
\DeclareMathOperator{\fl}{fl}
% Macros for letters/variables
\newcommand\ttilde{\kern -.15em\lower .7ex\hbox{\~{}}\kern .04em}
\newcommand{\N}{\ensuremath{\mathbb{N}}}
\newcommand{\Z}{\ensuremath{\mathbb{Z}}}
\newcommand{\Q}{\ensuremath{\mathbb{Q}}}
\newcommand{\R}{\ensuremath{\mathbb{R}}}
\newcommand{\C}{\ensuremath{\mathbb{C}}}
\newcommand{\F}{\ensuremath{\mathbb{F}}}
\newcommand{\M}{\ensuremath{\mathbb{M}}}
\renewcommand{\P}{\ensuremath{\mathbb{P}}}
\newcommand{\lam}{\ensuremath{\lambda}}
\newcommand{\nab}{\ensuremath{\nabla}}
\newcommand{\eps}{\ensuremath{\varepsilon}}
\newcommand{\es}{\ensuremath{\varnothing}}
% Macros for math symbols
\newcommand{\dx}[1]{\,\mathrm{d}#1}
\newcommand{\inv}{\ensuremath{^{-1}}}
\newcommand{\sm}{\setminus}
\newcommand{\sse}{\subseteq}
\newcommand{\ceq}{\coloneqq}
% Macros for pairs of math symbols
\newcommand{\abs}[1]{\ensuremath{\left\lvert #1 \right\rvert}}
\newcommand{\paren}[1]{\ensuremath{\left( #1 \right)}}
\newcommand{\norm}[1]{\ensuremath{\left\lVert #1\right\rVert}}
\newcommand{\set}[1]{\ensuremath{\left\{#1\right\}}}
\newcommand{\tup}[1]{\ensuremath{\left\langle #1 \right\rangle}}
\newcommand{\floor}[1]{\ensuremath{\left\lfloor #1 \right\rfloor}}
\newcommand{\ceil}[1]{\ensuremath{\left\lceil #1 \right\rceil}}
\newcommand{\eclass}[1]{\ensuremath{\left[ #1 \right]}}

\newcommand{\chapternum}{}
\newcommand{\ex}[1]{\noindent\textbf{Exercise \chapternum.{#1}.}}

\newcommand{\tsub}[1]{\textsubscript{#1}}
\newcommand{\tsup}[1]{\textsuperscript{#1}}

\renewcommand{\choose}[2]{\ensuremath{\phantom{}_{#1}C_{#2}}}
\newcommand{\perm}[2]{\ensuremath{\phantom{}_{#1}P_{#2}}}

% Include figures
\newcommand{\incfig}[2][1]{%
    \def\svgwidth{#1\columnwidth}
    \import{./figures/}{#2.pdf_tex}
}

\definecolor{problemBackground}{RGB}{212,232,246}
\newenvironment{problem}[1]
  {
    \begin{tcolorbox}[
      boxrule=.5pt,
      titlerule=.5pt,
      sharp corners,
      colback=problemBackground,
      breakable
    ]
    \ifx &#1& \textbf{Problem. }
    \else \textbf{Problem #1.} \fi
  }
  {
    \end{tcolorbox}
  }
\definecolor{exampleBackground}{RGB}{255,249,248}
\definecolor{exampleAccent}{RGB}{158,60,14}
\newenvironment{example}[1]
  {
    \begin{tcolorbox}[
      boxrule=.5pt,
      sharp corners,
      colback=exampleBackground,
      colframe=exampleAccent,
    ]
    \color{exampleAccent}\textbf{Example.} \emph{#1}\color{black}
  }
  {
    \end{tcolorbox}
  }
\definecolor{theoremBackground}{RGB}{234,243,251}
\definecolor{theoremAccent}{RGB}{0,116,183}
\newenvironment{theorem}[1]
  {
    \begin{tcolorbox}[
      boxrule=.5pt,
      titlerule=.5pt,
      sharp corners,
      colback=theoremBackground,
      colframe=theoremAccent,
      breakable
    ]
      % \color{theoremAccent}\textbf{Theorem --- }\emph{#1}\\\color{black}
    \ifx &#1& \color{theoremAccent}\textbf{Theorem. }\color{black}
    \else \color{theoremAccent}\textbf{Theorem --- }\emph{#1}\\\color{black} \fi
  }
  {
    \end{tcolorbox}
  }
\definecolor{noteBackground}{RGB}{244,249,244}
\definecolor{noteAccent}{RGB}{34,139,34}
\newenvironment{note}[1]
  {
  \begin{tcolorbox}[
    enhanced,
    boxrule=0pt,
    frame hidden,
    sharp corners,
    colback=noteBackground,
    borderline west={3pt}{-1.5pt}{noteAccent},
    breakable
    ]
    \ifx &#1& \color{noteAccent}\textbf{Note. }\color{black}
    \else \color{noteAccent}\textbf{Note (#1). }\color{black} \fi
    }
    {
  \end{tcolorbox}
  }
\definecolor{lemmaBackground}{RGB}{255,247,234}
\definecolor{lemmaAccent}{RGB}{255,153,0}
\newenvironment{lemma}[1]
  {
    \begin{tcolorbox}[
      enhanced,
      boxrule=0pt,
      frame hidden,
      sharp corners,
      colback=lemmaBackground,
      borderline west={3pt}{-1.5pt}{lemmaAccent},
      breakable
    ]
    \ifx &#1& \color{lemmaAccent}\textbf{Lemma. }\color{black}
    \else \color{lemmaAccent}\textbf{Lemma #1. }\color{black} \fi
  }
  {
    \end{tcolorbox}
  }
\definecolor{definitionBackground}{RGB}{246,246,246}
\newenvironment{definition}[1]
  {
    \begin{tcolorbox}[
      enhanced,
      boxrule=0pt,
      frame hidden,
      sharp corners,
      colback=definitionBackground,
      borderline west={3pt}{-1.5pt}{black},
      breakable
    ]
    \textbf{Definition. }\emph{#1}\\
  }
  {
    \end{tcolorbox}
  }

\newenvironment{amatrix}[2]{
    \left[
      \begin{array}{*{#1}{c}|*{#2}c}
  }
  {
      \end{array}
    \right]
  }

\definecolor{codeBackground}{RGB}{253,246,227}
\renewcommand\theFancyVerbLine{\arabic{FancyVerbLine}}
\newtcblisting{cpp}[1][]{
  boxrule=1pt,
  sharp corners,
  colback=codeBackground,
  listing only,
  breakable,
  minted language=cpp,
  minted style=material,
  minted options={
    xleftmargin=15pt,
    baselinestretch=1.2,
    linenos,
  }
}

\author{Kyle Chui}


\fancyhf{}
\lhead{Kyle Chui}
\rhead{Page \thepage}
\pagestyle{fancy}

\begin{document}
  \section{Lecture 13}
  \subsection{Uniform Distribution on an Interval}
  \begin{definition}{Uniform Distribution}
    Let $a < b$. I pick a point $X$ at random in the interval $[a, b]$. If I have an equal probability of picking every point in $[a, b]$, we say that $X$ is \emph{uniformly distributed on the interval $[a, b]$}. We write $X\sim \Uniform([a, b])$.
  \end{definition}
  \begin{theorem}{}
    If $a < b$ and $X\sim\Uniform([a, b])$ then it has PDF
    \[
      f_X(x) = \begin{cases} \frac{1}{b-a} & \text{if}\quad x\in (a, b)\\0 & \text{otherwise}\end{cases}
    \]
    \begin{proof}
      We know that $\P[X \leq x]$ should be 0 if $x \leq a$ and 1 if $x > b$. If $a < x \leq b$ then $\P[X \leq x] = \frac{x - a}{b - a}$. Hence
      \[
        f_X(x) = F_X'(x) = \frac{1}{b - a}\quad \text{for } a < x < b.
      \]
    \end{proof}
  \end{theorem}
  \subsection{Expected Value of a Continuous Random Variable}
  \textbf{Idea.} In order to find the expected value of a continuous random variable, we first cut up the interval into $n$ pieces, and then take the limit of the approximate discrete random variable as $n\to \infty$. \par
  Let $n \geq 1$ and define $X_n$ to be the discrete random variable taking values in the set $\set{0, \pm \frac{1}{n}, \pm \frac{2}{n},\dotsc}$ with PMF
  \[
    P_{X_n}(x) = \int_{\frac{j - 1}{n}}^{\frac{j}{n}}f_X(x) \,\mathrm dx\quad\text{if}\quad x = \frac{j}{n}.
  \]
  We can see that this is well-defined because when we add up $p_{X_n}(x)$ for all $x\in S$, we get the integral over the reals of $f_X(x)$, which yields 1.
  \begin{theorem}{}
    We have
    \[
      \E[X_n] \to \int_{-\infty}^{\infty}xf_X(x) \,\mathrm dx.
    \]
    \begin{proof}
      We compute that
      \begin{align*}
        \E[X_n] &= \sum_{x\in S} xp_{X_n}(x) \\
                &= \sum_{j=-\infty}^{\infty} \frac{j}{n}p_{X_n}\paren{\frac{j}{n}} \\
                &= \sum_{j=-\infty}^{\infty} \frac{j}{n} \int_{\frac{j - 1}{n}}^{\frac{j}{n}}f_X(x) \,\mathrm dx.
      \end{align*}
      Observe that if $x\in [\frac{j - 1}{n}, \frac{j}{n}]$, then $x \leq \frac{j}{n} \leq x + \frac{1}{n}$. Hence
      \begin{gather*}
        \sum_{j=-\infty}^{\infty} \int_{\frac{j - 1}{n}}^{\frac{j}{n}}xf_X(x) \,\mathrm dx \leq \sum_{j=-\infty}^{\infty} \int_{\frac{j - 1}{n}}^{\frac{j}{n}}\frac{j}{n}f_X(x) \,\mathrm dx \leq \sum_{j=-\infty}^{\infty} \int_{\frac{j - 1}{n}}^{\frac{j}{n}}\paren{x + \frac{1}{n}}f_X(x) \,\mathrm dx \\
        \int_{-\infty}^{\infty}xf_X(x) \,\mathrm dx \leq \E[X_n] \leq \int_{-\infty}^{\infty}\paren{x + \frac{1}{n}}f_X(x) \,\mathrm dx
      \end{gather*}
      If we take $n\to\infty$ and apply Squeeze Theorem, we have
      \[
        \lim_{n\to \infty} \E[X_n] = \int_{-\infty}^{\infty}xf_X(x) \,\mathrm dx.
      \]
    \end{proof}
  \end{theorem}
  \begin{definition}{Expected Value}
    If $X$ is a continuous random variable with PDF $f_X(x)$ we define its \emph{expected value} to be
    \[
      \E[X] = \int_{-\infty}^{\infty}xf_X(x) \,\mathrm dx.
    \]
    We still use the notation $\mu_X = \E[X]$. More generally, if $g\colon \R\to \R$ is any function, we define
    \[
      \E[g(X)] = \int_{-\infty}^{\infty}g(x)f_X(x) \,\mathrm dx.
    \]
  \end{definition}
  \begin{theorem}{}
    Let $X$ be a continuous random variable.
    \begin{itemize}
      \item If $a\in\R$ is a constant then
      \[
        \E[a] = a.
      \]
      \item If $a, b\in\R$ are constants and $g, h\colon \R\to \R$ then
      \[
        \E[ag(X) + bh(X)] = a\E[g(X)] + b\E[h(X)].
      \]
      \item If $g(x) \leq h(x)$ for all $x\in\R$ then
      \[
        \E[g(X)] \leq \E[h(X)].
      \]
    \end{itemize}
  \end{theorem}
\end{document}
