\documentclass[class=article, crop=false]{standalone}
% Import packages
\usepackage[margin=1in]{geometry}

\usepackage{amssymb, amsthm}
\usepackage{comment}
\usepackage{enumitem}
\usepackage{fancyhdr}
\usepackage{hyperref}
\usepackage{import}
\usepackage{mathrsfs, mathtools}
\usepackage{multicol}
\usepackage{pdfpages}
\usepackage{standalone}
\usepackage{transparent}
\usepackage{xcolor}

\usepackage{minted}
\usepackage[many]{tcolorbox}
\tcbuselibrary{minted}

% Declare math operators
\DeclareMathOperator{\lcm}{lcm}
\DeclareMathOperator{\proj}{proj}
\DeclareMathOperator{\vspan}{span}
\DeclareMathOperator{\im}{im}
\DeclareMathOperator{\Diff}{Diff}
\DeclareMathOperator{\Int}{Int}
\DeclareMathOperator{\fcn}{fcn}
\DeclareMathOperator{\id}{id}
\DeclareMathOperator{\rank}{rank}
\DeclareMathOperator{\tr}{tr}
\DeclareMathOperator{\dive}{div}
\DeclareMathOperator{\row}{row}
\DeclareMathOperator{\col}{col}
\DeclareMathOperator{\dom}{dom}
\DeclareMathOperator{\ran}{ran}
\DeclareMathOperator{\Dom}{Dom}
\DeclareMathOperator{\Ran}{Ran}
\DeclareMathOperator{\fl}{fl}
% Macros for letters/variables
\newcommand\ttilde{\kern -.15em\lower .7ex\hbox{\~{}}\kern .04em}
\newcommand{\N}{\ensuremath{\mathbb{N}}}
\newcommand{\Z}{\ensuremath{\mathbb{Z}}}
\newcommand{\Q}{\ensuremath{\mathbb{Q}}}
\newcommand{\R}{\ensuremath{\mathbb{R}}}
\newcommand{\C}{\ensuremath{\mathbb{C}}}
\newcommand{\F}{\ensuremath{\mathbb{F}}}
\newcommand{\M}{\ensuremath{\mathbb{M}}}
\renewcommand{\P}{\ensuremath{\mathbb{P}}}
\newcommand{\lam}{\ensuremath{\lambda}}
\newcommand{\nab}{\ensuremath{\nabla}}
\newcommand{\eps}{\ensuremath{\varepsilon}}
\newcommand{\es}{\ensuremath{\varnothing}}
% Macros for math symbols
\newcommand{\dx}[1]{\,\mathrm{d}#1}
\newcommand{\inv}{\ensuremath{^{-1}}}
\newcommand{\sm}{\setminus}
\newcommand{\sse}{\subseteq}
\newcommand{\ceq}{\coloneqq}
% Macros for pairs of math symbols
\newcommand{\abs}[1]{\ensuremath{\left\lvert #1 \right\rvert}}
\newcommand{\paren}[1]{\ensuremath{\left( #1 \right)}}
\newcommand{\norm}[1]{\ensuremath{\left\lVert #1\right\rVert}}
\newcommand{\set}[1]{\ensuremath{\left\{#1\right\}}}
\newcommand{\tup}[1]{\ensuremath{\left\langle #1 \right\rangle}}
\newcommand{\floor}[1]{\ensuremath{\left\lfloor #1 \right\rfloor}}
\newcommand{\ceil}[1]{\ensuremath{\left\lceil #1 \right\rceil}}
\newcommand{\eclass}[1]{\ensuremath{\left[ #1 \right]}}

\newcommand{\chapternum}{}
\newcommand{\ex}[1]{\noindent\textbf{Exercise \chapternum.{#1}.}}

\newcommand{\tsub}[1]{\textsubscript{#1}}
\newcommand{\tsup}[1]{\textsuperscript{#1}}

\renewcommand{\choose}[2]{\ensuremath{\phantom{}_{#1}C_{#2}}}
\newcommand{\perm}[2]{\ensuremath{\phantom{}_{#1}P_{#2}}}

% Include figures
\newcommand{\incfig}[2][1]{%
    \def\svgwidth{#1\columnwidth}
    \import{./figures/}{#2.pdf_tex}
}

\definecolor{problemBackground}{RGB}{212,232,246}
\newenvironment{problem}[1]
  {
    \begin{tcolorbox}[
      boxrule=.5pt,
      titlerule=.5pt,
      sharp corners,
      colback=problemBackground,
      breakable
    ]
    \ifx &#1& \textbf{Problem. }
    \else \textbf{Problem #1.} \fi
  }
  {
    \end{tcolorbox}
  }
\definecolor{exampleBackground}{RGB}{255,249,248}
\definecolor{exampleAccent}{RGB}{158,60,14}
\newenvironment{example}[1]
  {
    \begin{tcolorbox}[
      boxrule=.5pt,
      sharp corners,
      colback=exampleBackground,
      colframe=exampleAccent,
    ]
    \color{exampleAccent}\textbf{Example.} \emph{#1}\color{black}
  }
  {
    \end{tcolorbox}
  }
\definecolor{theoremBackground}{RGB}{234,243,251}
\definecolor{theoremAccent}{RGB}{0,116,183}
\newenvironment{theorem}[1]
  {
    \begin{tcolorbox}[
      boxrule=.5pt,
      titlerule=.5pt,
      sharp corners,
      colback=theoremBackground,
      colframe=theoremAccent,
      breakable
    ]
      % \color{theoremAccent}\textbf{Theorem --- }\emph{#1}\\\color{black}
    \ifx &#1& \color{theoremAccent}\textbf{Theorem. }\color{black}
    \else \color{theoremAccent}\textbf{Theorem --- }\emph{#1}\\\color{black} \fi
  }
  {
    \end{tcolorbox}
  }
\definecolor{noteBackground}{RGB}{244,249,244}
\definecolor{noteAccent}{RGB}{34,139,34}
\newenvironment{note}[1]
  {
  \begin{tcolorbox}[
    enhanced,
    boxrule=0pt,
    frame hidden,
    sharp corners,
    colback=noteBackground,
    borderline west={3pt}{-1.5pt}{noteAccent},
    breakable
    ]
    \ifx &#1& \color{noteAccent}\textbf{Note. }\color{black}
    \else \color{noteAccent}\textbf{Note (#1). }\color{black} \fi
    }
    {
  \end{tcolorbox}
  }
\definecolor{lemmaBackground}{RGB}{255,247,234}
\definecolor{lemmaAccent}{RGB}{255,153,0}
\newenvironment{lemma}[1]
  {
    \begin{tcolorbox}[
      enhanced,
      boxrule=0pt,
      frame hidden,
      sharp corners,
      colback=lemmaBackground,
      borderline west={3pt}{-1.5pt}{lemmaAccent},
      breakable
    ]
    \ifx &#1& \color{lemmaAccent}\textbf{Lemma. }\color{black}
    \else \color{lemmaAccent}\textbf{Lemma #1. }\color{black} \fi
  }
  {
    \end{tcolorbox}
  }
\definecolor{definitionBackground}{RGB}{246,246,246}
\newenvironment{definition}[1]
  {
    \begin{tcolorbox}[
      enhanced,
      boxrule=0pt,
      frame hidden,
      sharp corners,
      colback=definitionBackground,
      borderline west={3pt}{-1.5pt}{black},
      breakable
    ]
    \textbf{Definition. }\emph{#1}\\
  }
  {
    \end{tcolorbox}
  }

\newenvironment{amatrix}[2]{
    \left[
      \begin{array}{*{#1}{c}|*{#2}c}
  }
  {
      \end{array}
    \right]
  }

\definecolor{codeBackground}{RGB}{253,246,227}
\renewcommand\theFancyVerbLine{\arabic{FancyVerbLine}}
\newtcblisting{cpp}[1][]{
  boxrule=1pt,
  sharp corners,
  colback=codeBackground,
  listing only,
  breakable,
  minted language=cpp,
  minted style=material,
  minted options={
    xleftmargin=15pt,
    baselinestretch=1.2,
    linenos,
  }
}

\author{Kyle Chui}


\fancyhf{}
\lhead{Kyle Chui}
\rhead{Page \thepage}
\pagestyle{fancy}

\begin{document}
  \section{Lecture 12}
  To verify that the PMF that we found last lecture is valid, we observe that
  \begin{align*}
    \sum_{x=0}^{\infty} p_X(x) &= \sum_{x=0}^{\infty} \frac{\lam^x}{x!}e^{-\lam} \\
                               &= e^{-\lam} \sum_{x=0}^{\infty} \frac{\lam^x}{x!} \\
                               &= e^{-\lam}e^\lam \\
                               &= 1.
  \end{align*}
  \begin{theorem}{}
    Consider an approximate Poisson process with rate $\lam > 0$ per unit time. Let $X$ be the number of arrivals in a time interval of length $T > 0$ units. Then $X\sim \Poisson(\lam T)$.
    \begin{proof}
      As before, we cut up our time interval $T$ into $n$ sub-intervals, so $X\approx \Binomial(n, \frac{\lam T}{n})$. When we take $n\to\infty$, then we have $p_X(x) = e^{-\lam T} \frac{(\lam T)^x}{x!}$, so $X\sim \Poisson(\lam T)$.
    \end{proof}
  \end{theorem}
  \begin{example}{}
    \begin{itemize}
      \item I receive phone notifications according to an approximate Poisson process with rate $\frac{1}{15}$ notifications per minute.
      \item What is the probability that I receive at least one notification in an hour?
    \end{itemize}
    Let $X$ be the number of notifications I receive in an hour. Hence $X\sim \Poisson(4)$. Therefore $p_X(x \geq 1) = 1 - p_X(0) = 1 - e^{-4}$.
  \end{example}
  \begin{theorem}{}
    If $\lam > 0$ and $X\sim \Poisson(\lam)$ then its MGF is
    \[
      M_X(t) = e^{\lam(e^t - 1)}.
    \]
    \begin{proof}
      We compute
      \begin{align*}
        M_X(t) &= \E[e^{tX}] \\
               &= \sum_{x=0}^{\infty} e^{tx}p_X(x) \\
               &= \sum_{x=0}^{\infty} e^{tx} e^{-\lam}\cdot \frac{\lam^x}{x!} \\
               &= e^{-\lam}\sum_{x=0}^{\infty} \frac{(\lam e^t)^x}{x!} \\
               &= e^{-\lam}e^{\lam e^t} \\
               &= e^{\lam e^t - \lam} \\
               &= e^{\lam(e^t - 1)}.
      \end{align*}
    \end{proof}
  \end{theorem}
  \begin{theorem}{}
    If $\lam > 0$ and $X\sim \Poisson(\lam)$, then
    \begin{align*}
      \E[X] &= \lam \\
      \var(X) &= \lam.
    \end{align*}
    \begin{proof}
      Observe that $\ln M_X(t) = \lam (e^t - 1)$. Hence
      \[
        \E[X] = (\ln M_X(t))'(0) = \lam.
      \]
      Furthermore,
      \[
        \var(X) = (\ln M_X(t))''(0) = \lam.
      \]
    \end{proof}
  \end{theorem}
  \subsection{Random Variables of the Continuous Type}
  \begin{definition}{Continuous Random Variable}
    Let $X\colon \Omega\to \R$ be a random variable.
    \begin{itemize}
      \item We say that $X$ is a \emph{continuous random variable} if there exists a non-negative integrable function $f_X\colon \R\to [0, \infty)$ so that
      \[
        F_X(x) = \int_{-\infty}^{x}f_X(t) \,\mathrm dt.
      \]
      Note that this ensures that $F_X(x)$ is continuous.
      \item We call $f_X(x)$ a \emph{probability density function} for $X$.
    \end{itemize}
  \end{definition}
  \begin{theorem}{}
    If $X$ is a continuous random variable with PDF $f_X\colon \R\to [0, \infty)$ then
    \[
      \int_{-\infty}^{\infty}f_X(x) \,\mathrm dx = 1.
    \]
    \begin{proof}
      As $\displaystyle F_X(x) = \int_{-\infty}^{x}f_X(x) \,\mathrm dx$ and $\displaystyle \lim_{x\to +\infty} F_X(x) = 1$, then
      \begin{align*}
        1 &= \lim_{x\to +\infty} F_X(x) \\
          &= \lim_{x\to +\infty} \int_{-\infty}^{x}f_X(x) \,\mathrm dx \\
          &=\int_{-\infty}^{\infty}f_X(x) \,\mathrm dx.
      \end{align*}
    \end{proof}
  \end{theorem}
  \begin{note}{}
    This is analogous to
    \[
      \sum_{x\in S}p_X(x) = 1
    \]
    for a discrete random variable.
  \end{note}
  \phantom{}
  \begin{theorem}{}
    If $X$ is a continuous random variable with PDF $f_X\colon \R\to [0, \infty)$ and $a < b$ then
    \[
      \P[a < X \leq b] = \int_{a}^{b}f_X(x) \,\mathrm dx.
    \]
    \begin{proof}
      Observe that we have
      \begin{align*}
        \P[a < X \leq b] &= \P[\set{X \leq b}\sm \set{X \leq a}] \\
                         &= \P[X \leq b] - \P[X \leq a] \\
                         &= \int_{-\infty}^{b}f_X(x) \,\mathrm dx - \int_{-\infty}^{a}f_X(x) \,\mathrm dx \\
                         &= \int_{a}^{b}f_X(x) \,\mathrm dx.
      \end{align*}
    \end{proof}
  \end{theorem}
  \begin{theorem}{}
    If $X$ is a continuous random variable with PDF $f_X\colon \R\to [0, \infty)$ then for all $x\in\R$ we have
    \[
      \P[X = x] = 0.
    \]
    \begin{proof}
      Let $\delta > 0$. Then
      \begin{align*}
        \P[X = x] &\leq \P[x - \delta \leq X \leq x] \\
                  &\leq \int_{x - \delta}^{x}f_X(t) \,\mathrm dt.
      \end{align*}
      As we take $\delta \to 0$, we find that $\P[X = x]\to 0$. Hence $\P[X = x] = 0$.
    \end{proof}
  \end{theorem}
  A consequence of the above fact (if $X$ is a \emph{continuous} random variable):
  \begin{itemize}
    \item $\P[a < X < b] = \P[a \leq X < b] = \P[a < X \leq b] = \P[a \leq X \leq b]$.
  \end{itemize}
\end{document}
