\documentclass[class=article, crop=false]{standalone}
% Import packages
\usepackage[margin=1in]{geometry}

\usepackage{amssymb, amsthm}
\usepackage{comment}
\usepackage{enumitem}
\usepackage{fancyhdr}
\usepackage{hyperref}
\usepackage{import}
\usepackage{mathrsfs, mathtools}
\usepackage{multicol}
\usepackage{pdfpages}
\usepackage{standalone}
\usepackage{transparent}
\usepackage{xcolor}

\usepackage{minted}
\usepackage[many]{tcolorbox}
\tcbuselibrary{minted}

% Declare math operators
\DeclareMathOperator{\lcm}{lcm}
\DeclareMathOperator{\proj}{proj}
\DeclareMathOperator{\vspan}{span}
\DeclareMathOperator{\im}{im}
\DeclareMathOperator{\Diff}{Diff}
\DeclareMathOperator{\Int}{Int}
\DeclareMathOperator{\fcn}{fcn}
\DeclareMathOperator{\id}{id}
\DeclareMathOperator{\rank}{rank}
\DeclareMathOperator{\tr}{tr}
\DeclareMathOperator{\dive}{div}
\DeclareMathOperator{\row}{row}
\DeclareMathOperator{\col}{col}
\DeclareMathOperator{\dom}{dom}
\DeclareMathOperator{\ran}{ran}
\DeclareMathOperator{\Dom}{Dom}
\DeclareMathOperator{\Ran}{Ran}
\DeclareMathOperator{\fl}{fl}
% Macros for letters/variables
\newcommand\ttilde{\kern -.15em\lower .7ex\hbox{\~{}}\kern .04em}
\newcommand{\N}{\ensuremath{\mathbb{N}}}
\newcommand{\Z}{\ensuremath{\mathbb{Z}}}
\newcommand{\Q}{\ensuremath{\mathbb{Q}}}
\newcommand{\R}{\ensuremath{\mathbb{R}}}
\newcommand{\C}{\ensuremath{\mathbb{C}}}
\newcommand{\F}{\ensuremath{\mathbb{F}}}
\newcommand{\M}{\ensuremath{\mathbb{M}}}
\renewcommand{\P}{\ensuremath{\mathbb{P}}}
\newcommand{\lam}{\ensuremath{\lambda}}
\newcommand{\nab}{\ensuremath{\nabla}}
\newcommand{\eps}{\ensuremath{\varepsilon}}
\newcommand{\es}{\ensuremath{\varnothing}}
% Macros for math symbols
\newcommand{\dx}[1]{\,\mathrm{d}#1}
\newcommand{\inv}{\ensuremath{^{-1}}}
\newcommand{\sm}{\setminus}
\newcommand{\sse}{\subseteq}
\newcommand{\ceq}{\coloneqq}
% Macros for pairs of math symbols
\newcommand{\abs}[1]{\ensuremath{\left\lvert #1 \right\rvert}}
\newcommand{\paren}[1]{\ensuremath{\left( #1 \right)}}
\newcommand{\norm}[1]{\ensuremath{\left\lVert #1\right\rVert}}
\newcommand{\set}[1]{\ensuremath{\left\{#1\right\}}}
\newcommand{\tup}[1]{\ensuremath{\left\langle #1 \right\rangle}}
\newcommand{\floor}[1]{\ensuremath{\left\lfloor #1 \right\rfloor}}
\newcommand{\ceil}[1]{\ensuremath{\left\lceil #1 \right\rceil}}
\newcommand{\eclass}[1]{\ensuremath{\left[ #1 \right]}}

\newcommand{\chapternum}{}
\newcommand{\ex}[1]{\noindent\textbf{Exercise \chapternum.{#1}.}}

\newcommand{\tsub}[1]{\textsubscript{#1}}
\newcommand{\tsup}[1]{\textsuperscript{#1}}

\renewcommand{\choose}[2]{\ensuremath{\phantom{}_{#1}C_{#2}}}
\newcommand{\perm}[2]{\ensuremath{\phantom{}_{#1}P_{#2}}}

% Include figures
\newcommand{\incfig}[2][1]{%
    \def\svgwidth{#1\columnwidth}
    \import{./figures/}{#2.pdf_tex}
}

\definecolor{problemBackground}{RGB}{212,232,246}
\newenvironment{problem}[1]
  {
    \begin{tcolorbox}[
      boxrule=.5pt,
      titlerule=.5pt,
      sharp corners,
      colback=problemBackground,
      breakable
    ]
    \ifx &#1& \textbf{Problem. }
    \else \textbf{Problem #1.} \fi
  }
  {
    \end{tcolorbox}
  }
\definecolor{exampleBackground}{RGB}{255,249,248}
\definecolor{exampleAccent}{RGB}{158,60,14}
\newenvironment{example}[1]
  {
    \begin{tcolorbox}[
      boxrule=.5pt,
      sharp corners,
      colback=exampleBackground,
      colframe=exampleAccent,
    ]
    \color{exampleAccent}\textbf{Example.} \emph{#1}\color{black}
  }
  {
    \end{tcolorbox}
  }
\definecolor{theoremBackground}{RGB}{234,243,251}
\definecolor{theoremAccent}{RGB}{0,116,183}
\newenvironment{theorem}[1]
  {
    \begin{tcolorbox}[
      boxrule=.5pt,
      titlerule=.5pt,
      sharp corners,
      colback=theoremBackground,
      colframe=theoremAccent,
      breakable
    ]
      % \color{theoremAccent}\textbf{Theorem --- }\emph{#1}\\\color{black}
    \ifx &#1& \color{theoremAccent}\textbf{Theorem. }\color{black}
    \else \color{theoremAccent}\textbf{Theorem --- }\emph{#1}\\\color{black} \fi
  }
  {
    \end{tcolorbox}
  }
\definecolor{noteBackground}{RGB}{244,249,244}
\definecolor{noteAccent}{RGB}{34,139,34}
\newenvironment{note}[1]
  {
  \begin{tcolorbox}[
    enhanced,
    boxrule=0pt,
    frame hidden,
    sharp corners,
    colback=noteBackground,
    borderline west={3pt}{-1.5pt}{noteAccent},
    breakable
    ]
    \ifx &#1& \color{noteAccent}\textbf{Note. }\color{black}
    \else \color{noteAccent}\textbf{Note (#1). }\color{black} \fi
    }
    {
  \end{tcolorbox}
  }
\definecolor{lemmaBackground}{RGB}{255,247,234}
\definecolor{lemmaAccent}{RGB}{255,153,0}
\newenvironment{lemma}[1]
  {
    \begin{tcolorbox}[
      enhanced,
      boxrule=0pt,
      frame hidden,
      sharp corners,
      colback=lemmaBackground,
      borderline west={3pt}{-1.5pt}{lemmaAccent},
      breakable
    ]
    \ifx &#1& \color{lemmaAccent}\textbf{Lemma. }\color{black}
    \else \color{lemmaAccent}\textbf{Lemma #1. }\color{black} \fi
  }
  {
    \end{tcolorbox}
  }
\definecolor{definitionBackground}{RGB}{246,246,246}
\newenvironment{definition}[1]
  {
    \begin{tcolorbox}[
      enhanced,
      boxrule=0pt,
      frame hidden,
      sharp corners,
      colback=definitionBackground,
      borderline west={3pt}{-1.5pt}{black},
      breakable
    ]
    \textbf{Definition. }\emph{#1}\\
  }
  {
    \end{tcolorbox}
  }

\newenvironment{amatrix}[2]{
    \left[
      \begin{array}{*{#1}{c}|*{#2}c}
  }
  {
      \end{array}
    \right]
  }

\definecolor{codeBackground}{RGB}{253,246,227}
\renewcommand\theFancyVerbLine{\arabic{FancyVerbLine}}
\newtcblisting{cpp}[1][]{
  boxrule=1pt,
  sharp corners,
  colback=codeBackground,
  listing only,
  breakable,
  minted language=cpp,
  minted style=material,
  minted options={
    xleftmargin=15pt,
    baselinestretch=1.2,
    linenos,
  }
}

\author{Kyle Chui}


\fancyhf{}
\lhead{Kyle Chui}
\rhead{Page \thepage}
\pagestyle{fancy}

\begin{document}
  \section{Lecture 15}
  \subsection{The Gamma Distribution}
  \begin{definition}{Gamma Distribution}
    Consider an approximate Poisson process with rate $\lam > 0$ per unit time.
    \begin{itemize}
      \item Let $\alpha \geq 1$ be an integer and $X$ be the time of the $\alpha$\tsup{th} arrival.
      \item We say that $X$ is \emph{gamma distributed} with parameters $\alpha, \theta = \frac{1}{\lam}$ and write $X\sim \mathrm{Gamma}(\alpha, \theta)$.
    \end{itemize}
  \end{definition}
  \begin{note}{}
    By definition, we have that $\Exponential(\theta)\sim \mathrm{Gamma}(1, \theta)$.
  \end{note}
  \begin{theorem}{}
    Let $\alpha \geq 1$ be an integer and $\theta > 0$. If $X\sim \mathrm{Gamma}(\alpha, \theta)$ then its PDF is
    \[
      f_X(x) = \frac{1}{\theta^\alpha(\alpha - 1)!}x^{\alpha - 1}e^{-\frac{x}{\theta}}\quad\text{if}\quad x > 0.
    \]
    \begin{proof}
      Let $X > 0$ and $N$ be the number of arrivals in the time interval $[0, x]$. Then we know that $N\sim \Poisson(\lam x)$, where $\lam = \frac{1}{\theta}$. Hence
      \begin{align*}
        \P[X \leq x] &= 1 - \P[X > x] \\
                     &= 1 - \P[N \leq \alpha - 1] \\
                     &= 1 - \sum_{n=0}^{\alpha - 1} p_N(n) \\
                     &= 1 - \sum_{n=0}^{\alpha - 1} \frac{(\lam x)^n}{n!}e^{-\lam x}.
      \end{align*}
      Thus we have
      \begin{align*}
        f_X(x) &= F_X'(x) \\
               &= -\sum_{n=1}^{\alpha - 1} \lam\cdot \frac{(\lam x)^{n - 1}}{(n - 1)!}e^{-\lam x} + \sum_{n=0}^{\alpha - 1} \frac{(\lam x)^n}{n!} \lam e^{-\lam x} \\
               &= -\lam \sum_{n=0}^{\alpha - 2} \frac{(\lam x)^n}{n!}e^{-\lam x} + \lam \sum_{n=0}^{\alpha - 1} \frac{(\lam x)^n}{n!}e^{-\lam x} \\
               &= \lam \frac{(\lam x)^{\alpha - 1}}{(\alpha - 1)!}e^{-\lam x} \\
               &= \frac{x^{\alpha - 1}}{\theta^\alpha(\alpha - 1)!}e^{-\frac{x}{\theta}}\quad\text{if}\quad x > 0.
      \end{align*}
    \end{proof}
  \end{theorem}
  \begin{definition}{Gamma Function}
    We define
    \[
      \Gamma(\alpha) \ceq \int_{0}^{\infty}x^{\alpha - 1}e^{-x} \,\mathrm dx.
    \]
  \end{definition}
  \begin{theorem}{}
    We have that
    \begin{itemize}
      \item $\Gamma(1) = 1$
      \item For $\alpha > 1$ we have
      \[
        \Gamma(\alpha) = (\alpha - 1)\Gamma(\alpha - 1)
      \]
      \item If $\alpha \geq 1$ is an integer then
      \[
        \Gamma(\alpha) = (\alpha - 1)!
      \]
    \end{itemize}
    \begin{proof}
      \begin{itemize}
        \item Observe that
        \[
          \Gamma(1) = \int_{0}^{\infty}e^{-x} \,\mathrm dx = 1.
        \]
        \item We compute
        \begin{align*}
          \Gamma(\alpha) &= \int_{0}^{\infty}x^{\alpha - 1}e^{-x} \,\mathrm dx \\
                         &= [-x^{\alpha - 1}e^{-x}]_0^\infty + \int_{0}^{\infty}(\alpha - 1) x^{\alpha - 2}e^{-x}\,\mathrm dx \\
                         &= 0 + (\alpha - 1)\Gamma(\alpha - 1).
        \end{align*}
        \item This is a direct consequence of the previous two properties.
      \end{itemize}
    \end{proof}
  \end{theorem}
  \begin{note}{}
    Using the gamma function, we may extend our definition of the Gamma distribution to be
    \[
      f_X(x) = \frac{1}{\theta^\alpha\Gamma(\alpha)}x^{\alpha - 1}e^{-\frac{x}{\theta}}.
    \]
  \end{note}
  \begin{theorem}{}
    Let $\alpha, \theta > 0$ and $X\sim \mathrm{Gamma}(\alpha, \theta)$. Then $X$ has MGF
    \[
      M_X(t) = \frac{1}{(1 - \theta t)^\alpha}\quad\text{if}\quad t < \frac{1}{\theta}.
    \]
  \end{theorem}
  \begin{theorem}{}
    Let $\alpha, \theta > 0$ and $X\sim \mathrm{Gamma}(\alpha, \theta)$. Then
    \begin{align*}
      \E[X] &= \alpha\theta, \\
      \var(X) &= \alpha\theta^2.
    \end{align*}
  \end{theorem}
  \subsection{The Chi-Square Distribution}
  This is a special case of the Gamma distribution.
  \begin{definition}{Chi-Square Distribution}
    If $r\in \set{1,2,3,\dotsc}$ we call the $\Gamma(\frac{r}{2}, 2)$ distribution the \emph{chi-square distribution} with $r$ degrees of freedom. If $X\sim \chi^2(r)$ then it has PDF
    \[
      f_X(x) = \frac{1}{2^{\frac{r}{2}}\Gamma(\frac{r}{2})}x^{\frac{r}{2} - 1}e^{-\frac{x}{2}}\quad\text{if}\quad x > 0,
    \]
    as well as
    \[
      \E[X] = r\quad\text{and}\quad \var(X) = 2r.
    \]
  \end{definition}
  \begin{example}{}
    Suppose that $X$ is a continuous random variable with PDF
    \[
      f_X(x) = \frac{1}{\sqrt{2\pi}}e^{-\frac{x^2}{2}}.
    \]
    Then $X^2\sim \chi^2(1)$.
    \begin{proof}
      Let $Z = X^2$. Then
      \begin{align*}
        F_Z(z) &= \P[Z \leq z] \\
               &= \P[X^2 \leq z] \\
               &= \P[-\sqrt{z} \leq X \leq\sqrt{z}] \tag{if $z > 0$} \\
               &= \int_{-\sqrt{z}}^{\sqrt{z}}\frac{1}{\sqrt{2\pi}}e^{-\frac{x^2}{2}} \,\mathrm dx \\
               &= 2\int_{0}^{\sqrt{z}}\frac{1}{\sqrt{2\pi}}e^{-\frac{x^2}{2}} \,\mathrm dx \\
               &= \frac{2}{\sqrt{2\pi}} \int_{0}^{z}e^{-\frac{u}{2}}\cdot \frac{1}{2\sqrt{u}} \,\mathrm du \\
               &= \frac{1}{\sqrt{2\pi}} \int_{0}^{z}e^{-\frac{u}{2}}\cdot u^{-\frac{1}{2}} \,\mathrm du.
      \end{align*}
      Therefore we have $f_Z(z) = \frac{1}{\sqrt{2 \pi}}z^{-\frac{1}{2}}e^{-\frac{z}{2}}$ if $z > 0$, and $Z\sim \chi^2(1)$.
    \end{proof}
  \end{example}
\end{document}
