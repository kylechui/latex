\documentclass[class=article, crop=false]{standalone}
% Import packages
\usepackage[margin=1in]{geometry}

\usepackage{amssymb, amsthm}
\usepackage{comment}
\usepackage{enumitem}
\usepackage{fancyhdr}
\usepackage{hyperref}
\usepackage{import}
\usepackage{mathrsfs, mathtools}
\usepackage{multicol}
\usepackage{pdfpages}
\usepackage{standalone}
\usepackage{transparent}
\usepackage{xcolor}

\usepackage{minted}
\usepackage[many]{tcolorbox}
\tcbuselibrary{minted}

% Declare math operators
\DeclareMathOperator{\lcm}{lcm}
\DeclareMathOperator{\proj}{proj}
\DeclareMathOperator{\vspan}{span}
\DeclareMathOperator{\im}{im}
\DeclareMathOperator{\Diff}{Diff}
\DeclareMathOperator{\Int}{Int}
\DeclareMathOperator{\fcn}{fcn}
\DeclareMathOperator{\id}{id}
\DeclareMathOperator{\rank}{rank}
\DeclareMathOperator{\tr}{tr}
\DeclareMathOperator{\dive}{div}
\DeclareMathOperator{\row}{row}
\DeclareMathOperator{\col}{col}
\DeclareMathOperator{\dom}{dom}
\DeclareMathOperator{\ran}{ran}
\DeclareMathOperator{\Dom}{Dom}
\DeclareMathOperator{\Ran}{Ran}
\DeclareMathOperator{\fl}{fl}
% Macros for letters/variables
\newcommand\ttilde{\kern -.15em\lower .7ex\hbox{\~{}}\kern .04em}
\newcommand{\N}{\ensuremath{\mathbb{N}}}
\newcommand{\Z}{\ensuremath{\mathbb{Z}}}
\newcommand{\Q}{\ensuremath{\mathbb{Q}}}
\newcommand{\R}{\ensuremath{\mathbb{R}}}
\newcommand{\C}{\ensuremath{\mathbb{C}}}
\newcommand{\F}{\ensuremath{\mathbb{F}}}
\newcommand{\M}{\ensuremath{\mathbb{M}}}
\renewcommand{\P}{\ensuremath{\mathbb{P}}}
\newcommand{\lam}{\ensuremath{\lambda}}
\newcommand{\nab}{\ensuremath{\nabla}}
\newcommand{\eps}{\ensuremath{\varepsilon}}
\newcommand{\es}{\ensuremath{\varnothing}}
% Macros for math symbols
\newcommand{\dx}[1]{\,\mathrm{d}#1}
\newcommand{\inv}{\ensuremath{^{-1}}}
\newcommand{\sm}{\setminus}
\newcommand{\sse}{\subseteq}
\newcommand{\ceq}{\coloneqq}
% Macros for pairs of math symbols
\newcommand{\abs}[1]{\ensuremath{\left\lvert #1 \right\rvert}}
\newcommand{\paren}[1]{\ensuremath{\left( #1 \right)}}
\newcommand{\norm}[1]{\ensuremath{\left\lVert #1\right\rVert}}
\newcommand{\set}[1]{\ensuremath{\left\{#1\right\}}}
\newcommand{\tup}[1]{\ensuremath{\left\langle #1 \right\rangle}}
\newcommand{\floor}[1]{\ensuremath{\left\lfloor #1 \right\rfloor}}
\newcommand{\ceil}[1]{\ensuremath{\left\lceil #1 \right\rceil}}
\newcommand{\eclass}[1]{\ensuremath{\left[ #1 \right]}}

\newcommand{\chapternum}{}
\newcommand{\ex}[1]{\noindent\textbf{Exercise \chapternum.{#1}.}}

\newcommand{\tsub}[1]{\textsubscript{#1}}
\newcommand{\tsup}[1]{\textsuperscript{#1}}

\renewcommand{\choose}[2]{\ensuremath{\phantom{}_{#1}C_{#2}}}
\newcommand{\perm}[2]{\ensuremath{\phantom{}_{#1}P_{#2}}}

% Include figures
\newcommand{\incfig}[2][1]{%
    \def\svgwidth{#1\columnwidth}
    \import{./figures/}{#2.pdf_tex}
}

\definecolor{problemBackground}{RGB}{212,232,246}
\newenvironment{problem}[1]
  {
    \begin{tcolorbox}[
      boxrule=.5pt,
      titlerule=.5pt,
      sharp corners,
      colback=problemBackground,
      breakable
    ]
    \ifx &#1& \textbf{Problem. }
    \else \textbf{Problem #1.} \fi
  }
  {
    \end{tcolorbox}
  }
\definecolor{exampleBackground}{RGB}{255,249,248}
\definecolor{exampleAccent}{RGB}{158,60,14}
\newenvironment{example}[1]
  {
    \begin{tcolorbox}[
      boxrule=.5pt,
      sharp corners,
      colback=exampleBackground,
      colframe=exampleAccent,
    ]
    \color{exampleAccent}\textbf{Example.} \emph{#1}\color{black}
  }
  {
    \end{tcolorbox}
  }
\definecolor{theoremBackground}{RGB}{234,243,251}
\definecolor{theoremAccent}{RGB}{0,116,183}
\newenvironment{theorem}[1]
  {
    \begin{tcolorbox}[
      boxrule=.5pt,
      titlerule=.5pt,
      sharp corners,
      colback=theoremBackground,
      colframe=theoremAccent,
      breakable
    ]
      % \color{theoremAccent}\textbf{Theorem --- }\emph{#1}\\\color{black}
    \ifx &#1& \color{theoremAccent}\textbf{Theorem. }\color{black}
    \else \color{theoremAccent}\textbf{Theorem --- }\emph{#1}\\\color{black} \fi
  }
  {
    \end{tcolorbox}
  }
\definecolor{noteBackground}{RGB}{244,249,244}
\definecolor{noteAccent}{RGB}{34,139,34}
\newenvironment{note}[1]
  {
  \begin{tcolorbox}[
    enhanced,
    boxrule=0pt,
    frame hidden,
    sharp corners,
    colback=noteBackground,
    borderline west={3pt}{-1.5pt}{noteAccent},
    breakable
    ]
    \ifx &#1& \color{noteAccent}\textbf{Note. }\color{black}
    \else \color{noteAccent}\textbf{Note (#1). }\color{black} \fi
    }
    {
  \end{tcolorbox}
  }
\definecolor{lemmaBackground}{RGB}{255,247,234}
\definecolor{lemmaAccent}{RGB}{255,153,0}
\newenvironment{lemma}[1]
  {
    \begin{tcolorbox}[
      enhanced,
      boxrule=0pt,
      frame hidden,
      sharp corners,
      colback=lemmaBackground,
      borderline west={3pt}{-1.5pt}{lemmaAccent},
      breakable
    ]
    \ifx &#1& \color{lemmaAccent}\textbf{Lemma. }\color{black}
    \else \color{lemmaAccent}\textbf{Lemma #1. }\color{black} \fi
  }
  {
    \end{tcolorbox}
  }
\definecolor{definitionBackground}{RGB}{246,246,246}
\newenvironment{definition}[1]
  {
    \begin{tcolorbox}[
      enhanced,
      boxrule=0pt,
      frame hidden,
      sharp corners,
      colback=definitionBackground,
      borderline west={3pt}{-1.5pt}{black},
      breakable
    ]
    \textbf{Definition. }\emph{#1}\\
  }
  {
    \end{tcolorbox}
  }

\newenvironment{amatrix}[2]{
    \left[
      \begin{array}{*{#1}{c}|*{#2}c}
  }
  {
      \end{array}
    \right]
  }

\definecolor{codeBackground}{RGB}{253,246,227}
\renewcommand\theFancyVerbLine{\arabic{FancyVerbLine}}
\newtcblisting{cpp}[1][]{
  boxrule=1pt,
  sharp corners,
  colback=codeBackground,
  listing only,
  breakable,
  minted language=cpp,
  minted style=material,
  minted options={
    xleftmargin=15pt,
    baselinestretch=1.2,
    linenos,
  }
}

\author{Kyle Chui}


\fancyhf{}
\lhead{Kyle Chui}
\rhead{Page \thepage}
\pagestyle{fancy}

\begin{document}
  \section{Lecture 12}
  \subsection{Dimensional Analysis}
  \begin{note}{}
    For dimensional analysis:
    \begin{itemize}
      \item Express all of your time derivatives in a unit-less form by substituting another variable, i.e. $\tau\ceq \frac{t}{T}$.
      \item Make the equation truly dimensionless by dividing all sides by the common unit, i.e. a unit of force.
    \end{itemize}
  \end{note}
  Since we are not well-equipped to study second-order differential equations yet, we try to find certain conditions that make the second-order term vanish to zero (but not any of the other terms). Consider the following differential equation:
  \[
    \frac{r}{gT^2}\cdot \frac{\mathrm{d}^2\phi}{\mathrm{d}\tau^2} = -\frac{b}{mgT} \frac{\mathrm{d\phi}}{\mathrm{d}\tau} - \sin\phi + \frac{r\Omega^2}{g}\sin\phi\cos\phi.
  \]
  We wish to study a regime where
  \[
    \frac{b}{mgT} = 1,
  \]
  so we have that $T = \frac{b}{mg}$. Plugging this $T$ into the second-order term above, we get
  \begin{align*}
    \underbrace{\frac{r}{g\paren{\frac{b}{mg}}^2}}_{\eps} \frac{\mathrm{d}^2\phi}{\mathrm{d}\tau^2} &= -\frac{\mathrm{d}\phi}{\mathrm{d}\tau} - \sin\phi + \underbrace{\frac{r\Omega^2}{g}}_{\gamma}\sin\phi\cos\phi \\
    \eps \frac{\mathrm{d}^2\phi}{\mathrm{d}\tau^2} &= -\frac{\mathrm{d}\phi}{\mathrm{d}\tau} - \sin\phi + \gamma\sin\phi\cos\phi.
  \end{align*}
  If $\eps \ll 1$, then we get that
  \begin{align*}
    \frac{r}{g\paren{\frac{b}{mg}}^2} &\ll 1 \\
    m^2gr &\ll b^2.
  \end{align*}
  Thus we are left with the equation
  \begin{align*}
    0 &= -\frac{\mathrm{d}\phi}{\mathrm{d}\tau} - \sin\phi + \gamma\sin\phi\cos\phi \\
    \frac{\mathrm{d}\phi}{\mathrm{d}\tau} &= - \sin\phi + \gamma\sin\phi\cos\phi \\
    \frac{\mathrm{d}\phi}{\mathrm{d}\tau} &= \sin\phi(\gamma\cos\phi - 1).
  \end{align*}
  \begin{note}{}
    The non-dimensionalized form of our model is
    \[
      \eps \frac{\mathrm{d}^2\phi}{\mathrm{d}\tau^2} + \frac{\mathrm{d}\phi}{\mathrm{d}\tau} = \sin\phi(\gamma\cos\phi - 1).
    \]
    If $\eps \ll 1$, then
    \[
      \frac{\mathrm{d}\phi}{\mathrm{d}\tau} = \sin\phi(\gamma\cos\phi - 1).
    \]
  \end{note}
  \subsubsection{Bifurcation Analysis of the First-order Equation}
  We see that we have fixed points when $\phi = n\pi$ for all $n\in\Z$ or $\cos\phi = \frac{1}{\gamma}$. When $0 < \gamma < 1$, then there are no solutions to $\cos\gamma = \phi$. For $\gamma > 1$, we are going to have two solutions, symmetric about $\phi = 0$, given by $\phi^* = \pm \cos^{-1}(\frac{1}{\gamma})$. Finally, when $\gamma = 1$, then $\phi = 0$. \par
  We wish to find when $\frac{\partial f}{\partial \phi} = 0$, so
  \begin{align*}
    \cos\phi(\gamma\cos\phi - 1) + \sin\phi(-\gamma\sin\phi) &= \gamma\cos^2\phi - \cos\phi - \gamma\sin^2\phi \\
                                                             &= \gamma\cos^2\phi - \cos\phi - \gamma(1 - \cos^2\phi) \\
                                                             &= 2\gamma\cos^2\phi - \cos\phi - \gamma.
  \end{align*}
\end{document}
