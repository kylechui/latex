\documentclass[class=article, crop=false]{standalone}
\usepackage[margin=1in]{geometry}

\usepackage{amsmath}
\usepackage{amssymb}
\usepackage{amsthm}
\usepackage{comment}
\usepackage{enumitem}
\usepackage{fancyhdr}
\usepackage{hyperref}
\usepackage{mathrsfs}
\usepackage{mathtools}
\usepackage{standalone}
\usepackage{tcolorbox}
\usepackage{tikz}
\usepackage{xcolor}

\usetikzlibrary{decorations.pathreplacing}
\tcbuselibrary{skins}

\DeclareMathOperator{\lcm}{lcm}
\DeclareMathOperator{\proj}{proj}
\DeclareMathOperator{\vspan}{span}
\DeclareMathOperator{\im}{im}
\DeclareMathOperator{\range}{range}
\DeclareMathOperator{\Diff}{Diff}
\DeclareMathOperator{\Int}{Int}
\DeclareMathOperator{\fcn}{fcn}
\DeclareMathOperator{\id}{id}

\newcommand{\N}{\ensuremath{\mathbb{N}}}
\newcommand{\Z}{\ensuremath{\mathbb{Z}}}
\newcommand{\Q}{\ensuremath{\mathbb{Q}}}
\newcommand{\R}{\ensuremath{\mathbb{R}}}
\newcommand{\C}{\ensuremath{\mathbb{C}}}
\newcommand{\F}{\ensuremath{\mathbb{F}}}
\newcommand{\lam}{\ensuremath{\lambda}}
\newcommand{\nab}{\ensuremath{\nabla}}
\newcommand{\eps}{\ensuremath{\varepsilon}}
\newcommand{\es}{\ensuremath{\varnothing}}

\newcommand{\abs}[1]{\ensuremath{\left\lvert #1 \right\rvert}}
\newcommand{\bigpar}[1]{\ensuremath{\left( #1 \right)}}
\newcommand{\norm}[1]{\ensuremath{\lVert #1\rVert}}
\newcommand{\set}[1]{\ensuremath{\left\{#1\right\}}}
\newcommand{\tuple}[1]{\ensuremath{\left\langle #1 \right\rangle}}
\newcommand{\floor}[1]{\ensuremath{\left\lfloor #1 \right\rfloor}}
\newcommand{\ceil}[1]{\ensuremath{\left\lceil #1 \right\rceil}}

\newcommand{\chapternum}{}
\newcommand{\ex}[1]{\noindent\textbf{Exercise \chapternum.{#1}.}}

\newcommand{\dx}[1]{\,\mathrm{d}#1}
\newcommand{\inv}{\ensuremath{^{-1}}}
\newcommand{\sm}{\setminus}
\newcommand{\sse}{\subseteq}
\newcommand{\ceq}{\coloneqq}

\definecolor{problemBackground}{RGB}{212,232,246}
\newenvironment{problem}[1]
  {
    \begin{tcolorbox}[
      boxrule=.5pt,
      titlerule=.5pt,
      sharp corners,
      colback=problemBackground
    ]
    \ifx &#1& \textbf{Problem. }
    \else \textbf{Problem #1.} \fi
  }
  {
    \end{tcolorbox}
  }

\definecolor{exampleBackground}{RGB}{255,249,248}
\definecolor{exampleAccent}{RGB}{158,60,14}
\newenvironment{example}[1]
  {
    \begin{tcolorbox}[
      boxrule=.5pt,
      sharp corners,
      colback=exampleBackground,
      colframe=exampleAccent,
    ]
    \color{exampleAccent}\textbf{Example.} \emph{#1}\color{black}
  }
  {
    \end{tcolorbox}
  }

\definecolor{theoremBackground}{RGB}{234,243,251}
\definecolor{theoremAccent}{RGB}{0,116,183}
\newenvironment{theorem}[1]
  {
    \begin{tcolorbox}[
      boxrule=.5pt,
      titlerule=.5pt,
      sharp corners,
      colback=theoremBackground,
      colframe=theoremAccent
    ]
      \color{theoremAccent}\textbf{Theorem --- }\emph{#1}\\\color{black}
  }
  {
    \end{tcolorbox}
  }

\definecolor{noteBackground}{RGB}{244,249,244}
\definecolor{noteAccent}{RGB}{34,139,34}
\newenvironment{note}[1]
  {
  \begin{tcolorbox}[
    enhanced,
    boxrule=0pt,
    frame hidden,
    sharp corners,
    colback=noteBackground,
    borderline west={3pt}{-1.5pt}{noteAccent}
    ]
    \ifx &#1& \color{noteAccent}\textbf{Note. }\color{black}
    \else \color{noteAccent}\textbf{Note (#1). }\color{black} \fi
    }
    {
  \end{tcolorbox}
  }

\definecolor{definitionBackground}{RGB}{246,246,246}
\newenvironment{definition}[1]
  {
    \begin{tcolorbox}[
      enhanced,
      boxrule=0pt,
      frame hidden,
      sharp corners,
      colback=definitionBackground,
      borderline west={3pt}{-1.5pt}{black}
    ]
    \textbf{Definition. }\emph{#1}\\
  }
  {
    \end{tcolorbox}
  }
\newenvironment{amatrix}[2]{
    \left[
      \begin{array}{*{#1}{c}|*{#2}c}
  }
  {
      \end{array}
    \right]
  }
\date{\the\year-\the\month-\the\day}
\author{Kyle Chui}


\fancyhf{}
\lhead{Kyle Chui}
\rhead{Page \thepage}
\pagestyle{fancy}

\begin{document}
  \section{Lecture 12}
  \subsection{Dimensional Analysis}
  \begin{note}{}
    For dimensional analysis:
    \begin{itemize}
      \item Express all of your time derivatives in a unit-less form by substituting another variable, i.e. $\tau\ceq \frac{t}{T}$.
      \item Make the equation truly dimensionless by dividing all sides by the common unit, i.e. a unit of force.
    \end{itemize}
  \end{note}
  Since we are not well-equipped to study second-order differential equations yet, we try to find certain conditions that make the second-order term vanish to zero (but not any of the other terms). Consider the following differential equation:
  \[
    \frac{r}{gT^2}\cdot \frac{\mathrm{d}^2\phi}{\mathrm{d}\tau^2} = -\frac{b}{mgT} \frac{\mathrm{d\phi}}{\mathrm{d}\tau} - \sin\phi + \frac{r\Omega^2}{g}\sin\phi\cos\phi.
  \]
  We wish to study a regime where
  \[
    \frac{b}{mgT} = 1,
  \]
  so we have that $T = \frac{b}{mg}$. Plugging this $T$ into the second-order term above, we get
  \begin{align*}
    \underbrace{\frac{r}{g\paren{\frac{b}{mg}}^2}}_{\eps} \frac{\mathrm{d}^2\phi}{\mathrm{d}\tau^2} &= -\frac{\mathrm{d}\phi}{\mathrm{d}\tau} - \sin\phi + \underbrace{\frac{r\Omega^2}{g}}_{\gamma}\sin\phi\cos\phi \\
    \eps \frac{\mathrm{d}^2\phi}{\mathrm{d}\tau^2} &= -\frac{\mathrm{d}\phi}{\mathrm{d}\tau} - \sin\phi + \gamma\sin\phi\cos\phi.
  \end{align*}
  If $\eps \ll 1$, then we get that
  \begin{align*}
    \frac{r}{g\paren{\frac{b}{mg}}^2} &\ll 1 \\
    m^2gr &\ll b^2.
  \end{align*}
  Thus we are left with the equation
  \begin{align*}
    0 &= -\frac{\mathrm{d}\phi}{\mathrm{d}\tau} - \sin\phi + \gamma\sin\phi\cos\phi \\
    \frac{\mathrm{d}\phi}{\mathrm{d}\tau} &= - \sin\phi + \gamma\sin\phi\cos\phi \\
    \frac{\mathrm{d}\phi}{\mathrm{d}\tau} &= \sin\phi(\gamma\cos\phi - 1).
  \end{align*}
  \begin{note}{}
    The non-dimensionalized form of our model is
    \[
      \eps \frac{\mathrm{d}^2\phi}{\mathrm{d}\tau^2} + \frac{\mathrm{d}\phi}{\mathrm{d}\tau} = \sin\phi(\gamma\cos\phi - 1).
    \]
    If $\eps \ll 1$, then
    \[
      \frac{\mathrm{d}\phi}{\mathrm{d}\tau} = \sin\phi(\gamma\cos\phi - 1).
    \]
  \end{note}
  \subsubsection{Bifurcation Analysis of the First-order Equation}
  We see that we have fixed points when $\phi = n\pi$ for all $n\in\Z$ or $\cos\phi = \frac{1}{\gamma}$. When $0 < \gamma < 1$, then there are no solutions to $\cos\gamma = \phi$. For $\gamma > 1$, we are going to have two solutions, symmetric about $\phi = 0$, given by $\phi^* = \pm \cos^{-1}(\frac{1}{\gamma})$. Finally, when $\gamma = 1$, then $\phi = 0$. \par
  We wish to find when $\frac{\partial f}{\partial \phi} = 0$, so
  \begin{align*}
    \cos\phi(\gamma\cos\phi - 1) + \sin\phi(-\gamma\sin\phi) &= \gamma\cos^2\phi - \cos\phi - \gamma\sin^2\phi \\
                                                             &= \gamma\cos^2\phi - \cos\phi - \gamma(1 - \cos^2\phi) \\
                                                             &= 2\gamma\cos^2\phi - \cos\phi - \gamma.
  \end{align*}
\end{document}
