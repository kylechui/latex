\documentclass[class=article, crop=false]{standalone}
% Import packages
\usepackage[margin=1in]{geometry}

\usepackage{amssymb, amsthm}
\usepackage{comment}
\usepackage{enumitem}
\usepackage{fancyhdr}
\usepackage{hyperref}
\usepackage{import}
\usepackage{mathrsfs, mathtools}
\usepackage{multicol}
\usepackage{pdfpages}
\usepackage{standalone}
\usepackage{transparent}
\usepackage{xcolor}

\usepackage{minted}
\usepackage[many]{tcolorbox}
\tcbuselibrary{minted}

% Declare math operators
\DeclareMathOperator{\lcm}{lcm}
\DeclareMathOperator{\proj}{proj}
\DeclareMathOperator{\vspan}{span}
\DeclareMathOperator{\im}{im}
\DeclareMathOperator{\Diff}{Diff}
\DeclareMathOperator{\Int}{Int}
\DeclareMathOperator{\fcn}{fcn}
\DeclareMathOperator{\id}{id}
\DeclareMathOperator{\rank}{rank}
\DeclareMathOperator{\tr}{tr}
\DeclareMathOperator{\dive}{div}
\DeclareMathOperator{\row}{row}
\DeclareMathOperator{\col}{col}
\DeclareMathOperator{\dom}{dom}
\DeclareMathOperator{\ran}{ran}
\DeclareMathOperator{\Dom}{Dom}
\DeclareMathOperator{\Ran}{Ran}
\DeclareMathOperator{\fl}{fl}
% Macros for letters/variables
\newcommand\ttilde{\kern -.15em\lower .7ex\hbox{\~{}}\kern .04em}
\newcommand{\N}{\ensuremath{\mathbb{N}}}
\newcommand{\Z}{\ensuremath{\mathbb{Z}}}
\newcommand{\Q}{\ensuremath{\mathbb{Q}}}
\newcommand{\R}{\ensuremath{\mathbb{R}}}
\newcommand{\C}{\ensuremath{\mathbb{C}}}
\newcommand{\F}{\ensuremath{\mathbb{F}}}
\newcommand{\M}{\ensuremath{\mathbb{M}}}
\renewcommand{\P}{\ensuremath{\mathbb{P}}}
\newcommand{\lam}{\ensuremath{\lambda}}
\newcommand{\nab}{\ensuremath{\nabla}}
\newcommand{\eps}{\ensuremath{\varepsilon}}
\newcommand{\es}{\ensuremath{\varnothing}}
% Macros for math symbols
\newcommand{\dx}[1]{\,\mathrm{d}#1}
\newcommand{\inv}{\ensuremath{^{-1}}}
\newcommand{\sm}{\setminus}
\newcommand{\sse}{\subseteq}
\newcommand{\ceq}{\coloneqq}
% Macros for pairs of math symbols
\newcommand{\abs}[1]{\ensuremath{\left\lvert #1 \right\rvert}}
\newcommand{\paren}[1]{\ensuremath{\left( #1 \right)}}
\newcommand{\norm}[1]{\ensuremath{\left\lVert #1\right\rVert}}
\newcommand{\set}[1]{\ensuremath{\left\{#1\right\}}}
\newcommand{\tup}[1]{\ensuremath{\left\langle #1 \right\rangle}}
\newcommand{\floor}[1]{\ensuremath{\left\lfloor #1 \right\rfloor}}
\newcommand{\ceil}[1]{\ensuremath{\left\lceil #1 \right\rceil}}
\newcommand{\eclass}[1]{\ensuremath{\left[ #1 \right]}}

\newcommand{\chapternum}{}
\newcommand{\ex}[1]{\noindent\textbf{Exercise \chapternum.{#1}.}}

\newcommand{\tsub}[1]{\textsubscript{#1}}
\newcommand{\tsup}[1]{\textsuperscript{#1}}

\renewcommand{\choose}[2]{\ensuremath{\phantom{}_{#1}C_{#2}}}
\newcommand{\perm}[2]{\ensuremath{\phantom{}_{#1}P_{#2}}}

% Include figures
\newcommand{\incfig}[2][1]{%
    \def\svgwidth{#1\columnwidth}
    \import{./figures/}{#2.pdf_tex}
}

\definecolor{problemBackground}{RGB}{212,232,246}
\newenvironment{problem}[1]
  {
    \begin{tcolorbox}[
      boxrule=.5pt,
      titlerule=.5pt,
      sharp corners,
      colback=problemBackground,
      breakable
    ]
    \ifx &#1& \textbf{Problem. }
    \else \textbf{Problem #1.} \fi
  }
  {
    \end{tcolorbox}
  }
\definecolor{exampleBackground}{RGB}{255,249,248}
\definecolor{exampleAccent}{RGB}{158,60,14}
\newenvironment{example}[1]
  {
    \begin{tcolorbox}[
      boxrule=.5pt,
      sharp corners,
      colback=exampleBackground,
      colframe=exampleAccent,
    ]
    \color{exampleAccent}\textbf{Example.} \emph{#1}\color{black}
  }
  {
    \end{tcolorbox}
  }
\definecolor{theoremBackground}{RGB}{234,243,251}
\definecolor{theoremAccent}{RGB}{0,116,183}
\newenvironment{theorem}[1]
  {
    \begin{tcolorbox}[
      boxrule=.5pt,
      titlerule=.5pt,
      sharp corners,
      colback=theoremBackground,
      colframe=theoremAccent,
      breakable
    ]
      % \color{theoremAccent}\textbf{Theorem --- }\emph{#1}\\\color{black}
    \ifx &#1& \color{theoremAccent}\textbf{Theorem. }\color{black}
    \else \color{theoremAccent}\textbf{Theorem --- }\emph{#1}\\\color{black} \fi
  }
  {
    \end{tcolorbox}
  }
\definecolor{noteBackground}{RGB}{244,249,244}
\definecolor{noteAccent}{RGB}{34,139,34}
\newenvironment{note}[1]
  {
  \begin{tcolorbox}[
    enhanced,
    boxrule=0pt,
    frame hidden,
    sharp corners,
    colback=noteBackground,
    borderline west={3pt}{-1.5pt}{noteAccent},
    breakable
    ]
    \ifx &#1& \color{noteAccent}\textbf{Note. }\color{black}
    \else \color{noteAccent}\textbf{Note (#1). }\color{black} \fi
    }
    {
  \end{tcolorbox}
  }
\definecolor{lemmaBackground}{RGB}{255,247,234}
\definecolor{lemmaAccent}{RGB}{255,153,0}
\newenvironment{lemma}[1]
  {
    \begin{tcolorbox}[
      enhanced,
      boxrule=0pt,
      frame hidden,
      sharp corners,
      colback=lemmaBackground,
      borderline west={3pt}{-1.5pt}{lemmaAccent},
      breakable
    ]
    \ifx &#1& \color{lemmaAccent}\textbf{Lemma. }\color{black}
    \else \color{lemmaAccent}\textbf{Lemma #1. }\color{black} \fi
  }
  {
    \end{tcolorbox}
  }
\definecolor{definitionBackground}{RGB}{246,246,246}
\newenvironment{definition}[1]
  {
    \begin{tcolorbox}[
      enhanced,
      boxrule=0pt,
      frame hidden,
      sharp corners,
      colback=definitionBackground,
      borderline west={3pt}{-1.5pt}{black},
      breakable
    ]
    \textbf{Definition. }\emph{#1}\\
  }
  {
    \end{tcolorbox}
  }

\newenvironment{amatrix}[2]{
    \left[
      \begin{array}{*{#1}{c}|*{#2}c}
  }
  {
      \end{array}
    \right]
  }

\definecolor{codeBackground}{RGB}{253,246,227}
\renewcommand\theFancyVerbLine{\arabic{FancyVerbLine}}
\newtcblisting{cpp}[1][]{
  boxrule=1pt,
  sharp corners,
  colback=codeBackground,
  listing only,
  breakable,
  minted language=cpp,
  minted style=material,
  minted options={
    xleftmargin=15pt,
    baselinestretch=1.2,
    linenos,
  }
}

\author{Kyle Chui}


\fancyhf{}
\lhead{Kyle Chui}
\rhead{Page \thepage}
\pagestyle{fancy}

\begin{document}
  \section{Lecture 4}
  \begin{example}{}
    Consider the differential equation $\dot{x} = x - x^3$. Since we have $\dot{x} = -V'(x)$, we have
    \[
      V(x) = \frac{1}{4}x^4 - \frac{1}{2}x^2 = \frac{1}{4}(x^2 - 1)^2 - \frac{1}{4}.
    \]
    When we look at the graph of $V'(x)$, we may pretend that there is a ball rolling down a hill from every point, which tells us how to find the stability of points. Each point will settle in the first ``well'' that it meets.
  \end{example}
  \begin{theorem}{}
    Let $V\colon \R\to \R$ be smooth and consider the system
    \[
      \dot{x} = -V'(x).
    \]
    Then the \emph{potential energy} $V(x(t))$ is non-increasing (as a function of time). Furthermore, if $x(t)$ is not a fixed point for all $t\in (T_1, T_2)$, then the potential energy is strictly decreasing on $(T_1, T_2)$.
    \begin{proof}
      Observe that
      \begin{align*}
        \frac{\mathrm{d}}{\mathrm{d}t} V(x(t)) &= V'(x(t))\cdot \dot{x}(t) \\
                                               &= -V'(x(t))^2.
      \end{align*}
      Hence the potential energy is non-increasing, as its derivative is always non-positive. Thus if $V'(x_1) = 0$, then $x_1$ is a critical point!
    \end{proof}
  \end{theorem}
  \textbf{Corollary.} Let $V\colon \R\to \R$ be smooth and consider the system
  \[
    \dot{x} = -V'(x).
  \]
  If $x^*$ is an isolated critical point of $V$ then
  \begin{itemize}
    \item If it is a local minima of $V$, it is a stable fixed point.
    \item If it is a local maxima of $V$, it is a unstable fixed point.
    \item If it is an inflection point of $V$, it is a half-stable fixed point.
  \end{itemize}
  \begin{proof}
    If we imagine the ball analogy again, we can see that if $x^*$ is a local minima then points around it will tend towards $x^*$, and so it is stable. The opposite happens for divergence near a local maxima, and the analogy still holds for the half-stableness when $x^*$ is an inflection point.
  \end{proof}
  \begin{note}{}
    Every one dimensional system is a gradient flow, because if $f$ is smooth, then we can take the integral and define
    \[
      V(x)\ceq -\int_{0}^{x}f(s) \,\mathrm ds.
    \]
  \end{note}
  \subsection{Impossibility of Oscillations}
  \begin{definition}{Periodic Functions}
    If there exists a constant $p > 0$ so that for all $t$ we have
    \[
      x(t + p) = x(t),
    \]
    then we say that $p$ is \emph{periodic}.
  \end{definition}
  \begin{note}{}
    All constant functions are periodic.
  \end{note}
  \begin{theorem}{}
    There are no non-constant periodic solutions of the system
    \[
      \dot{x} = f(x).
    \]
    \begin{proof}
      Suppose that $x$ is a periodic solution, with period $p > 0$. If $0 \leq t \leq p$ then, as the potential energy is non-increasing,
      \[
        V[x(p)] \leq V[x(t)] \leq V(x(0)).
      \]
      Since $x(p) = x(0)$, we have $V[x(t)]$ is constant. Hence $x(t)$ is constant.
    \end{proof}
  \end{theorem}
  \subsection{Numerical Methods}
  \subsubsection{Integral Equations}
  We want to find a solution of the equation 
  \[
    \begin{cases}
      \dot{x} = f(x) \\
      x(0) = x_0
    \end{cases}
  \]
  Observe that
  \begin{align*}
    \dot{x} &= f(x) \\
    \int_{0}^{t} \frac{\mathrm{d}x(s)}{\mathrm{d}s} \,\mathrm ds &= \int_{0}^{t}f(x(s)) \,\mathrm ds \\
    x(t) - x(0) &= \int_{0}^{t} f(x(s)) \,\mathrm ds \\
    x(t) &= x_0 + \int_{0}^{t}f(x(s)) \,\mathrm ds.
  \end{align*}
  We call this an \emph{integral equation}, because the unknown now appears in the integral.
  \begin{example}{}
    Write the equation 
    \[
      \begin{cases}
        \dot{x} = \sin x \\
        x(0) = 1
      \end{cases}
    \]
    as an integral equation. \par
    We have that
    \[
      x(t) = 1 + \int_{0}^{t}\sin(x(s)) \,\mathrm ds.
    \]
  \end{example}
  \subsubsection{Numerical Approximation}
  Suppose we have the equation 
  \[
    \begin{cases}
      \dot{x} = f(x) \\
      x(0) = x_0
    \end{cases}
  \]
  Let's take $\Delta t > 0$ small. Then we have
  \[
    x(\Delta t) = x_0 + \int_{0}^{\Delta t}f(x(s)) \,\mathrm ds.
  \]
  We will approximate $x(s) \approx x_0$ on the interval $(0, \Delta t)$. Thus we have
  \begin{align*}
    x(\Delta t) &= x_0 + \int_{0}^{\Delta t}f(x(s)) \,\mathrm ds \\
                &= x_0 + \int_{0}^{\Delta t}f(x_0) \,\mathrm ds \\
                &= x_0 + f(x_0)\Delta t \tag{$x_1\ceq x_0 + f(x_0)\Delta t$}.
  \end{align*}
  Euler's method is to repeat the above to get $x_2\approx x(2\Delta t)$. We have
  \begin{align*}
    x_2 &= x_1 + f(x_1)\Delta t, \\
    x_3 &= x_2 + f(x_2)\Delta t, \\
        &\dotsb \\
    x_{n + 1} &= x_n + f(x_n)\Delta t.
  \end{align*}
\end{document}
