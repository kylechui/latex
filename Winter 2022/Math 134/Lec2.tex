\documentclass[class=article, crop=false]{standalone}
\usepackage[margin=1in]{geometry}

\usepackage{amsmath}
\usepackage{amssymb}
\usepackage{amsthm}
\usepackage{comment}
\usepackage{enumitem}
\usepackage{fancyhdr}
\usepackage{hyperref}
\usepackage{mathrsfs}
\usepackage{mathtools}
\usepackage{standalone}
\usepackage{tcolorbox}
\usepackage{tikz}
\usepackage{xcolor}

\usetikzlibrary{decorations.pathreplacing}
\tcbuselibrary{skins}

\DeclareMathOperator{\lcm}{lcm}
\DeclareMathOperator{\proj}{proj}
\DeclareMathOperator{\vspan}{span}
\DeclareMathOperator{\im}{im}
\DeclareMathOperator{\range}{range}
\DeclareMathOperator{\Diff}{Diff}
\DeclareMathOperator{\Int}{Int}
\DeclareMathOperator{\fcn}{fcn}
\DeclareMathOperator{\id}{id}

\newcommand{\N}{\ensuremath{\mathbb{N}}}
\newcommand{\Z}{\ensuremath{\mathbb{Z}}}
\newcommand{\Q}{\ensuremath{\mathbb{Q}}}
\newcommand{\R}{\ensuremath{\mathbb{R}}}
\newcommand{\C}{\ensuremath{\mathbb{C}}}
\newcommand{\F}{\ensuremath{\mathbb{F}}}
\newcommand{\lam}{\ensuremath{\lambda}}
\newcommand{\nab}{\ensuremath{\nabla}}
\newcommand{\eps}{\ensuremath{\varepsilon}}
\newcommand{\es}{\ensuremath{\varnothing}}

\newcommand{\abs}[1]{\ensuremath{\left\lvert #1 \right\rvert}}
\newcommand{\bigpar}[1]{\ensuremath{\left( #1 \right)}}
\newcommand{\norm}[1]{\ensuremath{\lVert #1\rVert}}
\newcommand{\set}[1]{\ensuremath{\left\{#1\right\}}}
\newcommand{\tuple}[1]{\ensuremath{\left\langle #1 \right\rangle}}
\newcommand{\floor}[1]{\ensuremath{\left\lfloor #1 \right\rfloor}}
\newcommand{\ceil}[1]{\ensuremath{\left\lceil #1 \right\rceil}}

\newcommand{\chapternum}{}
\newcommand{\ex}[1]{\noindent\textbf{Exercise \chapternum.{#1}.}}

\newcommand{\dx}[1]{\,\mathrm{d}#1}
\newcommand{\inv}{\ensuremath{^{-1}}}
\newcommand{\sm}{\setminus}
\newcommand{\sse}{\subseteq}
\newcommand{\ceq}{\coloneqq}

\definecolor{problemBackground}{RGB}{212,232,246}
\newenvironment{problem}[1]
  {
    \begin{tcolorbox}[
      boxrule=.5pt,
      titlerule=.5pt,
      sharp corners,
      colback=problemBackground
    ]
    \ifx &#1& \textbf{Problem. }
    \else \textbf{Problem #1.} \fi
  }
  {
    \end{tcolorbox}
  }

\definecolor{exampleBackground}{RGB}{255,249,248}
\definecolor{exampleAccent}{RGB}{158,60,14}
\newenvironment{example}[1]
  {
    \begin{tcolorbox}[
      boxrule=.5pt,
      sharp corners,
      colback=exampleBackground,
      colframe=exampleAccent,
    ]
    \color{exampleAccent}\textbf{Example.} \emph{#1}\color{black}
  }
  {
    \end{tcolorbox}
  }

\definecolor{theoremBackground}{RGB}{234,243,251}
\definecolor{theoremAccent}{RGB}{0,116,183}
\newenvironment{theorem}[1]
  {
    \begin{tcolorbox}[
      boxrule=.5pt,
      titlerule=.5pt,
      sharp corners,
      colback=theoremBackground,
      colframe=theoremAccent
    ]
      \color{theoremAccent}\textbf{Theorem --- }\emph{#1}\\\color{black}
  }
  {
    \end{tcolorbox}
  }

\definecolor{noteBackground}{RGB}{244,249,244}
\definecolor{noteAccent}{RGB}{34,139,34}
\newenvironment{note}[1]
  {
  \begin{tcolorbox}[
    enhanced,
    boxrule=0pt,
    frame hidden,
    sharp corners,
    colback=noteBackground,
    borderline west={3pt}{-1.5pt}{noteAccent}
    ]
    \ifx &#1& \color{noteAccent}\textbf{Note. }\color{black}
    \else \color{noteAccent}\textbf{Note (#1). }\color{black} \fi
    }
    {
  \end{tcolorbox}
  }

\definecolor{definitionBackground}{RGB}{246,246,246}
\newenvironment{definition}[1]
  {
    \begin{tcolorbox}[
      enhanced,
      boxrule=0pt,
      frame hidden,
      sharp corners,
      colback=definitionBackground,
      borderline west={3pt}{-1.5pt}{black}
    ]
    \textbf{Definition. }\emph{#1}\\
  }
  {
    \end{tcolorbox}
  }
\newenvironment{amatrix}[2]{
    \left[
      \begin{array}{*{#1}{c}|*{#2}c}
  }
  {
      \end{array}
    \right]
  }
\date{\the\year-\the\month-\the\day}
\author{Kyle Chui}


\fancyhf{}
\lhead{Kyle Chui}
\rhead{Page \thepage}
\pagestyle{fancy}

\begin{document}
  \section{Lecture 2}
  \subsection{Reducing ODEs to First Order Autonomous Systems}
  Consider the set of differential equations given by
  \[
    \begin{cases}
      \dot{x} = -\kappa(t)xy, \\
      \dot{y} = \kappa(t)xy - \delta y, \\
      \dot{z} = \delta y.
    \end{cases}
  \]
  Introduce a new variable, i.e. $\tau = \tau(t) = t$. Then we may rewrite the above as
  \[
    \begin{cases}
      \dot{x} = -\kappa(\tau)xy, \\
      \dot{y} = \kappa(\tau)xy - \delta y, \\
    \dot{z} = \delta y, \\
      \dot{\tau} = 1.
    \end{cases}
  \]
  Note that the above system is now autonomous.
  \begin{example}{The Pendulum} \\
    We can model the angle $\theta$ of a pendulum of length $L > 0$ by
    \[
      \ddot{\theta} + \frac{g}{L}\sin\theta = 0.
    \]
    Applying Newton's Second Law, we can get the equations
    \begin{align*}
      mL\ddot\theta &= -mg\sin\theta \\
      \theta &= \theta(t).
    \end{align*}
    Observe that if we let $x = \theta$ and $y = \dot{\theta}$, then we get
    \[
      \begin{cases}
        \dot{x} = y \\
        \dot{y} = -\frac{g}{L}\sin\theta.
      \end{cases}
    \]
  \end{example}
  \begin{example}{Pendulum with an external force} \\
    If we add an external force to our pendulum, then we get
    \[
      \ddot\theta + \frac{g}{L}\sin\theta = \frac{1}{m}F(t).
    \]
    Thus if we let $x = \theta$, $y = \dot{\theta}$, and $z = t$, then we get
    \[
      \begin{cases}
        \dot{x} = y \\
        \dot{y} = -\frac{g}{L}\sin x + \frac{1}{m}F(z) \\
        \dot{z} = 1.
      \end{cases}
    \]
  \end{example}
  \begin{note}{}
    In general, higher order ODEs of the form
    \[
      \frac{\mathrm{d}^kx}{\mathrm{d}t^k} = f(x, \frac{\mathrm{d}x}{\mathrm{d}t}, \frac{\mathrm{d}^2x}{\mathrm{d}t^2}, \dotsc, \frac{\mathrm{d}^{k - 1}x}{\mathrm{d}t^{k - 1}})
    \]
    can be converted into a first order system by taking
    \[
      z_1 = x, z_2 = \frac{\mathrm{d}x}{\mathrm{d}t},\dotsc, z_k = \frac{\mathrm{d}^{k - 1}x}{\mathrm{d}t^{k - 1}}.
    \]
    We get the system
    \[
      \begin{cases}
        \dot{z}_1 = \frac{\mathrm{d}x}{\mathrm{d}t} = z_2 \\
        \dot{z}_2 = \frac{\mathrm{d}^2x}{\mathrm{d}t^2} = z_3 \\
        \hspace{2pt}\vdots \\
        \dot{z}_k = f(z_1, z_2,\dotsc, z_k)
      \end{cases}
    \]
  \end{note}
  \subsubsection{Flows on the Line}
  We will now consider systems of the form
  \[
    \dot{x} = f(x)
  \]
  where $f\colon \R\to \R$ is a smooth function.
  \begin{example}{}
    Consider the ODE given by
    \[
      \dot{x} = x(x + 1)(x - 1)^2.
    \]
    We could use separation of variables to solve this.
  \end{example}
  \begin{note}{}
    Solutions to ODEs usually come in three different flavors:
    \begin{itemize}
      \item Analytic methods (separation of variables)
      \item Geometric methods (direction fields)
      \item Numerical methods (Euler's method)
    \end{itemize}
  \end{note}
  \begin{definition}{Phase Space}
    To help us analyze these differential equations, we can plot $\dot{x}$ against $x$ on a graph, and see the behavior around zeroes. This is called a \emph{phase space}. If some neighborhood of points around a zero $x$ tend towards $x$, then $x$ is called a \emph{stable point}. If they tend to move away from $x$, then $x$ is an \emph{unstable point}. On a phase space graph, we denote stable points with $\bullet$, unstable points with $\circ$, and other points with a half-filled circle.
  \end{definition}
  \subsubsection{Fixed Points}
  \begin{definition}{Fixed Point}
    We say that $x^*$ is a \emph{fixed point} of the system
    \[
      \dot{x} = f(x)
    \]
    if $f(x^*) = 0$. If $x^*$ is a fixed point then the system has a constant solution given by $x(t) = x^*$. These points are also known as equilibrium points, stationary points, rest points, critical points, and steady states.
  \end{definition}
  \subsubsection{Stability}
  \begin{definition}{Stability}
    Let $x^*$ be a fixed point of the system
    \[
      \dot{x} = f(x).
    \]
    For now, we say that $x^*$ is:
    \begin{itemize}
      \item \emph{Stable} if solutions starting close to $x^*$ approach $x^*$ as $t\to\infty$.
      \item \emph{Unstable} if solutions starting close to $x^*$ diverge from $x^*$ as $t\to\infty$.
      \item \emph{Half-stable} if solutions starting close to $x^*$ approach $x^*$ from one side, but diverge from the other side.
    \end{itemize}
  \end{definition}
\end{document}
