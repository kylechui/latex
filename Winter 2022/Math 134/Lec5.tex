\documentclass[class=article, crop=false]{standalone}
\usepackage[margin=1in]{geometry}

\usepackage{amsmath}
\usepackage{amssymb}
\usepackage{amsthm}
\usepackage{comment}
\usepackage{enumitem}
\usepackage{fancyhdr}
\usepackage{hyperref}
\usepackage{mathrsfs}
\usepackage{mathtools}
\usepackage{standalone}
\usepackage{tcolorbox}
\usepackage{tikz}
\usepackage{xcolor}

\usetikzlibrary{decorations.pathreplacing}
\tcbuselibrary{skins}

\DeclareMathOperator{\lcm}{lcm}
\DeclareMathOperator{\proj}{proj}
\DeclareMathOperator{\vspan}{span}
\DeclareMathOperator{\im}{im}
\DeclareMathOperator{\range}{range}
\DeclareMathOperator{\Diff}{Diff}
\DeclareMathOperator{\Int}{Int}
\DeclareMathOperator{\fcn}{fcn}
\DeclareMathOperator{\id}{id}

\newcommand{\N}{\ensuremath{\mathbb{N}}}
\newcommand{\Z}{\ensuremath{\mathbb{Z}}}
\newcommand{\Q}{\ensuremath{\mathbb{Q}}}
\newcommand{\R}{\ensuremath{\mathbb{R}}}
\newcommand{\C}{\ensuremath{\mathbb{C}}}
\newcommand{\F}{\ensuremath{\mathbb{F}}}
\newcommand{\lam}{\ensuremath{\lambda}}
\newcommand{\nab}{\ensuremath{\nabla}}
\newcommand{\eps}{\ensuremath{\varepsilon}}
\newcommand{\es}{\ensuremath{\varnothing}}

\newcommand{\abs}[1]{\ensuremath{\left\lvert #1 \right\rvert}}
\newcommand{\bigpar}[1]{\ensuremath{\left( #1 \right)}}
\newcommand{\norm}[1]{\ensuremath{\lVert #1\rVert}}
\newcommand{\set}[1]{\ensuremath{\left\{#1\right\}}}
\newcommand{\tuple}[1]{\ensuremath{\left\langle #1 \right\rangle}}
\newcommand{\floor}[1]{\ensuremath{\left\lfloor #1 \right\rfloor}}
\newcommand{\ceil}[1]{\ensuremath{\left\lceil #1 \right\rceil}}

\newcommand{\chapternum}{}
\newcommand{\ex}[1]{\noindent\textbf{Exercise \chapternum.{#1}.}}

\newcommand{\dx}[1]{\,\mathrm{d}#1}
\newcommand{\inv}{\ensuremath{^{-1}}}
\newcommand{\sm}{\setminus}
\newcommand{\sse}{\subseteq}
\newcommand{\ceq}{\coloneqq}

\definecolor{problemBackground}{RGB}{212,232,246}
\newenvironment{problem}[1]
  {
    \begin{tcolorbox}[
      boxrule=.5pt,
      titlerule=.5pt,
      sharp corners,
      colback=problemBackground
    ]
    \ifx &#1& \textbf{Problem. }
    \else \textbf{Problem #1.} \fi
  }
  {
    \end{tcolorbox}
  }

\definecolor{exampleBackground}{RGB}{255,249,248}
\definecolor{exampleAccent}{RGB}{158,60,14}
\newenvironment{example}[1]
  {
    \begin{tcolorbox}[
      boxrule=.5pt,
      sharp corners,
      colback=exampleBackground,
      colframe=exampleAccent,
    ]
    \color{exampleAccent}\textbf{Example.} \emph{#1}\color{black}
  }
  {
    \end{tcolorbox}
  }

\definecolor{theoremBackground}{RGB}{234,243,251}
\definecolor{theoremAccent}{RGB}{0,116,183}
\newenvironment{theorem}[1]
  {
    \begin{tcolorbox}[
      boxrule=.5pt,
      titlerule=.5pt,
      sharp corners,
      colback=theoremBackground,
      colframe=theoremAccent
    ]
      \color{theoremAccent}\textbf{Theorem --- }\emph{#1}\\\color{black}
  }
  {
    \end{tcolorbox}
  }

\definecolor{noteBackground}{RGB}{244,249,244}
\definecolor{noteAccent}{RGB}{34,139,34}
\newenvironment{note}[1]
  {
  \begin{tcolorbox}[
    enhanced,
    boxrule=0pt,
    frame hidden,
    sharp corners,
    colback=noteBackground,
    borderline west={3pt}{-1.5pt}{noteAccent}
    ]
    \ifx &#1& \color{noteAccent}\textbf{Note. }\color{black}
    \else \color{noteAccent}\textbf{Note (#1). }\color{black} \fi
    }
    {
  \end{tcolorbox}
  }

\definecolor{definitionBackground}{RGB}{246,246,246}
\newenvironment{definition}[1]
  {
    \begin{tcolorbox}[
      enhanced,
      boxrule=0pt,
      frame hidden,
      sharp corners,
      colback=definitionBackground,
      borderline west={3pt}{-1.5pt}{black}
    ]
    \textbf{Definition. }\emph{#1}\\
  }
  {
    \end{tcolorbox}
  }
\newenvironment{amatrix}[2]{
    \left[
      \begin{array}{*{#1}{c}|*{#2}c}
  }
  {
      \end{array}
    \right]
  }
\date{\the\year-\the\month-\the\day}
\author{Kyle Chui}


\fancyhf{}
\lhead{Kyle Chui}
\rhead{Page \thepage}
\pagestyle{fancy}

\begin{document}
  \section{Lecture 5}
  \subsection{Euler's Method}
  \begin{itemize}
    \item We want to approximate the solution of
    \[
      \begin{cases}
        \dot{x} = f(x) \\
        x(0) = x_0
      \end{cases}
    \]
    \item Given a time step $\Delta t$, for $n\geq 0$ define
    \[
      x_{n + 1} = x_n + f(x_n)\Delta t
    \]
    \item Thus we take $x_n$ to be our approximation to $x(n\Delta t)$.
  \end{itemize}
  \subsubsection{Truncation Error of a Numerical Method}
  \begin{itemize}
    \item Let $x_n\approx x(n\Delta t)$.
    \item We define the \emph{local truncation error} to be
    \[
      e_1 = x(\Delta t) - x_1.
    \]
    We will use Taylor's theorem with Lagrange residue to approximate the error:
    \begin{align*}
      x(\Delta t) &= x(0) + x'(0)\Delta t + \frac{x''(\xi)}{2}(\Delta t)^2 \tag{$\xi\in (0, \Delta t)$} \\
                  &= x_0 + f(x_0)\Delta t + \frac{f'(\xi)f(\xi)}{2}(\Delta t)^2. \tag{$\dot{x} = f$, Chain rule}
    \end{align*}
    Thus if $f$ and $f'$ are bounded and continuous, then $\abs{\text{Remainder}}\leq C(\Delta t)^2$. Substituting into our local truncation error definition, we have
    \begin{align*}
      e_1 &= x(\Delta t) - x_1 \\
          &= x(\Delta t) - (x_0 + f(x_0)\Delta t) \\
          &= \frac{x''(\xi)}{2}(\Delta t)^2.
    \end{align*}
    Hence $e_1$ is bounded above by $C(\Delta t)^2$.
    \begin{note}{}
      If we decrease $\Delta t$, then we decrease our error as well.
    \end{note}
    Euler's method is of first order because $\abs{e_1}\leq C(\Delta t)^2$. If we apply Euler's method $n$ times over some interval, then we will get $n$ errors:
    \[
      \abs{e_1} + \abs{e_2} + \dotsb + \abs{e_n}\leq Cn(\Delta t)^2 = Cn\paren{\frac{T}{n}}\Delta t = CT\Delta t.
    \]
  \end{itemize}
  \paragraph{How can we improve our results?}
  \begin{itemize}
    \item Take smaller time steps $\Delta t$.
    \begin{itemize}
      \item Unfortunately, this means that we need to perform more computations.
      \item There will be more ``round-off errors''.
    \end{itemize}
    \item Improve the approximation (see next section)
  \end{itemize}
  \subsubsection{Improved Euler's Method}
  \begin{itemize}
    \item We still want to approximate the solution of
    \[
      \begin{cases}
        \dot{x} = f(x) \\
        x(0) = x_0
      \end{cases}
    \]
    \item For $n\geq 0$:
    \begin{itemize}
      \item We make our first approximation
      \[
        \tilde x_{n + 1} = x_n + f(x_n)\Delta t.
      \]
      \item Use this to make a better approximation
      \[
        x_{n + 1} = x_n + \frac{1}{2} \paren{f(x_n) + f(\tilde x_{n + 1})}\Delta t.
      \]
    \end{itemize}
    \item Take $x_n$ to be our approximation for $x(n\Delta t)$.
  \end{itemize}
  The local truncation error for the improved Euler's method is of the form $C(\Delta t)^3$, and the global error is $CT(\Delta t)^2$.
  \subsubsection{Runge--Kutta 4th Order Method}
  \begin{itemize}
    \item We still want to approximate the solution of
    \[
      \begin{cases}
        \dot{x} = f(x) \\
        x(0) = x_0
      \end{cases}
    \]
    \item For $n\geq 0$, take:
    \begin{itemize}
      \item $k_n^{(1)} = f(x_n)\Delta t$
      \item $k_n^{(2)} = f(x_n + \frac{1}{2}k_n^{(1)})\Delta t$
      \item $k_n^{(3)} = f(x_n + \frac{1}{2}k_n^{(2)})\Delta t$
      \item $k_n^{(4)} = f(x_n + \frac{1}{2}k_n^{(3)})\Delta t$
    \end{itemize}
    \item Then we may set
    \[
      x_{n + 1} = x_n + \frac{1}{6}(k_n^{(1)} + 2k_n^{(2)} + 2k_n^{(3)} + k_n^{(4)}).
    \]
    \item The local truncation error satisfies
    \[
      \abs{e_1}\leq C(\Delta t)^5,
    \]
    and the global error is 4\tsup{th} order, satisfying $CT(\Delta t)^4$.
  \end{itemize}
  \subsection{Existence and Uniqueness of Solutions}
  \begin{theorem}{Cauchy--Peano Existence Theorem}
    Let $f\colon (a, b)\to \R$ be continuous and $x_0\in (a, b)$. Then there exists some $\delta > 0$ and a solution $x\colon [-\delta, \delta]\to \R$ of the IVP
    \[
      \begin{cases}
        \dot{x} = f(x) \\ 
        x(0) = x_0
      \end{cases}
    \]
  \end{theorem}
  \begin{theorem}{Picard--Lindel\"of Existence and Uniqueness Theorem}
    Let $f\colon (a, b)\to \R$ be continuous and $x_0\in (a, b)$. If $f$ is locally Lipschitz continuous, then there exists a \emph{unique local solution} $\bar{x}\in C^1(I, \R)$ of the IVP
    \[
      \begin{cases}
        \dot{x} = f(x) \\ 
        x(0) = x_0
      \end{cases}
    \]
    where $I$ is some interval around $0$.
  \end{theorem}
\end{document}
