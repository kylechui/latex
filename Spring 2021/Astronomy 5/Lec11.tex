\documentclass[class=article, crop=false]{standalone}
\usepackage[margin=1in]{geometry}

\usepackage{amsmath}
\usepackage{amssymb}
\usepackage{amsthm}
\usepackage{comment}
\usepackage{enumitem}
\usepackage{fancyhdr}
\usepackage{hyperref}
\usepackage{mathrsfs}
\usepackage{mathtools}
\usepackage{standalone}
\usepackage{tcolorbox}
\usepackage{tikz}
\usepackage{xcolor}

\usetikzlibrary{decorations.pathreplacing}
\tcbuselibrary{skins}

\DeclareMathOperator{\lcm}{lcm}
\DeclareMathOperator{\proj}{proj}
\DeclareMathOperator{\vspan}{span}
\DeclareMathOperator{\im}{im}
\DeclareMathOperator{\range}{range}
\DeclareMathOperator{\Diff}{Diff}
\DeclareMathOperator{\Int}{Int}
\DeclareMathOperator{\fcn}{fcn}
\DeclareMathOperator{\id}{id}

\newcommand{\N}{\ensuremath{\mathbb{N}}}
\newcommand{\Z}{\ensuremath{\mathbb{Z}}}
\newcommand{\Q}{\ensuremath{\mathbb{Q}}}
\newcommand{\R}{\ensuremath{\mathbb{R}}}
\newcommand{\C}{\ensuremath{\mathbb{C}}}
\newcommand{\F}{\ensuremath{\mathbb{F}}}
\newcommand{\lam}{\ensuremath{\lambda}}
\newcommand{\nab}{\ensuremath{\nabla}}
\newcommand{\eps}{\ensuremath{\varepsilon}}
\newcommand{\es}{\ensuremath{\varnothing}}

\newcommand{\abs}[1]{\ensuremath{\left\lvert #1 \right\rvert}}
\newcommand{\bigpar}[1]{\ensuremath{\left( #1 \right)}}
\newcommand{\norm}[1]{\ensuremath{\lVert #1\rVert}}
\newcommand{\set}[1]{\ensuremath{\left\{#1\right\}}}
\newcommand{\tuple}[1]{\ensuremath{\left\langle #1 \right\rangle}}
\newcommand{\floor}[1]{\ensuremath{\left\lfloor #1 \right\rfloor}}
\newcommand{\ceil}[1]{\ensuremath{\left\lceil #1 \right\rceil}}

\newcommand{\chapternum}{}
\newcommand{\ex}[1]{\noindent\textbf{Exercise \chapternum.{#1}.}}

\newcommand{\dx}[1]{\,\mathrm{d}#1}
\newcommand{\inv}{\ensuremath{^{-1}}}
\newcommand{\sm}{\setminus}
\newcommand{\sse}{\subseteq}
\newcommand{\ceq}{\coloneqq}

\definecolor{problemBackground}{RGB}{212,232,246}
\newenvironment{problem}[1]
  {
    \begin{tcolorbox}[
      boxrule=.5pt,
      titlerule=.5pt,
      sharp corners,
      colback=problemBackground
    ]
    \ifx &#1& \textbf{Problem. }
    \else \textbf{Problem #1.} \fi
  }
  {
    \end{tcolorbox}
  }

\definecolor{exampleBackground}{RGB}{255,249,248}
\definecolor{exampleAccent}{RGB}{158,60,14}
\newenvironment{example}[1]
  {
    \begin{tcolorbox}[
      boxrule=.5pt,
      sharp corners,
      colback=exampleBackground,
      colframe=exampleAccent,
    ]
    \color{exampleAccent}\textbf{Example.} \emph{#1}\color{black}
  }
  {
    \end{tcolorbox}
  }

\definecolor{theoremBackground}{RGB}{234,243,251}
\definecolor{theoremAccent}{RGB}{0,116,183}
\newenvironment{theorem}[1]
  {
    \begin{tcolorbox}[
      boxrule=.5pt,
      titlerule=.5pt,
      sharp corners,
      colback=theoremBackground,
      colframe=theoremAccent
    ]
      \color{theoremAccent}\textbf{Theorem --- }\emph{#1}\\\color{black}
  }
  {
    \end{tcolorbox}
  }

\definecolor{noteBackground}{RGB}{244,249,244}
\definecolor{noteAccent}{RGB}{34,139,34}
\newenvironment{note}[1]
  {
  \begin{tcolorbox}[
    enhanced,
    boxrule=0pt,
    frame hidden,
    sharp corners,
    colback=noteBackground,
    borderline west={3pt}{-1.5pt}{noteAccent}
    ]
    \ifx &#1& \color{noteAccent}\textbf{Note. }\color{black}
    \else \color{noteAccent}\textbf{Note (#1). }\color{black} \fi
    }
    {
  \end{tcolorbox}
  }

\definecolor{definitionBackground}{RGB}{246,246,246}
\newenvironment{definition}[1]
  {
    \begin{tcolorbox}[
      enhanced,
      boxrule=0pt,
      frame hidden,
      sharp corners,
      colback=definitionBackground,
      borderline west={3pt}{-1.5pt}{black}
    ]
    \textbf{Definition. }\emph{#1}\\
  }
  {
    \end{tcolorbox}
  }
\newenvironment{amatrix}[2]{
    \left[
      \begin{array}{*{#1}{c}|*{#2}c}
  }
  {
      \end{array}
    \right]
  }
\date{\the\year-\the\month-\the\day}
\author{Kyle Chui}


\fancyhf{}
\lhead{Kyle Chui}
\rhead{Page \thepage}
\pagestyle{fancy}

\begin{document}
  \section{Lecture 11}
  \subsection{Mars}
  \begin{itemize}
    \item It would take a long time to get there because it is $1$.5 AU away from the Sun (it would take a few years)
    \item You have little to no protection from radiation (little atmosphere)
  \end{itemize}
  \subsubsection{Mars vs The Earth}
  \begin{itemize}
    \item It has half Earth's radius and a tenth Earth's mass
    \item It gets 44\% less heat/light
    \item The axis tilt is about the same as the Earth's---has seasons but the tilt is more variable
    \item Similar rotation period
    \item Orbit is more elliptical than the Earth's
    \item Very thin CO\tsub{2} atmosphere: little greenhouse effect
    \item Cold deser; red look due to soil color from oxidation
    \item The main difference is that \emph{Mars is smaller}
  \end{itemize}
  \begin{note}{}
    Because Mars is so small, it is not geologically active (because it is not hot enough).
  \end{note}
  \subsubsection{Martian Seasons}
  \begin{itemize}
    \item Martian axial tilt is $25$ degrees compared to Earth's $23.5$
    \item Mars' orbit is significantly more elliptical, which leads to more variable seasons
    \item Martian winters are cold enough to freeze a portion of the atmosphere
    \item Winds associated with cycling of carbon dioxide gas lead to large, sometimes global wind storms
  \end{itemize}
  \subsubsection{Mars Today}
  \begin{itemize}
    \item It has atmospheric CO\tsub{2} just like Venus, but $10000$ times thinner
    \item Low atomspheric pressure causes water to sublimate
    \begin{note}{}
      Liquid water cannot exist on Mars.
    \end{note}
  \end{itemize}
  \subsubsection{Features of Mars}
  \begin{itemize}
    \item It seems to be split into Northern lowlands and Southern highlands
    \item The southern hemisphere seems more cratered (there is evidence of erosion in some craters)
    \item There are three regions of varying age on Mars
  \end{itemize}
  \subsubsection{Martian Volcanism and Tectonics}
  \begin{itemize}
    \item Evidence for Martian volcanism includes four massive volcanoes of the Tharsis Bulge
    \item Sedimentary walls are similar to the Grand canyon
  \end{itemize}
  \subsubsection{Evidence for Ancient Water}
  \begin{itemize}
    \item Orbital evidence---missions yield evidence of riverbeds
    \item Rover evidence---both orbiters and rovers have found hydrated minerals
    \item Phoenix lander found water ice on Mars
  \end{itemize}
  \subsubsection{Evidence for Recent Water}
  \begin{itemize}
    \item Dark streaks on crater walls seem to grow in the summer months
    \item All evidence points to a planet that was once much wetter and perhaps still contains water ice or subsurface liquid water
  \end{itemize}
  \subsubsection{The Climate History of Mars}
  \begin{itemize}
    \item Early Mars might have had a hot convecting core that would've protected it from the solar wind (too small to have a convection core)
    \item Mars' axial tilt could vary from $0$ to as much as $60$ degrees
  \end{itemize}
  \subsubsection{The Viking Experiments}
  \begin{itemize}
    \item Carbon assimilation experiment---mixed martian soil with CO\tsub{2} and CO to look for metabolic results (no indication of life)
    \item Gas exchange experiment---mixed martian soil with a nutrient ``broth'' (no indication of life)
    \item Labeled release experiment---mixed martian soil with a radioactive nutrient ``broth'' (expected signs of life)
    \item Gas chromatograph experiment---no organic material in martian soil
  \end{itemize}
  The results of these tests are inconclusive---more testing is needed.
  \begin{note}{}
    We need to make sure that we don't contaminate Mars with Earth life.
  \end{note}
  There are four lines of evidence for life from ALH84001:
  \begin{enumerate}
    \item Layered carbonate grains similar to biological activity on EArth
    \item Polycyclic aromatic hydrocarbonds produced by both living and non-living processes (high relative amounts)
    \item Crystals match similar crystals on Earth produced by bacteria
    \item Images of rod-shaped structures resemble fossilized bacteria
  \end{enumerate}
  However, this is equally refuted by:
  \begin{enumerate}
    \item Non-biological mechanisms also produce these carbonates
    \item PAHs are produced by both living and nonliving reactions
    \item Similarities between crystals may be coincidence
    \item These structures have also been seen on other meteorites and seem too small for RNA/DNA
  \end{enumerate}
  \subsubsection{Terraforming Mars}
  \begin{itemize}
    \item In the future, it might be possible to make Mars habitable
    \item However, this is currently impractical, because the scale is too massive
  \end{itemize}
\end{document}
