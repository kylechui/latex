\documentclass[class=article, crop=false]{standalone}
\usepackage[margin=1in]{geometry}

\usepackage{amsmath}
\usepackage{amssymb}
\usepackage{amsthm}
\usepackage{comment}
\usepackage{enumitem}
\usepackage{fancyhdr}
\usepackage{hyperref}
\usepackage{mathrsfs}
\usepackage{mathtools}
\usepackage{standalone}
\usepackage{tcolorbox}
\usepackage{tikz}
\usepackage{xcolor}

\usetikzlibrary{decorations.pathreplacing}
\tcbuselibrary{skins}

\DeclareMathOperator{\lcm}{lcm}
\DeclareMathOperator{\proj}{proj}
\DeclareMathOperator{\vspan}{span}
\DeclareMathOperator{\im}{im}
\DeclareMathOperator{\range}{range}
\DeclareMathOperator{\Diff}{Diff}
\DeclareMathOperator{\Int}{Int}
\DeclareMathOperator{\fcn}{fcn}
\DeclareMathOperator{\id}{id}

\newcommand{\N}{\ensuremath{\mathbb{N}}}
\newcommand{\Z}{\ensuremath{\mathbb{Z}}}
\newcommand{\Q}{\ensuremath{\mathbb{Q}}}
\newcommand{\R}{\ensuremath{\mathbb{R}}}
\newcommand{\C}{\ensuremath{\mathbb{C}}}
\newcommand{\F}{\ensuremath{\mathbb{F}}}
\newcommand{\lam}{\ensuremath{\lambda}}
\newcommand{\nab}{\ensuremath{\nabla}}
\newcommand{\eps}{\ensuremath{\varepsilon}}
\newcommand{\es}{\ensuremath{\varnothing}}

\newcommand{\abs}[1]{\ensuremath{\left\lvert #1 \right\rvert}}
\newcommand{\bigpar}[1]{\ensuremath{\left( #1 \right)}}
\newcommand{\norm}[1]{\ensuremath{\lVert #1\rVert}}
\newcommand{\set}[1]{\ensuremath{\left\{#1\right\}}}
\newcommand{\tuple}[1]{\ensuremath{\left\langle #1 \right\rangle}}
\newcommand{\floor}[1]{\ensuremath{\left\lfloor #1 \right\rfloor}}
\newcommand{\ceil}[1]{\ensuremath{\left\lceil #1 \right\rceil}}

\newcommand{\chapternum}{}
\newcommand{\ex}[1]{\noindent\textbf{Exercise \chapternum.{#1}.}}

\newcommand{\dx}[1]{\,\mathrm{d}#1}
\newcommand{\inv}{\ensuremath{^{-1}}}
\newcommand{\sm}{\setminus}
\newcommand{\sse}{\subseteq}
\newcommand{\ceq}{\coloneqq}

\definecolor{problemBackground}{RGB}{212,232,246}
\newenvironment{problem}[1]
  {
    \begin{tcolorbox}[
      boxrule=.5pt,
      titlerule=.5pt,
      sharp corners,
      colback=problemBackground
    ]
    \ifx &#1& \textbf{Problem. }
    \else \textbf{Problem #1.} \fi
  }
  {
    \end{tcolorbox}
  }

\definecolor{exampleBackground}{RGB}{255,249,248}
\definecolor{exampleAccent}{RGB}{158,60,14}
\newenvironment{example}[1]
  {
    \begin{tcolorbox}[
      boxrule=.5pt,
      sharp corners,
      colback=exampleBackground,
      colframe=exampleAccent,
    ]
    \color{exampleAccent}\textbf{Example.} \emph{#1}\color{black}
  }
  {
    \end{tcolorbox}
  }

\definecolor{theoremBackground}{RGB}{234,243,251}
\definecolor{theoremAccent}{RGB}{0,116,183}
\newenvironment{theorem}[1]
  {
    \begin{tcolorbox}[
      boxrule=.5pt,
      titlerule=.5pt,
      sharp corners,
      colback=theoremBackground,
      colframe=theoremAccent
    ]
      \color{theoremAccent}\textbf{Theorem --- }\emph{#1}\\\color{black}
  }
  {
    \end{tcolorbox}
  }

\definecolor{noteBackground}{RGB}{244,249,244}
\definecolor{noteAccent}{RGB}{34,139,34}
\newenvironment{note}[1]
  {
  \begin{tcolorbox}[
    enhanced,
    boxrule=0pt,
    frame hidden,
    sharp corners,
    colback=noteBackground,
    borderline west={3pt}{-1.5pt}{noteAccent}
    ]
    \ifx &#1& \color{noteAccent}\textbf{Note. }\color{black}
    \else \color{noteAccent}\textbf{Note (#1). }\color{black} \fi
    }
    {
  \end{tcolorbox}
  }

\definecolor{definitionBackground}{RGB}{246,246,246}
\newenvironment{definition}[1]
  {
    \begin{tcolorbox}[
      enhanced,
      boxrule=0pt,
      frame hidden,
      sharp corners,
      colback=definitionBackground,
      borderline west={3pt}{-1.5pt}{black}
    ]
    \textbf{Definition. }\emph{#1}\\
  }
  {
    \end{tcolorbox}
  }
\newenvironment{amatrix}[2]{
    \left[
      \begin{array}{*{#1}{c}|*{#2}c}
  }
  {
      \end{array}
    \right]
  }
\date{\the\year-\the\month-\the\day}
\author{Kyle Chui}


\fancyhf{}
\lhead{Kyle Chui}
\rhead{Page \thepage}
\pagestyle{fancy}

\begin{document}
  \section{Lecture 5}
  \subsection{Formation of the Solar System}
  An object's orbit cannot change spontaneously---it can only change if it gains or loses \emph{orbital energy}, the sum of kinetic and gravitational potential energy. \par
  \subsubsection{Kepler and Newton}
  Kepler's first two laws apply to \emph{all} orbiting objects, not just planets. In addition to being elliptic (bound paths), orbits can be unbound (hyperbolic or parabolic). Newton's Law of gravity permits all of these orbits. \par
  Newton generalised Kepler's Third Law, observing that: \par
  If a small object orbits a larger one, and if you measure the smaller object's orbital period and average orbital radius, then you can calculate the mass of the larger object. This is given by the equation:
  \[
    p^2 = \frac{4\pi^2}{G(M_1+M_2)}a^3,
  \]
  where $p$ is the orbital period, $a$ is the average orbital distance (between centres), and $M_1+M_2$ is the sum of the object masses.
  \begin{example}{Kepler's Third Law (Generalised)}
    \begin{itemize}
      \item You can calculate the mass of the Sun from the Earth's orbital period and average distance (1 year and $1$ AU, respectively).
      \item You can calculate the mass of Earth from the orbital period and average distance of \emph{any} orbiting satellite, including the Moon.
      \item You can calculate the mass of Jupiter from the orbital period and orbital radius of any one of Jupiter's moons.
    \end{itemize}
  \end{example}
  \subsubsection{Changing an Orbit}
  Conservation of angular momentum holds, always---orbits change either due to friction or gravitational encounters with another object.
  \begin{definition}{Escape Velocity}
    If an object gains enough orbital energy, it may \emph{escape} (change from a bound to an unbound orbit). For the Moon, this \emph{escape velocity} is about $2$ km/sec, for Mars it is about $5$ km/sec, and for the Earth, it's about $11.1$ km/sec.
  \end{definition}
  \begin{note}{}
    Escape velocity depends on the mass of the Earth, not the mass of the object.
  \end{note}
  \subsection{A Brief Tour of the Solar System}
  The planets are tiny in comparison to the Sun and the solar system is mostly empty space in between planetary orbits.
  \subsubsection{The Sun}
  \begin{itemize}
    \item Contains over $99.8$\% of the solar system's mass.
    \item Made mostly of \emph{ionized} Hydrogen and Helium gas (plasma).
    \item Converts $4$ million tons of mass into energy each second.
    \item Its radius is $696000$ km, approximately $108$ times the radius of the Earth.
  \end{itemize}
  \subsubsection{Mercury}
  \begin{itemize}
    \item Made of metal and rock; large \emph{iron} core.
    \item Desolate, cratered like our Moon; long, tall, steep cliffs.
    \item Temperatures fluctuate drastically: from $425^\circ$C (day) to $-170^\circ$C (night)
  \end{itemize}
  \subsubsection{Venus}
  \begin{itemize}
    \item Nearly identical in size to Earth; surface hidden by thick clouds.
    \item Hellish conditions due to an extreme \emph{greenhouse effect}.
    \item Even hotter than Mercury: $470^\circ$, both day and night.
    \item Atmospheric pressure is equivalent to $1$ km depth in Earth's oceans.
    \item No oxygen, no water.
    \item Venus may have been habitable (had liquid water) for $3$ billion years.
  \end{itemize}
  \subsubsection{Earth}
  \begin{itemize}
    \item An oasis of life; bio-generated oxygen in the atmosphere.
    \item The only planet with liquid water in the solar system; about 3/4 of our surface is covered in water.
    \item A surprisingly large moon (1/4 radius, 1/80 mass).
  \end{itemize}
  \subsubsection{Mars}
  \begin{itemize}
    \item Giant volcanoes, a huge canyon, polar caps, and more.
    \item Water flowed in the distant past; could there have been life?
    \item Thin atmosphere of carbon dioxide.
    \item Water probably flowed; surface habitable; $1$ billion years.
  \end{itemize}
  \subsubsection{Jupiter}
  \begin{itemize}
    \item Much farther from the Sun than the inner planets (5.2 AU).
    \item Very different composition---A large gas ball composed mostly of Hydrogen and Helium.
    \item No solid surface.
    \item Huge: $318$ times Earth's mass and over $1000$ times Earth's volume.
    \item Many moons and rings.
  \end{itemize}
  \paragraph{Moons of Jupiter}
  \begin{itemize}
    \item Io: Yellowish, with active volcanoes all over.
    \item \emph{Europa: Possible subsurface ocean; possible place to search for life.}
    \item Ganymede: Largest moon in the solar system---larger than Mercury.
    \item Callisto: a large, cratered ``ice ball'' with unexplained surface features.
  \end{itemize}
  \subsubsection{Saturn}
  \begin{itemize}
    \item Giant and gaseous like Jupiter.
    \item Most spectacular rings of the $4$ jovian planets.
    \item Many moons, including cloudy Titan.
    \item \emph{Enceladus has a warm, salty ocean.}
    \item The \textbf{Cassini} spacecraft spent $10$ years studying Saturn.
  \end{itemize}
  \subsubsection{Uranus}
  \begin{itemize}
    \item Smaller than Jupiter or Saturn, but still much larger than Earth (4x).
    \item Made of Hydrogen and Helium gas, and hydrogen compounds (H\tsub{2}O, NH\tsub{3}, CH\tsub{4}).
    \item Extreme axis tilt---nearly tipped on its ``side''. This causes extreme seasons during its $84$ year orbit.
    \item It has moons and rings.
  \end{itemize}
  \subsubsection{Neptune}
  \begin{itemize}
    \item Similar to Uranus (except for the axis tilt).
    \item Many moons, including unusual Triton (it orbits ``backwards'').
    \item Triton is larger than Pluto.
  \end{itemize}
  \subsubsection{Pluto and the Other Dwarf Planets}
  \begin{itemize}
    \item Much smaller than other planets (0.18 Earth's radius).
    \item Icy, comet-like composition.
    \item Pluto's largest moon (Charon) is similar in size to Pluto itself.
  \end{itemize}
  In January $2006$, the New Horizons probe was sent to investigate Pluto. It flew by Pluto on July $14$, $2015$. We now have incredible detail of Pluto's surface.
  \subsection{Clues to the Formation of the Solar System}
  \begin{itemize}
    \item The Sun, planets, and large moons orbit and rotate in an organised way.
    \item There are two major planet types: small, rocky planets close to the Sun, and large, gaseous (jovial) planets further away from the Sun.
  \end{itemize}
  A successful theory of solar system formation must allow for exceptions to general rules, i.e.:
  \begin{itemize}
    \item Earth's relatively large moon.
    \item Uranus's odd tilt.
  \end{itemize}
  \subsubsection{Nebular Theory}
  According to the \emph{nebular theory}, our solar system formed from a giant cloud of interstellar gas which also contained tiny solid grains of heavier elements. \par
  The conservation of angular momentum caused the gas cloud to rotate faster and faster as it shrank (due to gravity). The many particles in the cloud collided with each other, flattening the cloud.
  \subsubsection{The Formation of Planets}
  There are two types of planets because of temperature. Near the Sun, where it was extremely hot, only iron, nickel, and other heavy metals could condense out of the gas phase. Further away from the Sun, where it was cold, were the only regions where water, methane and ammonia to condense and make ``ice''. We call the border between these two areas the ``frost line''.
  \begin{definition}{Planetesimals}
    Little pieces of matter that condensed out of the nebula.
  \end{definition}
  \begin{definition}{Accretion}
    The assembly of planets from planetesimals due to gravity, with ice sticking pieces together, is called \emph{accretion}.
  \end{definition}
  \begin{definition}{Fragmentation}
    The process by which small, denser regions within the nebula can collapse more quickly than the rest.
  \end{definition}
  There is further evidence for accretion:
  \begin{itemize}
    \item Asteroids are planetesimals that formed inside the frost line.
    \item Comets are planetesimals that formed outside the frost line.
  \end{itemize}
  Earth's moon was probably created when a large planetesimal crashed into the young Earth (nearly $4.5$ billion years ago).
  \subsubsection{Finding the Age of the Solar System}
  We find the age of rocks that compose a planet via radioactive dating (which works by calculating the proportion of a radioactive isotope). Each radioactive isotope has a \emph{half-life}, the time it takes for approximately half of the isotope to decay into something stable. \par
  Using radioactive dating, we have found meteorites that are approximately $4.6$ billion years old, which is also about the age of the Sun (found via separate analysis).
\end{document}
