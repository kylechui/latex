\documentclass[class=article, crop=false]{standalone}
\usepackage[margin=1in]{geometry}

\usepackage{amsmath}
\usepackage{amssymb}
\usepackage{amsthm}
\usepackage{comment}
\usepackage{enumitem}
\usepackage{fancyhdr}
\usepackage{hyperref}
\usepackage{mathrsfs}
\usepackage{mathtools}
\usepackage{standalone}
\usepackage{tcolorbox}
\usepackage{tikz}
\usepackage{xcolor}

\usetikzlibrary{decorations.pathreplacing}
\tcbuselibrary{skins}

\DeclareMathOperator{\lcm}{lcm}
\DeclareMathOperator{\proj}{proj}
\DeclareMathOperator{\vspan}{span}
\DeclareMathOperator{\im}{im}
\DeclareMathOperator{\range}{range}
\DeclareMathOperator{\Diff}{Diff}
\DeclareMathOperator{\Int}{Int}
\DeclareMathOperator{\fcn}{fcn}
\DeclareMathOperator{\id}{id}

\newcommand{\N}{\ensuremath{\mathbb{N}}}
\newcommand{\Z}{\ensuremath{\mathbb{Z}}}
\newcommand{\Q}{\ensuremath{\mathbb{Q}}}
\newcommand{\R}{\ensuremath{\mathbb{R}}}
\newcommand{\C}{\ensuremath{\mathbb{C}}}
\newcommand{\F}{\ensuremath{\mathbb{F}}}
\newcommand{\lam}{\ensuremath{\lambda}}
\newcommand{\nab}{\ensuremath{\nabla}}
\newcommand{\eps}{\ensuremath{\varepsilon}}
\newcommand{\es}{\ensuremath{\varnothing}}

\newcommand{\abs}[1]{\ensuremath{\left\lvert #1 \right\rvert}}
\newcommand{\bigpar}[1]{\ensuremath{\left( #1 \right)}}
\newcommand{\norm}[1]{\ensuremath{\lVert #1\rVert}}
\newcommand{\set}[1]{\ensuremath{\left\{#1\right\}}}
\newcommand{\tuple}[1]{\ensuremath{\left\langle #1 \right\rangle}}
\newcommand{\floor}[1]{\ensuremath{\left\lfloor #1 \right\rfloor}}
\newcommand{\ceil}[1]{\ensuremath{\left\lceil #1 \right\rceil}}

\newcommand{\chapternum}{}
\newcommand{\ex}[1]{\noindent\textbf{Exercise \chapternum.{#1}.}}

\newcommand{\dx}[1]{\,\mathrm{d}#1}
\newcommand{\inv}{\ensuremath{^{-1}}}
\newcommand{\sm}{\setminus}
\newcommand{\sse}{\subseteq}
\newcommand{\ceq}{\coloneqq}

\definecolor{problemBackground}{RGB}{212,232,246}
\newenvironment{problem}[1]
  {
    \begin{tcolorbox}[
      boxrule=.5pt,
      titlerule=.5pt,
      sharp corners,
      colback=problemBackground
    ]
    \ifx &#1& \textbf{Problem. }
    \else \textbf{Problem #1.} \fi
  }
  {
    \end{tcolorbox}
  }

\definecolor{exampleBackground}{RGB}{255,249,248}
\definecolor{exampleAccent}{RGB}{158,60,14}
\newenvironment{example}[1]
  {
    \begin{tcolorbox}[
      boxrule=.5pt,
      sharp corners,
      colback=exampleBackground,
      colframe=exampleAccent,
    ]
    \color{exampleAccent}\textbf{Example.} \emph{#1}\color{black}
  }
  {
    \end{tcolorbox}
  }

\definecolor{theoremBackground}{RGB}{234,243,251}
\definecolor{theoremAccent}{RGB}{0,116,183}
\newenvironment{theorem}[1]
  {
    \begin{tcolorbox}[
      boxrule=.5pt,
      titlerule=.5pt,
      sharp corners,
      colback=theoremBackground,
      colframe=theoremAccent
    ]
      \color{theoremAccent}\textbf{Theorem --- }\emph{#1}\\\color{black}
  }
  {
    \end{tcolorbox}
  }

\definecolor{noteBackground}{RGB}{244,249,244}
\definecolor{noteAccent}{RGB}{34,139,34}
\newenvironment{note}[1]
  {
  \begin{tcolorbox}[
    enhanced,
    boxrule=0pt,
    frame hidden,
    sharp corners,
    colback=noteBackground,
    borderline west={3pt}{-1.5pt}{noteAccent}
    ]
    \ifx &#1& \color{noteAccent}\textbf{Note. }\color{black}
    \else \color{noteAccent}\textbf{Note (#1). }\color{black} \fi
    }
    {
  \end{tcolorbox}
  }

\definecolor{definitionBackground}{RGB}{246,246,246}
\newenvironment{definition}[1]
  {
    \begin{tcolorbox}[
      enhanced,
      boxrule=0pt,
      frame hidden,
      sharp corners,
      colback=definitionBackground,
      borderline west={3pt}{-1.5pt}{black}
    ]
    \textbf{Definition. }\emph{#1}\\
  }
  {
    \end{tcolorbox}
  }
\newenvironment{amatrix}[2]{
    \left[
      \begin{array}{*{#1}{c}|*{#2}c}
  }
  {
      \end{array}
    \right]
  }
\date{\the\year-\the\month-\the\day}
\author{Kyle Chui}


\fancyhf{}
\lhead{Kyle Chui}
\rhead{Page \thepage}
\pagestyle{fancy}

\begin{document}
  \section{Lecture 2}
  \subsection{Biology}
  \begin{definition}{Biology}
    \emph{Biology} is the study of the formation and evolution of life.
  \end{definition}
  \begin{itemize}
    \item Planetary science and astronomy yield context for life.
    \item Biological research is limited to Earth-based life, yielding poor context for possibilities of universal life.
    \item Understanding the conditions that led to life on Earth helps us identify potential locations for extraterrestrial life.
  \end{itemize}
  \subsection{The Human Adventure}
  \begin{itemize}
    \item The development of Astronomy is deeply intertwined with the development of civilisation and changes in society.
    \item Revolutions in astronomy have gone hand in hand with giant leaps in technology and science.
    \item Astronomy is the science that asks the deep question about the origins of humanity.
  \end{itemize}
  \subsection{Modern View of the Universe}
  \begin{itemize}
    \item What is our physical place in the universe?
    \item What is known about planets, stars, galaxies, space and time?
    \item How do we know what we know?
  \end{itemize}
  \subsection{Universal Objects}
  \begin{definition}{Star}
    A \emph{star} is a large, glowing ball of mostly hydrogen gas that generated heat and light by nuclear fusion. Larger stars generally have shorter lifespans.
  \end{definition}
  \begin{note}{}
    Hydrogen is the lightest and most abundant element in the universe.
  \end{note}
  \begin{definition}{Planet}
    A \emph{planet} is a moderately large object which orbits a star; it shines by reflected light. They may be rocky, icy, or gaseous in composition.
  \end{definition}
  Pluto and the family of objects beyond Neptune are now called ``dwarf planets''.
  \begin{definition}{Moon}
    A \emph{moon} is an object which orbits a planet---may also be referred to as a \emph{satellite}.
  \end{definition}
  \begin{definition}{Asteroid}
    An \emph{asteroid} is a relatively small and rocky object which orbits a star. They usually do not have the mass (and thus the gravity) to be spherical in shape.
  \end{definition}
  \begin{definition}{Comet}
    Comets are icy dust balls that get vaporised when they get too close to a star, leaving a bright trail in the sky. They usually originate outside the solar system.
  \end{definition}
  \begin{definition}{Solar (Star) System}
    A star and all the material that orbits it, including planets, asteroids and comets, and all the moons that orbit those planets.
  \end{definition}
  \begin{definition}{Nebula}
    A \emph{nebula} is a cloud of gas that is gravitationally attracted to itself, and will eventually become a star (system). Like most things, it is mostly composed of hydrogen, but also contains small solid particulates of carbon and silicon.
  \end{definition}
  \begin{definition}{Galaxy}
    An enormous ``island of stars'' far out in space, all held together by gravity and orbiting a common centre.
  \end{definition}
  \begin{definition}{Universe}
    The sum total of all matter and energy; that is, everything within and between all galaxies.
  \end{definition}
  \begin{definition}{Astronomical Unit}
    The (average) distance from the Earth to the Sun is $150$ million km or $93$ million miles, and is called an \emph{astronomical unit}.
  \end{definition}
  \begin{definition}{Atom}
    \emph{Atoms} are the microscopic ``building blocks'' of all chemical elements---92 of which occur in nature.
  \end{definition}
  \subsection{Where Do We Come From?}
  \begin{itemize}
    \item The first (and simplest) atoms (hydrogen and helium) were created during the \emph{Big Bang}.
    \item More complex atoms (like carbon and oxygen) were created much later, inside stars.
    \item As stars age and die, they expel matter into space, which in turn forms new stars and planets.
  \end{itemize}
  \subsection{Light is the Measure of All Things}
  \begin{itemize}
    \item The speed of light is the absolute speed limit of the universe.
    \item As far as we know, there are no methods of travelling faster than light.
  \end{itemize}
  \subsection{A Universe in Motion}
  \begin{itemize}
    \item We are constantly moving in space---on ``spaceship Earth''. The earth rotates around its axis once per day, which results in a speed of around $1000$ mph at the equator.
  \end{itemize}
\end{document}
