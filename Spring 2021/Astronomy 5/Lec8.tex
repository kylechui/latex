\documentclass[class=article, crop=false]{standalone}
\usepackage[margin=1in]{geometry}

\usepackage{amsmath}
\usepackage{amssymb}
\usepackage{amsthm}
\usepackage{comment}
\usepackage{enumitem}
\usepackage{fancyhdr}
\usepackage{hyperref}
\usepackage{mathrsfs}
\usepackage{mathtools}
\usepackage{standalone}
\usepackage{tcolorbox}
\usepackage{tikz}
\usepackage{xcolor}

\usetikzlibrary{decorations.pathreplacing}
\tcbuselibrary{skins}

\DeclareMathOperator{\lcm}{lcm}
\DeclareMathOperator{\proj}{proj}
\DeclareMathOperator{\vspan}{span}
\DeclareMathOperator{\im}{im}
\DeclareMathOperator{\range}{range}
\DeclareMathOperator{\Diff}{Diff}
\DeclareMathOperator{\Int}{Int}
\DeclareMathOperator{\fcn}{fcn}
\DeclareMathOperator{\id}{id}

\newcommand{\N}{\ensuremath{\mathbb{N}}}
\newcommand{\Z}{\ensuremath{\mathbb{Z}}}
\newcommand{\Q}{\ensuremath{\mathbb{Q}}}
\newcommand{\R}{\ensuremath{\mathbb{R}}}
\newcommand{\C}{\ensuremath{\mathbb{C}}}
\newcommand{\F}{\ensuremath{\mathbb{F}}}
\newcommand{\lam}{\ensuremath{\lambda}}
\newcommand{\nab}{\ensuremath{\nabla}}
\newcommand{\eps}{\ensuremath{\varepsilon}}
\newcommand{\es}{\ensuremath{\varnothing}}

\newcommand{\abs}[1]{\ensuremath{\left\lvert #1 \right\rvert}}
\newcommand{\bigpar}[1]{\ensuremath{\left( #1 \right)}}
\newcommand{\norm}[1]{\ensuremath{\lVert #1\rVert}}
\newcommand{\set}[1]{\ensuremath{\left\{#1\right\}}}
\newcommand{\tuple}[1]{\ensuremath{\left\langle #1 \right\rangle}}
\newcommand{\floor}[1]{\ensuremath{\left\lfloor #1 \right\rfloor}}
\newcommand{\ceil}[1]{\ensuremath{\left\lceil #1 \right\rceil}}

\newcommand{\chapternum}{}
\newcommand{\ex}[1]{\noindent\textbf{Exercise \chapternum.{#1}.}}

\newcommand{\dx}[1]{\,\mathrm{d}#1}
\newcommand{\inv}{\ensuremath{^{-1}}}
\newcommand{\sm}{\setminus}
\newcommand{\sse}{\subseteq}
\newcommand{\ceq}{\coloneqq}

\definecolor{problemBackground}{RGB}{212,232,246}
\newenvironment{problem}[1]
  {
    \begin{tcolorbox}[
      boxrule=.5pt,
      titlerule=.5pt,
      sharp corners,
      colback=problemBackground
    ]
    \ifx &#1& \textbf{Problem. }
    \else \textbf{Problem #1.} \fi
  }
  {
    \end{tcolorbox}
  }

\definecolor{exampleBackground}{RGB}{255,249,248}
\definecolor{exampleAccent}{RGB}{158,60,14}
\newenvironment{example}[1]
  {
    \begin{tcolorbox}[
      boxrule=.5pt,
      sharp corners,
      colback=exampleBackground,
      colframe=exampleAccent,
    ]
    \color{exampleAccent}\textbf{Example.} \emph{#1}\color{black}
  }
  {
    \end{tcolorbox}
  }

\definecolor{theoremBackground}{RGB}{234,243,251}
\definecolor{theoremAccent}{RGB}{0,116,183}
\newenvironment{theorem}[1]
  {
    \begin{tcolorbox}[
      boxrule=.5pt,
      titlerule=.5pt,
      sharp corners,
      colback=theoremBackground,
      colframe=theoremAccent
    ]
      \color{theoremAccent}\textbf{Theorem --- }\emph{#1}\\\color{black}
  }
  {
    \end{tcolorbox}
  }

\definecolor{noteBackground}{RGB}{244,249,244}
\definecolor{noteAccent}{RGB}{34,139,34}
\newenvironment{note}[1]
  {
  \begin{tcolorbox}[
    enhanced,
    boxrule=0pt,
    frame hidden,
    sharp corners,
    colback=noteBackground,
    borderline west={3pt}{-1.5pt}{noteAccent}
    ]
    \ifx &#1& \color{noteAccent}\textbf{Note. }\color{black}
    \else \color{noteAccent}\textbf{Note (#1). }\color{black} \fi
    }
    {
  \end{tcolorbox}
  }

\definecolor{definitionBackground}{RGB}{246,246,246}
\newenvironment{definition}[1]
  {
    \begin{tcolorbox}[
      enhanced,
      boxrule=0pt,
      frame hidden,
      sharp corners,
      colback=definitionBackground,
      borderline west={3pt}{-1.5pt}{black}
    ]
    \textbf{Definition. }\emph{#1}\\
  }
  {
    \end{tcolorbox}
  }
\newenvironment{amatrix}[2]{
    \left[
      \begin{array}{*{#1}{c}|*{#2}c}
  }
  {
      \end{array}
    \right]
  }
\date{\the\year-\the\month-\the\day}
\author{Kyle Chui}


\fancyhf{}
\lhead{Kyle Chui}
\rhead{Page \thepage}
\pagestyle{fancy}

\begin{document}
  \section{Lecture 8}
  \subsection{The Nature of Life on Earth}
  \subsubsection{Defining Life}
  In defining life, we look to six key properties shared by most or all living organisms on Earth.
  \begin{enumerate}
    \item Order
    \item Reproduction
    \item Growth and Development
    \item Energy utilisation
    \item Response to the environment
    \item Evolutionary adaptation
  \end{enumerate}
  \subsubsection{Necessities of Life}
  \begin{itemize}
    \item Nutrient source
    \item Energy (sunlight, chemical reactions, internal heat)
    \item Liquid water (or possibly some other liquid)
    \begin{note}{}
      Finding liquid water is the hardest thing.
    \end{note}
  \end{itemize}
  \subsubsection{Order}
  \begin{definition}{Order}
    The idea that life tends to be non-random.
  \end{definition}
  \begin{itemize}
    \item Cells of different purpose and nature are spread throughout your body non-randomly, performing tasks in an ordered manner
    \item Order alone fails to categorise life, as order can be found in books, crystals, etc.
    \item Order is a \emph{necessary but not sufficient condition for life}
  \end{itemize}
  \subsubsection{Reproduction}
  \begin{itemize}
    \item All life reproduce or are the product of reproduction
    \begin{definition}{Reproduction}
      Cell division (almost exact copy) and sex (semi-random combination of genetic codes)
    \end{definition}
    \item Insufficient for life: a mule is sterile, but alive
    \item Viruses: incapable of reproduction on their own. They infect a host in order to reproduce
  \end{itemize}
  \subsubsection{Grown and Development}
  \begin{itemize}
    \item Similar to the idea of heredity, the passing of traits from parent to offspring, regardless of reproductive pathology
    \item Insufficient for life: Fire also ``grows and develops''
    \item Heredity touches upon DNA, which might bring viruses and prions back into our ``life'' column, but they use RNA
  \end{itemize}
  \subsubsection{Energy Utilisation}
  \begin{itemize}
    \item First two laws of thermodynamics:
    \begin{itemize}
      \item Any kind of energy can be turned into any other kind of energy.
      \item There is always a cost to the first law.
    \end{itemize}
    \item The second law also tells us about entropy; going from order to disorder
    \item If we place an organism in a sealed environment, it will use up all its energy and die
    \item Living organisms must have continual energy use to counter the effects of entropy
    \item Indefinite life without energy utilisation is impossible
    \item Insufficient for life: Any appliance at home uses energy to function
    \item Organisms obtain energy from the environment, which gets its energy from an internal or external source (i.e. heat of the planetary core or sunlight)
  \end{itemize}
  \subsubsection{Response to Environment}
  \begin{itemize}
    \item All life interacts with its surroundings
    \item Insufficient for life: A thermostat changes with its surroundings but is not alive
  \end{itemize}
  \subsection{The Chemistry of Life}
  \begin{note}{}
    Life prefers carbon over silicon because the bonds in silicon are single bonds and weaker. Carbon-based molecules are also more versatile---they are gaseous in comparison to silicon, which is solid.
  \end{note}
  \subsubsection{Proteins}
  Proteins are made from chains of amino acid monomers connected to form a polypeptide chain of monomers that fold.
  \begin{itemize}
    \item They can take the form of enzymes---catalysing or speeding up chemical reactions
    \item They can form the scaffolding and structure of tissue
    \item They can transport molecules to new locations
    \item Muscle contraction and cell motion
    \item Some hormones \emph{are} proteins
  \end{itemize}
  \begin{note}{}
    Life only uses $20$ amino acids to form all proteins.
  \end{note}
  \subsubsection{Nucleic Acids}
  \begin{itemize}
    \item DNA---Deoxyribonucleic acid, contains the genetic code
    \item RNA---Ribonucleic acid, active in the synthesis/transport of proteins
  \end{itemize}
  \begin{note}{}
    The failure of copying things perfectly results in mutations and evolution.
  \end{note}
  \subsection{Mechanism of Evolution}
  Landmark work described by ``two undeniable facts and an inescapable conclusion''
  \begin{itemize}
    \item Overproduction and competition for survival
    \item Individual variation
  \end{itemize}
  Conclusion: Unequal reproductive success: ``survival of the fittest''.
  \begin{note}{}
    Humans have taken advantage of this by breeding many plants and animals into forms that very little resemble their wild ancestors.
  \end{note}
  \subsection{Cells---The Basic Units of Life}
  \begin{itemize}
    \item All living organisms are made of cells; the basic structure of life separated from one another via a membrane.
    \item The vast majority of organisms are single-celled.
    \item All cells share numerous similarities:
    \begin{itemize}
      \item All pass hereditary information via DNA.
      \item All have the same basic components for building cells.
      \item All life is made from roughly $20$ elements (96\% of which are oxygen, carbon, hydrogen, nitrogen).
    \end{itemize}
  \end{itemize}
\end{document}
