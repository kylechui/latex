\documentclass[class=article, crop=false]{standalone}
\usepackage[margin=1in]{geometry}

\usepackage{amsmath}
\usepackage{amssymb}
\usepackage{amsthm}
\usepackage{comment}
\usepackage{enumitem}
\usepackage{fancyhdr}
\usepackage{hyperref}
\usepackage{mathrsfs}
\usepackage{mathtools}
\usepackage{standalone}
\usepackage{tcolorbox}
\usepackage{tikz}
\usepackage{xcolor}

\usetikzlibrary{decorations.pathreplacing}
\tcbuselibrary{skins}

\DeclareMathOperator{\lcm}{lcm}
\DeclareMathOperator{\proj}{proj}
\DeclareMathOperator{\vspan}{span}
\DeclareMathOperator{\im}{im}
\DeclareMathOperator{\range}{range}
\DeclareMathOperator{\Diff}{Diff}
\DeclareMathOperator{\Int}{Int}
\DeclareMathOperator{\fcn}{fcn}
\DeclareMathOperator{\id}{id}

\newcommand{\N}{\ensuremath{\mathbb{N}}}
\newcommand{\Z}{\ensuremath{\mathbb{Z}}}
\newcommand{\Q}{\ensuremath{\mathbb{Q}}}
\newcommand{\R}{\ensuremath{\mathbb{R}}}
\newcommand{\C}{\ensuremath{\mathbb{C}}}
\newcommand{\F}{\ensuremath{\mathbb{F}}}
\newcommand{\lam}{\ensuremath{\lambda}}
\newcommand{\nab}{\ensuremath{\nabla}}
\newcommand{\eps}{\ensuremath{\varepsilon}}
\newcommand{\es}{\ensuremath{\varnothing}}

\newcommand{\abs}[1]{\ensuremath{\left\lvert #1 \right\rvert}}
\newcommand{\bigpar}[1]{\ensuremath{\left( #1 \right)}}
\newcommand{\norm}[1]{\ensuremath{\lVert #1\rVert}}
\newcommand{\set}[1]{\ensuremath{\left\{#1\right\}}}
\newcommand{\tuple}[1]{\ensuremath{\left\langle #1 \right\rangle}}
\newcommand{\floor}[1]{\ensuremath{\left\lfloor #1 \right\rfloor}}
\newcommand{\ceil}[1]{\ensuremath{\left\lceil #1 \right\rceil}}

\newcommand{\chapternum}{}
\newcommand{\ex}[1]{\noindent\textbf{Exercise \chapternum.{#1}.}}

\newcommand{\dx}[1]{\,\mathrm{d}#1}
\newcommand{\inv}{\ensuremath{^{-1}}}
\newcommand{\sm}{\setminus}
\newcommand{\sse}{\subseteq}
\newcommand{\ceq}{\coloneqq}

\definecolor{problemBackground}{RGB}{212,232,246}
\newenvironment{problem}[1]
  {
    \begin{tcolorbox}[
      boxrule=.5pt,
      titlerule=.5pt,
      sharp corners,
      colback=problemBackground
    ]
    \ifx &#1& \textbf{Problem. }
    \else \textbf{Problem #1.} \fi
  }
  {
    \end{tcolorbox}
  }

\definecolor{exampleBackground}{RGB}{255,249,248}
\definecolor{exampleAccent}{RGB}{158,60,14}
\newenvironment{example}[1]
  {
    \begin{tcolorbox}[
      boxrule=.5pt,
      sharp corners,
      colback=exampleBackground,
      colframe=exampleAccent,
    ]
    \color{exampleAccent}\textbf{Example.} \emph{#1}\color{black}
  }
  {
    \end{tcolorbox}
  }

\definecolor{theoremBackground}{RGB}{234,243,251}
\definecolor{theoremAccent}{RGB}{0,116,183}
\newenvironment{theorem}[1]
  {
    \begin{tcolorbox}[
      boxrule=.5pt,
      titlerule=.5pt,
      sharp corners,
      colback=theoremBackground,
      colframe=theoremAccent
    ]
      \color{theoremAccent}\textbf{Theorem --- }\emph{#1}\\\color{black}
  }
  {
    \end{tcolorbox}
  }

\definecolor{noteBackground}{RGB}{244,249,244}
\definecolor{noteAccent}{RGB}{34,139,34}
\newenvironment{note}[1]
  {
  \begin{tcolorbox}[
    enhanced,
    boxrule=0pt,
    frame hidden,
    sharp corners,
    colback=noteBackground,
    borderline west={3pt}{-1.5pt}{noteAccent}
    ]
    \ifx &#1& \color{noteAccent}\textbf{Note. }\color{black}
    \else \color{noteAccent}\textbf{Note (#1). }\color{black} \fi
    }
    {
  \end{tcolorbox}
  }

\definecolor{definitionBackground}{RGB}{246,246,246}
\newenvironment{definition}[1]
  {
    \begin{tcolorbox}[
      enhanced,
      boxrule=0pt,
      frame hidden,
      sharp corners,
      colback=definitionBackground,
      borderline west={3pt}{-1.5pt}{black}
    ]
    \textbf{Definition. }\emph{#1}\\
  }
  {
    \end{tcolorbox}
  }
\newenvironment{amatrix}[2]{
    \left[
      \begin{array}{*{#1}{c}|*{#2}c}
  }
  {
      \end{array}
    \right]
  }
\date{\the\year-\the\month-\the\day}
\author{Kyle Chui}


\fancyhf{}
\lhead{Kyle Chui}
\rhead{Page \thepage}
\pagestyle{fancy}

\begin{document}
  \section{Warring States Period}
  \subsection{Warring States and Azuchi-Momoyama Key Dates}
  \begin{itemize}
    \item $1467$--$1477$ \=Onin war
    \item $1542$ Portuguese reach Tanegashima
    \item $1568$ Nobunaga installs last Ashikaga shogun in Ky\=oto (which begins the Azuchi-Momoyama Period)
    \item $1568$ Nobunaga dies
    \item $1573$ Last Ashikaga shogun driven out of Ky\=oto
    \item $1582$ Invasion of Choson (second invasion in 1597)---also known as the Imjin war
    \item $1598$ Hideyoshi dies
    \item $1600$ Battle of Sekigahara
    \item $1603$ Establishments of Tokugawa shogunate
  \end{itemize}
  \subsection{\=Onin War}
  \begin{itemize}
    \item Begins as a conflict over Yoshimasa succession between his son (supported by Yamana clan) and brother (supported by Hosokawa clan)
    \item Shogunate loses all authority
    \item Completely destroys Ky\=oto
    \item Yamana clan leaves the capital in 1477
    \item No clear victor, no one left in charge to guarantee peace between warrior clans
    \begin{note}{}
      The battle started in Ky\=oto, which was supposed to be the one place where there was peace. Once it was destroyed, there was no more centralised authority to keep the peace.
    \end{note}
    \item War sanctions conflicts between warrior domains
    \item No more ``Retired emperors'' (Imperial court impoverished and loses everything)
  \end{itemize}
  \subsection{Socio-economic Developments}
  \begin{itemize}
    \item No effective central authority---many alliances rise up between the numerous warrior clans.
    \item Sh\=oen system has disappeared (in practice but not in name).
    \begin{note}{}
      There are still locations that are called by their estate name, but the estate doesn't really exist any more.
    \end{note}
    \item Increased urbanisation: emergence of villages and towns.
    \item Age of the daimy\=o (warlords).
    \begin{note}{}
      There is economic decentralisation as well---each warlord is receiving taxes from those who live on their land, and doesn't pay taxes to the imperial government.
    \end{note}
    \item Gekokujo: Wealthy peasants become landowners and take samurai status.
    \item Peasant uprisings and village communes.
    \begin{note}{}
      Women no longer own land.
    \end{note}
  \end{itemize}
  \subsection{Socio-military Developments}
  \begin{itemize}
    \item The daimy\=o are completely independent of the government, they issue their own laws for their lands, and form regional alliances.
    \item There are also militarised Buddhist sects:
    \begin{itemize}
      \item Ikk\=o sect (extreme Pure Land sect)
      \item Lotus leagues (followers of Nichiren)
      \item Enyakuji (Tendai sect)
    \end{itemize}
    \item Introduction of firearms in the mid 1500s.
    \item Use of foot soldiers.
    \item Development of castle building---linked to the development of towns.
  \end{itemize}
  \subsection{Buddhist Lineages in the Warring States}
  \begin{itemize}
    \item Most Heian temples are destroyed during the \=Onin war.
    \item Buddhist sects all had armies and fought against domain lords and each other---Lotus v. True Pure Land and Tendai v. Lotus Leagues
  \end{itemize}
  \begin{note}{}
    During the battles, the warriors aren't just fighting for more land---many have the vision of a unified Japan.
  \end{note}
  \subsection{Azuchi-Momoyama Period (1568--1600)}
  \begin{itemize}
    \item Oda Nobunaga
    \item Toyotomi Hideyoshi
    \item Tokugawa Ieyasu
  \end{itemize}
  \subsubsection{Nobunaga}
  \begin{itemize}
    \item $1560$ Wins crucial battle against Imagawa clan (outnumbered $8$ to $1$)
    \item $1561$ Forms alliance with Matsudaira (later Tokugawa)
    \item $1567$ Starts using personal seal: ``Take all under heaven by force''
    \item $1568$ Installs last Ashikaga shogun in Ky\=oto
    \item $1571$ Destroys Enryakuji (Tendai sect)
    \item $1573$ Takeda Shingen dies; Nobunaga drives last Ashikaga shogun out of Ky\=oto
    \item $1575$ Battle of Nagashino (Oda and Tokugawa defeat Takeda)
    \item $1576$ Battle of Tedorigawa (Uesugi defeat Oda)
    \item $1578$ Uesugi Kenshin dies
    \item $1582$ Nobunaga dies at Honn\=oji
  \end{itemize}
  \subsubsection{Hideyoshi (Nobunaga's Successor)}
  Hideyoshi was born a peasant, and rose through the samurai ranks.
  \begin{itemize}
    \item $1585$ Hideyoshi appointed Regent
    \item $1585$--$1587$ Conquers Etch\=u province and Ky\=ush\=u (Shimazu clan)
    \item $1587$ Hideyoshi's edict expelling Christian missionaries
    \item $1588$ Hideyoshi edict prohibiting farmers from bearing arms
    \item $1590$ Conquers H\=oj\=o in Eastern Japan---offers territory to Ieyasu
    \item $1591$ Hideyoshi edicts on census and on restrictions on change of social status
    \item $1592$ Hideyoshi sends army to invade Korea
    \item $1595$ Edict controlling domain lords
    \item $1597$ Second invasion of Korea
    \item $1598$ Hideyoshi dies
  \end{itemize}
  \subsubsection{Tokugawa Ieyasu}
  Son of head of Matsudaira clan (minor warrior clan)
  \begin{itemize}
    \item $1561$ Ieyasu allies with Nobunaga
    \item $1571--1584$ Wars with Takeda clan
    \item $1584$ Ieyasu supports Nobunaga's son Nobuo against Hideyoshi, which ends in a truce
    \item $1590$ Ieyasu accepts Hideyoshi's offer of land exhange
    \item $1598$ Hideyoshi dies
    \item $1600$ Battle of Sekigahara
    \item $1603$ Tokugawa Ieyasu takes title of shogun
  \end{itemize}
\end{document}
