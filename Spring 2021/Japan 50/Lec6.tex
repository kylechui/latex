\documentclass[class=article, crop=false]{standalone}
\usepackage[margin=1in]{geometry}

\usepackage{amsmath}
\usepackage{amssymb}
\usepackage{amsthm}
\usepackage{comment}
\usepackage{enumitem}
\usepackage{fancyhdr}
\usepackage{hyperref}
\usepackage{mathrsfs}
\usepackage{mathtools}
\usepackage{standalone}
\usepackage{tcolorbox}
\usepackage{tikz}
\usepackage{xcolor}

\usetikzlibrary{decorations.pathreplacing}
\tcbuselibrary{skins}

\DeclareMathOperator{\lcm}{lcm}
\DeclareMathOperator{\proj}{proj}
\DeclareMathOperator{\vspan}{span}
\DeclareMathOperator{\im}{im}
\DeclareMathOperator{\range}{range}
\DeclareMathOperator{\Diff}{Diff}
\DeclareMathOperator{\Int}{Int}
\DeclareMathOperator{\fcn}{fcn}
\DeclareMathOperator{\id}{id}

\newcommand{\N}{\ensuremath{\mathbb{N}}}
\newcommand{\Z}{\ensuremath{\mathbb{Z}}}
\newcommand{\Q}{\ensuremath{\mathbb{Q}}}
\newcommand{\R}{\ensuremath{\mathbb{R}}}
\newcommand{\C}{\ensuremath{\mathbb{C}}}
\newcommand{\F}{\ensuremath{\mathbb{F}}}
\newcommand{\lam}{\ensuremath{\lambda}}
\newcommand{\nab}{\ensuremath{\nabla}}
\newcommand{\eps}{\ensuremath{\varepsilon}}
\newcommand{\es}{\ensuremath{\varnothing}}

\newcommand{\abs}[1]{\ensuremath{\left\lvert #1 \right\rvert}}
\newcommand{\bigpar}[1]{\ensuremath{\left( #1 \right)}}
\newcommand{\norm}[1]{\ensuremath{\lVert #1\rVert}}
\newcommand{\set}[1]{\ensuremath{\left\{#1\right\}}}
\newcommand{\tuple}[1]{\ensuremath{\left\langle #1 \right\rangle}}
\newcommand{\floor}[1]{\ensuremath{\left\lfloor #1 \right\rfloor}}
\newcommand{\ceil}[1]{\ensuremath{\left\lceil #1 \right\rceil}}

\newcommand{\chapternum}{}
\newcommand{\ex}[1]{\noindent\textbf{Exercise \chapternum.{#1}.}}

\newcommand{\dx}[1]{\,\mathrm{d}#1}
\newcommand{\inv}{\ensuremath{^{-1}}}
\newcommand{\sm}{\setminus}
\newcommand{\sse}{\subseteq}
\newcommand{\ceq}{\coloneqq}

\definecolor{problemBackground}{RGB}{212,232,246}
\newenvironment{problem}[1]
  {
    \begin{tcolorbox}[
      boxrule=.5pt,
      titlerule=.5pt,
      sharp corners,
      colback=problemBackground
    ]
    \ifx &#1& \textbf{Problem. }
    \else \textbf{Problem #1.} \fi
  }
  {
    \end{tcolorbox}
  }

\definecolor{exampleBackground}{RGB}{255,249,248}
\definecolor{exampleAccent}{RGB}{158,60,14}
\newenvironment{example}[1]
  {
    \begin{tcolorbox}[
      boxrule=.5pt,
      sharp corners,
      colback=exampleBackground,
      colframe=exampleAccent,
    ]
    \color{exampleAccent}\textbf{Example.} \emph{#1}\color{black}
  }
  {
    \end{tcolorbox}
  }

\definecolor{theoremBackground}{RGB}{234,243,251}
\definecolor{theoremAccent}{RGB}{0,116,183}
\newenvironment{theorem}[1]
  {
    \begin{tcolorbox}[
      boxrule=.5pt,
      titlerule=.5pt,
      sharp corners,
      colback=theoremBackground,
      colframe=theoremAccent
    ]
      \color{theoremAccent}\textbf{Theorem --- }\emph{#1}\\\color{black}
  }
  {
    \end{tcolorbox}
  }

\definecolor{noteBackground}{RGB}{244,249,244}
\definecolor{noteAccent}{RGB}{34,139,34}
\newenvironment{note}[1]
  {
  \begin{tcolorbox}[
    enhanced,
    boxrule=0pt,
    frame hidden,
    sharp corners,
    colback=noteBackground,
    borderline west={3pt}{-1.5pt}{noteAccent}
    ]
    \ifx &#1& \color{noteAccent}\textbf{Note. }\color{black}
    \else \color{noteAccent}\textbf{Note (#1). }\color{black} \fi
    }
    {
  \end{tcolorbox}
  }

\definecolor{definitionBackground}{RGB}{246,246,246}
\newenvironment{definition}[1]
  {
    \begin{tcolorbox}[
      enhanced,
      boxrule=0pt,
      frame hidden,
      sharp corners,
      colback=definitionBackground,
      borderline west={3pt}{-1.5pt}{black}
    ]
    \textbf{Definition. }\emph{#1}\\
  }
  {
    \end{tcolorbox}
  }
\newenvironment{amatrix}[2]{
    \left[
      \begin{array}{*{#1}{c}|*{#2}c}
  }
  {
      \end{array}
    \right]
  }
\date{\the\year-\the\month-\the\day}
\author{Kyle Chui}


\fancyhf{}
\lhead{Kyle Chui}
\rhead{Page \thepage}
\pagestyle{fancy}

\begin{document}
  \section{Muromachi Period (1333-1477)}
  \subsection{Historical Overview}
  \begin{itemize}
    \item Kenmu Restoration (1333--1336), where an emperor tries to restore imperial rule
    \item Northern and Southern Courts (1336--1392)
    \item Ashikaga Shogunate (1336--1573)
    \item Onin War (1467--1477): fall of Ashikaga Shogunate
  \end{itemize}
  \subsection{Key Points of the Muromachi Period}
  \begin{itemize}
    \item Political Decentralisation
    \begin{itemize}
      \item Weakened Kamakura shogunate
      \item Go-Daigo is never in a position of real power
      \item Ashikaga shogunate also has limited power
      \item Emergence of Daimyo
    \end{itemize}
    \begin{note}{}
      During this time period, we have two governments---main capital and warrior capital. The Kamakura shogunate falls because they are no longer able to maintain order. Instead of power centralisation in these two capitals, the emergence of many smaller warrior lineages fractures power.
    \end{note}
    \item Economic Development
    \begin{itemize}
      \item Improvements in farming technology, increased commerce
      \item Land rights of absentee landlord reduced
      \item Military governors obtain taxation rights
      \item Land stewards consolidate land rights
      \item Vibrant trade with Ming China, Choson Korea, within Japan
    \end{itemize}
    \begin{note}{}
      The Sh\=oen system gradually disappears.
    \end{note}
    \item Cultural and Social Change
    \begin{itemize}
      \item Beginning of ``Shint\=o''
      \item Zen institutions and culture
      \item New forms of aristocratic and warrior culture
      \item Time of great social upheaval
    \end{itemize}
  \end{itemize}
  \subsection{Medieval Shint\=o}
  \begin{note}{}
    Shint\=o develops in tandem with Buddhism.
  \end{note}
  \subsubsection{Dual Shint\=o}
  \begin{itemize}
    \item Buddhas manifest locally as Kami (Shint\=o gods)
    \item Great Sun Buddha = Sun Goddess
    \item Great Sun-origin (Japan) Land = Great-Sun [Buddha] Original Land
  \end{itemize}
  \subsection{Ashikaga Shogunate (1336--1447) [1573]}
  \begin{itemize}
    \item Takauji (r. 1338--1358) Establishes Northern court in Kyoto, where he also sets up the shogunate.
    \item Yoshimitsu (r. 1368--1408) Unifies northern and southern courts in $1392$, which is at the height of Ashikaga influence.
    \item Ashikaga Yoshimasa (1449--1490) Collapse of regime.
  \end{itemize}
  \subsection{Economic Growth and Social Upheaval}
  \begin{itemize}
    \item Technological improvements in farming, mining, and various crafts
    \item Robust trade with Korea and China
    \item Emergence of merchant trade guilds in towns and cities, and the beginning of a cash economy (used Chinese currency)
    \item Bandits and ``Evil Parties'': marginal people and local leaders
    \item Minor warrior groups organise into egalitarian federations (ikki)
    \item Peasant leaders acquire rights to rent from land---sometimes revolt
  \end{itemize}
  The capital changes to have northern and southern hemispheres for aristocratic and commercial sectors.
  \subsection{Zen (Chan)}
  \begin{itemize}
    \item Emphasis on meditation---rejection of prayer, study, and complex rituals.
    \item Emphasis on teacher-disciple relation.
    \item In Song China, Zen had become the dominant form of Buddhism. They employed the Gozan (five mountain) system to control Zen temples (five in Kyoto, five in Kamakura).
    \item Two main schools in Japan: Rinzai zen and S\=ot\=o zen.
  \end{itemize}
  Rinzai Zen accommodated Zen to Tendai, Shingon, and Pure Land practices. It emphasised sever monastic discipline (which appealed to warriors). On the other hand, S\=ot\=o Zen was more about silent meditation, emphasising teacher-disciple relations. \par
  The Zen monks become an elite class---diplomats, advisors, poets, painters, etc.
  \subsection{Ashikaga Yoshimitsu}
  \begin{itemize}
    \item Third Ashikaga Shogun
    \item Gold pavilion
    \item Established tributary relationship with Ming Dynasty in 1397
    \item Patron of the arts---collector of paintings, enjoyed poetry and theatre
  \end{itemize}
  \subsection{Ashikaga Yoshimasa}
  \begin{itemize}
    \item Eighth Ashikaga shogun
    \item Silver pavilion
    \item Conflict to succeed him destroyed the shogunate
    \item Patron of the Tea ceremony (Sad\=o)
    \item Also enjoyed theatre and ink painting
  \end{itemize}
  \subsection{Late Medieval Literary Culture}
  \begin{itemize}
    \item Linked verse (aristocrats)
    \item N\=o theatre (warriors)
    \item Ky\=ogen Theatre (warriors and commoners)
    \item Popular Tales (Otogi zoshi) (from aristocrats to commoners)
  \end{itemize}
  \subsubsection{N\=o Theatre}
  \begin{itemize}
    \item Origins obscure (actors are from the outcast class)
    \item Elements
    \begin{itemize}
      \item Main actor (shite)
      \item Secondary actor (waki)
      \item Chorus
      \item Musicians
    \end{itemize}
    \item Zeami Motokiyo was patronised by Yoshimitsu
    \item There were plays about Gods, warriors, women, demons, etc.
  \end{itemize}
\end{document}
