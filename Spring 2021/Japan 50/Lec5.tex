\documentclass[class=article, crop=false]{standalone}
\usepackage[margin=1in]{geometry}

\usepackage{amsmath}
\usepackage{amssymb}
\usepackage{amsthm}
\usepackage{comment}
\usepackage{enumitem}
\usepackage{fancyhdr}
\usepackage{hyperref}
\usepackage{mathrsfs}
\usepackage{mathtools}
\usepackage{standalone}
\usepackage{tcolorbox}
\usepackage{tikz}
\usepackage{xcolor}

\usetikzlibrary{decorations.pathreplacing}
\tcbuselibrary{skins}

\DeclareMathOperator{\lcm}{lcm}
\DeclareMathOperator{\proj}{proj}
\DeclareMathOperator{\vspan}{span}
\DeclareMathOperator{\im}{im}
\DeclareMathOperator{\range}{range}
\DeclareMathOperator{\Diff}{Diff}
\DeclareMathOperator{\Int}{Int}
\DeclareMathOperator{\fcn}{fcn}
\DeclareMathOperator{\id}{id}

\newcommand{\N}{\ensuremath{\mathbb{N}}}
\newcommand{\Z}{\ensuremath{\mathbb{Z}}}
\newcommand{\Q}{\ensuremath{\mathbb{Q}}}
\newcommand{\R}{\ensuremath{\mathbb{R}}}
\newcommand{\C}{\ensuremath{\mathbb{C}}}
\newcommand{\F}{\ensuremath{\mathbb{F}}}
\newcommand{\lam}{\ensuremath{\lambda}}
\newcommand{\nab}{\ensuremath{\nabla}}
\newcommand{\eps}{\ensuremath{\varepsilon}}
\newcommand{\es}{\ensuremath{\varnothing}}

\newcommand{\abs}[1]{\ensuremath{\left\lvert #1 \right\rvert}}
\newcommand{\bigpar}[1]{\ensuremath{\left( #1 \right)}}
\newcommand{\norm}[1]{\ensuremath{\lVert #1\rVert}}
\newcommand{\set}[1]{\ensuremath{\left\{#1\right\}}}
\newcommand{\tuple}[1]{\ensuremath{\left\langle #1 \right\rangle}}
\newcommand{\floor}[1]{\ensuremath{\left\lfloor #1 \right\rfloor}}
\newcommand{\ceil}[1]{\ensuremath{\left\lceil #1 \right\rceil}}

\newcommand{\chapternum}{}
\newcommand{\ex}[1]{\noindent\textbf{Exercise \chapternum.{#1}.}}

\newcommand{\dx}[1]{\,\mathrm{d}#1}
\newcommand{\inv}{\ensuremath{^{-1}}}
\newcommand{\sm}{\setminus}
\newcommand{\sse}{\subseteq}
\newcommand{\ceq}{\coloneqq}

\definecolor{problemBackground}{RGB}{212,232,246}
\newenvironment{problem}[1]
  {
    \begin{tcolorbox}[
      boxrule=.5pt,
      titlerule=.5pt,
      sharp corners,
      colback=problemBackground
    ]
    \ifx &#1& \textbf{Problem. }
    \else \textbf{Problem #1.} \fi
  }
  {
    \end{tcolorbox}
  }

\definecolor{exampleBackground}{RGB}{255,249,248}
\definecolor{exampleAccent}{RGB}{158,60,14}
\newenvironment{example}[1]
  {
    \begin{tcolorbox}[
      boxrule=.5pt,
      sharp corners,
      colback=exampleBackground,
      colframe=exampleAccent,
    ]
    \color{exampleAccent}\textbf{Example.} \emph{#1}\color{black}
  }
  {
    \end{tcolorbox}
  }

\definecolor{theoremBackground}{RGB}{234,243,251}
\definecolor{theoremAccent}{RGB}{0,116,183}
\newenvironment{theorem}[1]
  {
    \begin{tcolorbox}[
      boxrule=.5pt,
      titlerule=.5pt,
      sharp corners,
      colback=theoremBackground,
      colframe=theoremAccent
    ]
      \color{theoremAccent}\textbf{Theorem --- }\emph{#1}\\\color{black}
  }
  {
    \end{tcolorbox}
  }

\definecolor{noteBackground}{RGB}{244,249,244}
\definecolor{noteAccent}{RGB}{34,139,34}
\newenvironment{note}[1]
  {
  \begin{tcolorbox}[
    enhanced,
    boxrule=0pt,
    frame hidden,
    sharp corners,
    colback=noteBackground,
    borderline west={3pt}{-1.5pt}{noteAccent}
    ]
    \ifx &#1& \color{noteAccent}\textbf{Note. }\color{black}
    \else \color{noteAccent}\textbf{Note (#1). }\color{black} \fi
    }
    {
  \end{tcolorbox}
  }

\definecolor{definitionBackground}{RGB}{246,246,246}
\newenvironment{definition}[1]
  {
    \begin{tcolorbox}[
      enhanced,
      boxrule=0pt,
      frame hidden,
      sharp corners,
      colback=definitionBackground,
      borderline west={3pt}{-1.5pt}{black}
    ]
    \textbf{Definition. }\emph{#1}\\
  }
  {
    \end{tcolorbox}
  }
\newenvironment{amatrix}[2]{
    \left[
      \begin{array}{*{#1}{c}|*{#2}c}
  }
  {
      \end{array}
    \right]
  }
\date{\the\year-\the\month-\the\day}
\author{Kyle Chui}


\fancyhf{}
\lhead{Kyle Chui}
\rhead{Page \thepage}
\pagestyle{fancy}

\begin{document}
  \subsection{The Era of Cloistered Emperors (1086-1160)}
  \begin{itemize}
    \item The Emperor Go-Sanj\=o accedes to throne in $1068$, and is the first emperor for $170$ years whose mother was not the daughter of a Fujiwara master.
    \item There are attempts by the imperial family to control the sh\=oen system. Retired emperors become private estate owners, and the Imperial household becomes the largest landholder.
    \item Retired emperors become a real power at court.
    \item There is competition between various factions (emperors and retired emperors).
    \item 1156: Warrior clans involved in a court dispute.
    \item 1160-1185: Head of warrior clan Taira Kiyomori becomes main power at court.
  \end{itemize}
  \section{Kamakura Period (1185-1333)}
  The main events leading to the Kamakura Period are:
  \begin{itemize}
    \item 1156: H\=ogen War
    \item 1159: Heiji War
    \item 1160-1185: Taira effectively control the court
    \item 1180-1185: Genpei wars between two warrior clans: the Minamoto and the Taira
  \end{itemize}
  ``Kamakura'' is the city in Eastern Japan where Minamoto no Yoritomo's military government was located.
  \subsection{Origin of the Warrior Class}
  In the Heian period, provincial governors are appointed by the imperial court. Later on, local lords begin grouping together to control their peasants and protect their land. Some of these groups later on rebel against the court, which allies with other warrior lineages to quash rebellions. Finally, we see court political factions ally with different warrior lineages to fight each other.
  \subsection{H\=oj\=oki}
  \begin{itemize}
    \item It is about Kamo no Ch\=omei, who is from a family of Shint\=o priests but later became a Tendai monk.
    \item Description of disasters in the capital in the late 12\tsup{th} century.
    \item Exemplar of a genre known as ``recluse literature''.
    \item Best known opening lines of any work of pre modern Japanese literature.
  \end{itemize}
  \begin{note}{}
    The battles between different warrior lineages span the entirety of Japan, and everyone is affected.
  \end{note}
  \subsection{Events of the Kamakura Period}
  \begin{itemize}
    \item 1185: Minamoto no Yoritomo's victory in Genpei War
    \item 1199: Yoritomo dies
    \item 1199-1333: H\=oj\=o regents rule Kamakura, which name Fujiwara as shoguns (instead of Minamoto)
    \item 1221: Joky\=u disturbance: Emperor Go-Toba rebellion (further decline of imperial court)
    \item 1232: J\=oei code: first codification of warrior law
    \item 1274: First Mongol Invasion ($30000$ destroyed by typhoon)
    \item 1281: Second Mongol Invasion ($140000$ held back for seven weeks until destroyed by typhoon again)
  \end{itemize}
  \begin{note}{}
    The court still has the majority of the power, but is no longer omnipotent. Think 60-40 power split with the warrior clans.
  \end{note}
  \subsection{Kamakura Key Points}
  \begin{itemize}
    \item Loss of power of the Retired Emperors
    \item System of dual government
    \item Two courts (Ky\=oto and Kamakura), and the shogunate issues legal codes
    \item Kamakura control of provinces
    \item Mongol invasions allowed Kamakura shogunate to extend influence into Western Japan
    \begin{note}{}
      The invasions also weakened the shogunate, as it could not reward those who defended Japan with land (which was usually the case for expansion/conquest).
    \end{note}
  \end{itemize}
  \subsection{Sh\=oen System}
  In the Kamakura period, the shogunate appointed
  \begin{itemize}
    \item A military governor (shug\=o) to protect the land/keep peace
    \item A steward (jit\=o) to administrate the land, collect taxes and rent
  \end{itemize}
  Both systems coexist, until the imperial system gradually disappears.
  \subsection{Religion in the Kamakura Period}
  \begin{itemize}
    \item Tendai and Shingon Buddhism are still powerful
    \item New sects devlop from Tendai (Amidism and Nichiren)
    \item Zen (attractive to elite warrior class)
    \item Deity worship (Shinto): Develops alongside Buddhism
  \end{itemize}
  \subsubsection{Pure Land Buddhism (Amidism)}
  \begin{itemize}
    \item Belief in Mapp\=o, or ``the last age''
    \item Worship of Amida's Pure Land, belief that you can save yourself through the recitation of Nenbutsu (Amida's name)
    \item Genshin (942--1017) was a Tendai Priest, who wrote the text \emph{Essentials of Salvation}
  \end{itemize}
  \subsubsection{Six Realms of Existence in Buddhism}
  \begin{enumerate}
    \item Gods/Heaven (pleasure)
    \item Humans (desire)
    \item Asura (anger, jealousy, war)
    \item Beasts (stupidity/servitude)
    \item Hungry Ghosts (starvation)
    \item Hell (torture)
  \end{enumerate}
  \begin{note}{}
    At this point in time, the Japanese people think of the world as centred around ``Three Lands'' (India, China, Japan).
  \end{note}
  \subsection{Warrior Class Writes History}
  They begin to write stories about all of the conflicts that have happened (H\=ogen, Heiji, Genpei). The tales of the Heike are usually told accompanied by a lute (biwa), often by blind priest-performers.
\end{document}
