\documentclass[class=article, crop=false]{standalone}
\usepackage[margin=1in]{geometry}

\usepackage{amsmath}
\usepackage{amssymb}
\usepackage{amsthm}
\usepackage{comment}
\usepackage{enumitem}
\usepackage{fancyhdr}
\usepackage{hyperref}
\usepackage{mathrsfs}
\usepackage{mathtools}
\usepackage{standalone}
\usepackage{tcolorbox}
\usepackage{tikz}
\usepackage{xcolor}

\usetikzlibrary{decorations.pathreplacing}
\tcbuselibrary{skins}

\DeclareMathOperator{\lcm}{lcm}
\DeclareMathOperator{\proj}{proj}
\DeclareMathOperator{\vspan}{span}
\DeclareMathOperator{\im}{im}
\DeclareMathOperator{\range}{range}
\DeclareMathOperator{\Diff}{Diff}
\DeclareMathOperator{\Int}{Int}
\DeclareMathOperator{\fcn}{fcn}
\DeclareMathOperator{\id}{id}

\newcommand{\N}{\ensuremath{\mathbb{N}}}
\newcommand{\Z}{\ensuremath{\mathbb{Z}}}
\newcommand{\Q}{\ensuremath{\mathbb{Q}}}
\newcommand{\R}{\ensuremath{\mathbb{R}}}
\newcommand{\C}{\ensuremath{\mathbb{C}}}
\newcommand{\F}{\ensuremath{\mathbb{F}}}
\newcommand{\lam}{\ensuremath{\lambda}}
\newcommand{\nab}{\ensuremath{\nabla}}
\newcommand{\eps}{\ensuremath{\varepsilon}}
\newcommand{\es}{\ensuremath{\varnothing}}

\newcommand{\abs}[1]{\ensuremath{\left\lvert #1 \right\rvert}}
\newcommand{\bigpar}[1]{\ensuremath{\left( #1 \right)}}
\newcommand{\norm}[1]{\ensuremath{\lVert #1\rVert}}
\newcommand{\set}[1]{\ensuremath{\left\{#1\right\}}}
\newcommand{\tuple}[1]{\ensuremath{\left\langle #1 \right\rangle}}
\newcommand{\floor}[1]{\ensuremath{\left\lfloor #1 \right\rfloor}}
\newcommand{\ceil}[1]{\ensuremath{\left\lceil #1 \right\rceil}}

\newcommand{\chapternum}{}
\newcommand{\ex}[1]{\noindent\textbf{Exercise \chapternum.{#1}.}}

\newcommand{\dx}[1]{\,\mathrm{d}#1}
\newcommand{\inv}{\ensuremath{^{-1}}}
\newcommand{\sm}{\setminus}
\newcommand{\sse}{\subseteq}
\newcommand{\ceq}{\coloneqq}

\definecolor{problemBackground}{RGB}{212,232,246}
\newenvironment{problem}[1]
  {
    \begin{tcolorbox}[
      boxrule=.5pt,
      titlerule=.5pt,
      sharp corners,
      colback=problemBackground
    ]
    \ifx &#1& \textbf{Problem. }
    \else \textbf{Problem #1.} \fi
  }
  {
    \end{tcolorbox}
  }

\definecolor{exampleBackground}{RGB}{255,249,248}
\definecolor{exampleAccent}{RGB}{158,60,14}
\newenvironment{example}[1]
  {
    \begin{tcolorbox}[
      boxrule=.5pt,
      sharp corners,
      colback=exampleBackground,
      colframe=exampleAccent,
    ]
    \color{exampleAccent}\textbf{Example.} \emph{#1}\color{black}
  }
  {
    \end{tcolorbox}
  }

\definecolor{theoremBackground}{RGB}{234,243,251}
\definecolor{theoremAccent}{RGB}{0,116,183}
\newenvironment{theorem}[1]
  {
    \begin{tcolorbox}[
      boxrule=.5pt,
      titlerule=.5pt,
      sharp corners,
      colback=theoremBackground,
      colframe=theoremAccent
    ]
      \color{theoremAccent}\textbf{Theorem --- }\emph{#1}\\\color{black}
  }
  {
    \end{tcolorbox}
  }

\definecolor{noteBackground}{RGB}{244,249,244}
\definecolor{noteAccent}{RGB}{34,139,34}
\newenvironment{note}[1]
  {
  \begin{tcolorbox}[
    enhanced,
    boxrule=0pt,
    frame hidden,
    sharp corners,
    colback=noteBackground,
    borderline west={3pt}{-1.5pt}{noteAccent}
    ]
    \ifx &#1& \color{noteAccent}\textbf{Note. }\color{black}
    \else \color{noteAccent}\textbf{Note (#1). }\color{black} \fi
    }
    {
  \end{tcolorbox}
  }

\definecolor{definitionBackground}{RGB}{246,246,246}
\newenvironment{definition}[1]
  {
    \begin{tcolorbox}[
      enhanced,
      boxrule=0pt,
      frame hidden,
      sharp corners,
      colback=definitionBackground,
      borderline west={3pt}{-1.5pt}{black}
    ]
    \textbf{Definition. }\emph{#1}\\
  }
  {
    \end{tcolorbox}
  }
\newenvironment{amatrix}[2]{
    \left[
      \begin{array}{*{#1}{c}|*{#2}c}
  }
  {
      \end{array}
    \right]
  }
\date{\the\year-\the\month-\the\day}
\author{Kyle Chui}


\fancyhf{}
\lhead{Kyle Chui}
\rhead{Page \thepage}
\pagestyle{fancy}

\begin{document}
  \section{Machine-Level Programming I: Basics}
  \subsection{History of Intel processors and architectures}
  \begin{itemize}
    \item Dominate laptop/desktop/server market
    \item Evolutionary design
    \begin{itemize}
      \item Backwards compatible (up until $8086$, introduced in 1978).
      \item More features were added as time went on.
    \end{itemize}
    \item Complex instruction set computer (CISC)
    \begin{itemize}
      \item Many different instructions with many different formats.
      \item Hard to match the performance of Reduced Instruction Set Computers (RISC).
      \item Intel has performance, but not great at power efficiency.
    \end{itemize}
  \end{itemize}
  \subsection{C, assembly, machine code}
  \begin{definition}{Instruction Set Architecture (ISA)}
    An \emph{Instruction Set Architecture} is the parts of a processor design that one needs to understand or write assembly/machine code, i.e. instruction set specification, registers.
  \end{definition}
  \begin{definition}{Microarchitecture}
    The \emph{microarchitecture} is the implementation of the architecture, i.e. cache sizes and core frequency.
  \end{definition}
  \begin{definition}{Code Forms}
    There are two code forms---\emph{machine code}, which are the byte-level programs that a processor executes, and \emph{assembly code}, which is a text representation of machine code.
  \end{definition}
  Some examples of ISAs include x86, IA32, x86-64, and ARM.
  \subsubsection{Assembly/Machine Code View}
  CPU:
  \begin{itemize}
    \item PC: Program Counter
    \begin{itemize}
      \item Keeps track of which instruction will be run next, and is called the ``RIP'' for x86-64 architectures.
    \end{itemize}
    \item The Register file is used for heavily used program data (because it is very fast).
    \item Condition codes are special registers that store status information about the most recent arithmetic or logical operation, and are used for conditional branching.
  \end{itemize}
  Memory:
  \begin{itemize}
    \item We think of memory as a large array where each element is a byte.
    \item It holds code and user data, and uses a stack to support procedure calls.
  \end{itemize}
  \begin{center}\resizebox{9cm}{!}{\import{./Figures}{AssemblyMachineCodeView.pdf_tex}}\end{center}
  \subsubsection{Turning C into Object Code}
  We first code in human-readable files, such as \texttt{.c} files, and then compile them into an executable program (which is a binary and not human-readable).
  \begin{enumerate}
    \item We start with a C program, which is written in text.
    \item The code gets compiled into an Asm program, which is still in text.
    \item The Asm program goes through an assembler, which turns it into an object program (a binary).
    \item The object program finally goes through a linker, which turns it into an executable program (also a binary).
  \end{enumerate}
  \subsubsection{Assembly Characteristics: Data Types}
  \begin{itemize}
    \item ``Integer'' data of $1$, $2$, $4$, or $8$ bytes (includes data values and addresses).
    \item Floating point data of $4$, $8$, or $10$ bytes.
    \item Code: Byte sequences encoding series of instructions.
    \item No aggregate types such as arrays or structures (just contiguously allocated bytes in memory).
  \end{itemize}
  \subsubsection{Assembly Characteristics: Operations}
  \begin{itemize}
    \item Perform arithmetic functions on register or memory data.
    \item Transfer data between memory and register.
    \item Transfer control (changing the instruction pointer of the program counter).
  \end{itemize}
  \subsubsection{Object Code}
  Assembler:
  \begin{itemize}
    \item Translates \texttt{.s} into \texttt{.o}.
    \item Binary encoding of each instruction.
    \item Nearly-complete image of executable code.
    \item Missing linkages between code in different files.
  \end{itemize}
  Linker:
  \begin{itemize}
    \item Resolves references between files.
    \item Combines with static run-time libraries (e.g. code for \texttt{malloc}, \texttt{printf}).
    \item Some libraries are \emph{dynamically linked} (but linking occurs when program begins execution).
  \end{itemize}
\end{document}
