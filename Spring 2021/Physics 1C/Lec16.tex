\documentclass[class=article, crop=false]{standalone}
\usepackage[margin=1in]{geometry}

\usepackage{amsmath}
\usepackage{amssymb}
\usepackage{amsthm}
\usepackage{comment}
\usepackage{enumitem}
\usepackage{fancyhdr}
\usepackage{hyperref}
\usepackage{mathrsfs}
\usepackage{mathtools}
\usepackage{standalone}
\usepackage{tcolorbox}
\usepackage{tikz}
\usepackage{xcolor}

\usetikzlibrary{decorations.pathreplacing}
\tcbuselibrary{skins}

\DeclareMathOperator{\lcm}{lcm}
\DeclareMathOperator{\proj}{proj}
\DeclareMathOperator{\vspan}{span}
\DeclareMathOperator{\im}{im}
\DeclareMathOperator{\range}{range}
\DeclareMathOperator{\Diff}{Diff}
\DeclareMathOperator{\Int}{Int}
\DeclareMathOperator{\fcn}{fcn}
\DeclareMathOperator{\id}{id}

\newcommand{\N}{\ensuremath{\mathbb{N}}}
\newcommand{\Z}{\ensuremath{\mathbb{Z}}}
\newcommand{\Q}{\ensuremath{\mathbb{Q}}}
\newcommand{\R}{\ensuremath{\mathbb{R}}}
\newcommand{\C}{\ensuremath{\mathbb{C}}}
\newcommand{\F}{\ensuremath{\mathbb{F}}}
\newcommand{\lam}{\ensuremath{\lambda}}
\newcommand{\nab}{\ensuremath{\nabla}}
\newcommand{\eps}{\ensuremath{\varepsilon}}
\newcommand{\es}{\ensuremath{\varnothing}}

\newcommand{\abs}[1]{\ensuremath{\left\lvert #1 \right\rvert}}
\newcommand{\bigpar}[1]{\ensuremath{\left( #1 \right)}}
\newcommand{\norm}[1]{\ensuremath{\lVert #1\rVert}}
\newcommand{\set}[1]{\ensuremath{\left\{#1\right\}}}
\newcommand{\tuple}[1]{\ensuremath{\left\langle #1 \right\rangle}}
\newcommand{\floor}[1]{\ensuremath{\left\lfloor #1 \right\rfloor}}
\newcommand{\ceil}[1]{\ensuremath{\left\lceil #1 \right\rceil}}

\newcommand{\chapternum}{}
\newcommand{\ex}[1]{\noindent\textbf{Exercise \chapternum.{#1}.}}

\newcommand{\dx}[1]{\,\mathrm{d}#1}
\newcommand{\inv}{\ensuremath{^{-1}}}
\newcommand{\sm}{\setminus}
\newcommand{\sse}{\subseteq}
\newcommand{\ceq}{\coloneqq}

\definecolor{problemBackground}{RGB}{212,232,246}
\newenvironment{problem}[1]
  {
    \begin{tcolorbox}[
      boxrule=.5pt,
      titlerule=.5pt,
      sharp corners,
      colback=problemBackground
    ]
    \ifx &#1& \textbf{Problem. }
    \else \textbf{Problem #1.} \fi
  }
  {
    \end{tcolorbox}
  }

\definecolor{exampleBackground}{RGB}{255,249,248}
\definecolor{exampleAccent}{RGB}{158,60,14}
\newenvironment{example}[1]
  {
    \begin{tcolorbox}[
      boxrule=.5pt,
      sharp corners,
      colback=exampleBackground,
      colframe=exampleAccent,
    ]
    \color{exampleAccent}\textbf{Example.} \emph{#1}\color{black}
  }
  {
    \end{tcolorbox}
  }

\definecolor{theoremBackground}{RGB}{234,243,251}
\definecolor{theoremAccent}{RGB}{0,116,183}
\newenvironment{theorem}[1]
  {
    \begin{tcolorbox}[
      boxrule=.5pt,
      titlerule=.5pt,
      sharp corners,
      colback=theoremBackground,
      colframe=theoremAccent
    ]
      \color{theoremAccent}\textbf{Theorem --- }\emph{#1}\\\color{black}
  }
  {
    \end{tcolorbox}
  }

\definecolor{noteBackground}{RGB}{244,249,244}
\definecolor{noteAccent}{RGB}{34,139,34}
\newenvironment{note}[1]
  {
  \begin{tcolorbox}[
    enhanced,
    boxrule=0pt,
    frame hidden,
    sharp corners,
    colback=noteBackground,
    borderline west={3pt}{-1.5pt}{noteAccent}
    ]
    \ifx &#1& \color{noteAccent}\textbf{Note. }\color{black}
    \else \color{noteAccent}\textbf{Note (#1). }\color{black} \fi
    }
    {
  \end{tcolorbox}
  }

\definecolor{definitionBackground}{RGB}{246,246,246}
\newenvironment{definition}[1]
  {
    \begin{tcolorbox}[
      enhanced,
      boxrule=0pt,
      frame hidden,
      sharp corners,
      colback=definitionBackground,
      borderline west={3pt}{-1.5pt}{black}
    ]
    \textbf{Definition. }\emph{#1}\\
  }
  {
    \end{tcolorbox}
  }
\newenvironment{amatrix}[2]{
    \left[
      \begin{array}{*{#1}{c}|*{#2}c}
  }
  {
      \end{array}
    \right]
  }
\date{\the\year-\the\month-\the\day}
\author{Kyle Chui}


\fancyhf{}
\lhead{Kyle Chui}
\rhead{Page \thepage}
\pagestyle{fancy}

\begin{document}
  \section{Lecture 16}
  \subsection{AC Circuits---Motivation and History}
  There is power loss in normal circuits when the voltage is low, so we use transformers with ``high-voltage lines'' in order to reduce the power loss (not possible with DC power) when getting electricity from the source to destination. We have the equation
  \[
    \frac{P_\text{loss}}{P} = \frac{PR}{\eps^2}.
  \]
  \begin{note}{}
    The power loss decreases with the square of the voltage, so high-voltage lines are used.
  \end{note}
  \subsection{RMS Voltages and Currents}
  \begin{note}{}
    RMS stands for ``root mean square''.
  \end{note}
  Alternating currents follow a sinusoidal path, in other words
  \[
    I = I_0\sin (\omega t+\phi),
  \]
  where $f = \frac{\omega}{2\pi}$, which is $60$ Hz for Americans or $50$ Hz for Europeans. If we square the current, we have
  \begin{align*}
    I^2 &= I_0^2\sin^2(\omega t + \phi) \\
        &= \frac{I_0^2}{2}\paren{1 - \cos(2\omega t + 2\phi)},
  \end{align*}
  which is also sinusoidal (but stays positive). Furthermore, we know that the power in a resistor is $P = I^2R$. The average power is thus
  \[
    P_\text{avg} = (I^2)_\text{avg}R = \frac{I_0^2}{2}\cdot R = I_\text{RMS}^2R. \tag{$I_\text{RMS} \ceq \frac{I_0}{\sqrt{2}}$}
  \]
  Similarly, we define $V_\text{RMS}$ by $V_\text{RMS} \ceq \frac{V_0}{\sqrt{2}}$. \par
  For $V_\text{RMS} = 120$ V, we have $V_0 = V_\text{RMS}\sqrt{2}\approx 170$ V (the voltage fluctuates between $-170$ and $170$ volts).
  \subsection{The \emph{Driven} RLC Circuit}
  We take a typical RLC circuit and add a time-varying EMF to it. There are two solutions to this:
  \begin{enumerate}
    \item Transitory (sinusoidal but amplitude approaching zero)
    \item Steady-state (oscillates sinusoidally forever)
  \end{enumerate}
  To make things simple for ourselves, we choose a phase such that $q = A\sin (\omega t)$. Then we know the voltage across the capacitor,
  \[
    V_C = \frac{q}{C} = \frac{A}{C}\sin (\omega t).
  \]
  The current can then be found by differentiating:
  \[
    I = \frac{\mathrm{d}q}{\mathrm{d}t} = A\omega \cos(\omega t) = A\omega\sin \paren{\omega t + \frac{\pi}{2}}.
  \]
  And so the voltage across the resistor is given by
  \[
    V_R = RI = RA\omega\sin \left(\omega t + \frac{\pi}{2}\right).
  \]
  Finally, we find the voltage across the inductor:
  \[
    V_L = L\cdot \frac{\mathrm{d}I}{\mathrm{d}t} = -LA\omega^2\sin(\omega t) = LA\omega^2\sin(\omega t + \pi).
  \]
  We use the mnemonic ``ELI the ICE man''. For ``ELI'', we see that in an inductor (L) the EMF precedes the current. And for ``ICE'', we see that in a capacitor (C) the current precedes the EMF.
\end{document}
