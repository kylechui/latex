\documentclass[class=article, crop=false]{standalone}
\usepackage[margin=1in]{geometry}

\usepackage{amsmath}
\usepackage{amssymb}
\usepackage{amsthm}
\usepackage{comment}
\usepackage{enumitem}
\usepackage{fancyhdr}
\usepackage{hyperref}
\usepackage{mathrsfs}
\usepackage{mathtools}
\usepackage{standalone}
\usepackage{tcolorbox}
\usepackage{tikz}
\usepackage{xcolor}

\usetikzlibrary{decorations.pathreplacing}
\tcbuselibrary{skins}

\DeclareMathOperator{\lcm}{lcm}
\DeclareMathOperator{\proj}{proj}
\DeclareMathOperator{\vspan}{span}
\DeclareMathOperator{\im}{im}
\DeclareMathOperator{\range}{range}
\DeclareMathOperator{\Diff}{Diff}
\DeclareMathOperator{\Int}{Int}
\DeclareMathOperator{\fcn}{fcn}
\DeclareMathOperator{\id}{id}

\newcommand{\N}{\ensuremath{\mathbb{N}}}
\newcommand{\Z}{\ensuremath{\mathbb{Z}}}
\newcommand{\Q}{\ensuremath{\mathbb{Q}}}
\newcommand{\R}{\ensuremath{\mathbb{R}}}
\newcommand{\C}{\ensuremath{\mathbb{C}}}
\newcommand{\F}{\ensuremath{\mathbb{F}}}
\newcommand{\lam}{\ensuremath{\lambda}}
\newcommand{\nab}{\ensuremath{\nabla}}
\newcommand{\eps}{\ensuremath{\varepsilon}}
\newcommand{\es}{\ensuremath{\varnothing}}

\newcommand{\abs}[1]{\ensuremath{\left\lvert #1 \right\rvert}}
\newcommand{\bigpar}[1]{\ensuremath{\left( #1 \right)}}
\newcommand{\norm}[1]{\ensuremath{\lVert #1\rVert}}
\newcommand{\set}[1]{\ensuremath{\left\{#1\right\}}}
\newcommand{\tuple}[1]{\ensuremath{\left\langle #1 \right\rangle}}
\newcommand{\floor}[1]{\ensuremath{\left\lfloor #1 \right\rfloor}}
\newcommand{\ceil}[1]{\ensuremath{\left\lceil #1 \right\rceil}}

\newcommand{\chapternum}{}
\newcommand{\ex}[1]{\noindent\textbf{Exercise \chapternum.{#1}.}}

\newcommand{\dx}[1]{\,\mathrm{d}#1}
\newcommand{\inv}{\ensuremath{^{-1}}}
\newcommand{\sm}{\setminus}
\newcommand{\sse}{\subseteq}
\newcommand{\ceq}{\coloneqq}

\definecolor{problemBackground}{RGB}{212,232,246}
\newenvironment{problem}[1]
  {
    \begin{tcolorbox}[
      boxrule=.5pt,
      titlerule=.5pt,
      sharp corners,
      colback=problemBackground
    ]
    \ifx &#1& \textbf{Problem. }
    \else \textbf{Problem #1.} \fi
  }
  {
    \end{tcolorbox}
  }

\definecolor{exampleBackground}{RGB}{255,249,248}
\definecolor{exampleAccent}{RGB}{158,60,14}
\newenvironment{example}[1]
  {
    \begin{tcolorbox}[
      boxrule=.5pt,
      sharp corners,
      colback=exampleBackground,
      colframe=exampleAccent,
    ]
    \color{exampleAccent}\textbf{Example.} \emph{#1}\color{black}
  }
  {
    \end{tcolorbox}
  }

\definecolor{theoremBackground}{RGB}{234,243,251}
\definecolor{theoremAccent}{RGB}{0,116,183}
\newenvironment{theorem}[1]
  {
    \begin{tcolorbox}[
      boxrule=.5pt,
      titlerule=.5pt,
      sharp corners,
      colback=theoremBackground,
      colframe=theoremAccent
    ]
      \color{theoremAccent}\textbf{Theorem --- }\emph{#1}\\\color{black}
  }
  {
    \end{tcolorbox}
  }

\definecolor{noteBackground}{RGB}{244,249,244}
\definecolor{noteAccent}{RGB}{34,139,34}
\newenvironment{note}[1]
  {
  \begin{tcolorbox}[
    enhanced,
    boxrule=0pt,
    frame hidden,
    sharp corners,
    colback=noteBackground,
    borderline west={3pt}{-1.5pt}{noteAccent}
    ]
    \ifx &#1& \color{noteAccent}\textbf{Note. }\color{black}
    \else \color{noteAccent}\textbf{Note (#1). }\color{black} \fi
    }
    {
  \end{tcolorbox}
  }

\definecolor{definitionBackground}{RGB}{246,246,246}
\newenvironment{definition}[1]
  {
    \begin{tcolorbox}[
      enhanced,
      boxrule=0pt,
      frame hidden,
      sharp corners,
      colback=definitionBackground,
      borderline west={3pt}{-1.5pt}{black}
    ]
    \textbf{Definition. }\emph{#1}\\
  }
  {
    \end{tcolorbox}
  }
\newenvironment{amatrix}[2]{
    \left[
      \begin{array}{*{#1}{c}|*{#2}c}
  }
  {
      \end{array}
    \right]
  }
\date{\the\year-\the\month-\the\day}
\author{Kyle Chui}


\fancyhf{}
\lhead{Kyle Chui}
\rhead{Page \thepage}
\pagestyle{fancy}

\begin{document}
  \section{Lecture 17}
  \subsection{Reactance in the Driven RLC Circuit}
  Using the equations from last class, notice that
  \[
    \frac{V_C (\text{max})}{I(\text{max})} = \frac{A/C}{A\omega} = \frac{1}{\omega C} = \frac{V_C(\text{RMS})}{I(\text{RMS})}.
  \]
  \begin{definition}{Capacitive Reactance}
    We define the \emph{capacitive reactance} by the equation:
    \[
      X_C\ceq \frac{1}{\omega C}.
    \]
  \end{definition}
  Similarly for an inductor, we have
  \[
    \frac{V_L (\text{max})}{I (\text{max})} = \frac{LA\omega^2}{A\omega} = \omega L.
  \]
  \begin{definition}{Inductive Reactance}
    We define the \emph{inductive reactance} by the equation:
    \[
      X_L\ceq \omega L.
    \]
  \end{definition}
  By Kirchoff's Loop Rule, we have
  \begin{align*}
    \eps &= V_C + V_R + V_L \\
         &= X_CI_\text{max}\sin(\omega t) + I_\text{max}R\sin\paren{\omega t + \frac{\pi}{2}} + X_LI_\text{max}\sin(\omega t + \pi) \\
         &= I_\text{max}\paren{X_C\sin(\omega t) + R\sin\paren{\omega t + \frac{\pi}{2}} + X_L\sin(\omega t + \pi)}.
   \intertext{Dividing both sides by $I_\text{max}$, we get}
   \frac{\eps}{I_\text{max}} &= X_C\sin(\omega t) + R\sin\paren{\omega t + \frac{\pi}{2}} + X_L\sin(\omega t + \pi).
  \end{align*}
  \subsection{Phasors in the Driven RLC Circuit}
  If we graphed out all of these values as vectors and added them up, we would see that
  \[
    \frac{\eps_\text{max}}{I_\text{max}} = \sqrt{R^2 + (X_C - X_L)^2}.
  \]
  \begin{note}{}
    This is because if you pay attention to the angles in the original equation for $\frac{\eps}{I_\text{max}}$, you see that the middle term is perpendicular to the $X_C$ and $X_L$ terms, which are acting in opposition. Thus we may use the Pythagorean Theorem to get the above.
  \end{note}
  \newpage
  \subsection{Impedance Z in the Driven RLC Circuit}
  \begin{definition}{Impedance}
    We define the \emph{impedance} (units given in $\Omega$) to be
    \[
      Z\ceq \sqrt{R^2 + (X_C - X_L)^2}.
    \]
  \end{definition}
  Furthermore, for the initial phase, we have
  \[
    \tan\phi = \frac{X_L-X_C}{R}.
  \]
  The power consumed by this circuit is given by
  \[
    P_\text{avg} = I_\text{RMS}V_\text{RMS}\cdot \underbrace{\cos\phi}_{\mathclap{\text{``Power factor''}}}.
  \]
  \begin{note}{}
    To maximise the power of the motor, we want $\cos\phi$ to be as close to $0$ as possible. Thus we should add just the right amount of capacitance.
  \end{note}
\end{document}
