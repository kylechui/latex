\documentclass[class=article, crop=false]{standalone}
\usepackage[margin=1in]{geometry}

\usepackage{amsmath}
\usepackage{amssymb}
\usepackage{amsthm}
\usepackage{comment}
\usepackage{enumitem}
\usepackage{fancyhdr}
\usepackage{hyperref}
\usepackage{mathrsfs}
\usepackage{mathtools}
\usepackage{standalone}
\usepackage{tcolorbox}
\usepackage{tikz}
\usepackage{xcolor}

\usetikzlibrary{decorations.pathreplacing}
\tcbuselibrary{skins}

\DeclareMathOperator{\lcm}{lcm}
\DeclareMathOperator{\proj}{proj}
\DeclareMathOperator{\vspan}{span}
\DeclareMathOperator{\im}{im}
\DeclareMathOperator{\range}{range}
\DeclareMathOperator{\Diff}{Diff}
\DeclareMathOperator{\Int}{Int}
\DeclareMathOperator{\fcn}{fcn}
\DeclareMathOperator{\id}{id}

\newcommand{\N}{\ensuremath{\mathbb{N}}}
\newcommand{\Z}{\ensuremath{\mathbb{Z}}}
\newcommand{\Q}{\ensuremath{\mathbb{Q}}}
\newcommand{\R}{\ensuremath{\mathbb{R}}}
\newcommand{\C}{\ensuremath{\mathbb{C}}}
\newcommand{\F}{\ensuremath{\mathbb{F}}}
\newcommand{\lam}{\ensuremath{\lambda}}
\newcommand{\nab}{\ensuremath{\nabla}}
\newcommand{\eps}{\ensuremath{\varepsilon}}
\newcommand{\es}{\ensuremath{\varnothing}}

\newcommand{\abs}[1]{\ensuremath{\left\lvert #1 \right\rvert}}
\newcommand{\bigpar}[1]{\ensuremath{\left( #1 \right)}}
\newcommand{\norm}[1]{\ensuremath{\lVert #1\rVert}}
\newcommand{\set}[1]{\ensuremath{\left\{#1\right\}}}
\newcommand{\tuple}[1]{\ensuremath{\left\langle #1 \right\rangle}}
\newcommand{\floor}[1]{\ensuremath{\left\lfloor #1 \right\rfloor}}
\newcommand{\ceil}[1]{\ensuremath{\left\lceil #1 \right\rceil}}

\newcommand{\chapternum}{}
\newcommand{\ex}[1]{\noindent\textbf{Exercise \chapternum.{#1}.}}

\newcommand{\dx}[1]{\,\mathrm{d}#1}
\newcommand{\inv}{\ensuremath{^{-1}}}
\newcommand{\sm}{\setminus}
\newcommand{\sse}{\subseteq}
\newcommand{\ceq}{\coloneqq}

\definecolor{problemBackground}{RGB}{212,232,246}
\newenvironment{problem}[1]
  {
    \begin{tcolorbox}[
      boxrule=.5pt,
      titlerule=.5pt,
      sharp corners,
      colback=problemBackground
    ]
    \ifx &#1& \textbf{Problem. }
    \else \textbf{Problem #1.} \fi
  }
  {
    \end{tcolorbox}
  }

\definecolor{exampleBackground}{RGB}{255,249,248}
\definecolor{exampleAccent}{RGB}{158,60,14}
\newenvironment{example}[1]
  {
    \begin{tcolorbox}[
      boxrule=.5pt,
      sharp corners,
      colback=exampleBackground,
      colframe=exampleAccent,
    ]
    \color{exampleAccent}\textbf{Example.} \emph{#1}\color{black}
  }
  {
    \end{tcolorbox}
  }

\definecolor{theoremBackground}{RGB}{234,243,251}
\definecolor{theoremAccent}{RGB}{0,116,183}
\newenvironment{theorem}[1]
  {
    \begin{tcolorbox}[
      boxrule=.5pt,
      titlerule=.5pt,
      sharp corners,
      colback=theoremBackground,
      colframe=theoremAccent
    ]
      \color{theoremAccent}\textbf{Theorem --- }\emph{#1}\\\color{black}
  }
  {
    \end{tcolorbox}
  }

\definecolor{noteBackground}{RGB}{244,249,244}
\definecolor{noteAccent}{RGB}{34,139,34}
\newenvironment{note}[1]
  {
  \begin{tcolorbox}[
    enhanced,
    boxrule=0pt,
    frame hidden,
    sharp corners,
    colback=noteBackground,
    borderline west={3pt}{-1.5pt}{noteAccent}
    ]
    \ifx &#1& \color{noteAccent}\textbf{Note. }\color{black}
    \else \color{noteAccent}\textbf{Note (#1). }\color{black} \fi
    }
    {
  \end{tcolorbox}
  }

\definecolor{definitionBackground}{RGB}{246,246,246}
\newenvironment{definition}[1]
  {
    \begin{tcolorbox}[
      enhanced,
      boxrule=0pt,
      frame hidden,
      sharp corners,
      colback=definitionBackground,
      borderline west={3pt}{-1.5pt}{black}
    ]
    \textbf{Definition. }\emph{#1}\\
  }
  {
    \end{tcolorbox}
  }
\newenvironment{amatrix}[2]{
    \left[
      \begin{array}{*{#1}{c}|*{#2}c}
  }
  {
      \end{array}
    \right]
  }
\date{\the\year-\the\month-\the\day}
\author{Kyle Chui}


\fancyhf{}
\lhead{Kyle Chui}
\rhead{Page \thepage}
\pagestyle{fancy}

\begin{document}
  \section{Lecture 33}
  \subsection{Galilean Relativity}
  Inertial frames are frames in which if $\vec{F} = m\vec{a} = 0$, then $\vec{a} = 0$. Consider two different coordinate systems, one called $S$ which is the ``stationary'' coordinate system, and another that is shifted over from $S$ labelled $S'$, which is a ``moving'' coordinate system that is moving to the right at speed $v$. Then we have
  \begin{align*}
    x' &= x - vt \\
    y' &= y \\
    z' &= z
  \end{align*}
  Differentiating with respect to $t$, we have
  \begin{align*}
    v_x &= v_{x'} + v \\
    v_y &= v_{y'} \\
    v_z &= v_{z'}
  \end{align*}
  Assuming $v$ is constant, then the accelerations for any object is the same in either coordinate system.
  \subsection{Einstein Relativity}
  We know that the speed of light is
  \[
    c = \frac{1}{\sqrt{\mu_0\eps_0}} \approx 3\cdot 10^8\ \frac{\mathrm{m}}{\mathrm{s}}.
  \]
  What is the inertial frame in which this applies?
  \subsubsection{Michelson--Morley Experiment}
  The Michelson--Morley experiment measured the speed of light from a distant star from two different points in Earth's orbit (6 months apart), and measured the same speed of light. \par
  Einstein proposed that there is no ``preferred'' inertial frame, i.e. that Galilean relativity is wrong. Let's assume that electromagnetism laws are correct, and that the speed of light is constant.
  \subsubsection{The ``Light Clock''}
  Consider a container that holds a light emitter, a detector, and a mirror, where the emitter emits a beam of light that reflects off of the mirror and into the detector. If the distance from the emitter/detector to the mirror is $D$, then the time is takes for the light to be detected is $\Delta \tau = \frac{2D}{c}$. \par
  Now consider the situation in which the container is moving to the right extremely quickly at a speed $v$. The light beam must be angled forwards in order to reflect back and hit the detector. Denoting the total distance travelled by the light by $s$, we have
  \begin{align*}
    s^2 &= D^2 + \paren{\frac{v\Delta t}{2}}^2 \\
    \paren{c\cdot \frac{\Delta t}{2}}^2 &= \paren{\frac{c\Delta\tau}{2}}^2 + \paren{\frac{v\Delta t}{2}}^2 \\
    c^2(\Delta t)^2 &= c^2(\Delta\tau)^2 + v^2(\Delta t)^2 \\
    (\Delta\tau)^2 &= \frac{c^2-v^2}{c^2}\cdot (\Delta t)^2 \\
    \Delta t &= \frac{c}{\sqrt{c^2-v^2}}\cdot \Delta\tau.
  \end{align*}
  From this we can see that assuming that the speed of light is constant, that faster objects experience time slower than slower or stationary objects.
\end{document}
