\documentclass[class=article, crop=false]{standalone}
\usepackage[margin=1in]{geometry}

\usepackage{amsmath}
\usepackage{amssymb}
\usepackage{amsthm}
\usepackage{comment}
\usepackage{enumitem}
\usepackage{fancyhdr}
\usepackage{hyperref}
\usepackage{mathrsfs}
\usepackage{mathtools}
\usepackage{standalone}
\usepackage{tcolorbox}
\usepackage{tikz}
\usepackage{xcolor}

\usetikzlibrary{decorations.pathreplacing}
\tcbuselibrary{skins}

\DeclareMathOperator{\lcm}{lcm}
\DeclareMathOperator{\proj}{proj}
\DeclareMathOperator{\vspan}{span}
\DeclareMathOperator{\im}{im}
\DeclareMathOperator{\range}{range}
\DeclareMathOperator{\Diff}{Diff}
\DeclareMathOperator{\Int}{Int}
\DeclareMathOperator{\fcn}{fcn}
\DeclareMathOperator{\id}{id}

\newcommand{\N}{\ensuremath{\mathbb{N}}}
\newcommand{\Z}{\ensuremath{\mathbb{Z}}}
\newcommand{\Q}{\ensuremath{\mathbb{Q}}}
\newcommand{\R}{\ensuremath{\mathbb{R}}}
\newcommand{\C}{\ensuremath{\mathbb{C}}}
\newcommand{\F}{\ensuremath{\mathbb{F}}}
\newcommand{\lam}{\ensuremath{\lambda}}
\newcommand{\nab}{\ensuremath{\nabla}}
\newcommand{\eps}{\ensuremath{\varepsilon}}
\newcommand{\es}{\ensuremath{\varnothing}}

\newcommand{\abs}[1]{\ensuremath{\left\lvert #1 \right\rvert}}
\newcommand{\bigpar}[1]{\ensuremath{\left( #1 \right)}}
\newcommand{\norm}[1]{\ensuremath{\lVert #1\rVert}}
\newcommand{\set}[1]{\ensuremath{\left\{#1\right\}}}
\newcommand{\tuple}[1]{\ensuremath{\left\langle #1 \right\rangle}}
\newcommand{\floor}[1]{\ensuremath{\left\lfloor #1 \right\rfloor}}
\newcommand{\ceil}[1]{\ensuremath{\left\lceil #1 \right\rceil}}

\newcommand{\chapternum}{}
\newcommand{\ex}[1]{\noindent\textbf{Exercise \chapternum.{#1}.}}

\newcommand{\dx}[1]{\,\mathrm{d}#1}
\newcommand{\inv}{\ensuremath{^{-1}}}
\newcommand{\sm}{\setminus}
\newcommand{\sse}{\subseteq}
\newcommand{\ceq}{\coloneqq}

\definecolor{problemBackground}{RGB}{212,232,246}
\newenvironment{problem}[1]
  {
    \begin{tcolorbox}[
      boxrule=.5pt,
      titlerule=.5pt,
      sharp corners,
      colback=problemBackground
    ]
    \ifx &#1& \textbf{Problem. }
    \else \textbf{Problem #1.} \fi
  }
  {
    \end{tcolorbox}
  }

\definecolor{exampleBackground}{RGB}{255,249,248}
\definecolor{exampleAccent}{RGB}{158,60,14}
\newenvironment{example}[1]
  {
    \begin{tcolorbox}[
      boxrule=.5pt,
      sharp corners,
      colback=exampleBackground,
      colframe=exampleAccent,
    ]
    \color{exampleAccent}\textbf{Example.} \emph{#1}\color{black}
  }
  {
    \end{tcolorbox}
  }

\definecolor{theoremBackground}{RGB}{234,243,251}
\definecolor{theoremAccent}{RGB}{0,116,183}
\newenvironment{theorem}[1]
  {
    \begin{tcolorbox}[
      boxrule=.5pt,
      titlerule=.5pt,
      sharp corners,
      colback=theoremBackground,
      colframe=theoremAccent
    ]
      \color{theoremAccent}\textbf{Theorem --- }\emph{#1}\\\color{black}
  }
  {
    \end{tcolorbox}
  }

\definecolor{noteBackground}{RGB}{244,249,244}
\definecolor{noteAccent}{RGB}{34,139,34}
\newenvironment{note}[1]
  {
  \begin{tcolorbox}[
    enhanced,
    boxrule=0pt,
    frame hidden,
    sharp corners,
    colback=noteBackground,
    borderline west={3pt}{-1.5pt}{noteAccent}
    ]
    \ifx &#1& \color{noteAccent}\textbf{Note. }\color{black}
    \else \color{noteAccent}\textbf{Note (#1). }\color{black} \fi
    }
    {
  \end{tcolorbox}
  }

\definecolor{definitionBackground}{RGB}{246,246,246}
\newenvironment{definition}[1]
  {
    \begin{tcolorbox}[
      enhanced,
      boxrule=0pt,
      frame hidden,
      sharp corners,
      colback=definitionBackground,
      borderline west={3pt}{-1.5pt}{black}
    ]
    \textbf{Definition. }\emph{#1}\\
  }
  {
    \end{tcolorbox}
  }
\newenvironment{amatrix}[2]{
    \left[
      \begin{array}{*{#1}{c}|*{#2}c}
  }
  {
      \end{array}
    \right]
  }
\date{\the\year-\the\month-\the\day}
\author{Kyle Chui}


\fancyhf{}
\lhead{Kyle Chui}
\rhead{Page \thepage}
\pagestyle{fancy}

\begin{document}
  \section{Lecture 19}
  \subsection{Displacement Current}
  Recall Ampere's Law:
  \[
    \oint_\text{loop}\vec{B}\cdot \mathrm{d}\vec{\ell} = \mu_0I_\text{encl}.
  \]
  When we say $I_\text{encl}$ to mean enclosed current, we mean that the current goes through the surface. The main thing to take note of is that the surface doesn't need to be planar---it can be any surface. \par
  Consider a current flowing into a capacitor, with an amperium loop that goes around half of the capacitor. There is seemingly no current flowing across the capacitor, a contradiction. What Maxwell notices is that charges build up on the plates of the capacitor---maybe a changing $\vec{E}$ field could produce a current. He modified Ampere's Law to get
  \[
    \oint_\text{loop}\vec{B}\cdot \mathrm{d}\vec{\ell} = \mu_0(I_\text{encl} + I_\text{disp}).
  \]
  His idea was that $I_\text{disp}$ should be proportional to $\frac{\mathrm{d}E}{\mathrm{d}t}$ and proportional to the area $A$ of the parallel plates. \par
  Remember that for capacitors, $E = \frac{V}{d}$ and $C = \eps_0\cdot \frac{A}{d}$. Thus
  \begin{align*}
    A\cdot \frac{\mathrm{d}E}{\mathrm{d}t} &= \paren{\frac{Cd}{\eps_0}}\cdot \frac{1}{d}\cdot \frac{\mathrm{d}V}{\mathrm{d}t} \\
                                           &= \frac{1}{\eps_0}C\cdot \frac{\mathrm{d}Q}{\mathrm{d}t}\cdot \frac{1}{C} \\
                                           &= \frac{1}{\eps_0}\cdot \frac{\mathrm{d}Q}{\mathrm{d}t}.
  \end{align*}
  Rearranging the above, we get
  \[
    \frac{\mathrm{d}Q}{\mathrm{d}t} = I_\text{disp} = \eps_0\cdot A\cdot \frac{\mathrm{d}E}{\mathrm{d}t}.
  \]
  Thus Maxwell's modification of Ampere's Law takes the form
  \[
    \oint_\text{loop}\vec{B}\cdot \mathrm{d}\vec{\ell} = \mu_0\paren{I_\text{encl} + \eps_0A\cdot \frac{\mathrm{d}E}{\mathrm{d}t}}.
  \]
  \begin{note}{}
    We can substitute $A\cdot \frac{\mathrm{d}E}{\mathrm{d}t}$ for $\frac{\mathrm{d}\Phi_E}{\mathrm{d}t}$.
  \end{note}
  This forms a parallel with Faraday's Law, which states that a changing $\Phi_B$ generates $\vec{E}$, whereas this says that a changing $\Phi_E$ generates $\vec{B}$.
  \begin{note}{}
    A magnetic field (although possibly not large) is generated when charging up a capacitor, parallel to the plates of the capacitor.
  \end{note}
  \begin{note}{}
    This exhibits \emph{exactly} the same behaviour as the wire of radius $R$ from Lecture $6$.
  \end{note}
  \subsection{Maxwell's Equations}
  \begin{enumerate}
    \item Gauss' Law for $\vec{E}$:
    \[
      \oint \vec{E}\cdot \mathrm{d}\vec{A} = \frac{Q_\text{encl}}{\eps_0}
    \]
    \item Gauss' Law for $\vec{B}$:
    \[
      \oint \vec{B}\cdot \mathrm{d}\vec{A} = 0
    \]
    \item Faraday's Law for a stationary integration path:
    \[
      \oint \vec{E}\cdot \mathrm{d}\vec{\ell} = -\frac{\mathrm{d}\Phi_B}{\mathrm{d}t}
    \]
    \item Ampere's law for a stationary integration path:
    \[
      \oint \vec{B}\cdot \mathrm{d}\vec{\ell} = \mu_0\paren{i_C + \eps_0 \frac{\mathrm{d}\Phi_E}{\mathrm{d}t}}_\text{encl}
    \]
  \end{enumerate}
  \begin{note}{}
    In empty space (with no charges nor currents), Maxwell's equations are highly symmetric.
  \end{note}
  \subsection{Plane Electromagnetic Waves (in Vacuum)}
  \subsubsection{Recap on Waves}
  Given a ``wave number'' $k$, we usually describe sinusoidal waves by
  \[
    f(x) = A\sin (kx).
  \]
  We know that the wavelength must satisfy $k\lam = 2\pi$, so $\lam = \frac{2\pi}{k}$. If we take $f(x - vt)$, the wave is moving to the \emph{right}. If the function is $f(x + vt)$, then the function is moving to the \emph{left}. Waves can take other shapes, for example
  \[
    f(x, t) = \frac{1}{1 + (x - vt)^2}. \tag{Moves to the right}
  \]
\end{document}
