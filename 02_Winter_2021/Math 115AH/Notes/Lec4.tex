\documentclass[class=article, crop=false]{standalone}
% Import packages
\usepackage[margin=1in]{geometry}

\usepackage{amssymb, amsthm}
\usepackage{comment}
\usepackage{enumitem}
\usepackage{fancyhdr}
\usepackage{hyperref}
\usepackage{import}
\usepackage{mathrsfs, mathtools}
\usepackage{multicol}
\usepackage{pdfpages}
\usepackage{standalone}
\usepackage{transparent}
\usepackage{xcolor}

\usepackage{minted}
\usepackage[many]{tcolorbox}
\tcbuselibrary{minted}

% Declare math operators
\DeclareMathOperator{\lcm}{lcm}
\DeclareMathOperator{\proj}{proj}
\DeclareMathOperator{\vspan}{span}
\DeclareMathOperator{\im}{im}
\DeclareMathOperator{\Diff}{Diff}
\DeclareMathOperator{\Int}{Int}
\DeclareMathOperator{\fcn}{fcn}
\DeclareMathOperator{\id}{id}
\DeclareMathOperator{\rank}{rank}
\DeclareMathOperator{\tr}{tr}
\DeclareMathOperator{\dive}{div}
\DeclareMathOperator{\row}{row}
\DeclareMathOperator{\col}{col}
\DeclareMathOperator{\dom}{dom}
\DeclareMathOperator{\ran}{ran}
\DeclareMathOperator{\Dom}{Dom}
\DeclareMathOperator{\Ran}{Ran}
\DeclareMathOperator{\fl}{fl}
% Macros for letters/variables
\newcommand\ttilde{\kern -.15em\lower .7ex\hbox{\~{}}\kern .04em}
\newcommand{\N}{\ensuremath{\mathbb{N}}}
\newcommand{\Z}{\ensuremath{\mathbb{Z}}}
\newcommand{\Q}{\ensuremath{\mathbb{Q}}}
\newcommand{\R}{\ensuremath{\mathbb{R}}}
\newcommand{\C}{\ensuremath{\mathbb{C}}}
\newcommand{\F}{\ensuremath{\mathbb{F}}}
\newcommand{\M}{\ensuremath{\mathbb{M}}}
\renewcommand{\P}{\ensuremath{\mathbb{P}}}
\newcommand{\lam}{\ensuremath{\lambda}}
\newcommand{\nab}{\ensuremath{\nabla}}
\newcommand{\eps}{\ensuremath{\varepsilon}}
\newcommand{\es}{\ensuremath{\varnothing}}
% Macros for math symbols
\newcommand{\dx}[1]{\,\mathrm{d}#1}
\newcommand{\inv}{\ensuremath{^{-1}}}
\newcommand{\sm}{\setminus}
\newcommand{\sse}{\subseteq}
\newcommand{\ceq}{\coloneqq}
% Macros for pairs of math symbols
\newcommand{\abs}[1]{\ensuremath{\left\lvert #1 \right\rvert}}
\newcommand{\paren}[1]{\ensuremath{\left( #1 \right)}}
\newcommand{\norm}[1]{\ensuremath{\left\lVert #1\right\rVert}}
\newcommand{\set}[1]{\ensuremath{\left\{#1\right\}}}
\newcommand{\tup}[1]{\ensuremath{\left\langle #1 \right\rangle}}
\newcommand{\floor}[1]{\ensuremath{\left\lfloor #1 \right\rfloor}}
\newcommand{\ceil}[1]{\ensuremath{\left\lceil #1 \right\rceil}}
\newcommand{\eclass}[1]{\ensuremath{\left[ #1 \right]}}

\newcommand{\chapternum}{}
\newcommand{\ex}[1]{\noindent\textbf{Exercise \chapternum.{#1}.}}

\newcommand{\tsub}[1]{\textsubscript{#1}}
\newcommand{\tsup}[1]{\textsuperscript{#1}}

\renewcommand{\choose}[2]{\ensuremath{\phantom{}_{#1}C_{#2}}}
\newcommand{\perm}[2]{\ensuremath{\phantom{}_{#1}P_{#2}}}

% Include figures
\newcommand{\incfig}[2][1]{%
    \def\svgwidth{#1\columnwidth}
    \import{./figures/}{#2.pdf_tex}
}

\definecolor{problemBackground}{RGB}{212,232,246}
\newenvironment{problem}[1]
  {
    \begin{tcolorbox}[
      boxrule=.5pt,
      titlerule=.5pt,
      sharp corners,
      colback=problemBackground,
      breakable
    ]
    \ifx &#1& \textbf{Problem. }
    \else \textbf{Problem #1.} \fi
  }
  {
    \end{tcolorbox}
  }
\definecolor{exampleBackground}{RGB}{255,249,248}
\definecolor{exampleAccent}{RGB}{158,60,14}
\newenvironment{example}[1]
  {
    \begin{tcolorbox}[
      boxrule=.5pt,
      sharp corners,
      colback=exampleBackground,
      colframe=exampleAccent,
    ]
    \color{exampleAccent}\textbf{Example.} \emph{#1}\color{black}
  }
  {
    \end{tcolorbox}
  }
\definecolor{theoremBackground}{RGB}{234,243,251}
\definecolor{theoremAccent}{RGB}{0,116,183}
\newenvironment{theorem}[1]
  {
    \begin{tcolorbox}[
      boxrule=.5pt,
      titlerule=.5pt,
      sharp corners,
      colback=theoremBackground,
      colframe=theoremAccent,
      breakable
    ]
      % \color{theoremAccent}\textbf{Theorem --- }\emph{#1}\\\color{black}
    \ifx &#1& \color{theoremAccent}\textbf{Theorem. }\color{black}
    \else \color{theoremAccent}\textbf{Theorem --- }\emph{#1}\\\color{black} \fi
  }
  {
    \end{tcolorbox}
  }
\definecolor{noteBackground}{RGB}{244,249,244}
\definecolor{noteAccent}{RGB}{34,139,34}
\newenvironment{note}[1]
  {
  \begin{tcolorbox}[
    enhanced,
    boxrule=0pt,
    frame hidden,
    sharp corners,
    colback=noteBackground,
    borderline west={3pt}{-1.5pt}{noteAccent},
    breakable
    ]
    \ifx &#1& \color{noteAccent}\textbf{Note. }\color{black}
    \else \color{noteAccent}\textbf{Note (#1). }\color{black} \fi
    }
    {
  \end{tcolorbox}
  }
\definecolor{lemmaBackground}{RGB}{255,247,234}
\definecolor{lemmaAccent}{RGB}{255,153,0}
\newenvironment{lemma}[1]
  {
    \begin{tcolorbox}[
      enhanced,
      boxrule=0pt,
      frame hidden,
      sharp corners,
      colback=lemmaBackground,
      borderline west={3pt}{-1.5pt}{lemmaAccent},
      breakable
    ]
    \ifx &#1& \color{lemmaAccent}\textbf{Lemma. }\color{black}
    \else \color{lemmaAccent}\textbf{Lemma #1. }\color{black} \fi
  }
  {
    \end{tcolorbox}
  }
\definecolor{definitionBackground}{RGB}{246,246,246}
\newenvironment{definition}[1]
  {
    \begin{tcolorbox}[
      enhanced,
      boxrule=0pt,
      frame hidden,
      sharp corners,
      colback=definitionBackground,
      borderline west={3pt}{-1.5pt}{black},
      breakable
    ]
    \textbf{Definition. }\emph{#1}\\
  }
  {
    \end{tcolorbox}
  }

\newenvironment{amatrix}[2]{
    \left[
      \begin{array}{*{#1}{c}|*{#2}c}
  }
  {
      \end{array}
    \right]
  }

\definecolor{codeBackground}{RGB}{253,246,227}
\renewcommand\theFancyVerbLine{\arabic{FancyVerbLine}}
\newtcblisting{cpp}[1][]{
  boxrule=1pt,
  sharp corners,
  colback=codeBackground,
  listing only,
  breakable,
  minted language=cpp,
  minted style=material,
  minted options={
    xleftmargin=15pt,
    baselinestretch=1.2,
    linenos,
  }
}

\author{Kyle Chui}


\fancyhf{}
\lhead{Kyle Chui}
\rhead{Page \thepage}
\pagestyle{fancy}
\pagenumbering{gobble}

\title{Lecture 4 Notes}

\begin{document}
  \maketitle
  \newpage
  \pagenumbering{arabic}
  \section{Subspaces}
  \begin{note}{Subset $\neq$ Subspace}
    Most subsets of vector spaces are not subspaces!
  \end{note}
  \begin{example}{Subspaces}
    \begin{enumerate}
      \item Let $F$ be a field. Then $F = F[t]_0 < F[t]_1 < \dotsb < F[t]_n < \dotsb < F[t]$ are vector spaces over $F$, each a subspace of the vector space that contains them.
      \item If $W_1 \sse W_2 \sse V$, where $W_1, W_2$ are subspaces of $V$, then $W_1 \sse W_2$ is a subspace.
      \item If $W_1 \sse W_2$ is a subspace and $W_2 \sse V$ is a subspace, then $W \sse V$ is a subspace.
      \item Let $W\ceq \set{(0, \alpha_1, \dotsc, \alpha_n)\mid \alpha_i \in F, 2 \leq i \leq n} \sse \F^n$.
      \item Every line or plane through the origin in $\R^3$ is a subspace.
    \end{enumerate}
  \end{example}
  \subsection{Linear Combinations and Span}
  \begin{definition}{Linear Combination}
    Let $V$ be a vector space over a field $F$, $v_1, v_2, \dotsc, v_n \in V$. We say $v\in V$ is a \emph{linear combination} of $v_1, \dotsc, v_n$ if there exists $\alpha_1, \dotsc, \alpha_n\in F$ such that $v = \alpha_1v_1 + \dotsb + \alpha_nv_n$.
  \end{definition}
  Let $\vspan(v_1,\dotsc,v_n)\ceq\set{\text{All linear combinations of }v_1, v_2, \dotsc, v_n}$. We may also write
  \[
    \vspan(v_1,\dotsc,v_n) = \set{\sum_{i=1}^{n}\alpha_iv_i\mid \alpha_1, \dotsc, \alpha_n \in F}.
  \]
  This is a subspace of $V$ (by Subspace Theorem) called the \emph{span} of $v_1, \dotsc, v_n$. It is the (unique) ``smallest" subspace of $V$ containing $v_1, \dotsc, v_n$. i.e. if $W \sse V$ is a subspace and $v_1, \dotsc, v_n \in W$, then $\vspan(v_1,\dotsc, v_n)\sse W$. We also let $\vspan(\es) \ceq \set{0_V} = 0$. The smallest vector space containing no vectors. \par
  \textbf{Question.} If we view $\C$ as a vector space over $\R$, then $\R$ is a subspace of $\C$, but if we view $\C$ as a vector space over $\C$, then $\R$ is \emph{not} a subspace of $\C$ (why?). What is going on?
  \begin{definition}{Span of a Set of Vectors}
    Let $V$ be a vector space over a field $F$, $\es \neq S \sse V$ a subset. Then
    \[
      \vspan(S) \ceq \text{the set of all \emph{finite} linear combinations of vectors in }S.
    \]
  \end{definition}
  i.e. if $v\in \vspan(S)$, then
  \[
  \exists v_1, \dotsc, v_n\in S, \alpha_1, \dotsc, \alpha_n \in F \text{ such that } v = \alpha_1v_1 + \dotsb + \alpha_nv_n.
  \]
  Thus $\vspan(S)\sse V$ is a subspace. \par
  \textbf{Question.} What is the smallest set of vectors in $V$ such that $\vspan(S) = V$?
  \begin{note}{Vector spaces}
    There exist sets that are not vector spaces over one field, but are vector spaces over other fields.
  \end{note}
  \begin{definition}{Kernel of a Matrix}
    The \emph{kernel} of a matrix $A$ is the set of all vectors that get mapped to the zero vector. In other words,
    \[
      \ker A \ceq \set{x\in F^{n\times1}\mid Ax = 0}.
    \]
  \end{definition}
  \begin{theorem}{Intersection of Subspaces is a Subspace}
    Let $W_i \in V$, $i\in I$ be subspaces. Then
    \[
      \bigcap_I W = \bigcap_{i\in I}W_i = \set{x\in V\mid x\in W_i, \forall i\in I}
    \]
    is a subspace of $V$.
  \end{theorem}
  \begin{note}{Union of Subspaces}
    The union of subspaces is not necessarily a subspace. For instance, consider two lines going through the origin in $\R^2$ that are not the same line.
  \end{note}
  \begin{definition}{Sum of Subspaces}
    Let $W_1, W_2\sse V$ be subspaces. Define
    \begin{align*}
      W_1 + W_2 &\ceq \set{w_1 + w_2\mid w_1\in W_1, w_2\in W_2} \\
      &= \vspan(W_1 \cup W_2).
    \end{align*}
  \end{definition}
  So $W_1 + W_2\sse V$ is a subspace and the smallest subspace of $V$ containing $W_1$ and $W_2$.
  \subsection{Linear Independence}
  We know that $\R^n$ is an $n$-dimensional vector space over $\R$. Since we need $n$ coordinates to describe all vectors in $\R^n$ but no fewer will do. We want something like the following: \par
  Let $V$ be a vector space over a field $F$ with $V \neq 0$. Can we find distinct vectors $v_1, \dotsc, v_n$ in $V$, some $n$ with the following properties:
  \begin{enumerate}
    \item $V = \vspan(v_1, \dotsc, v_n)$
    \item No $v_i$ is a linear combination of $v_1, \dotsc, v_{i-1}, v_{i+1}, \dotsc, v_n$
  \end{enumerate}
  Then we want to call $V$ an $n$-dimensional vector space over $F$. \par
  \textbf{Lemma.} Let $V$ be a vector space over $F$, and $n > 1$. Suppose $v_1, \dotsc, v_n$ are distinct. Then $(2)$ is equivalent to if
  \[
    \alpha_1v_1 + \dotsb + \alpha_nv_n = \beta_1v_1 + \dotsb + \beta_nv_n, \alpha_i,\beta_i\in F,\forall i,
  \]
  then $\alpha_i = \beta_i,\forall i$. In other words, the ``coordinates" are unique.
  \begin{proof}
    (Contradiction) Without loss of generality, suppose $\alpha_1 \neq \beta_1$. Then $\alpha_1 - \beta_1 \neq 0$, so it has a multiplicative inverse. Thus
    \begin{align*}
      (\alpha_1 - \beta_1)v_1 &= \sum_{i=2}^{n}(\beta_i-\alpha_i)v_i  \\
      v_1 &= \frac{1}{\alpha_1-\beta_1} \cdot \sum_{i=2}^{n}(\beta_i-\alpha_i)v_i\tag{$\alpha_1-\beta_1\neq0$}
    \end{align*}
     so $v_1 \in \vspan(v_2,\dotsc,v_n)$, which is a contradiction.
  \end{proof}
\end{document}
