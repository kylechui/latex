\documentclass[class=article, crop=false]{standalone}
% Import packages
\usepackage[margin=1in]{geometry}

\usepackage{amssymb, amsthm}
\usepackage{comment}
\usepackage{enumitem}
\usepackage{fancyhdr}
\usepackage{hyperref}
\usepackage{import}
\usepackage{mathrsfs, mathtools}
\usepackage{multicol}
\usepackage{pdfpages}
\usepackage{standalone}
\usepackage{transparent}
\usepackage{xcolor}

\usepackage{minted}
\usepackage[many]{tcolorbox}
\tcbuselibrary{minted}

% Declare math operators
\DeclareMathOperator{\lcm}{lcm}
\DeclareMathOperator{\proj}{proj}
\DeclareMathOperator{\vspan}{span}
\DeclareMathOperator{\im}{im}
\DeclareMathOperator{\Diff}{Diff}
\DeclareMathOperator{\Int}{Int}
\DeclareMathOperator{\fcn}{fcn}
\DeclareMathOperator{\id}{id}
\DeclareMathOperator{\rank}{rank}
\DeclareMathOperator{\tr}{tr}
\DeclareMathOperator{\dive}{div}
\DeclareMathOperator{\row}{row}
\DeclareMathOperator{\col}{col}
\DeclareMathOperator{\dom}{dom}
\DeclareMathOperator{\ran}{ran}
\DeclareMathOperator{\Dom}{Dom}
\DeclareMathOperator{\Ran}{Ran}
\DeclareMathOperator{\fl}{fl}
% Macros for letters/variables
\newcommand\ttilde{\kern -.15em\lower .7ex\hbox{\~{}}\kern .04em}
\newcommand{\N}{\ensuremath{\mathbb{N}}}
\newcommand{\Z}{\ensuremath{\mathbb{Z}}}
\newcommand{\Q}{\ensuremath{\mathbb{Q}}}
\newcommand{\R}{\ensuremath{\mathbb{R}}}
\newcommand{\C}{\ensuremath{\mathbb{C}}}
\newcommand{\F}{\ensuremath{\mathbb{F}}}
\newcommand{\M}{\ensuremath{\mathbb{M}}}
\renewcommand{\P}{\ensuremath{\mathbb{P}}}
\newcommand{\lam}{\ensuremath{\lambda}}
\newcommand{\nab}{\ensuremath{\nabla}}
\newcommand{\eps}{\ensuremath{\varepsilon}}
\newcommand{\es}{\ensuremath{\varnothing}}
% Macros for math symbols
\newcommand{\dx}[1]{\,\mathrm{d}#1}
\newcommand{\inv}{\ensuremath{^{-1}}}
\newcommand{\sm}{\setminus}
\newcommand{\sse}{\subseteq}
\newcommand{\ceq}{\coloneqq}
% Macros for pairs of math symbols
\newcommand{\abs}[1]{\ensuremath{\left\lvert #1 \right\rvert}}
\newcommand{\paren}[1]{\ensuremath{\left( #1 \right)}}
\newcommand{\norm}[1]{\ensuremath{\left\lVert #1\right\rVert}}
\newcommand{\set}[1]{\ensuremath{\left\{#1\right\}}}
\newcommand{\tup}[1]{\ensuremath{\left\langle #1 \right\rangle}}
\newcommand{\floor}[1]{\ensuremath{\left\lfloor #1 \right\rfloor}}
\newcommand{\ceil}[1]{\ensuremath{\left\lceil #1 \right\rceil}}
\newcommand{\eclass}[1]{\ensuremath{\left[ #1 \right]}}

\newcommand{\chapternum}{}
\newcommand{\ex}[1]{\noindent\textbf{Exercise \chapternum.{#1}.}}

\newcommand{\tsub}[1]{\textsubscript{#1}}
\newcommand{\tsup}[1]{\textsuperscript{#1}}

\renewcommand{\choose}[2]{\ensuremath{\phantom{}_{#1}C_{#2}}}
\newcommand{\perm}[2]{\ensuremath{\phantom{}_{#1}P_{#2}}}

% Include figures
\newcommand{\incfig}[2][1]{%
    \def\svgwidth{#1\columnwidth}
    \import{./figures/}{#2.pdf_tex}
}

\definecolor{problemBackground}{RGB}{212,232,246}
\newenvironment{problem}[1]
  {
    \begin{tcolorbox}[
      boxrule=.5pt,
      titlerule=.5pt,
      sharp corners,
      colback=problemBackground,
      breakable
    ]
    \ifx &#1& \textbf{Problem. }
    \else \textbf{Problem #1.} \fi
  }
  {
    \end{tcolorbox}
  }
\definecolor{exampleBackground}{RGB}{255,249,248}
\definecolor{exampleAccent}{RGB}{158,60,14}
\newenvironment{example}[1]
  {
    \begin{tcolorbox}[
      boxrule=.5pt,
      sharp corners,
      colback=exampleBackground,
      colframe=exampleAccent,
    ]
    \color{exampleAccent}\textbf{Example.} \emph{#1}\color{black}
  }
  {
    \end{tcolorbox}
  }
\definecolor{theoremBackground}{RGB}{234,243,251}
\definecolor{theoremAccent}{RGB}{0,116,183}
\newenvironment{theorem}[1]
  {
    \begin{tcolorbox}[
      boxrule=.5pt,
      titlerule=.5pt,
      sharp corners,
      colback=theoremBackground,
      colframe=theoremAccent,
      breakable
    ]
      % \color{theoremAccent}\textbf{Theorem --- }\emph{#1}\\\color{black}
    \ifx &#1& \color{theoremAccent}\textbf{Theorem. }\color{black}
    \else \color{theoremAccent}\textbf{Theorem --- }\emph{#1}\\\color{black} \fi
  }
  {
    \end{tcolorbox}
  }
\definecolor{noteBackground}{RGB}{244,249,244}
\definecolor{noteAccent}{RGB}{34,139,34}
\newenvironment{note}[1]
  {
  \begin{tcolorbox}[
    enhanced,
    boxrule=0pt,
    frame hidden,
    sharp corners,
    colback=noteBackground,
    borderline west={3pt}{-1.5pt}{noteAccent},
    breakable
    ]
    \ifx &#1& \color{noteAccent}\textbf{Note. }\color{black}
    \else \color{noteAccent}\textbf{Note (#1). }\color{black} \fi
    }
    {
  \end{tcolorbox}
  }
\definecolor{lemmaBackground}{RGB}{255,247,234}
\definecolor{lemmaAccent}{RGB}{255,153,0}
\newenvironment{lemma}[1]
  {
    \begin{tcolorbox}[
      enhanced,
      boxrule=0pt,
      frame hidden,
      sharp corners,
      colback=lemmaBackground,
      borderline west={3pt}{-1.5pt}{lemmaAccent},
      breakable
    ]
    \ifx &#1& \color{lemmaAccent}\textbf{Lemma. }\color{black}
    \else \color{lemmaAccent}\textbf{Lemma #1. }\color{black} \fi
  }
  {
    \end{tcolorbox}
  }
\definecolor{definitionBackground}{RGB}{246,246,246}
\newenvironment{definition}[1]
  {
    \begin{tcolorbox}[
      enhanced,
      boxrule=0pt,
      frame hidden,
      sharp corners,
      colback=definitionBackground,
      borderline west={3pt}{-1.5pt}{black},
      breakable
    ]
    \textbf{Definition. }\emph{#1}\\
  }
  {
    \end{tcolorbox}
  }

\newenvironment{amatrix}[2]{
    \left[
      \begin{array}{*{#1}{c}|*{#2}c}
  }
  {
      \end{array}
    \right]
  }

\definecolor{codeBackground}{RGB}{253,246,227}
\renewcommand\theFancyVerbLine{\arabic{FancyVerbLine}}
\newtcblisting{cpp}[1][]{
  boxrule=1pt,
  sharp corners,
  colback=codeBackground,
  listing only,
  breakable,
  minted language=cpp,
  minted style=material,
  minted options={
    xleftmargin=15pt,
    baselinestretch=1.2,
    linenos,
  }
}

\author{Kyle Chui}


\fancyhf{}
\lhead{Kyle Chui}
\rhead{Page \thepage}
\pagestyle{fancy}

\begin{document}
  \section{Waves}
  Waves can be categorized into two different kinds: longitudinal and transverse. Solids are able to transmit both longitudinal and transverse waves, whereas fluids are only able to handle longitudinal waves.
  \begin{note}{}
    Ocean waves are surface waves and live at the interface between two different media. They are neither longitudinal nor transverse!
  \end{note}
  \subsection{Wave Velocity}
  A wave is an oscillation in both space and time. At a fixed position, the wave experienced is the same as a simple harmonic oscillator. Waves are described by the following equation:
  \[
    y = A\sin \Big(\underbrace{\frac{2\pi}{\lam}}_{k}x-\underbrace{\frac{2\pi}{T}}_\omega t\Big),
  \]
  where $k$ is the wave number and $\omega$ is the angular velocity. Furthermore, just like with SHOs, the energy scales with the square of the frequency and amplitude: $E\sim f^2A^2$. The wave travels one wavelength each period, so $v = \frac{\lam}{T}=f\cdot\lam = \frac{\omega}{k}$.
  \begin{note}{Two different velocities}
    There are \emph{two different velocities} in a wave. It has a phase velocity (how it moves from left to right), given by $v = f\lam = \frac{\omega}{k}$, and a transverse velocity (the medium oscillating up and down), given by $v = \omega A\sin(\omega t)$ or $v_{\text{max}} = \omega A$.
  \end{note}
  For transverse waves passing through a string/rope, we have
  \[
    v = \sqrt{\frac{F_{\text{tension}}}{\mu}}.
  \]
  \begin{note}{}
    When a wave travels from one medium to another, the frequency \emph{does not change}. The wavelength and velocity will change accordingly, but the frequency is determined by the source oscillator.
  \end{note}
  Assuming we have a wave $y(x, t) = A\cos(kx - \omega t)$, we can derive the following by taking partial derivatives and dividing.
  \[
    \frac{\partial^2y}{\partial x^2} = \frac{1}{v^2}\frac{\partial^2y}{\partial t^2}
  \]
  \subsection{Elasticity}
  The bulk modulus of a material is the inverse of its compressibilty. If something is easy to compress, it has a low bulk modulus. The equation is
  \[
    \frac{\Delta V}{V_0} = -\frac{1}{B}\Delta p.
  \]
  \subsection{Wave Velocity in Fluids}
  \begin{itemize}
    \item Let $v$ be the speed of a longitudinal wave in a fluid, $B$ the bulk modulus of the fluid, and $\rho$ the density of the fluid. Then we have
    \[
      v = \sqrt{\frac{B}{\rho}}. 
    \]
    \item Let $v$ be the speed of a longitudinal wave in a solid rod, $Y$ be the Young's modulus of the material, and $\rho$ the density of the rod. Then we have
    \[
      v=\sqrt{\frac{Y}{\rho}}
    \]
  \end{itemize}
  \begin{note}{}
    Both the bulk modulus and Young's modulus depend on the temperature of the material. However, most of it is due to the density of the material.
  \end{note}
  \begin{itemize}
    \item The speed of sound in gases is related to the average speed of particles in the gas:
    \[
      v_{\text{rms}} = \sqrt{\frac{3kT}{m}},
    \]
    where $k$ is the Boltzmann constant and $m$ is the molar mass of the gas.
    \item For sound in air at sea-level, the velocity of sound is
    \[
      v_{\text{sound}} = 331 \sqrt{\frac{T}{273}},
    \]
    where $T$ is in units of Kelvin.
  \end{itemize}
  \subsection{Standing Waves}
  A standing wave is created by two identical waves that are superimposed on one another, that are travelling in opposite directions. If the original wave is described by $y = A\sin (kx - \omega t)$, then the standing wave's equation is
  \[
    y = 2A\sin(kx)\cos(\omega t).
  \]
  \begin{note}{}
    We usually denote the amplitude of the standing wave by $A$, because we don't care about the original travelling waves.
  \end{note}
  The \emph{order} of a standing wave tells you the shape of the wave and gives you nice equations to find the wavelength and frequency. The equations are:
  \[
    \lam = \frac{2L}{n} \quad \text{and}\quad f = \frac{v}{2L}n,
  \]
  where $n$ is the order. Furthermore, the energy of the standing wave scales with the square of the order. \\[10pt]
  For a standing wave, the locations where the string remains at rest are called \emph{nodes}, and the locations where the string reaches the highest displacement are called \emph{antinodes}. \\[10pt]
  The wavelength of a standing wave depends on $L$ only: $\lam=2L$(if both ends are closed), $\lam=4L$(if one end is open).\\[10pt]
  Thus
  \[
    f=\frac{v}{\lam} \quad\text{where}\quad v=\sqrt{\frac{F_T}{\mu}},
  \]
  where $F_T$ is the tension in the string and $\mu$ is the density of the string.\\[10pt]
  The pitch of \emph{wind} instruments depends on the temperature of the air: As temperature goes down, so do the velocity of the sound and the frequency of the sound. However, temperature does not affect string instruments.
  \begin{note}{}
    All instruments are based on standing waves. The overtones of an instrument determine the sound characteristics of it (why two instruments playing the same note sound so different).
  \end{note}
  \subsection{Sound Intensity}
  Sound is emitted isotropically, that is to say, equally in all directions. Intensity is defined by 
  \[
    I = \frac{P}{A},
  \]
  and is measured in W/m\textsuperscript{2}. The power received by a sensor of area $A$ is $P_{\text{received}} = I\cdot A$. Additionally, we also have
  \[
    I = \frac{(p_{\text{max}})^2}{2\rho v_w},
  \]
  where $\Delta p$ is the pressure variation, $\rho$ is the density, and $v_w$ is the wave velocity. \\[10pt]
  Furthermore, we also have that
  \[
    p_{\text{max}} = BkA,
  \]
  where $B$ is the bulk modulus of the medium, $k$ is the wave number, and A is the displacement amplitude.\\[10pt]
  We use a logarithmic scale, called \emph{sound intensity level $\beta$} to quantify loudness, which is measured in Bell. 1dB = 1 decibel = 10 bel. To convert sound intensity level in dB to intensity in W/m\textsuperscript{2}, we have
  \[
    I = I_0\cdot 10^{\frac{\beta}{10}},
  \]
  where $I_0$ is the threshold for human hearing, about $10^{-12}$ W/m\textsuperscript{2}. The pain threshold for human hearing is 1W/m\textsuperscript{2}, or about 120dB.
  \subsection{Doppler Effect}
  The shift depends on the component of the source's velocity in your direction. \\[10pt]
  We have the following four cases and equations for finding the frequency heard by the observer, using the Doppler effect:
  \begin{enumerate}[label=(\alph*)]
    \item The source is coming towards you
    \[
      f'=\frac{f}{1-\cfrac{v_{\text{source}}}{v_{\text{sound}}}}
    \]
    \item The source is moving away from you
    \[
      f' = \frac{f}{1+\cfrac{v_{\text{source}}}{v_{\text{sound}}}}
    \]
    \item The observer is moving towards the source
    \[
      f' = f \paren{1+\frac{v_{\text{observer}}}{v_{\text{sound}}}}
    \]
    \item The observer is moving away from the source
    \[
      f' = f \paren{1 - \frac{v_{\text{observer}}}{v_{\text{sound}}}}
    \]
  \end{enumerate}
  When you have both objects moving, you calculate things sequentially by considering the frequency at a stationary point between the two objects.
  \subsection{Beats}
  Beats are created by the superposition of two waves of slightly different frequencies.\\[10pt]
  The equation of a superposition of two sound waves is 
  \[
    y = 2A \cdot \sin \paren{2\pi \frac{f_A - f_B}{2}t}\cos \paren{2\pi \frac{f_A + f_B}{2}t}.
  \]
  The frequency of the beat is $f_{\text{beat}} = f_1 - f_2$, because your ear cannot distinguish between the crest and the trough, so you hear it twice as often as you normally would. The frequency of the envelope is $\frac{1}{2}(f_1+f_2)$. \\[10pt]
  In a shock wave, the pressure and temperature go up, as the speed of the wave goes down.
\end{document}
