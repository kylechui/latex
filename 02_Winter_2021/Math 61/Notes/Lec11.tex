\documentclass[class=article, crop=false]{standalone}
% Import packages
\usepackage[margin=1in]{geometry}

\usepackage{amssymb, amsthm}
\usepackage{comment}
\usepackage{enumitem}
\usepackage{fancyhdr}
\usepackage{hyperref}
\usepackage{import}
\usepackage{mathrsfs, mathtools}
\usepackage{multicol}
\usepackage{pdfpages}
\usepackage{standalone}
\usepackage{transparent}
\usepackage{xcolor}

\usepackage{minted}
\usepackage[many]{tcolorbox}
\tcbuselibrary{minted}

% Declare math operators
\DeclareMathOperator{\lcm}{lcm}
\DeclareMathOperator{\proj}{proj}
\DeclareMathOperator{\vspan}{span}
\DeclareMathOperator{\im}{im}
\DeclareMathOperator{\Diff}{Diff}
\DeclareMathOperator{\Int}{Int}
\DeclareMathOperator{\fcn}{fcn}
\DeclareMathOperator{\id}{id}
\DeclareMathOperator{\rank}{rank}
\DeclareMathOperator{\tr}{tr}
\DeclareMathOperator{\dive}{div}
\DeclareMathOperator{\row}{row}
\DeclareMathOperator{\col}{col}
\DeclareMathOperator{\dom}{dom}
\DeclareMathOperator{\ran}{ran}
\DeclareMathOperator{\Dom}{Dom}
\DeclareMathOperator{\Ran}{Ran}
\DeclareMathOperator{\fl}{fl}
% Macros for letters/variables
\newcommand\ttilde{\kern -.15em\lower .7ex\hbox{\~{}}\kern .04em}
\newcommand{\N}{\ensuremath{\mathbb{N}}}
\newcommand{\Z}{\ensuremath{\mathbb{Z}}}
\newcommand{\Q}{\ensuremath{\mathbb{Q}}}
\newcommand{\R}{\ensuremath{\mathbb{R}}}
\newcommand{\C}{\ensuremath{\mathbb{C}}}
\newcommand{\F}{\ensuremath{\mathbb{F}}}
\newcommand{\M}{\ensuremath{\mathbb{M}}}
\renewcommand{\P}{\ensuremath{\mathbb{P}}}
\newcommand{\lam}{\ensuremath{\lambda}}
\newcommand{\nab}{\ensuremath{\nabla}}
\newcommand{\eps}{\ensuremath{\varepsilon}}
\newcommand{\es}{\ensuremath{\varnothing}}
% Macros for math symbols
\newcommand{\dx}[1]{\,\mathrm{d}#1}
\newcommand{\inv}{\ensuremath{^{-1}}}
\newcommand{\sm}{\setminus}
\newcommand{\sse}{\subseteq}
\newcommand{\ceq}{\coloneqq}
% Macros for pairs of math symbols
\newcommand{\abs}[1]{\ensuremath{\left\lvert #1 \right\rvert}}
\newcommand{\paren}[1]{\ensuremath{\left( #1 \right)}}
\newcommand{\norm}[1]{\ensuremath{\left\lVert #1\right\rVert}}
\newcommand{\set}[1]{\ensuremath{\left\{#1\right\}}}
\newcommand{\tup}[1]{\ensuremath{\left\langle #1 \right\rangle}}
\newcommand{\floor}[1]{\ensuremath{\left\lfloor #1 \right\rfloor}}
\newcommand{\ceil}[1]{\ensuremath{\left\lceil #1 \right\rceil}}
\newcommand{\eclass}[1]{\ensuremath{\left[ #1 \right]}}

\newcommand{\chapternum}{}
\newcommand{\ex}[1]{\noindent\textbf{Exercise \chapternum.{#1}.}}

\newcommand{\tsub}[1]{\textsubscript{#1}}
\newcommand{\tsup}[1]{\textsuperscript{#1}}

\renewcommand{\choose}[2]{\ensuremath{\phantom{}_{#1}C_{#2}}}
\newcommand{\perm}[2]{\ensuremath{\phantom{}_{#1}P_{#2}}}

% Include figures
\newcommand{\incfig}[2][1]{%
    \def\svgwidth{#1\columnwidth}
    \import{./figures/}{#2.pdf_tex}
}

\definecolor{problemBackground}{RGB}{212,232,246}
\newenvironment{problem}[1]
  {
    \begin{tcolorbox}[
      boxrule=.5pt,
      titlerule=.5pt,
      sharp corners,
      colback=problemBackground,
      breakable
    ]
    \ifx &#1& \textbf{Problem. }
    \else \textbf{Problem #1.} \fi
  }
  {
    \end{tcolorbox}
  }
\definecolor{exampleBackground}{RGB}{255,249,248}
\definecolor{exampleAccent}{RGB}{158,60,14}
\newenvironment{example}[1]
  {
    \begin{tcolorbox}[
      boxrule=.5pt,
      sharp corners,
      colback=exampleBackground,
      colframe=exampleAccent,
    ]
    \color{exampleAccent}\textbf{Example.} \emph{#1}\color{black}
  }
  {
    \end{tcolorbox}
  }
\definecolor{theoremBackground}{RGB}{234,243,251}
\definecolor{theoremAccent}{RGB}{0,116,183}
\newenvironment{theorem}[1]
  {
    \begin{tcolorbox}[
      boxrule=.5pt,
      titlerule=.5pt,
      sharp corners,
      colback=theoremBackground,
      colframe=theoremAccent,
      breakable
    ]
      % \color{theoremAccent}\textbf{Theorem --- }\emph{#1}\\\color{black}
    \ifx &#1& \color{theoremAccent}\textbf{Theorem. }\color{black}
    \else \color{theoremAccent}\textbf{Theorem --- }\emph{#1}\\\color{black} \fi
  }
  {
    \end{tcolorbox}
  }
\definecolor{noteBackground}{RGB}{244,249,244}
\definecolor{noteAccent}{RGB}{34,139,34}
\newenvironment{note}[1]
  {
  \begin{tcolorbox}[
    enhanced,
    boxrule=0pt,
    frame hidden,
    sharp corners,
    colback=noteBackground,
    borderline west={3pt}{-1.5pt}{noteAccent},
    breakable
    ]
    \ifx &#1& \color{noteAccent}\textbf{Note. }\color{black}
    \else \color{noteAccent}\textbf{Note (#1). }\color{black} \fi
    }
    {
  \end{tcolorbox}
  }
\definecolor{lemmaBackground}{RGB}{255,247,234}
\definecolor{lemmaAccent}{RGB}{255,153,0}
\newenvironment{lemma}[1]
  {
    \begin{tcolorbox}[
      enhanced,
      boxrule=0pt,
      frame hidden,
      sharp corners,
      colback=lemmaBackground,
      borderline west={3pt}{-1.5pt}{lemmaAccent},
      breakable
    ]
    \ifx &#1& \color{lemmaAccent}\textbf{Lemma. }\color{black}
    \else \color{lemmaAccent}\textbf{Lemma #1. }\color{black} \fi
  }
  {
    \end{tcolorbox}
  }
\definecolor{definitionBackground}{RGB}{246,246,246}
\newenvironment{definition}[1]
  {
    \begin{tcolorbox}[
      enhanced,
      boxrule=0pt,
      frame hidden,
      sharp corners,
      colback=definitionBackground,
      borderline west={3pt}{-1.5pt}{black},
      breakable
    ]
    \textbf{Definition. }\emph{#1}\\
  }
  {
    \end{tcolorbox}
  }

\newenvironment{amatrix}[2]{
    \left[
      \begin{array}{*{#1}{c}|*{#2}c}
  }
  {
      \end{array}
    \right]
  }

\definecolor{codeBackground}{RGB}{253,246,227}
\renewcommand\theFancyVerbLine{\arabic{FancyVerbLine}}
\newtcblisting{cpp}[1][]{
  boxrule=1pt,
  sharp corners,
  colback=codeBackground,
  listing only,
  breakable,
  minted language=cpp,
  minted style=material,
  minted options={
    xleftmargin=15pt,
    baselinestretch=1.2,
    linenos,
  }
}

\author{Kyle Chui}


\fancyhf{}
\lhead{Kyle Chui}
\rhead{Page \thepage}
\pagestyle{fancy}

\begin{document}
  \textbf{Warm Up.} Find the number of solutions to $x_1+x_2+x_3+x_4=20$, where $x_i\in\N$.\\[10pt]
  We have $20$ ``stars" and $3$ bars, so the number of possible solutions to this is $\binom{23}{3}$.\\[10pt]
  What if $x_4\geq 5$? \\[10pt]
  We will first allocate $5$ stars to the compartment for $x_4$, so this is equivalent to finding the number of solutions to the equation $x_1+x_2+x_3+x_4=15$. There are $\binom{18}{3}$ such solutions. \\[10pt]
  What if $x_4\leq 4$? (and all $x_i\in\N$) \\[10pt]
  We can do complementary counting--we subtract the solutions to $x_1+x_2+x_3+x_4=20$ with $x_4\geq 5$ from the solutions to $x_1+x_2+x_3+x_4=20$.
  Thus we have $\binom{23}{3}-\binom{18}{3}$. \\
  \emph{or} we may count the cases where $x_4=0, 1, 2, 3, 4$. In other words, adding the number of solutions to the equations $x_1+x_2+x_3=20,19,18,17,16$. Thus we have
  \[
    \binom{22}{2}+\binom{21}{2}+\binom{20}{2}+\binom{19}{2}+\binom{18}{2}=\binom{23}{3}+\binom{18}{3}.
  \]
  \subsection{Combinatorial Identities}
  \begin{example}{}
    Let's expand $(x+y)^5 = (x+y)(x+y)(x+y)(x+y)(x+y)$. \par
    When creating a term in their product, observe that we either choose an $x$ term or a $y$ term from each $x+y$. Thus every term is a sum of monomials of the form $x^iy^j$ where $i+j=5$. There are $2^5$ terms before gathering like terms. We ask ourselves, what is the coefficient on an arbitrary monomial $x^iy^j$? In other words, how many ways are there to choose an $i$ number of $x$ terms and a $j$ number of $y$ terms from the $i+j$ total terms? There are $(i+j)!$ ways to order all $i+j$ terms, but we have $i$ identical $x$ terms and $j$ identical $y$ terms. Thus we must divide by $i!$ and $j!$ to account for the redundancies. Thus the coefficient on the $x^iy^j$ term is $\frac{(i+j)!}{i!j!}=\binom{i+j}{i}=\binom{i+j}{j}$. 
  \end{example}
  \begin{theorem}{Binomial Theorem}
    For $n\in\N$, we have
    \[
      (a+b)^n = \sum_{k=0}^n \binom{n}{k} a^{n-k}b^k.
    \]
    The proof uses the same argument as the above.
  \end{theorem}
  \begin{example}{}
    \[
      2^n = (1+1)^2 = \sum_{k=0}^n \binom{n}{k}1^{n-k}1^k = \sum_{k=0}^n \binom{n}{k}
    \]
    Additionally, observe that $2^n$ is the total number of subsets of a set $X$ with cardinality $n$. Furthermore, the right hand side is the sum of the number of subsets with cardinality $k$ of $X$, for all $0\leq k\leq n$. Thus they are equal.
  \end{example}
  \begin{example}{}
    \[
      3^n = (2+1)^n = \sum_{k=0}^n\binom{n}{k} 2^k 1^{n-k}  = \sum_{k=0}^n \binom{n}{k}2^k 
    \]
    \textbf{Exercise.} Try to find a counting argument to prove this identity. Hint: If $\abs{X}=n$, there are $3^n$ pairs of subsets of $X$ of the form $(A, B)$ where $A\sse B$.
  \end{example}
  \subsection{Pascal's Triangle}
  \begin{center}
    1\\
    1\quad 1\\
    1\quad 2\quad 1\\
    1\quad 3\quad 3\quad 1\\
    1\quad 4\quad 6\quad 4\quad 1\\
    1\quad 5\quad 10\quad 10\quad 5\quad 1
  \end{center}
  Each term is the sum of the two terms above it, and the outermost diagonals are all $1$'s.
  \begin{note}{Pascal's Triangle Observations}
    \begin{itemize}
      \item It is symmetrical about the vertical axis
      \item $\binom{n}{1}=\binom{n}{n-1}=n$
      \item $\binom{n}{k}=\binom{n}{n-k}$
      \item $\binom{n}{i} + \binom{n}{i-1} = \binom{n+1}{i}$
      \item The sum of the $n$th row is $2^{n-1}$
      \item $\sum_{i-k}^n \binom{i}{k} = \binom{n+1}{k+1}$
    \end{itemize}
  \end{note}
  \textbf{Claim.} $\binom{n+1}{i}=\binom{n}{i}+\binom{n}{i-1}$.
  \begin{proof}
    Let $X$ be a set with $n+1$ elements. The left hand side counts the number of ways to choose an $i$-element subset from $X$. Observe that $\binom{n}{i}$ counts the number of $i$-element subsets that don't contain a particular element $x_1$ of $X$, while $\binom{n-1}{i}$ counts the number of $i$-element subsets that do contain $x_1$. Thus these two sets form a partition for the subsets of $X$ with $i$ elements, so $\binom{n+1}{i}=\binom{n}{i}+\binom{n}{i-1}$.
  \end{proof}
  \begin{note}{}
    The above proof may be thought of more concretely by letting $X$ be the set $\set{0, 1, \dotsc, n}$, and choosing subsets of $X$ with and without $0$.
  \end{note}
\end{document}
