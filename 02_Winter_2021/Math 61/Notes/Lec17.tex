\documentclass[class=article, crop=false]{standalone}
% Import packages
\usepackage[margin=1in]{geometry}

\usepackage{amssymb, amsthm}
\usepackage{comment}
\usepackage{enumitem}
\usepackage{fancyhdr}
\usepackage{hyperref}
\usepackage{import}
\usepackage{mathrsfs, mathtools}
\usepackage{multicol}
\usepackage{pdfpages}
\usepackage{standalone}
\usepackage{transparent}
\usepackage{xcolor}

\usepackage{minted}
\usepackage[many]{tcolorbox}
\tcbuselibrary{minted}

% Declare math operators
\DeclareMathOperator{\lcm}{lcm}
\DeclareMathOperator{\proj}{proj}
\DeclareMathOperator{\vspan}{span}
\DeclareMathOperator{\im}{im}
\DeclareMathOperator{\Diff}{Diff}
\DeclareMathOperator{\Int}{Int}
\DeclareMathOperator{\fcn}{fcn}
\DeclareMathOperator{\id}{id}
\DeclareMathOperator{\rank}{rank}
\DeclareMathOperator{\tr}{tr}
\DeclareMathOperator{\dive}{div}
\DeclareMathOperator{\row}{row}
\DeclareMathOperator{\col}{col}
\DeclareMathOperator{\dom}{dom}
\DeclareMathOperator{\ran}{ran}
\DeclareMathOperator{\Dom}{Dom}
\DeclareMathOperator{\Ran}{Ran}
\DeclareMathOperator{\fl}{fl}
% Macros for letters/variables
\newcommand\ttilde{\kern -.15em\lower .7ex\hbox{\~{}}\kern .04em}
\newcommand{\N}{\ensuremath{\mathbb{N}}}
\newcommand{\Z}{\ensuremath{\mathbb{Z}}}
\newcommand{\Q}{\ensuremath{\mathbb{Q}}}
\newcommand{\R}{\ensuremath{\mathbb{R}}}
\newcommand{\C}{\ensuremath{\mathbb{C}}}
\newcommand{\F}{\ensuremath{\mathbb{F}}}
\newcommand{\M}{\ensuremath{\mathbb{M}}}
\renewcommand{\P}{\ensuremath{\mathbb{P}}}
\newcommand{\lam}{\ensuremath{\lambda}}
\newcommand{\nab}{\ensuremath{\nabla}}
\newcommand{\eps}{\ensuremath{\varepsilon}}
\newcommand{\es}{\ensuremath{\varnothing}}
% Macros for math symbols
\newcommand{\dx}[1]{\,\mathrm{d}#1}
\newcommand{\inv}{\ensuremath{^{-1}}}
\newcommand{\sm}{\setminus}
\newcommand{\sse}{\subseteq}
\newcommand{\ceq}{\coloneqq}
% Macros for pairs of math symbols
\newcommand{\abs}[1]{\ensuremath{\left\lvert #1 \right\rvert}}
\newcommand{\paren}[1]{\ensuremath{\left( #1 \right)}}
\newcommand{\norm}[1]{\ensuremath{\left\lVert #1\right\rVert}}
\newcommand{\set}[1]{\ensuremath{\left\{#1\right\}}}
\newcommand{\tup}[1]{\ensuremath{\left\langle #1 \right\rangle}}
\newcommand{\floor}[1]{\ensuremath{\left\lfloor #1 \right\rfloor}}
\newcommand{\ceil}[1]{\ensuremath{\left\lceil #1 \right\rceil}}
\newcommand{\eclass}[1]{\ensuremath{\left[ #1 \right]}}

\newcommand{\chapternum}{}
\newcommand{\ex}[1]{\noindent\textbf{Exercise \chapternum.{#1}.}}

\newcommand{\tsub}[1]{\textsubscript{#1}}
\newcommand{\tsup}[1]{\textsuperscript{#1}}

\renewcommand{\choose}[2]{\ensuremath{\phantom{}_{#1}C_{#2}}}
\newcommand{\perm}[2]{\ensuremath{\phantom{}_{#1}P_{#2}}}

% Include figures
\newcommand{\incfig}[2][1]{%
    \def\svgwidth{#1\columnwidth}
    \import{./figures/}{#2.pdf_tex}
}

\definecolor{problemBackground}{RGB}{212,232,246}
\newenvironment{problem}[1]
  {
    \begin{tcolorbox}[
      boxrule=.5pt,
      titlerule=.5pt,
      sharp corners,
      colback=problemBackground,
      breakable
    ]
    \ifx &#1& \textbf{Problem. }
    \else \textbf{Problem #1.} \fi
  }
  {
    \end{tcolorbox}
  }
\definecolor{exampleBackground}{RGB}{255,249,248}
\definecolor{exampleAccent}{RGB}{158,60,14}
\newenvironment{example}[1]
  {
    \begin{tcolorbox}[
      boxrule=.5pt,
      sharp corners,
      colback=exampleBackground,
      colframe=exampleAccent,
    ]
    \color{exampleAccent}\textbf{Example.} \emph{#1}\color{black}
  }
  {
    \end{tcolorbox}
  }
\definecolor{theoremBackground}{RGB}{234,243,251}
\definecolor{theoremAccent}{RGB}{0,116,183}
\newenvironment{theorem}[1]
  {
    \begin{tcolorbox}[
      boxrule=.5pt,
      titlerule=.5pt,
      sharp corners,
      colback=theoremBackground,
      colframe=theoremAccent,
      breakable
    ]
      % \color{theoremAccent}\textbf{Theorem --- }\emph{#1}\\\color{black}
    \ifx &#1& \color{theoremAccent}\textbf{Theorem. }\color{black}
    \else \color{theoremAccent}\textbf{Theorem --- }\emph{#1}\\\color{black} \fi
  }
  {
    \end{tcolorbox}
  }
\definecolor{noteBackground}{RGB}{244,249,244}
\definecolor{noteAccent}{RGB}{34,139,34}
\newenvironment{note}[1]
  {
  \begin{tcolorbox}[
    enhanced,
    boxrule=0pt,
    frame hidden,
    sharp corners,
    colback=noteBackground,
    borderline west={3pt}{-1.5pt}{noteAccent},
    breakable
    ]
    \ifx &#1& \color{noteAccent}\textbf{Note. }\color{black}
    \else \color{noteAccent}\textbf{Note (#1). }\color{black} \fi
    }
    {
  \end{tcolorbox}
  }
\definecolor{lemmaBackground}{RGB}{255,247,234}
\definecolor{lemmaAccent}{RGB}{255,153,0}
\newenvironment{lemma}[1]
  {
    \begin{tcolorbox}[
      enhanced,
      boxrule=0pt,
      frame hidden,
      sharp corners,
      colback=lemmaBackground,
      borderline west={3pt}{-1.5pt}{lemmaAccent},
      breakable
    ]
    \ifx &#1& \color{lemmaAccent}\textbf{Lemma. }\color{black}
    \else \color{lemmaAccent}\textbf{Lemma #1. }\color{black} \fi
  }
  {
    \end{tcolorbox}
  }
\definecolor{definitionBackground}{RGB}{246,246,246}
\newenvironment{definition}[1]
  {
    \begin{tcolorbox}[
      enhanced,
      boxrule=0pt,
      frame hidden,
      sharp corners,
      colback=definitionBackground,
      borderline west={3pt}{-1.5pt}{black},
      breakable
    ]
    \textbf{Definition. }\emph{#1}\\
  }
  {
    \end{tcolorbox}
  }

\newenvironment{amatrix}[2]{
    \left[
      \begin{array}{*{#1}{c}|*{#2}c}
  }
  {
      \end{array}
    \right]
  }

\definecolor{codeBackground}{RGB}{253,246,227}
\renewcommand\theFancyVerbLine{\arabic{FancyVerbLine}}
\newtcblisting{cpp}[1][]{
  boxrule=1pt,
  sharp corners,
  colback=codeBackground,
  listing only,
  breakable,
  minted language=cpp,
  minted style=material,
  minted options={
    xleftmargin=15pt,
    baselinestretch=1.2,
    linenos,
  }
}

\author{Kyle Chui}


\fancyhf{}
\lhead{Kyle Chui}
\rhead{Page \thepage}
\pagestyle{fancy}

\begin{document}
  \subsection{Euler Cycles}
  \begin{definition}{Euler Cycle}
    An \emph{Euler cycle} in a graph $G$ is a cycle including each edge in the graph (without repeating edges). Also, we'll say that the trivial path $v$ in a graph with one vertex $v$ and no edges is an Euler cycle.
  \end{definition}
  \textbf{Proposition.} If a graph $G$ has an Euler cycle, all vertices of $G$ must have even degree. \par
  \begin{proof}
    Observe that to pass through a vertex you must have an edge ``entering" it, and an edge ``leaving it", so each path through a vertex contributes degree $2$ to the degree of the vertex. Thus they must be even.
  \end{proof}
  However, consider the graph with six vertices composed of two triangles. It does not have an Euler cycle because it is disconnected. Thus our proposition is a necessary but not sufficient condition for an Euler cycle.
  \begin{theorem}{Euler Cycles}
    Let $G$ be a connected graph, then $G$ has an Euler cycle if and only if all vertices of $G$ have even degree.
  \end{theorem}
  \begin{proof}
    We have already shown that if $G$ has an Euler cycle, that all vertices have even degree. We must now show that if all vertices have even degree, then $G$ has an Euler cycle. We can check that this is true for any graph of $1$ or two vertices. If $G$ has three or more vertices, then somewhere in $G$, the following configuration exists.
    \begin{center}\begin{tikzpicture}
      \draw (0, 0) circle (10pt) node (v1) {$v_1$};
      \draw (1.5in, -1in) circle (10pt) node (v2) {$v_2$};
      \draw (2in, 0) circle (10pt) node (v3) {$v_3$};
      \draw (v1) -- (v2) -- (v3);
    \end{tikzpicture}\end{center}
    Assuming the original graph has all vertices with even degree, we know that if we remove edges $v_1v_2$ and $v_2v_3$ and connect $v_1v_3$, all vertices still have even degree.
    \begin{center}\begin{tikzpicture}
      \draw (0, 0) circle (10pt) node (v1) {$v_1$};
      \draw (1.5in, -1in) circle (10pt) node (v2) {$v_2$};
      \draw (2in, 0) circle (10pt) node (v3) {$v_3$};
      \draw[dashed] (v1) -- (v2) -- (v3);
      \draw (v1) -- (v3);
    \end{tikzpicture}\end{center}
    Let's call $G'$ the graph with the new configuration. Because $G$ was connected, we know that $G'$ is either connected or has two comopnents. If $G'$ is connected, then we just take the Euler cycle of $G'$ until we get to $v_1$, and then visit $v_2$ instead of directly going to $v_3$. \par
    We now consider the case where $G'$ is not connected. We have two components for $G'$, and we can construct an Euler cycle through all of them by taking the Euler cycles through the subcomponents and connecting them using the edges we removed earlier.
  \end{proof}
  \subsection{Shortest Paths}
  \textbf{Question.} Given a weighted graph, how can we find the shortest path/the length of a shortest path between two vertices?
  \begin{theorem}{Dijkstra's Algorithm}
    Given a weighted graph $G$, a start vertex $A$ and an end vertex $B$, we use the following process:
    \begin{enumerate}[label=\arabic*)]
      \item Label vertex $A$ with zero, and all other vertices with $\infty$
      \item While vertex $B$ is uncircled: \\
      Circle the vertex with the smallest label, out of the uncircled vertices (call this vertex $v$) \\
      Update the label of all uncircled vertices $w$ by
      \[
        w \text{'s new label} = \begin{cases}v \text{'s label + weight of edge from $v$ to $w$} \\ w \text{'s current label}\end{cases}
      \]
      We claim that at each step, the label of every circled vertex is the length of a shortest path from $A$ to that vertex.
    \end{enumerate} 
  \end{theorem}
  \begin{proof}
    We proceed by induction on the number of labels circled. For our base case, consider when only $A$ is circled. Suppose that the claim is true when $k$ vertices have been circled. \par 
    Suppose towards a contradiction that vertex $v$ has the smallest label $\ell$ (so $v$ is about to get circled), but there is a path from $A$ to $v$ of length less than $\ell$. Suppose the path takes the form
    \[
      Ae_0v_1e_2\ldots v_me_mv.
    \]
    Consider the first vertex in this path that \emph{hasn't} been circled, and call it $x$ ($x$ could be equal to $v$). Let us denote the vertex before $x$ by $y$. The part of the path from $A$ to $y$ has length greater than or equal to $\ell(y)$. Note that $x$'s label is less than or equal to $\ell(y) + w(x, y)$, where $w(x, y)$ is the weight of the edge between $x$ and $y$. Thus we have what $\ell(y) + w(x, y) < \ell$, so 
    \[
      x \text{'s label}\leq \ell(y) + w(x, y) < \ell = v \text{'s label}.
    \]
    Therefore $x$'s label is less than $v$'s label, contradicting our assumption that $v$ has the smallest label.
  \end{proof}
\end{document}
