\documentclass[class=article, crop=false]{standalone}
% Import packages
\usepackage[margin=1in]{geometry}

\usepackage{amssymb, amsthm}
\usepackage{comment}
\usepackage{enumitem}
\usepackage{fancyhdr}
\usepackage{hyperref}
\usepackage{import}
\usepackage{mathrsfs, mathtools}
\usepackage{multicol}
\usepackage{pdfpages}
\usepackage{standalone}
\usepackage{transparent}
\usepackage{xcolor}

\usepackage{minted}
\usepackage[many]{tcolorbox}
\tcbuselibrary{minted}

% Declare math operators
\DeclareMathOperator{\lcm}{lcm}
\DeclareMathOperator{\proj}{proj}
\DeclareMathOperator{\vspan}{span}
\DeclareMathOperator{\im}{im}
\DeclareMathOperator{\Diff}{Diff}
\DeclareMathOperator{\Int}{Int}
\DeclareMathOperator{\fcn}{fcn}
\DeclareMathOperator{\id}{id}
\DeclareMathOperator{\rank}{rank}
\DeclareMathOperator{\tr}{tr}
\DeclareMathOperator{\dive}{div}
\DeclareMathOperator{\row}{row}
\DeclareMathOperator{\col}{col}
\DeclareMathOperator{\dom}{dom}
\DeclareMathOperator{\ran}{ran}
\DeclareMathOperator{\Dom}{Dom}
\DeclareMathOperator{\Ran}{Ran}
\DeclareMathOperator{\fl}{fl}
% Macros for letters/variables
\newcommand\ttilde{\kern -.15em\lower .7ex\hbox{\~{}}\kern .04em}
\newcommand{\N}{\ensuremath{\mathbb{N}}}
\newcommand{\Z}{\ensuremath{\mathbb{Z}}}
\newcommand{\Q}{\ensuremath{\mathbb{Q}}}
\newcommand{\R}{\ensuremath{\mathbb{R}}}
\newcommand{\C}{\ensuremath{\mathbb{C}}}
\newcommand{\F}{\ensuremath{\mathbb{F}}}
\newcommand{\M}{\ensuremath{\mathbb{M}}}
\renewcommand{\P}{\ensuremath{\mathbb{P}}}
\newcommand{\lam}{\ensuremath{\lambda}}
\newcommand{\nab}{\ensuremath{\nabla}}
\newcommand{\eps}{\ensuremath{\varepsilon}}
\newcommand{\es}{\ensuremath{\varnothing}}
% Macros for math symbols
\newcommand{\dx}[1]{\,\mathrm{d}#1}
\newcommand{\inv}{\ensuremath{^{-1}}}
\newcommand{\sm}{\setminus}
\newcommand{\sse}{\subseteq}
\newcommand{\ceq}{\coloneqq}
% Macros for pairs of math symbols
\newcommand{\abs}[1]{\ensuremath{\left\lvert #1 \right\rvert}}
\newcommand{\paren}[1]{\ensuremath{\left( #1 \right)}}
\newcommand{\norm}[1]{\ensuremath{\left\lVert #1\right\rVert}}
\newcommand{\set}[1]{\ensuremath{\left\{#1\right\}}}
\newcommand{\tup}[1]{\ensuremath{\left\langle #1 \right\rangle}}
\newcommand{\floor}[1]{\ensuremath{\left\lfloor #1 \right\rfloor}}
\newcommand{\ceil}[1]{\ensuremath{\left\lceil #1 \right\rceil}}
\newcommand{\eclass}[1]{\ensuremath{\left[ #1 \right]}}

\newcommand{\chapternum}{}
\newcommand{\ex}[1]{\noindent\textbf{Exercise \chapternum.{#1}.}}

\newcommand{\tsub}[1]{\textsubscript{#1}}
\newcommand{\tsup}[1]{\textsuperscript{#1}}

\renewcommand{\choose}[2]{\ensuremath{\phantom{}_{#1}C_{#2}}}
\newcommand{\perm}[2]{\ensuremath{\phantom{}_{#1}P_{#2}}}

% Include figures
\newcommand{\incfig}[2][1]{%
    \def\svgwidth{#1\columnwidth}
    \import{./figures/}{#2.pdf_tex}
}

\definecolor{problemBackground}{RGB}{212,232,246}
\newenvironment{problem}[1]
  {
    \begin{tcolorbox}[
      boxrule=.5pt,
      titlerule=.5pt,
      sharp corners,
      colback=problemBackground,
      breakable
    ]
    \ifx &#1& \textbf{Problem. }
    \else \textbf{Problem #1.} \fi
  }
  {
    \end{tcolorbox}
  }
\definecolor{exampleBackground}{RGB}{255,249,248}
\definecolor{exampleAccent}{RGB}{158,60,14}
\newenvironment{example}[1]
  {
    \begin{tcolorbox}[
      boxrule=.5pt,
      sharp corners,
      colback=exampleBackground,
      colframe=exampleAccent,
    ]
    \color{exampleAccent}\textbf{Example.} \emph{#1}\color{black}
  }
  {
    \end{tcolorbox}
  }
\definecolor{theoremBackground}{RGB}{234,243,251}
\definecolor{theoremAccent}{RGB}{0,116,183}
\newenvironment{theorem}[1]
  {
    \begin{tcolorbox}[
      boxrule=.5pt,
      titlerule=.5pt,
      sharp corners,
      colback=theoremBackground,
      colframe=theoremAccent,
      breakable
    ]
      % \color{theoremAccent}\textbf{Theorem --- }\emph{#1}\\\color{black}
    \ifx &#1& \color{theoremAccent}\textbf{Theorem. }\color{black}
    \else \color{theoremAccent}\textbf{Theorem --- }\emph{#1}\\\color{black} \fi
  }
  {
    \end{tcolorbox}
  }
\definecolor{noteBackground}{RGB}{244,249,244}
\definecolor{noteAccent}{RGB}{34,139,34}
\newenvironment{note}[1]
  {
  \begin{tcolorbox}[
    enhanced,
    boxrule=0pt,
    frame hidden,
    sharp corners,
    colback=noteBackground,
    borderline west={3pt}{-1.5pt}{noteAccent},
    breakable
    ]
    \ifx &#1& \color{noteAccent}\textbf{Note. }\color{black}
    \else \color{noteAccent}\textbf{Note (#1). }\color{black} \fi
    }
    {
  \end{tcolorbox}
  }
\definecolor{lemmaBackground}{RGB}{255,247,234}
\definecolor{lemmaAccent}{RGB}{255,153,0}
\newenvironment{lemma}[1]
  {
    \begin{tcolorbox}[
      enhanced,
      boxrule=0pt,
      frame hidden,
      sharp corners,
      colback=lemmaBackground,
      borderline west={3pt}{-1.5pt}{lemmaAccent},
      breakable
    ]
    \ifx &#1& \color{lemmaAccent}\textbf{Lemma. }\color{black}
    \else \color{lemmaAccent}\textbf{Lemma #1. }\color{black} \fi
  }
  {
    \end{tcolorbox}
  }
\definecolor{definitionBackground}{RGB}{246,246,246}
\newenvironment{definition}[1]
  {
    \begin{tcolorbox}[
      enhanced,
      boxrule=0pt,
      frame hidden,
      sharp corners,
      colback=definitionBackground,
      borderline west={3pt}{-1.5pt}{black},
      breakable
    ]
    \textbf{Definition. }\emph{#1}\\
  }
  {
    \end{tcolorbox}
  }

\newenvironment{amatrix}[2]{
    \left[
      \begin{array}{*{#1}{c}|*{#2}c}
  }
  {
      \end{array}
    \right]
  }

\definecolor{codeBackground}{RGB}{253,246,227}
\renewcommand\theFancyVerbLine{\arabic{FancyVerbLine}}
\newtcblisting{cpp}[1][]{
  boxrule=1pt,
  sharp corners,
  colback=codeBackground,
  listing only,
  breakable,
  minted language=cpp,
  minted style=material,
  minted options={
    xleftmargin=15pt,
    baselinestretch=1.2,
    linenos,
  }
}

\author{Kyle Chui}


\fancyhf{}
\lhead{Kyle Chui}
\rhead{Page \thepage}
\pagestyle{fancy}

\begin{document}
  \begin{note}{}
    For the theorem above, we need to verify:
    \begin{itemize}
      \item $F\inv(\mathcal{S})$ is an equivalence relation,
      \item $F\circ F\inv(\mathcal{S}) = \mathcal{S}$ for all partitions $\mathcal{S}$,
      \item $F\inv\circ F(R) = R$ for all equivalence relations $R$.
    \end{itemize}
  \end{note}
  \begin{example}{Equivalence relations $\iff$ partitions}\\
    Let
    \begin{align*}
      X &= \set{1, 2, 3, 4, 5} \\
      R &= \set{(1, 1), (2, 2), (3, 3), (4, 4), (5, 5), (1, 5), (5, 1), (1, 3), (3, 1), (3, 5), (5, 3), (2, 4), (4, 2)}.
    \end{align*}
    What is the corresponding partition, i.e. what is $F(R)$? \par
    The corresponding partition to the aforementioned relation $R$ is $\set{\set{1,3,5}, \set{2,4}}$.\\[10pt]
    Let $\mathcal{S} = \set{\set{1, 2, 3}, \set{4, 5}}$ what is the corresponding equivalence relation? i.e. what is $F\inv(\mathcal{S})$?\par
    The corresponding equivalence relation for $S$ is
    \[
      \set{(1, 1), (2, 2), (3, 3), (4, 4), (5, 5), (1, 2), (2, 1), (1, 3), (3, 1), (2, 3), (3, 2), (4, 5), (5, 4)}.
    \]
  \end{example}
  \section{Counting}
  \textbf{Question.} If $\abs{X} = n$, and $\abs{Y} = m$, how many functions are there from $X$ to $Y$? \par
  Every element in the domain must get mapped to something in the codomain, but it does not matter \emph{which} element in the codomain. Furthermore, the function maps each element in the domain to \emph{exactly one} element in the codomain. Observe that for each $x\in X$, there are $m$ ``choices" for where $x$ gets mapped. Thus there are $\underbrace{m \cdot m \dotsm m}_{n \text{ times}} = m^n$ total functions.
  \begin{theorem}{Multiplication principle}
    If a set can be enumerated/constructed in $t$ steps (where each step is independent of the other steps) and each step has $n_i$ choices/outcomes, then the set has $n_1n_2\dotsm n_t$ elements.
  \end{theorem}
  \begin{example}{}
    I am going to get a pizza from Vito's or D'more's:
    \begin{center}\begin{tabular}{cc}
      Vito's: & D'more's: \\
      2 crusts & 1 crust \\
      6 toppings & 2 sauces \\
      2 cheeses & 3 toppings
    \end{tabular}\end{center}
    The total number of pizzas I could order is: $2 \cdot 6 \cdot 2 + 1 \cdot 2 \cdot 3 = 30$.
  \end{example}
  \begin{theorem}{Addition principle}
    If $X$ and $Y$ are disjoint finite sets, then $\abs{X\cup Y} = \abs{X} + \abs{Y}$.
  \end{theorem}
  \begin{example}{}
    How many strings of length $5$ in $\set{0, 1}$ start with $10$ or end with $01$?\\[10pt]
    By the multiplication principle, we know there are $2^3$ strings that start with $10$. By similar reasoning, there are $2^3$ strings that end with $01$. Furthermore, there are $2$ strings that satisfy both of these properties. Thus the total number of strings satisfying the statement above is $2^3 + 2^3 - 2 = 16$.
  \end{example}
  \begin{theorem}{Inclusion/Exclusion principle}
    If $X, Y$ are finite sets, then
    \[
      \abs{X\cup Y} = \abs{X} + \abs{Y} - \abs{X\cap Y}.
    \]
  \end{theorem}
  \begin{note}{}
    The addition principle is just a special case of the inclusion/exclusion principle where $X\cap Y = \es$.
  \end{note}
\end{document}
