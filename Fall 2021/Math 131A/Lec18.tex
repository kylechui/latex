\documentclass[class=article, crop=false]{standalone}
% Import packages
\usepackage[margin=1in]{geometry}

\usepackage{amssymb, amsthm}
\usepackage{comment}
\usepackage{enumitem}
\usepackage{fancyhdr}
\usepackage{hyperref}
\usepackage{import}
\usepackage{mathrsfs, mathtools}
\usepackage{multicol}
\usepackage{pdfpages}
\usepackage{standalone}
\usepackage{transparent}
\usepackage{xcolor}

\usepackage{minted}
\usepackage[many]{tcolorbox}
\tcbuselibrary{minted}

% Declare math operators
\DeclareMathOperator{\lcm}{lcm}
\DeclareMathOperator{\proj}{proj}
\DeclareMathOperator{\vspan}{span}
\DeclareMathOperator{\im}{im}
\DeclareMathOperator{\Diff}{Diff}
\DeclareMathOperator{\Int}{Int}
\DeclareMathOperator{\fcn}{fcn}
\DeclareMathOperator{\id}{id}
\DeclareMathOperator{\rank}{rank}
\DeclareMathOperator{\tr}{tr}
\DeclareMathOperator{\dive}{div}
\DeclareMathOperator{\row}{row}
\DeclareMathOperator{\col}{col}
\DeclareMathOperator{\dom}{dom}
\DeclareMathOperator{\ran}{ran}
\DeclareMathOperator{\Dom}{Dom}
\DeclareMathOperator{\Ran}{Ran}
\DeclareMathOperator{\fl}{fl}
% Macros for letters/variables
\newcommand\ttilde{\kern -.15em\lower .7ex\hbox{\~{}}\kern .04em}
\newcommand{\N}{\ensuremath{\mathbb{N}}}
\newcommand{\Z}{\ensuremath{\mathbb{Z}}}
\newcommand{\Q}{\ensuremath{\mathbb{Q}}}
\newcommand{\R}{\ensuremath{\mathbb{R}}}
\newcommand{\C}{\ensuremath{\mathbb{C}}}
\newcommand{\F}{\ensuremath{\mathbb{F}}}
\newcommand{\M}{\ensuremath{\mathbb{M}}}
\renewcommand{\P}{\ensuremath{\mathbb{P}}}
\newcommand{\lam}{\ensuremath{\lambda}}
\newcommand{\nab}{\ensuremath{\nabla}}
\newcommand{\eps}{\ensuremath{\varepsilon}}
\newcommand{\es}{\ensuremath{\varnothing}}
% Macros for math symbols
\newcommand{\dx}[1]{\,\mathrm{d}#1}
\newcommand{\inv}{\ensuremath{^{-1}}}
\newcommand{\sm}{\setminus}
\newcommand{\sse}{\subseteq}
\newcommand{\ceq}{\coloneqq}
% Macros for pairs of math symbols
\newcommand{\abs}[1]{\ensuremath{\left\lvert #1 \right\rvert}}
\newcommand{\paren}[1]{\ensuremath{\left( #1 \right)}}
\newcommand{\norm}[1]{\ensuremath{\left\lVert #1\right\rVert}}
\newcommand{\set}[1]{\ensuremath{\left\{#1\right\}}}
\newcommand{\tup}[1]{\ensuremath{\left\langle #1 \right\rangle}}
\newcommand{\floor}[1]{\ensuremath{\left\lfloor #1 \right\rfloor}}
\newcommand{\ceil}[1]{\ensuremath{\left\lceil #1 \right\rceil}}
\newcommand{\eclass}[1]{\ensuremath{\left[ #1 \right]}}

\newcommand{\chapternum}{}
\newcommand{\ex}[1]{\noindent\textbf{Exercise \chapternum.{#1}.}}

\newcommand{\tsub}[1]{\textsubscript{#1}}
\newcommand{\tsup}[1]{\textsuperscript{#1}}

\renewcommand{\choose}[2]{\ensuremath{\phantom{}_{#1}C_{#2}}}
\newcommand{\perm}[2]{\ensuremath{\phantom{}_{#1}P_{#2}}}

% Include figures
\newcommand{\incfig}[2][1]{%
    \def\svgwidth{#1\columnwidth}
    \import{./figures/}{#2.pdf_tex}
}

\definecolor{problemBackground}{RGB}{212,232,246}
\newenvironment{problem}[1]
  {
    \begin{tcolorbox}[
      boxrule=.5pt,
      titlerule=.5pt,
      sharp corners,
      colback=problemBackground,
      breakable
    ]
    \ifx &#1& \textbf{Problem. }
    \else \textbf{Problem #1.} \fi
  }
  {
    \end{tcolorbox}
  }
\definecolor{exampleBackground}{RGB}{255,249,248}
\definecolor{exampleAccent}{RGB}{158,60,14}
\newenvironment{example}[1]
  {
    \begin{tcolorbox}[
      boxrule=.5pt,
      sharp corners,
      colback=exampleBackground,
      colframe=exampleAccent,
    ]
    \color{exampleAccent}\textbf{Example.} \emph{#1}\color{black}
  }
  {
    \end{tcolorbox}
  }
\definecolor{theoremBackground}{RGB}{234,243,251}
\definecolor{theoremAccent}{RGB}{0,116,183}
\newenvironment{theorem}[1]
  {
    \begin{tcolorbox}[
      boxrule=.5pt,
      titlerule=.5pt,
      sharp corners,
      colback=theoremBackground,
      colframe=theoremAccent,
      breakable
    ]
      % \color{theoremAccent}\textbf{Theorem --- }\emph{#1}\\\color{black}
    \ifx &#1& \color{theoremAccent}\textbf{Theorem. }\color{black}
    \else \color{theoremAccent}\textbf{Theorem --- }\emph{#1}\\\color{black} \fi
  }
  {
    \end{tcolorbox}
  }
\definecolor{noteBackground}{RGB}{244,249,244}
\definecolor{noteAccent}{RGB}{34,139,34}
\newenvironment{note}[1]
  {
  \begin{tcolorbox}[
    enhanced,
    boxrule=0pt,
    frame hidden,
    sharp corners,
    colback=noteBackground,
    borderline west={3pt}{-1.5pt}{noteAccent},
    breakable
    ]
    \ifx &#1& \color{noteAccent}\textbf{Note. }\color{black}
    \else \color{noteAccent}\textbf{Note (#1). }\color{black} \fi
    }
    {
  \end{tcolorbox}
  }
\definecolor{lemmaBackground}{RGB}{255,247,234}
\definecolor{lemmaAccent}{RGB}{255,153,0}
\newenvironment{lemma}[1]
  {
    \begin{tcolorbox}[
      enhanced,
      boxrule=0pt,
      frame hidden,
      sharp corners,
      colback=lemmaBackground,
      borderline west={3pt}{-1.5pt}{lemmaAccent},
      breakable
    ]
    \ifx &#1& \color{lemmaAccent}\textbf{Lemma. }\color{black}
    \else \color{lemmaAccent}\textbf{Lemma #1. }\color{black} \fi
  }
  {
    \end{tcolorbox}
  }
\definecolor{definitionBackground}{RGB}{246,246,246}
\newenvironment{definition}[1]
  {
    \begin{tcolorbox}[
      enhanced,
      boxrule=0pt,
      frame hidden,
      sharp corners,
      colback=definitionBackground,
      borderline west={3pt}{-1.5pt}{black},
      breakable
    ]
    \textbf{Definition. }\emph{#1}\\
  }
  {
    \end{tcolorbox}
  }

\newenvironment{amatrix}[2]{
    \left[
      \begin{array}{*{#1}{c}|*{#2}c}
  }
  {
      \end{array}
    \right]
  }

\definecolor{codeBackground}{RGB}{253,246,227}
\renewcommand\theFancyVerbLine{\arabic{FancyVerbLine}}
\newtcblisting{cpp}[1][]{
  boxrule=1pt,
  sharp corners,
  colback=codeBackground,
  listing only,
  breakable,
  minted language=cpp,
  minted style=material,
  minted options={
    xleftmargin=15pt,
    baselinestretch=1.2,
    linenos,
  }
}

\author{Kyle Chui}


\fancyhf{}
\lhead{Kyle Chui}
\rhead{Page \thepage}
\pagestyle{fancy}

\begin{document}
  \section{Lecture 18}
  \subsection{Introduction to Continuity}
  We now want to discuss functions on the reals,
  \[
    f\colon A\sse \R\to \R, \quad\dom(A) = f.
  \]
  We have a basic understanding of continuity---``if $x\in A$ is \emph{close} to $c\in A$, then $f(x)$ is \emph{close} to $f(c)$''. Written a bit more rigorously, we can say that ``$f$ is continuous at a point $c$ if $\lim\limits_{x\to c} f(x) = f(c)$''. However, we haven't defined the limit of a function yet. We'll call the limit above a ``functional limit''.
  \begin{example}{Dirichlet's Function} \\
    Let us define
    \[
      g(x) = \begin{cases}
        1 & \text{if }x\in \Q, \\
        0 & \text{if }x\notin \Q.
      \end{cases}
    \]
    \textbf{Question.} What number should we associate with
    \[
      \lim_{x\to \frac{1}{2}} g(x)?
    \]
    We take a sequence of points in $\R$, apply $g$ to them, and see what that approaches as those points approach $\frac{1}{2}$. There exists some $(x_n)\in\Q$ such that $x_n\to \frac{1}{2}$, so $g(x_n) = 1$. Thus
    \[
      \lim_{n\to \infty} g(x_n) = 1.
    \]
    However, by the density of the irrationals, we know there exists some $(y_n)\in \R\sm\Q$ such that $y_n\to \frac{1}{2}$, so $g(y_n) = 0$. Thus
    \[
      \lim_{n\to \infty} g(y_n) = 0.
    \]
    Since these two values disagree with each other, we say that the limit does not exist. In particular for Dirichlet's function, we say that ``$g$ is discontinuous \emph{everywhere}''.
  \end{example}
  \begin{example}{}
    Let us consider the function given by
    \[
      h(x) = \begin{cases}
        x & x\in\Q, \\
        0 & x\in\R\sm\Q.
      \end{cases}
    \]
    Consider what the value of
    \[
      \lim_{x\to 0} h(x)
    \]
    should be. Let $(x_n)\sse\R$ such that $x_n\to 0$. Thus we have
    \[
      \abs{h(x_n)} \leq \abs{x_n} \to 0.
    \]
    Thus by Squeeze Theorem we have that $h(x_n)\to 0$ as $n\to\infty$. Therefore
    ``$\displaystyle \lim_{x\to 0} h(x) = 0$'' and ``$h$ is continuous at $x = 0$ and discontinuous elsewhere''.
  \end{example}
  \subsection{Functional Limits}
  \begin{definition}{4.1---Limit Points}
    Let $A\sse \R$. We say $c\in\R$ is a \emph{limit point} of $A$ if there exists a sequence such that
    \begin{itemize}
      \item $(x_n)\sse A$
      \item $x_n\neq c$ for all $n\in\N$
      \item $\displaystyle\lim_{n\to \infty} x_n = c$
    \end{itemize}
    In other words, $c$ is a limit point if a sequence of points in $A$ (not equal to $c$) converges to $c$.
  \end{definition}
  \begin{note}{}
    A limit point $c$ does not necessarily have to belong to the set $A$.
  \end{note}
  \begin{example}{}
    \begin{itemize}
      \item Let $A = (0, 1)$. If $x\in A$, then $x_n = x - \frac{1}{n}$ converges to $x$ when $n$ is sufficiently large, so all points in $A$ are limit points. Note that if we let $c = 1$, we can still define $x_n = 1 - \frac{1}{n}\to 1$ to see that $1$ is also a limit point of $A$ despite not belonging to $A$.
      \item If $A = [0, 1]$, then $1\in A$ and every point in $A$ is a limit point of $A$.
      \item Every interval for the form $[a, b] = \set{x\in\R, a\leq x\leq b}$, $a < b$ contains all of its limit points. We write $L_A\sse\R$ to be the set of all limit points.
    \end{itemize}
  \end{example}
  \begin{note}{}
    The set of limit points $L_A$ of a set $A$ in general can look very different from $A$ itself, and we can't draw any conclusions about one being the subset of the other.
  \end{note}
  \begin{definition}{Isolated Point}
    A point in $A$ that is not a limit point of $A$ is called a \emph{isolated point}.
  \end{definition}
  \begin{definition}{4.2---Closed Set}
    A set containing its limit points is called a \emph{closed set}.
  \end{definition}
  \begin{definition}{4.3---Functional Limit}
    Let $A\sse\R$, $c\in\R$ be a limit point of $A$, and $f\colon A\to \R\; (\dom(f) = A)$. Then we define
    \[
      \lim_{x\to c} f(x) = L
    \]
    for some $L\in \R$, if for \emph{every} sequence $(x_n)$ in $A$ which converges to $c$, we have
    \[
      \lim_{n\to \infty} f(x_n) = L.
    \]
  \end{definition}
  \begin{note}{}
    We extend the definition to allow $c\in\R\cup \set{+\infty, -\infty}$ so we may have limits of the form
    \[
      \lim_{x\to +\infty} f(x)\quad \text{and}\quad \lim_{x\to -\infty} f(x). \tag{$L_A\sse \overline\R$}
    \]
    Furthermore, we extend $L\in\R\cup \set{+\infty, -\infty}$ so we may have limits of the form
    \[
      \lim_{x\to c} f(x) = \pm\infty.
    \]
  \end{note}
  \begin{example}{}
    \begin{enumerate}[label=\arabic*)]
      \item Let $f(x) = x^2$, where $A = \R$. Furthermore, let $c\in\R$ and $x_n\to c$. Then $f(x_n) = x_n^2\to c^2$ by Algebraic Limit Theorem. Hence,
      \[
        \lim_{x\to c} f(x) = c^2
      \]
      \item Let $f(x) = \frac{x^2 - 4}{x - 2}$. We have that $A = (-\infty, 2)\cup (2, \infty)$. We can see that $2$ is a limit point of $A$. Let $(x_n)\sse A$ such that $x_n\to 2$, and $x_n\neq 2$ for all $n$. Then
      \begin{align*}
        f(x_n) &= \frac{x_n^2 - 4}{x_n - 2} \\
               &= \frac{(x_n + 2)(x_n - 2)}{x_n - 2} \\
               &= x_n + 2,
      \end{align*}
      which converges to $4$. Hence, $\displaystyle \lim_{x\to 2} f(x) = 4$.
      \item Let $f(x) = \frac{1}{(x - 2)^3}$, so $A = \R\sm \set{2}$. We see that $\pm\infty$ are limit points of $A$, and want to show that
      \[
        \lim_{x\to +\infty} f(x) = \lim_{x\to -\infty} f(x) = 0.
      \]
      Let $x_n\to+\infty$, and $(x_n)\sse A$. Then $f(x_n) = \frac{1}{(x_n - 2)^3}\to 0$. \par
      What about $\displaystyle \lim_{x\to 2} f(x)$? Looking at the graph, we suspect that the limit does not exist. If $(x_n)\sse A$ such that $x_n > 2$ and $x_n\to 2$, then
      \[
        f(x_n) = \frac{1}{(x_n - 2)^3}\to+\infty.
      \]
      If $(y_n)\sse A$ such that $y_n < 2$ and $y_n\to 2$, then
      \[
        f(y_n) = \frac{1}{(y_n - 2)^3}\to-\infty.
      \]
      We found two sequences such that when $f$ is applied, they converge to different values, and so $\displaystyle \lim_{x\to 2} f(x)$ does not exist. However, we take the notion of ``one-sided'' limits:
      \begin{itemize}
        \item $\displaystyle\lim_{x\to 2^+} f(x) = +\infty$.
        \item $\displaystyle\lim_{x\to 2^-} f(x) = -\infty$.
      \end{itemize}
    \end{enumerate}
  \end{example}
  \begin{definition}{4.4---Left and Right Hand Limits}
    Let $c$ be a limit point of $A\sse\R$. We define the \emph{right-hand limit} of $f\colon A\to \R$ by
    \[
      \lim_{x\to c^+} f(x) = c
    \]
    if for every $(x_n)\sse A$, $x_n > c$ and $x_n\to c$ we have
    \[
      \lim_{n\to \infty} f(x_n) = L.
    \]
    Similarly, we define the \emph{left-hand limit} of $f\colon A\to \R$ by
    \[
      \lim_{x\to c^-} f(x) = c
    \]
    if for every $(x_n)\sse A$, $x_n < c$ and $x_n\to c$ we have
    \[
      \lim_{n\to \infty} f(x_n) = L.
    \]
  \end{definition}
  \begin{theorem}{4.5}
    If $f\colon A\to \R$, $c$ a limit point of $A$, Then
    \[
      \lim_{x\to c} f(x) = L \tag{exists}
    \]
    if and only if
    \[
      \lim_{x\to c^+} f(x) = L\quad \text{and}\quad \lim_{x\to c^-} f(x) = L.
    \]
  \end{theorem}
\end{document}
