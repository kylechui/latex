\documentclass[class=article, crop=false]{standalone}
\usepackage[margin=1in]{geometry}

\usepackage{amsmath}
\usepackage{amssymb}
\usepackage{amsthm}
\usepackage{comment}
\usepackage{enumitem}
\usepackage{fancyhdr}
\usepackage{hyperref}
\usepackage{mathrsfs}
\usepackage{mathtools}
\usepackage{standalone}
\usepackage{tcolorbox}
\usepackage{tikz}
\usepackage{xcolor}

\usetikzlibrary{decorations.pathreplacing}
\tcbuselibrary{skins}

\DeclareMathOperator{\lcm}{lcm}
\DeclareMathOperator{\proj}{proj}
\DeclareMathOperator{\vspan}{span}
\DeclareMathOperator{\im}{im}
\DeclareMathOperator{\range}{range}
\DeclareMathOperator{\Diff}{Diff}
\DeclareMathOperator{\Int}{Int}
\DeclareMathOperator{\fcn}{fcn}
\DeclareMathOperator{\id}{id}

\newcommand{\N}{\ensuremath{\mathbb{N}}}
\newcommand{\Z}{\ensuremath{\mathbb{Z}}}
\newcommand{\Q}{\ensuremath{\mathbb{Q}}}
\newcommand{\R}{\ensuremath{\mathbb{R}}}
\newcommand{\C}{\ensuremath{\mathbb{C}}}
\newcommand{\F}{\ensuremath{\mathbb{F}}}
\newcommand{\lam}{\ensuremath{\lambda}}
\newcommand{\nab}{\ensuremath{\nabla}}
\newcommand{\eps}{\ensuremath{\varepsilon}}
\newcommand{\es}{\ensuremath{\varnothing}}

\newcommand{\abs}[1]{\ensuremath{\left\lvert #1 \right\rvert}}
\newcommand{\bigpar}[1]{\ensuremath{\left( #1 \right)}}
\newcommand{\norm}[1]{\ensuremath{\lVert #1\rVert}}
\newcommand{\set}[1]{\ensuremath{\left\{#1\right\}}}
\newcommand{\tuple}[1]{\ensuremath{\left\langle #1 \right\rangle}}
\newcommand{\floor}[1]{\ensuremath{\left\lfloor #1 \right\rfloor}}
\newcommand{\ceil}[1]{\ensuremath{\left\lceil #1 \right\rceil}}

\newcommand{\chapternum}{}
\newcommand{\ex}[1]{\noindent\textbf{Exercise \chapternum.{#1}.}}

\newcommand{\dx}[1]{\,\mathrm{d}#1}
\newcommand{\inv}{\ensuremath{^{-1}}}
\newcommand{\sm}{\setminus}
\newcommand{\sse}{\subseteq}
\newcommand{\ceq}{\coloneqq}

\definecolor{problemBackground}{RGB}{212,232,246}
\newenvironment{problem}[1]
  {
    \begin{tcolorbox}[
      boxrule=.5pt,
      titlerule=.5pt,
      sharp corners,
      colback=problemBackground
    ]
    \ifx &#1& \textbf{Problem. }
    \else \textbf{Problem #1.} \fi
  }
  {
    \end{tcolorbox}
  }

\definecolor{exampleBackground}{RGB}{255,249,248}
\definecolor{exampleAccent}{RGB}{158,60,14}
\newenvironment{example}[1]
  {
    \begin{tcolorbox}[
      boxrule=.5pt,
      sharp corners,
      colback=exampleBackground,
      colframe=exampleAccent,
    ]
    \color{exampleAccent}\textbf{Example.} \emph{#1}\color{black}
  }
  {
    \end{tcolorbox}
  }

\definecolor{theoremBackground}{RGB}{234,243,251}
\definecolor{theoremAccent}{RGB}{0,116,183}
\newenvironment{theorem}[1]
  {
    \begin{tcolorbox}[
      boxrule=.5pt,
      titlerule=.5pt,
      sharp corners,
      colback=theoremBackground,
      colframe=theoremAccent
    ]
      \color{theoremAccent}\textbf{Theorem --- }\emph{#1}\\\color{black}
  }
  {
    \end{tcolorbox}
  }

\definecolor{noteBackground}{RGB}{244,249,244}
\definecolor{noteAccent}{RGB}{34,139,34}
\newenvironment{note}[1]
  {
  \begin{tcolorbox}[
    enhanced,
    boxrule=0pt,
    frame hidden,
    sharp corners,
    colback=noteBackground,
    borderline west={3pt}{-1.5pt}{noteAccent}
    ]
    \ifx &#1& \color{noteAccent}\textbf{Note. }\color{black}
    \else \color{noteAccent}\textbf{Note (#1). }\color{black} \fi
    }
    {
  \end{tcolorbox}
  }

\definecolor{definitionBackground}{RGB}{246,246,246}
\newenvironment{definition}[1]
  {
    \begin{tcolorbox}[
      enhanced,
      boxrule=0pt,
      frame hidden,
      sharp corners,
      colback=definitionBackground,
      borderline west={3pt}{-1.5pt}{black}
    ]
    \textbf{Definition. }\emph{#1}\\
  }
  {
    \end{tcolorbox}
  }
\newenvironment{amatrix}[2]{
    \left[
      \begin{array}{*{#1}{c}|*{#2}c}
  }
  {
      \end{array}
    \right]
  }
\date{\the\year-\the\month-\the\day}
\author{Kyle Chui}


\fancyhf{}
\lhead{Kyle Chui}
\rhead{Page \thepage}
\pagestyle{fancy}

\begin{document}
  \section{Lecture 1}
  \subsection{Goals of This Class}
  The end goal of this class is to go over the Fundamental Theorem of Calculus, and to do that we must first cover:
  \begin{itemize}
    \item What is a real number and what makes them special?
    \item How do we define the convergence of a sequence of real numbers?
    \item What is a limit?
    \item What is continuity?
    \item What is a derivative?
    \item What is an integral?
  \end{itemize}
  \subsection{Mathematical Arguments and Logic}
  We have the following components to reasoning about solutions to a mathematical problem:
  \begin{itemize}
    \item Assumptions
    \item Logical steps
    \item Conclusions
  \end{itemize}
  \subsection{Logical Connections}
  We usually use the letters $P$ and $Q$ to denote logical statements (either true or false). We can also use conjunctions to connect logical statements to one another.
  \begin{enumerate}
    \item Conjunctions: ``$P$ and $Q$'', $P\land Q$
    \item Disjunctions: ``$P$ or $Q$'', $P\lor Q$
    \item Implications: ``If $P$, then $Q$'', $P\implies Q$
    \begin{enumerate}
      \item If the proposition is false (i.e. if $P$ is false) then the whole statement is true.
      \begin{definition}{}
        We say that the statement is \emph{vacuously true}.
      \end{definition}
    \end{enumerate}
    \item Negations: ``Not $P$'', $\neg P$
  \end{enumerate}
  \subsubsection{Truth Tables}
  \begin{center}\begin{tabular}{c|c|c}
    $P$ & $Q$ & $P\implies Q$ \\
    \hline
    $T$ & $T$ & $T$ \\
    $T$ & $F$ & $F$ \\
    $F$ & $T$ & $T$ \\
    $F$ & $F$ & $T$
  \end{tabular}\end{center}
  \begin{example}{}
    Prove that if $n$ is an integer, then $n(n + 1)$ is even.
    \begin{proof}
      Suppose that $n$ is an integer. Then we have two cases, where either $n$ is even or $n$ is odd. Let $n$ be an even integer such that $n = 2k$ where $k\in \Z$. Then we have
      \begin{align*}
        n(n + 1) &= 2k(2k + 1) \\
                 &= 2(2k^2 + k).
      \end{align*}
      Thus we see that $n(n + 1)$ is even when $n$ is even. Now let $n$ be odd such that $n = 2m + 1$ where $m\in \Z$. Then we have
      \begin{align*}
        n(n + 1) &= (2m + 1)(2m + 1 + 1) \\
                 &= (2m + 1)(2m + 2) \\
                 &= 2(m + 1)(2m + 1).
      \end{align*}
      Thus $n(n + 1)$ is also even when $n$ is odd, and so is even for all integers $n$.
    \end{proof}
  \end{example}
\end{document}
