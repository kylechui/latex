\documentclass[class=article, crop=false]{standalone}
% Import packages
\usepackage[margin=1in]{geometry}

\usepackage{amssymb, amsthm}
\usepackage{comment}
\usepackage{enumitem}
\usepackage{fancyhdr}
\usepackage{hyperref}
\usepackage{import}
\usepackage{mathrsfs, mathtools}
\usepackage{multicol}
\usepackage{pdfpages}
\usepackage{standalone}
\usepackage{transparent}
\usepackage{xcolor}

\usepackage{minted}
\usepackage[many]{tcolorbox}
\tcbuselibrary{minted}

% Declare math operators
\DeclareMathOperator{\lcm}{lcm}
\DeclareMathOperator{\proj}{proj}
\DeclareMathOperator{\vspan}{span}
\DeclareMathOperator{\im}{im}
\DeclareMathOperator{\Diff}{Diff}
\DeclareMathOperator{\Int}{Int}
\DeclareMathOperator{\fcn}{fcn}
\DeclareMathOperator{\id}{id}
\DeclareMathOperator{\rank}{rank}
\DeclareMathOperator{\tr}{tr}
\DeclareMathOperator{\dive}{div}
\DeclareMathOperator{\row}{row}
\DeclareMathOperator{\col}{col}
\DeclareMathOperator{\dom}{dom}
\DeclareMathOperator{\ran}{ran}
\DeclareMathOperator{\Dom}{Dom}
\DeclareMathOperator{\Ran}{Ran}
\DeclareMathOperator{\fl}{fl}
% Macros for letters/variables
\newcommand\ttilde{\kern -.15em\lower .7ex\hbox{\~{}}\kern .04em}
\newcommand{\N}{\ensuremath{\mathbb{N}}}
\newcommand{\Z}{\ensuremath{\mathbb{Z}}}
\newcommand{\Q}{\ensuremath{\mathbb{Q}}}
\newcommand{\R}{\ensuremath{\mathbb{R}}}
\newcommand{\C}{\ensuremath{\mathbb{C}}}
\newcommand{\F}{\ensuremath{\mathbb{F}}}
\newcommand{\M}{\ensuremath{\mathbb{M}}}
\renewcommand{\P}{\ensuremath{\mathbb{P}}}
\newcommand{\lam}{\ensuremath{\lambda}}
\newcommand{\nab}{\ensuremath{\nabla}}
\newcommand{\eps}{\ensuremath{\varepsilon}}
\newcommand{\es}{\ensuremath{\varnothing}}
% Macros for math symbols
\newcommand{\dx}[1]{\,\mathrm{d}#1}
\newcommand{\inv}{\ensuremath{^{-1}}}
\newcommand{\sm}{\setminus}
\newcommand{\sse}{\subseteq}
\newcommand{\ceq}{\coloneqq}
% Macros for pairs of math symbols
\newcommand{\abs}[1]{\ensuremath{\left\lvert #1 \right\rvert}}
\newcommand{\paren}[1]{\ensuremath{\left( #1 \right)}}
\newcommand{\norm}[1]{\ensuremath{\left\lVert #1\right\rVert}}
\newcommand{\set}[1]{\ensuremath{\left\{#1\right\}}}
\newcommand{\tup}[1]{\ensuremath{\left\langle #1 \right\rangle}}
\newcommand{\floor}[1]{\ensuremath{\left\lfloor #1 \right\rfloor}}
\newcommand{\ceil}[1]{\ensuremath{\left\lceil #1 \right\rceil}}
\newcommand{\eclass}[1]{\ensuremath{\left[ #1 \right]}}

\newcommand{\chapternum}{}
\newcommand{\ex}[1]{\noindent\textbf{Exercise \chapternum.{#1}.}}

\newcommand{\tsub}[1]{\textsubscript{#1}}
\newcommand{\tsup}[1]{\textsuperscript{#1}}

\renewcommand{\choose}[2]{\ensuremath{\phantom{}_{#1}C_{#2}}}
\newcommand{\perm}[2]{\ensuremath{\phantom{}_{#1}P_{#2}}}

% Include figures
\newcommand{\incfig}[2][1]{%
    \def\svgwidth{#1\columnwidth}
    \import{./figures/}{#2.pdf_tex}
}

\definecolor{problemBackground}{RGB}{212,232,246}
\newenvironment{problem}[1]
  {
    \begin{tcolorbox}[
      boxrule=.5pt,
      titlerule=.5pt,
      sharp corners,
      colback=problemBackground,
      breakable
    ]
    \ifx &#1& \textbf{Problem. }
    \else \textbf{Problem #1.} \fi
  }
  {
    \end{tcolorbox}
  }
\definecolor{exampleBackground}{RGB}{255,249,248}
\definecolor{exampleAccent}{RGB}{158,60,14}
\newenvironment{example}[1]
  {
    \begin{tcolorbox}[
      boxrule=.5pt,
      sharp corners,
      colback=exampleBackground,
      colframe=exampleAccent,
    ]
    \color{exampleAccent}\textbf{Example.} \emph{#1}\color{black}
  }
  {
    \end{tcolorbox}
  }
\definecolor{theoremBackground}{RGB}{234,243,251}
\definecolor{theoremAccent}{RGB}{0,116,183}
\newenvironment{theorem}[1]
  {
    \begin{tcolorbox}[
      boxrule=.5pt,
      titlerule=.5pt,
      sharp corners,
      colback=theoremBackground,
      colframe=theoremAccent,
      breakable
    ]
      % \color{theoremAccent}\textbf{Theorem --- }\emph{#1}\\\color{black}
    \ifx &#1& \color{theoremAccent}\textbf{Theorem. }\color{black}
    \else \color{theoremAccent}\textbf{Theorem --- }\emph{#1}\\\color{black} \fi
  }
  {
    \end{tcolorbox}
  }
\definecolor{noteBackground}{RGB}{244,249,244}
\definecolor{noteAccent}{RGB}{34,139,34}
\newenvironment{note}[1]
  {
  \begin{tcolorbox}[
    enhanced,
    boxrule=0pt,
    frame hidden,
    sharp corners,
    colback=noteBackground,
    borderline west={3pt}{-1.5pt}{noteAccent},
    breakable
    ]
    \ifx &#1& \color{noteAccent}\textbf{Note. }\color{black}
    \else \color{noteAccent}\textbf{Note (#1). }\color{black} \fi
    }
    {
  \end{tcolorbox}
  }
\definecolor{lemmaBackground}{RGB}{255,247,234}
\definecolor{lemmaAccent}{RGB}{255,153,0}
\newenvironment{lemma}[1]
  {
    \begin{tcolorbox}[
      enhanced,
      boxrule=0pt,
      frame hidden,
      sharp corners,
      colback=lemmaBackground,
      borderline west={3pt}{-1.5pt}{lemmaAccent},
      breakable
    ]
    \ifx &#1& \color{lemmaAccent}\textbf{Lemma. }\color{black}
    \else \color{lemmaAccent}\textbf{Lemma #1. }\color{black} \fi
  }
  {
    \end{tcolorbox}
  }
\definecolor{definitionBackground}{RGB}{246,246,246}
\newenvironment{definition}[1]
  {
    \begin{tcolorbox}[
      enhanced,
      boxrule=0pt,
      frame hidden,
      sharp corners,
      colback=definitionBackground,
      borderline west={3pt}{-1.5pt}{black},
      breakable
    ]
    \textbf{Definition. }\emph{#1}\\
  }
  {
    \end{tcolorbox}
  }

\newenvironment{amatrix}[2]{
    \left[
      \begin{array}{*{#1}{c}|*{#2}c}
  }
  {
      \end{array}
    \right]
  }

\definecolor{codeBackground}{RGB}{253,246,227}
\renewcommand\theFancyVerbLine{\arabic{FancyVerbLine}}
\newtcblisting{cpp}[1][]{
  boxrule=1pt,
  sharp corners,
  colback=codeBackground,
  listing only,
  breakable,
  minted language=cpp,
  minted style=material,
  minted options={
    xleftmargin=15pt,
    baselinestretch=1.2,
    linenos,
  }
}

\author{Kyle Chui}


\fancyhf{}
\lhead{Kyle Chui}
\rhead{Page \thepage}
\pagestyle{fancy}

\begin{document}
  \section{Lecture 4}
  \subsection{Cartesian Product}
  If I have two sets $A$ and $B$, then we may form their \emph{Cartesian Product}, which is
  \[
    A\times B = \set{(x, y)\mid x_0in ?a\land y\in B}.
  \]
  \begin{definition}{Binary Relation}
    A \emph{binary relation} is a subset $R\sse A\times B$. We say $x\in A$ is in relation to $y\in B$ if $(x, y)\in R$. We denote this by
    \[
      xRy\iff (x,y)\in R.
    \]
  \end{definition}
  \begin{example}{}
    Consider the relation 
    \[
      \R\times \R\supseteq R = \set{(x, y)\in \R\times \R\mid x \leq y}.
    \]
    Then the relation is \emph{reflexive}, because $xRx$. It is also antisymmetric, because $xRy\land yRx\implies x = y$. Finally, this relation is transitive, because $xRy\land yRz\implies xRz$.
  \end{example}
  These properties only make sense if $A = B$, i.e. $R\sse A\times A$, and we say that ``$R$ is a relation on $A$''.
  \begin{definition}{Partial Order}
    If a relation is reflexive, antisymmetric, and transitive on $A$, then it is a \emph{partial order} on $A$.
  \end{definition}
  The notion of ``less than or equal to'' is a partial order for $\N$, $\Z$, $\Q$, and $\R$, but there exists no partial order for $\C$.
  \begin{definition}{Power Set}
    For a set $A$, we may define its \emph{power set} by
    \[
      \mathscr{P}(A) = \set{C\mid C\sse A}.
    \]
    Note that set inclusion is a partial order on $\mathscr{P}(A)$.
  \end{definition}
  \begin{definition}{Equivalence Relation}
    An \emph{equivalence relation} $R$ over $A$ is a relation that is reflexive, symmetric, and transitive.
  \end{definition}
  \begin{note}{}
    Just like a partial order behaves much like $\leq$, an equivalence relation behaves much like $=$.
  \end{note}
  \begin{definition}{Equivalence Class}
    Given an equivalence relation on $A$.0, we define a new set
    \[
      [x] \ceq \set{y\in A\mid x\sim y}.
    \]
    We call $[x]$ the \emph{equivalence class} of $x$. Any $z\in [x]$ is called a \emph{representative} of the equivalence class $[x]$. In particular, $x$ is a representative of its own equivalence class.
  \end{definition}
  Let $A$ be a set with equivalence relation $\sim$. Then for any $x, y\in A$,
  \[
    [x] = [y] \quad\text{or}\quad [x] \cap [y] = \es.
  \]
  \begin{proof}
    Let $x, y\in A$. We know that $x$ is either equivalent to $y$ or it is not. Suppose the former is true and let $z\in [x]$. Thus we know that $z\sim x$ and $x\sim y$, and by transitivity we have $z\sim y$. Thus $z\in [y]$ and $[x]\sse[y]$. The reverse argument is the same. \par
    If $x$ is not equivalent to $y$, then suppose towards a contradiction that $[x] \cap [y] \neq \es$. Let $x\in [x] \cap [y]$. Then $z\sim x$ and $z\sim y$. By symmetry we know that $x\sim z$ and by transitivity we have $x\sim y$. We have arrived at the contradiction that $x$ is both equivalent and not equivalent to $y$.
  \end{proof}
  \begin{definition}{Function}
    A relation $R\sse A\times B$ is a \emph{function} if for all $x\in A$ and all $y, z\in B$, we have the following:
    \begin{itemize}
      \item $xRy\land xRz\implies y = z$.
    \end{itemize}
    In other words, every input $x$ has only one output.
  \end{definition}
  \begin{definition}{Injective Functions}
    A function $f$ is \emph{injective} if $f(x_1) = f(x_2)\implies x_1 = x_2$.
  \end{definition}
  \begin{definition}{Surjective Functions}
    A function $f$ is \emph{surjective} if for every $y\in B$, there exists some $x\in A$ such that $f(x) = y$.
  \end{definition}
\end{document}
