\documentclass[class=article, crop=false]{standalone}
% Import packages
\usepackage[margin=1in]{geometry}

\usepackage{amssymb, amsthm}
\usepackage{comment}
\usepackage{enumitem}
\usepackage{fancyhdr}
\usepackage{hyperref}
\usepackage{import}
\usepackage{mathrsfs, mathtools}
\usepackage{multicol}
\usepackage{pdfpages}
\usepackage{standalone}
\usepackage{transparent}
\usepackage{xcolor}

\usepackage{minted}
\usepackage[many]{tcolorbox}
\tcbuselibrary{minted}

% Declare math operators
\DeclareMathOperator{\lcm}{lcm}
\DeclareMathOperator{\proj}{proj}
\DeclareMathOperator{\vspan}{span}
\DeclareMathOperator{\im}{im}
\DeclareMathOperator{\Diff}{Diff}
\DeclareMathOperator{\Int}{Int}
\DeclareMathOperator{\fcn}{fcn}
\DeclareMathOperator{\id}{id}
\DeclareMathOperator{\rank}{rank}
\DeclareMathOperator{\tr}{tr}
\DeclareMathOperator{\dive}{div}
\DeclareMathOperator{\row}{row}
\DeclareMathOperator{\col}{col}
\DeclareMathOperator{\dom}{dom}
\DeclareMathOperator{\ran}{ran}
\DeclareMathOperator{\Dom}{Dom}
\DeclareMathOperator{\Ran}{Ran}
\DeclareMathOperator{\fl}{fl}
% Macros for letters/variables
\newcommand\ttilde{\kern -.15em\lower .7ex\hbox{\~{}}\kern .04em}
\newcommand{\N}{\ensuremath{\mathbb{N}}}
\newcommand{\Z}{\ensuremath{\mathbb{Z}}}
\newcommand{\Q}{\ensuremath{\mathbb{Q}}}
\newcommand{\R}{\ensuremath{\mathbb{R}}}
\newcommand{\C}{\ensuremath{\mathbb{C}}}
\newcommand{\F}{\ensuremath{\mathbb{F}}}
\newcommand{\M}{\ensuremath{\mathbb{M}}}
\renewcommand{\P}{\ensuremath{\mathbb{P}}}
\newcommand{\lam}{\ensuremath{\lambda}}
\newcommand{\nab}{\ensuremath{\nabla}}
\newcommand{\eps}{\ensuremath{\varepsilon}}
\newcommand{\es}{\ensuremath{\varnothing}}
% Macros for math symbols
\newcommand{\dx}[1]{\,\mathrm{d}#1}
\newcommand{\inv}{\ensuremath{^{-1}}}
\newcommand{\sm}{\setminus}
\newcommand{\sse}{\subseteq}
\newcommand{\ceq}{\coloneqq}
% Macros for pairs of math symbols
\newcommand{\abs}[1]{\ensuremath{\left\lvert #1 \right\rvert}}
\newcommand{\paren}[1]{\ensuremath{\left( #1 \right)}}
\newcommand{\norm}[1]{\ensuremath{\left\lVert #1\right\rVert}}
\newcommand{\set}[1]{\ensuremath{\left\{#1\right\}}}
\newcommand{\tup}[1]{\ensuremath{\left\langle #1 \right\rangle}}
\newcommand{\floor}[1]{\ensuremath{\left\lfloor #1 \right\rfloor}}
\newcommand{\ceil}[1]{\ensuremath{\left\lceil #1 \right\rceil}}
\newcommand{\eclass}[1]{\ensuremath{\left[ #1 \right]}}

\newcommand{\chapternum}{}
\newcommand{\ex}[1]{\noindent\textbf{Exercise \chapternum.{#1}.}}

\newcommand{\tsub}[1]{\textsubscript{#1}}
\newcommand{\tsup}[1]{\textsuperscript{#1}}

\renewcommand{\choose}[2]{\ensuremath{\phantom{}_{#1}C_{#2}}}
\newcommand{\perm}[2]{\ensuremath{\phantom{}_{#1}P_{#2}}}

% Include figures
\newcommand{\incfig}[2][1]{%
    \def\svgwidth{#1\columnwidth}
    \import{./figures/}{#2.pdf_tex}
}

\definecolor{problemBackground}{RGB}{212,232,246}
\newenvironment{problem}[1]
  {
    \begin{tcolorbox}[
      boxrule=.5pt,
      titlerule=.5pt,
      sharp corners,
      colback=problemBackground,
      breakable
    ]
    \ifx &#1& \textbf{Problem. }
    \else \textbf{Problem #1.} \fi
  }
  {
    \end{tcolorbox}
  }
\definecolor{exampleBackground}{RGB}{255,249,248}
\definecolor{exampleAccent}{RGB}{158,60,14}
\newenvironment{example}[1]
  {
    \begin{tcolorbox}[
      boxrule=.5pt,
      sharp corners,
      colback=exampleBackground,
      colframe=exampleAccent,
    ]
    \color{exampleAccent}\textbf{Example.} \emph{#1}\color{black}
  }
  {
    \end{tcolorbox}
  }
\definecolor{theoremBackground}{RGB}{234,243,251}
\definecolor{theoremAccent}{RGB}{0,116,183}
\newenvironment{theorem}[1]
  {
    \begin{tcolorbox}[
      boxrule=.5pt,
      titlerule=.5pt,
      sharp corners,
      colback=theoremBackground,
      colframe=theoremAccent,
      breakable
    ]
      % \color{theoremAccent}\textbf{Theorem --- }\emph{#1}\\\color{black}
    \ifx &#1& \color{theoremAccent}\textbf{Theorem. }\color{black}
    \else \color{theoremAccent}\textbf{Theorem --- }\emph{#1}\\\color{black} \fi
  }
  {
    \end{tcolorbox}
  }
\definecolor{noteBackground}{RGB}{244,249,244}
\definecolor{noteAccent}{RGB}{34,139,34}
\newenvironment{note}[1]
  {
  \begin{tcolorbox}[
    enhanced,
    boxrule=0pt,
    frame hidden,
    sharp corners,
    colback=noteBackground,
    borderline west={3pt}{-1.5pt}{noteAccent},
    breakable
    ]
    \ifx &#1& \color{noteAccent}\textbf{Note. }\color{black}
    \else \color{noteAccent}\textbf{Note (#1). }\color{black} \fi
    }
    {
  \end{tcolorbox}
  }
\definecolor{lemmaBackground}{RGB}{255,247,234}
\definecolor{lemmaAccent}{RGB}{255,153,0}
\newenvironment{lemma}[1]
  {
    \begin{tcolorbox}[
      enhanced,
      boxrule=0pt,
      frame hidden,
      sharp corners,
      colback=lemmaBackground,
      borderline west={3pt}{-1.5pt}{lemmaAccent},
      breakable
    ]
    \ifx &#1& \color{lemmaAccent}\textbf{Lemma. }\color{black}
    \else \color{lemmaAccent}\textbf{Lemma #1. }\color{black} \fi
  }
  {
    \end{tcolorbox}
  }
\definecolor{definitionBackground}{RGB}{246,246,246}
\newenvironment{definition}[1]
  {
    \begin{tcolorbox}[
      enhanced,
      boxrule=0pt,
      frame hidden,
      sharp corners,
      colback=definitionBackground,
      borderline west={3pt}{-1.5pt}{black},
      breakable
    ]
    \textbf{Definition. }\emph{#1}\\
  }
  {
    \end{tcolorbox}
  }

\newenvironment{amatrix}[2]{
    \left[
      \begin{array}{*{#1}{c}|*{#2}c}
  }
  {
      \end{array}
    \right]
  }

\definecolor{codeBackground}{RGB}{253,246,227}
\renewcommand\theFancyVerbLine{\arabic{FancyVerbLine}}
\newtcblisting{cpp}[1][]{
  boxrule=1pt,
  sharp corners,
  colback=codeBackground,
  listing only,
  breakable,
  minted language=cpp,
  minted style=material,
  minted options={
    xleftmargin=15pt,
    baselinestretch=1.2,
    linenos,
  }
}

\author{Kyle Chui}


\fancyhf{}
\lhead{Kyle Chui}
\rhead{Page \thepage}
\pagestyle{fancy}

\begin{document}
  \section{Lecture 25}
  \subsection{Mean Value Theorem}
  Using the Mean Value Theorem, we are able to show the link between differentiation and integration.\begin{definition}{Lipschitz bound}
    A function that has a \emph{Lipschitz bound} is a function of the form
    \[
      \abs{f(x) - f(y)}\leq M\abs{x - y},
    \]
    where $M\in\R$.
  \end{definition}
  We use monotonicity to prove very useful inequalities, and verify assumptions for the Inverse of a Function Theorem.
  \begin{theorem}{5.9 (Rolle's Theorem)}
    Let $f\colon [a, b]\to \R$ be continuous on $[a, b]$ and differentiable on $(a, b)$. If $f(a) = f(b)$, then there exists $c\in (a, b)$ such that $f'(c) = 0$.
    \begin{proof}
      By Proposition $5.7$, we know that $f$ has extrema at $a, b$ or points $c\in (a, b)$ such that $f'(c) = 0$. If maxima and minima occur at the endpoints, ($a$ and $b$), then $f$ must be constant, and $f'(c) = 0$ for all $c\in [a, b]$. If not, then there exists some extrema in the interval, and so there exists some $c\in [a, b]$ such that $f'(c) = 0$.
    \end{proof}
  \end{theorem}
  \begin{theorem}{5.10 (Mean Value Theorem)}
    Let $f\colon [a, b]\to \R$, $f$ continuous on $[a, b]$, differentiable on $(a, b)$. Then there exists $c\in (a, b)$ such that
    \[
      f'(c) = \frac{f(b) - f(a)}{b - a}.
    \]
    \begin{proof}
      Notice that the line going through $(a, f(a))$ and $(b, f(b))$ can be expressed as:
      \[
        g(x) = \frac{f(b) - f(a)}{b - a}\cdot (x - a) + f(a).
      \]
      Let us also define
      \[
        h(x)\ceq (f - g)(x),
      \]
      so $h(a) = f(a) - f(a) = 0$ and $h(b) = f(b) - f(b) = 0$. Thus by Rolle's Theorem there exists some point $c$ in $(a, b)$ such that $h'(c) = 0$. Hence there exists $c\in (a, b)$ such that
      \[
        0 = f'(c) - g'(c),
      \]
      so $\displaystyle f'(c) = g'(c) = \frac{f(b) - f(a)}{b - a}$, and we are done.
    \end{proof}
  \end{theorem}
  \subsubsection{Monotonicity Applications}
  \textbf{Proposition 5.11} Let $f\colon [a, b]\to \R$, continuous on $[a, b]$ and differentiable on $(a, b)$. Then:
  \begin{enumerate}[label=(\roman*)]
    \item If $f'(x) = 0$ for all $x\in [a, b]$ then $f$ is constant.
    \item If $f'(x) \geq 0$ for all $x\in [a, b]$ then $f$ is non-decreasing.
    \item If $f'(x) \leq 0$ for all $x\in [a, b]$ then $f$ is non-increasing.
  \end{enumerate}
  \begin{enumerate}[label=(\roman*)]
    \item
    \begin{proof}
      Let $a < x < y < b$ and apply Mean Value Theorem on every subinterval $[x, y]$:
      \[
        0 = f'(c) = \frac{f(y) - f(x)}{y - x},
      \]
      so $f(x) = f(y)$ so $f$ is constant.
    \end{proof}
    \item 
    \begin{proof}
      Let $a < x < y < b$. Then by Mean Value Theorem we know that
      \[
        \frac{f(y) - f(x)}{y - x} = f'(c)\geq 0,
      \]
      so $f(y) - f(x)\geq 0$ for all $x, y\in [a, b]$.
    \end{proof}
    \item 
    \begin{proof}
      Let $a < x < y < b$. Then by Mean Value Theorem we know that
      \[
        \frac{f(y) - f(x)}{y - x} = f'(c)\leq 0,
      \]
      so $f(y) - f(x)\leq 0$ for all $x, y\in [a, b]$.
    \end{proof}
  \end{enumerate}
  \begin{example}{}
    Prove $\log x\leq x$ for all $x\geq x_0\geq 1$. Let us define $f(x) = x - \log x$, so
    \[
      f'(x) = 1 - \frac{1}{x} = \frac{x - 1}{x},
    \]
    which is positive for all $x > 1$. Hence $f$ is non-decreasing on $[1, \infty)$, so $f(x)\geq 1 > 0$. Thus $x\geq \log x$ for $x$ sufficiently large.
  \end{example}
\end{document}
