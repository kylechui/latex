\documentclass[class=article, crop=false]{standalone}
% Import packages
\usepackage[margin=1in]{geometry}

\usepackage{amssymb, amsthm}
\usepackage{comment}
\usepackage{enumitem}
\usepackage{fancyhdr}
\usepackage{hyperref}
\usepackage{import}
\usepackage{mathrsfs, mathtools}
\usepackage{multicol}
\usepackage{pdfpages}
\usepackage{standalone}
\usepackage{transparent}
\usepackage{xcolor}

\usepackage{minted}
\usepackage[many]{tcolorbox}
\tcbuselibrary{minted}

% Declare math operators
\DeclareMathOperator{\lcm}{lcm}
\DeclareMathOperator{\proj}{proj}
\DeclareMathOperator{\vspan}{span}
\DeclareMathOperator{\im}{im}
\DeclareMathOperator{\Diff}{Diff}
\DeclareMathOperator{\Int}{Int}
\DeclareMathOperator{\fcn}{fcn}
\DeclareMathOperator{\id}{id}
\DeclareMathOperator{\rank}{rank}
\DeclareMathOperator{\tr}{tr}
\DeclareMathOperator{\dive}{div}
\DeclareMathOperator{\row}{row}
\DeclareMathOperator{\col}{col}
\DeclareMathOperator{\dom}{dom}
\DeclareMathOperator{\ran}{ran}
\DeclareMathOperator{\Dom}{Dom}
\DeclareMathOperator{\Ran}{Ran}
\DeclareMathOperator{\fl}{fl}
% Macros for letters/variables
\newcommand\ttilde{\kern -.15em\lower .7ex\hbox{\~{}}\kern .04em}
\newcommand{\N}{\ensuremath{\mathbb{N}}}
\newcommand{\Z}{\ensuremath{\mathbb{Z}}}
\newcommand{\Q}{\ensuremath{\mathbb{Q}}}
\newcommand{\R}{\ensuremath{\mathbb{R}}}
\newcommand{\C}{\ensuremath{\mathbb{C}}}
\newcommand{\F}{\ensuremath{\mathbb{F}}}
\newcommand{\M}{\ensuremath{\mathbb{M}}}
\renewcommand{\P}{\ensuremath{\mathbb{P}}}
\newcommand{\lam}{\ensuremath{\lambda}}
\newcommand{\nab}{\ensuremath{\nabla}}
\newcommand{\eps}{\ensuremath{\varepsilon}}
\newcommand{\es}{\ensuremath{\varnothing}}
% Macros for math symbols
\newcommand{\dx}[1]{\,\mathrm{d}#1}
\newcommand{\inv}{\ensuremath{^{-1}}}
\newcommand{\sm}{\setminus}
\newcommand{\sse}{\subseteq}
\newcommand{\ceq}{\coloneqq}
% Macros for pairs of math symbols
\newcommand{\abs}[1]{\ensuremath{\left\lvert #1 \right\rvert}}
\newcommand{\paren}[1]{\ensuremath{\left( #1 \right)}}
\newcommand{\norm}[1]{\ensuremath{\left\lVert #1\right\rVert}}
\newcommand{\set}[1]{\ensuremath{\left\{#1\right\}}}
\newcommand{\tup}[1]{\ensuremath{\left\langle #1 \right\rangle}}
\newcommand{\floor}[1]{\ensuremath{\left\lfloor #1 \right\rfloor}}
\newcommand{\ceil}[1]{\ensuremath{\left\lceil #1 \right\rceil}}
\newcommand{\eclass}[1]{\ensuremath{\left[ #1 \right]}}

\newcommand{\chapternum}{}
\newcommand{\ex}[1]{\noindent\textbf{Exercise \chapternum.{#1}.}}

\newcommand{\tsub}[1]{\textsubscript{#1}}
\newcommand{\tsup}[1]{\textsuperscript{#1}}

\renewcommand{\choose}[2]{\ensuremath{\phantom{}_{#1}C_{#2}}}
\newcommand{\perm}[2]{\ensuremath{\phantom{}_{#1}P_{#2}}}

% Include figures
\newcommand{\incfig}[2][1]{%
    \def\svgwidth{#1\columnwidth}
    \import{./figures/}{#2.pdf_tex}
}

\definecolor{problemBackground}{RGB}{212,232,246}
\newenvironment{problem}[1]
  {
    \begin{tcolorbox}[
      boxrule=.5pt,
      titlerule=.5pt,
      sharp corners,
      colback=problemBackground,
      breakable
    ]
    \ifx &#1& \textbf{Problem. }
    \else \textbf{Problem #1.} \fi
  }
  {
    \end{tcolorbox}
  }
\definecolor{exampleBackground}{RGB}{255,249,248}
\definecolor{exampleAccent}{RGB}{158,60,14}
\newenvironment{example}[1]
  {
    \begin{tcolorbox}[
      boxrule=.5pt,
      sharp corners,
      colback=exampleBackground,
      colframe=exampleAccent,
    ]
    \color{exampleAccent}\textbf{Example.} \emph{#1}\color{black}
  }
  {
    \end{tcolorbox}
  }
\definecolor{theoremBackground}{RGB}{234,243,251}
\definecolor{theoremAccent}{RGB}{0,116,183}
\newenvironment{theorem}[1]
  {
    \begin{tcolorbox}[
      boxrule=.5pt,
      titlerule=.5pt,
      sharp corners,
      colback=theoremBackground,
      colframe=theoremAccent,
      breakable
    ]
      % \color{theoremAccent}\textbf{Theorem --- }\emph{#1}\\\color{black}
    \ifx &#1& \color{theoremAccent}\textbf{Theorem. }\color{black}
    \else \color{theoremAccent}\textbf{Theorem --- }\emph{#1}\\\color{black} \fi
  }
  {
    \end{tcolorbox}
  }
\definecolor{noteBackground}{RGB}{244,249,244}
\definecolor{noteAccent}{RGB}{34,139,34}
\newenvironment{note}[1]
  {
  \begin{tcolorbox}[
    enhanced,
    boxrule=0pt,
    frame hidden,
    sharp corners,
    colback=noteBackground,
    borderline west={3pt}{-1.5pt}{noteAccent},
    breakable
    ]
    \ifx &#1& \color{noteAccent}\textbf{Note. }\color{black}
    \else \color{noteAccent}\textbf{Note (#1). }\color{black} \fi
    }
    {
  \end{tcolorbox}
  }
\definecolor{lemmaBackground}{RGB}{255,247,234}
\definecolor{lemmaAccent}{RGB}{255,153,0}
\newenvironment{lemma}[1]
  {
    \begin{tcolorbox}[
      enhanced,
      boxrule=0pt,
      frame hidden,
      sharp corners,
      colback=lemmaBackground,
      borderline west={3pt}{-1.5pt}{lemmaAccent},
      breakable
    ]
    \ifx &#1& \color{lemmaAccent}\textbf{Lemma. }\color{black}
    \else \color{lemmaAccent}\textbf{Lemma #1. }\color{black} \fi
  }
  {
    \end{tcolorbox}
  }
\definecolor{definitionBackground}{RGB}{246,246,246}
\newenvironment{definition}[1]
  {
    \begin{tcolorbox}[
      enhanced,
      boxrule=0pt,
      frame hidden,
      sharp corners,
      colback=definitionBackground,
      borderline west={3pt}{-1.5pt}{black},
      breakable
    ]
    \textbf{Definition. }\emph{#1}\\
  }
  {
    \end{tcolorbox}
  }

\newenvironment{amatrix}[2]{
    \left[
      \begin{array}{*{#1}{c}|*{#2}c}
  }
  {
      \end{array}
    \right]
  }

\definecolor{codeBackground}{RGB}{253,246,227}
\renewcommand\theFancyVerbLine{\arabic{FancyVerbLine}}
\newtcblisting{cpp}[1][]{
  boxrule=1pt,
  sharp corners,
  colback=codeBackground,
  listing only,
  breakable,
  minted language=cpp,
  minted style=material,
  minted options={
    xleftmargin=15pt,
    baselinestretch=1.2,
    linenos,
  }
}

\author{Kyle Chui}


\fancyhf{}
\lhead{Kyle Chui}
\rhead{Page \thepage}
\pagestyle{fancy}

\begin{document}
  \section{Lecture 12}
  \begin{example}{}
    Consider the sequence defined by $x_1 = 2$, and for $n\geq 2$
    \[
      x_{n + 1} = \frac{1}{2}\paren{x_n + \frac{2}{x_n}}.
    \]
    We don't have a nice function for $x_n$ in terms of $n$, so proving convergence by normal means is non-ideal. To find the limit of $(x_n)$, we:
    \begin{itemize}
      \item Show that $(x_n)$ is monotonic (decreasing in this case) and bounded
      \item Apply the Monotonic Convergence Theorem, and say that there exists some $x\in\R$ such that $x_n\to x$
      \item Apply limit laws to actually find the value of $x$
      \begin{note}{}
        If $x_n\to x$, we know that $x_{n + 1}\to x$.
      \end{note}
    \end{itemize}
    Taking the first few terms, we see
    \[
      x_1 = 2, x_2 = \frac{3}{2}, \dotsc
    \]
    We guess that the sequence is always bounded, namely $1\leq x_n\leq 2$ for all $n\in\N$. We can prove this by induction. To show that the sequence is decreasing, we just need to show that for all $n\in\N$, we have $x_{n + 1} - x_n\leq 0$. We can then use induction again, along with the fact that the previous inequality can be expressed as a function of $x_n$, to show that the sequence is decreasing. \par
    We can see that the sequence $(x_n)$ is strictly greater than zero, so we may use the Algebraic Limit Theorem. Thus we have
    \begin{align*}
      x &= \frac{1}{2}\paren{x + \frac{2}{x}} \\
      2x &= x + \frac{2}{x} \\
      x &= \frac{2}{x} \\
      x^2 &= 2 \\
      x &= \sqrt{2}.
    \end{align*}
  \end{example}
  \subsection{Subsequences}
  Consider the diverging sequence given by $x_n = (-1)^n$. If we take the even terms, we see that $x_{2n} = 1$, and if we take the odd terms, we have $x_{2n + 1} = -1$. Thus we see that ``parts'' of our diverging sequence are actually convergent sequences.
  \begin{definition}{Subsequences}
    Let $(x_n)$ be a sequence and
    \[
      n_1 < n_2 < n_3 < \dotsb
    \]
    be a strictly increasing sequence of natural numbers. Then for $k\in\N$, the sequence $(x_{n_k})$ is a \emph{subsequence} of the original sequence $(x_n)$.
  \end{definition}
  \begin{example}{Subsequences}
    \begin{enumerate}
      \item Consider $x_n = (-1)^n$. Then $x_{2k} = 1$ is a subsequence. Similarly, $x_{2k + 1} = -1$ is also a subsequence.
      \item Consider $x_n = \frac{1}{n} = \paren{1, \frac{1}{2}, \frac{1}{3}, \dotsc}$. A valid subsequence could be $x_{n_k} = \frac{1}{2k}$ or $x_{n_k} = \frac{1}{10^k}$. However, the sequence given by
      \[
        n_1 = 10, n_2 = 50, n_3 = 20, n_4 = 5, n_5 = 1000
      \]
      is not a subsequence because the indices are not strictly increasing.
    \end{enumerate}
  \end{example}
  \textbf{Observation.} Suppose we have $(n_k)$ is strictly increasing. Then we know that $n_k\geq k$, because it is strictly increasing and starts at $1$. Then if $x_{n_k} = \frac{1}{n_k}$, then we see
  \[
    x_{n_k}\leq \frac{1}{k}\to 0.
  \]
  Thus we see that for \emph{any} subsequence of our convergent sequence $x_n$, it converges to the same limit as $(x_n)$. \\[10pt]
  \textbf{Proposition 3.12} Every subsequence of a converging sequence converges and does so to the same limit as the original sequence.
  \begin{lemma}{}
    We will show that $n_k\geq k$ for all $k\in\N$, assuming that $n_k$ is a strictly increasing sequence.
    \begin{proof}
      We proceed via induction. Observe that for $n_1\in\N$, we have $n_1\geq 1$. Suppose $n_k\geq k$ Then since $(n_k)$ is striclty increasing, we have $n_{k + 1} > n_k\geq k$, so $n_{k + 1}\geq k + 1$.
    \end{proof}
  \end{lemma}
  \begin{proof}
    Suppose $(x_n)$ converges to $x$. Let $(x_{n_k})$ be a subsequence of $(x_n)$. We claim that $x_{n_k}\to x$. Let $\eps > 0$. Then there exists some $N\in\N$ such that $\abs{x_n - x} < \eps$ for all $n > N$. Since $n_k\geq k$ for $k > N$, we have $\abs{x_{n_k} - x} < \eps$ for all $n_k > N$. Thus $x_{n_k}\to x$ as $k\to \infty$.
  \end{proof}
  \begin{note}{}
    If $x_n\to x$, then $x_{n + 1}\to x$ because $(x_{n + 1})$ is a subsequence of $(x_n)$.
  \end{note}
  \begin{note}{}
    Proposition $3.12$ can be used to prove divergence, by showing that two subsequences of $(x_n)$ converge to different values. For example, if $(x_n) = (-1)^n$, then
    \begin{align*}
      x_{2n} &= (-1)^{2n} = 1 \\
      x_{2n + 1} &= (-1)^{2n + 1} = -1,
    \end{align*}
    so $(x_n)$ diverges.
  \end{note}
  \textbf{Proposition 3.13} Every sequence has a monotonic subsequence.
  \begin{proof}
    We need to carefully select our subsequence. Let $(x_n)$ be a sequence and
    \[
      D = \set{n\in\N \mid x_n > x_m \text{ for all }m > n}\sse \N.
    \]
    If $n\in D$, then $x_n > x_m$ for all $m > n$. We say that if $n\in D$, that $x_n$ is \emph{dominant}. We now consider when $D$ is finite and when $D$ is infinite.
    \begin{enumerate}
      \item If $D$ is infinite, then there exists $\set{n_k}\sse D$ such that $x_{n_k}$ is dominant. Then $x_{n_{k + 1}} < x_{n_k}$ and so $(x_{n_k})$ is a subsequence which is decreasing, and so monotonic.
      \item If $D$ is finite, then there exists some $N$ such that $\max D = N$. Thus the last dominant term of our sequence is $x_N$. Hence there exists some $n_1 > N$ such that $x_{n_1} > x_{N + 1}$. However, since $n_1 > N$, we have $n_1\notin D$, so there must exist some $x_{n_2} > x_{n_1 + 1}$. By induction on $k\in\N$, we get $(n_k)$, so there exist
      \[
        n_1 < n_2 < n_3 < \dotsb
      \]
      such that
      \[
        x_{n_1} < x_{n_2} < \dotsb < x_{n_k} < x_N.
      \]
      Therefore $(x_{n_k})$ is an increasing subsequence, and so is monotonic.
    \end{enumerate}
  \end{proof}
  \begin{theorem}{3.14 (Bolzano--Weierstrass)}
    Every bounded sequence in $\R$ has a converging subsequence.
    \begin{proof}
      Suppose we have a bounded sequence $(x_n)$ in $\R$. Then by Proposition $3.13$ we know that there exists some subsequence $(x_{n_k})$ that is monotonic, which is also bounded. Thus by the Monotone Convergence Theorem we have that $(x_{n_k})$ converges in $\R$.
    \end{proof}
  \end{theorem}
  \begin{note}{}
    Bolzano--Weierstrass doesn't tell us anything about the original sequence, just that there are converging subsequences.
  \end{note}
\end{document}
