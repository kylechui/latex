\documentclass[class=article, crop=false]{standalone}
% Import packages
\usepackage[margin=1in]{geometry}

\usepackage{amssymb, amsthm}
\usepackage{comment}
\usepackage{enumitem}
\usepackage{fancyhdr}
\usepackage{hyperref}
\usepackage{import}
\usepackage{mathrsfs, mathtools}
\usepackage{multicol}
\usepackage{pdfpages}
\usepackage{standalone}
\usepackage{transparent}
\usepackage{xcolor}

\usepackage{minted}
\usepackage[many]{tcolorbox}
\tcbuselibrary{minted}

% Declare math operators
\DeclareMathOperator{\lcm}{lcm}
\DeclareMathOperator{\proj}{proj}
\DeclareMathOperator{\vspan}{span}
\DeclareMathOperator{\im}{im}
\DeclareMathOperator{\Diff}{Diff}
\DeclareMathOperator{\Int}{Int}
\DeclareMathOperator{\fcn}{fcn}
\DeclareMathOperator{\id}{id}
\DeclareMathOperator{\rank}{rank}
\DeclareMathOperator{\tr}{tr}
\DeclareMathOperator{\dive}{div}
\DeclareMathOperator{\row}{row}
\DeclareMathOperator{\col}{col}
\DeclareMathOperator{\dom}{dom}
\DeclareMathOperator{\ran}{ran}
\DeclareMathOperator{\Dom}{Dom}
\DeclareMathOperator{\Ran}{Ran}
\DeclareMathOperator{\fl}{fl}
% Macros for letters/variables
\newcommand\ttilde{\kern -.15em\lower .7ex\hbox{\~{}}\kern .04em}
\newcommand{\N}{\ensuremath{\mathbb{N}}}
\newcommand{\Z}{\ensuremath{\mathbb{Z}}}
\newcommand{\Q}{\ensuremath{\mathbb{Q}}}
\newcommand{\R}{\ensuremath{\mathbb{R}}}
\newcommand{\C}{\ensuremath{\mathbb{C}}}
\newcommand{\F}{\ensuremath{\mathbb{F}}}
\newcommand{\M}{\ensuremath{\mathbb{M}}}
\renewcommand{\P}{\ensuremath{\mathbb{P}}}
\newcommand{\lam}{\ensuremath{\lambda}}
\newcommand{\nab}{\ensuremath{\nabla}}
\newcommand{\eps}{\ensuremath{\varepsilon}}
\newcommand{\es}{\ensuremath{\varnothing}}
% Macros for math symbols
\newcommand{\dx}[1]{\,\mathrm{d}#1}
\newcommand{\inv}{\ensuremath{^{-1}}}
\newcommand{\sm}{\setminus}
\newcommand{\sse}{\subseteq}
\newcommand{\ceq}{\coloneqq}
% Macros for pairs of math symbols
\newcommand{\abs}[1]{\ensuremath{\left\lvert #1 \right\rvert}}
\newcommand{\paren}[1]{\ensuremath{\left( #1 \right)}}
\newcommand{\norm}[1]{\ensuremath{\left\lVert #1\right\rVert}}
\newcommand{\set}[1]{\ensuremath{\left\{#1\right\}}}
\newcommand{\tup}[1]{\ensuremath{\left\langle #1 \right\rangle}}
\newcommand{\floor}[1]{\ensuremath{\left\lfloor #1 \right\rfloor}}
\newcommand{\ceil}[1]{\ensuremath{\left\lceil #1 \right\rceil}}
\newcommand{\eclass}[1]{\ensuremath{\left[ #1 \right]}}

\newcommand{\chapternum}{}
\newcommand{\ex}[1]{\noindent\textbf{Exercise \chapternum.{#1}.}}

\newcommand{\tsub}[1]{\textsubscript{#1}}
\newcommand{\tsup}[1]{\textsuperscript{#1}}

\renewcommand{\choose}[2]{\ensuremath{\phantom{}_{#1}C_{#2}}}
\newcommand{\perm}[2]{\ensuremath{\phantom{}_{#1}P_{#2}}}

% Include figures
\newcommand{\incfig}[2][1]{%
    \def\svgwidth{#1\columnwidth}
    \import{./figures/}{#2.pdf_tex}
}

\definecolor{problemBackground}{RGB}{212,232,246}
\newenvironment{problem}[1]
  {
    \begin{tcolorbox}[
      boxrule=.5pt,
      titlerule=.5pt,
      sharp corners,
      colback=problemBackground,
      breakable
    ]
    \ifx &#1& \textbf{Problem. }
    \else \textbf{Problem #1.} \fi
  }
  {
    \end{tcolorbox}
  }
\definecolor{exampleBackground}{RGB}{255,249,248}
\definecolor{exampleAccent}{RGB}{158,60,14}
\newenvironment{example}[1]
  {
    \begin{tcolorbox}[
      boxrule=.5pt,
      sharp corners,
      colback=exampleBackground,
      colframe=exampleAccent,
    ]
    \color{exampleAccent}\textbf{Example.} \emph{#1}\color{black}
  }
  {
    \end{tcolorbox}
  }
\definecolor{theoremBackground}{RGB}{234,243,251}
\definecolor{theoremAccent}{RGB}{0,116,183}
\newenvironment{theorem}[1]
  {
    \begin{tcolorbox}[
      boxrule=.5pt,
      titlerule=.5pt,
      sharp corners,
      colback=theoremBackground,
      colframe=theoremAccent,
      breakable
    ]
      % \color{theoremAccent}\textbf{Theorem --- }\emph{#1}\\\color{black}
    \ifx &#1& \color{theoremAccent}\textbf{Theorem. }\color{black}
    \else \color{theoremAccent}\textbf{Theorem --- }\emph{#1}\\\color{black} \fi
  }
  {
    \end{tcolorbox}
  }
\definecolor{noteBackground}{RGB}{244,249,244}
\definecolor{noteAccent}{RGB}{34,139,34}
\newenvironment{note}[1]
  {
  \begin{tcolorbox}[
    enhanced,
    boxrule=0pt,
    frame hidden,
    sharp corners,
    colback=noteBackground,
    borderline west={3pt}{-1.5pt}{noteAccent},
    breakable
    ]
    \ifx &#1& \color{noteAccent}\textbf{Note. }\color{black}
    \else \color{noteAccent}\textbf{Note (#1). }\color{black} \fi
    }
    {
  \end{tcolorbox}
  }
\definecolor{lemmaBackground}{RGB}{255,247,234}
\definecolor{lemmaAccent}{RGB}{255,153,0}
\newenvironment{lemma}[1]
  {
    \begin{tcolorbox}[
      enhanced,
      boxrule=0pt,
      frame hidden,
      sharp corners,
      colback=lemmaBackground,
      borderline west={3pt}{-1.5pt}{lemmaAccent},
      breakable
    ]
    \ifx &#1& \color{lemmaAccent}\textbf{Lemma. }\color{black}
    \else \color{lemmaAccent}\textbf{Lemma #1. }\color{black} \fi
  }
  {
    \end{tcolorbox}
  }
\definecolor{definitionBackground}{RGB}{246,246,246}
\newenvironment{definition}[1]
  {
    \begin{tcolorbox}[
      enhanced,
      boxrule=0pt,
      frame hidden,
      sharp corners,
      colback=definitionBackground,
      borderline west={3pt}{-1.5pt}{black},
      breakable
    ]
    \textbf{Definition. }\emph{#1}\\
  }
  {
    \end{tcolorbox}
  }

\newenvironment{amatrix}[2]{
    \left[
      \begin{array}{*{#1}{c}|*{#2}c}
  }
  {
      \end{array}
    \right]
  }

\definecolor{codeBackground}{RGB}{253,246,227}
\renewcommand\theFancyVerbLine{\arabic{FancyVerbLine}}
\newtcblisting{cpp}[1][]{
  boxrule=1pt,
  sharp corners,
  colback=codeBackground,
  listing only,
  breakable,
  minted language=cpp,
  minted style=material,
  minted options={
    xleftmargin=15pt,
    baselinestretch=1.2,
    linenos,
  }
}

\author{Kyle Chui}


\fancyhf{}
\lhead{Kyle Chui}
\rhead{Page \thepage}
\pagestyle{fancy}

\begin{document}
  \section{Lecture 21}
  \begin{note}{Why do we like compact sets?}
    \begin{itemize}
      \item In general, continuous functions don't map closed sets to closed sets.
      \item Continuous functions map compact sets to compact sets.
      \item Over compact sets, continuous functions have both minima and maxima.
    \end{itemize}
  \end{note}
  \begin{theorem}{4.16}
    Let $f\colon A\to \R$ be continuous on $A$. If $K\sse A$ is compact, then $f(K)$ is also compact. In particular, $f$ is bounded on $K$ (by Heine--Borel):
    \[
      \exists M > 0 \text{ such that } \abs{f(x)}\leq M,
    \]
    for all $x\in K$.
    \begin{proof}
      Let $(y_n)\sse f(K)$, so there exists some sequence $(x_n)\sse K$ such that $f(x_n) = y_n$. We know that $K$ is compact, so there exists a subsequence $(x_{n_k})$ such that $x_{n_k}\to x\in K$. Thus we have $y_{n_k} = f(x_{n_k})\to f(x)\in f(K)$.
    \end{proof}
  \end{theorem}
  \begin{theorem}{4.17 (Extreme Value Theorem)}
    If $f\colon K\to \R$ is continuous on compact $K$, then there exists $x_1, x_2\in K$ such that
    \[
      f(x_1) \leq f(x) \leq f(x_2),
    \]
    for all $x\in K$. In other words, $f$ has a maxima and minima over $K$.
    \begin{lemma}{}
      If a non-empty set $K\sse\R$ is compact, then it has a maxima and a minima.
      \begin{proof}
        We begin by showing that $K$ has a maxima. By Heine--Borel, if a set is compact it must be closed and bounded, so $K$ is closed and bounded. Since $K$ is bounded and non-empty, it must have a supremum, and so $\sup K$ exists. We know that there exists a sequence $(x_n)\sse K$ that converges to $\sup K$. However, $K$ is closed so $\sup K\in K$, and so $K$ has a maxima. \par
        We may apply similar logic to show that $K$ has a minima. Since $K$ is bounded and non-empty, it must have an infimum, so $\inf K$ exists. We know that there exists a sequence $(x_n)\sse K$ that converges to $\inf K$. Because $K$ is closed we have $\inf K\in K$, so $K$ has a minima.
      \end{proof}
    \end{lemma}
    \begin{proof}
      By Theorem $4.16$, we have that $f(K)$ is compact. Thus $f(K)$ has maxima and minima.
    \end{proof}
  \end{theorem}
  \subsection{The Intermediate Value Theorem}
  \begin{theorem}{4.18 (Intermediate Value Theorem)}
    Let $f\colon [a, b]\to \R$ be continuous on $[a, b]$ and let $L\in \R$ such that
    \begin{align*}
      f(a) < L &< f(b) \quad \text{or} \\
      f(b) < L &< f(a).
    \end{align*}
    Then, there exists some $c\in (a, b)$ such that $f(c) = L$.
    \begin{proof}
      We have some simplifying assumptions that make the proof easier:
      \begin{itemize}
        \item By shifting $f$, we may assume that $L = 0$. What this means is that we shift the function $f$ up or down until our $L$ becomes $0$.
        \item We suppose that
        \[
          f(a) < 0 < f(b),
        \]
        as the other case is similar.
      \end{itemize}
      Consider the set $A = \set{x\in [a, b]\mid f(x)\leq 0}$. We claim that $\sup A$ not only exists, but also is in $[a, b]$. Since $f(a) < 0$, we have $a\in A$ and so $A\neq \es$. Furthermore, $A$ is bounded above by $b$, and so by the least upper bound property we have $c = \sup A$ exists. Note that $c\in (a, b)$ because $a\leq c\leq b$. \par
      We now show that $f(c) = 0$. Suppose towards a contradiction that $f(c) > 0$. Since $c = \sup A$, there exists some sequence $(x_n)\sse A$ such that $x_n\to c$. Since $f$ is continuous we have $f(x_n)\to f(c)$. Furthermore, as $(x_n)\sse A$ we have $f(x_n) \leq 0$ for all $n$, so $f(c) \leq 0$, a contradiction. \par
      Now suppose towards a contradiction that $f(c) < 0$. Let $z_n = \min(b, c + \frac{1}{n})$, so $(z_n)\sse [a, b]\sm A$. We know that $z_n\to c$, and by continuity of $f$, $f(z_n)\to f(c)$. Since $(z_n)$ is not a subset of $A$, $f(z_n) > 0$. Thus by Ordered Limit Theorem we have $f(c) \geq 0$, a contradiction. \par
      Therefore $f(c) = 0$.
    \end{proof}
  \end{theorem}
  \begin{note}{}
    The zero found by the Intermediate Value Theorem is \emph{not necessarily unique}.
  \end{note}
  \subsubsection{Consequences of Intermediate Value Theorem}
  \textbf{Corollary 4.19} If $f$ is continuous on an interval $I$, then $f(I)$ is an interval or a single point. \par
  \begin{theorem}{Inverse Function Theorem}
    Let $I$ be an interval and $f\colon I\to \R$ be strictly increasing (so $f(x) < f(y)$ if $x < y$) and continuous on $I$. Then $f\inv\colon f(I)\to \R$ exists, and is strictly increasing/decreasing and continuous on $f(I)$.
  \end{theorem}
\end{document}
