\documentclass[class=article, crop=false]{standalone}
% Import packages
\usepackage[margin=1in]{geometry}

\usepackage{amssymb, amsthm}
\usepackage{comment}
\usepackage{enumitem}
\usepackage{fancyhdr}
\usepackage{hyperref}
\usepackage{import}
\usepackage{mathrsfs, mathtools}
\usepackage{multicol}
\usepackage{pdfpages}
\usepackage{standalone}
\usepackage{transparent}
\usepackage{xcolor}

\usepackage{minted}
\usepackage[many]{tcolorbox}
\tcbuselibrary{minted}

% Declare math operators
\DeclareMathOperator{\lcm}{lcm}
\DeclareMathOperator{\proj}{proj}
\DeclareMathOperator{\vspan}{span}
\DeclareMathOperator{\im}{im}
\DeclareMathOperator{\Diff}{Diff}
\DeclareMathOperator{\Int}{Int}
\DeclareMathOperator{\fcn}{fcn}
\DeclareMathOperator{\id}{id}
\DeclareMathOperator{\rank}{rank}
\DeclareMathOperator{\tr}{tr}
\DeclareMathOperator{\dive}{div}
\DeclareMathOperator{\row}{row}
\DeclareMathOperator{\col}{col}
\DeclareMathOperator{\dom}{dom}
\DeclareMathOperator{\ran}{ran}
\DeclareMathOperator{\Dom}{Dom}
\DeclareMathOperator{\Ran}{Ran}
\DeclareMathOperator{\fl}{fl}
% Macros for letters/variables
\newcommand\ttilde{\kern -.15em\lower .7ex\hbox{\~{}}\kern .04em}
\newcommand{\N}{\ensuremath{\mathbb{N}}}
\newcommand{\Z}{\ensuremath{\mathbb{Z}}}
\newcommand{\Q}{\ensuremath{\mathbb{Q}}}
\newcommand{\R}{\ensuremath{\mathbb{R}}}
\newcommand{\C}{\ensuremath{\mathbb{C}}}
\newcommand{\F}{\ensuremath{\mathbb{F}}}
\newcommand{\M}{\ensuremath{\mathbb{M}}}
\renewcommand{\P}{\ensuremath{\mathbb{P}}}
\newcommand{\lam}{\ensuremath{\lambda}}
\newcommand{\nab}{\ensuremath{\nabla}}
\newcommand{\eps}{\ensuremath{\varepsilon}}
\newcommand{\es}{\ensuremath{\varnothing}}
% Macros for math symbols
\newcommand{\dx}[1]{\,\mathrm{d}#1}
\newcommand{\inv}{\ensuremath{^{-1}}}
\newcommand{\sm}{\setminus}
\newcommand{\sse}{\subseteq}
\newcommand{\ceq}{\coloneqq}
% Macros for pairs of math symbols
\newcommand{\abs}[1]{\ensuremath{\left\lvert #1 \right\rvert}}
\newcommand{\paren}[1]{\ensuremath{\left( #1 \right)}}
\newcommand{\norm}[1]{\ensuremath{\left\lVert #1\right\rVert}}
\newcommand{\set}[1]{\ensuremath{\left\{#1\right\}}}
\newcommand{\tup}[1]{\ensuremath{\left\langle #1 \right\rangle}}
\newcommand{\floor}[1]{\ensuremath{\left\lfloor #1 \right\rfloor}}
\newcommand{\ceil}[1]{\ensuremath{\left\lceil #1 \right\rceil}}
\newcommand{\eclass}[1]{\ensuremath{\left[ #1 \right]}}

\newcommand{\chapternum}{}
\newcommand{\ex}[1]{\noindent\textbf{Exercise \chapternum.{#1}.}}

\newcommand{\tsub}[1]{\textsubscript{#1}}
\newcommand{\tsup}[1]{\textsuperscript{#1}}

\renewcommand{\choose}[2]{\ensuremath{\phantom{}_{#1}C_{#2}}}
\newcommand{\perm}[2]{\ensuremath{\phantom{}_{#1}P_{#2}}}

% Include figures
\newcommand{\incfig}[2][1]{%
    \def\svgwidth{#1\columnwidth}
    \import{./figures/}{#2.pdf_tex}
}

\definecolor{problemBackground}{RGB}{212,232,246}
\newenvironment{problem}[1]
  {
    \begin{tcolorbox}[
      boxrule=.5pt,
      titlerule=.5pt,
      sharp corners,
      colback=problemBackground,
      breakable
    ]
    \ifx &#1& \textbf{Problem. }
    \else \textbf{Problem #1.} \fi
  }
  {
    \end{tcolorbox}
  }
\definecolor{exampleBackground}{RGB}{255,249,248}
\definecolor{exampleAccent}{RGB}{158,60,14}
\newenvironment{example}[1]
  {
    \begin{tcolorbox}[
      boxrule=.5pt,
      sharp corners,
      colback=exampleBackground,
      colframe=exampleAccent,
    ]
    \color{exampleAccent}\textbf{Example.} \emph{#1}\color{black}
  }
  {
    \end{tcolorbox}
  }
\definecolor{theoremBackground}{RGB}{234,243,251}
\definecolor{theoremAccent}{RGB}{0,116,183}
\newenvironment{theorem}[1]
  {
    \begin{tcolorbox}[
      boxrule=.5pt,
      titlerule=.5pt,
      sharp corners,
      colback=theoremBackground,
      colframe=theoremAccent,
      breakable
    ]
      % \color{theoremAccent}\textbf{Theorem --- }\emph{#1}\\\color{black}
    \ifx &#1& \color{theoremAccent}\textbf{Theorem. }\color{black}
    \else \color{theoremAccent}\textbf{Theorem --- }\emph{#1}\\\color{black} \fi
  }
  {
    \end{tcolorbox}
  }
\definecolor{noteBackground}{RGB}{244,249,244}
\definecolor{noteAccent}{RGB}{34,139,34}
\newenvironment{note}[1]
  {
  \begin{tcolorbox}[
    enhanced,
    boxrule=0pt,
    frame hidden,
    sharp corners,
    colback=noteBackground,
    borderline west={3pt}{-1.5pt}{noteAccent},
    breakable
    ]
    \ifx &#1& \color{noteAccent}\textbf{Note. }\color{black}
    \else \color{noteAccent}\textbf{Note (#1). }\color{black} \fi
    }
    {
  \end{tcolorbox}
  }
\definecolor{lemmaBackground}{RGB}{255,247,234}
\definecolor{lemmaAccent}{RGB}{255,153,0}
\newenvironment{lemma}[1]
  {
    \begin{tcolorbox}[
      enhanced,
      boxrule=0pt,
      frame hidden,
      sharp corners,
      colback=lemmaBackground,
      borderline west={3pt}{-1.5pt}{lemmaAccent},
      breakable
    ]
    \ifx &#1& \color{lemmaAccent}\textbf{Lemma. }\color{black}
    \else \color{lemmaAccent}\textbf{Lemma #1. }\color{black} \fi
  }
  {
    \end{tcolorbox}
  }
\definecolor{definitionBackground}{RGB}{246,246,246}
\newenvironment{definition}[1]
  {
    \begin{tcolorbox}[
      enhanced,
      boxrule=0pt,
      frame hidden,
      sharp corners,
      colback=definitionBackground,
      borderline west={3pt}{-1.5pt}{black},
      breakable
    ]
    \textbf{Definition. }\emph{#1}\\
  }
  {
    \end{tcolorbox}
  }

\newenvironment{amatrix}[2]{
    \left[
      \begin{array}{*{#1}{c}|*{#2}c}
  }
  {
      \end{array}
    \right]
  }

\definecolor{codeBackground}{RGB}{253,246,227}
\renewcommand\theFancyVerbLine{\arabic{FancyVerbLine}}
\newtcblisting{cpp}[1][]{
  boxrule=1pt,
  sharp corners,
  colback=codeBackground,
  listing only,
  breakable,
  minted language=cpp,
  minted style=material,
  minted options={
    xleftmargin=15pt,
    baselinestretch=1.2,
    linenos,
  }
}

\author{Kyle Chui}


\fancyhf{}
\lhead{Kyle Chui}
\rhead{Page \thepage}
\pagestyle{fancy}

\begin{document}
  \section{Lecture 6}
  \subsection{The Rationals}
  On the natural numbers, we have a notion of addition, multiplication, and comparison ($\leq$). We constructed the integers and so we now have:
  \begin{itemize}
    \item A notion of an additive identity, $0\in\Z$
    \item Additive inverses
  \end{itemize}
  However, we \emph{don't} have:
  \begin{itemize}
    \item Multiplicative inverses
  \end{itemize}
  When we think of the rationals, we consider the set
  \[
    \Q = \set{\frac{p}{q}\mid p, q\in\Z, q\neq 0}.
  \]
  \begin{note}{}
    When dealing with $\Q$, we now have multiplicative inverses.
  \end{note}
  In particular, $(\Q, +, \cdot, \leq)$ is an \emph{ordered field}.
  \begin{definition}{Ordered Field}
    An \emph{ordered field} is a field $\F$ which is also an ordered set ($\leq$) such that:
    \begin{enumerate}[label=(\roman*)]
      \item If $x, y, z\in\F$ and $y < z$, then $x + y < x + z$
      \item If $x, y\in\F$, $x > 0$ and $y > 0$, then $x\cdot y > 0$
    \end{enumerate}
  \end{definition}
  Unfortunately, the rational numbers still don't allow us to solve polynomial equations (i.e. $x^2 = 2$).
  \begin{definition}{Algebraic Numbers}
    A number is called \emph{algebraic} if it solves
    \[
      c_nx^n + c_{n - 1}x^{n - 1} + \dotsb + c_1x + c_0 = 0,
    \]
    where $c_0,\dotsc,c_n\in\Z$, $c_n\neq 0$, $n\in\N$.
  \end{definition}
  \begin{example}{}
    Every rational number is algebraic, because it solves the equation
    \[
      qx - p = 0.
    \]
  \end{example}
  \begin{definition}{Dividing}
    We say $k\in\Z$ \emph{divides} $m\in\Z$ if $\frac{m}{k}\in\Z$.
  \end{definition}
  \begin{theorem}{Rational Zeros Theorem}
    Suppose $c_0,\dotsc,c_n\in\Z$, $c_n\neq 0$ and $r\in\Q$ satisfies
    \[
      c_nr^n + c_{n-1}r^{n-1} + \dotsb + c_1r + c_0 = 0.
    \]
    Then writing $r = \frac{c}{d}$ with $c, d$ having no common factors, $d\neq 0$, we have:
    \begin{align*}
      c \text{ divides } c_0 \\
      d \text{ divides } c_n
    \end{align*}
    \begin{proof}
      Since $r$ solves the equation, we have
      \[
        c_n \paren{\frac{c}{d}}^n + c_{n-1} \paren{\frac{c}{d}}^{n-1} + \dotsb + c_1 \paren{\frac{c}{d}} + c_0 = 0.
      \]
      Multiplying both sides by $d^n$, we get
      \[
        c_nc^n + c_{n-1}c^{n-1}d + \dotsb + c_1cd^{n-1} + c_0d^n = 0.
      \]
      Rearranging, we have
      \[
        c_nc^n =  -d(c_{n-1}c^{n-1} - \dotsb + c_1cd^{n-2} - c_0d^{n-1}).
      \]
      We know that $d\mid c_nc^n$. Since $c$ and $d$ have no common factors, we know $d\mid c_n$.
      Rearranging terms again,
      \[
        -c(c_nc^{n-1} + c_{n-1}c^{n-2}d + \dotsb + c_1d^{n-1}) = c_0d^n.
      \]
      By the same reasoning as before, we have that $c\mid c_0$.
    \end{proof}
  \end{theorem}
  \textbf{Corollary.} Suppose $r\in\Q$ solves
  \[
    r^n + c_{n-1}r^{n-1} + \dotsb + c_1r + c_0 = 0.
  \]
  Then $r\in\Z$ and $r\mid c_0$.
  \begin{proof}
    Since $r\in\Q$, we may express $r = \frac{c}{d}$, $c\mid c_0$ and $d\mid 1$. From this we know that $d = 1$, so $r = c$ and $c\mid c_0$. Therefore $r\in\Z$ and $r\mid c_0$.
  \end{proof}
  We have some deficiencies for $\Q$:
  \begin{itemize}
    \item There seem to be some ``gaps'' in $\Q$.
  \end{itemize}
  \textbf{Proposition.} We consider the sets $A = \set{p\in\Q\mid p\geq 0 \text{ and }p^2 < 2}$ and $B = \set{p\in\Q\mid p\geq 0 \text{ and } p^2 > 2}$. Notice that $A$ has no largest element and $B$ has no smallest element.
  \begin{proof}
    Given $p\in\Q$, let
    \[
      q = p - \frac{p^2-2}{p^2+2} = \frac{2(p + 1)}{p + 2}.
    \]
    We also have $q^2 - 2 = \frac{2(p^2-2)}{(p+2)^2}$. If $p\in A$, $p^2 - 2 < 0$ so $q > p$ and $q^2 - 2 > 0$. If $p\in B$, $p^2 - 2 > 0$ so $q < p$ and $q^2 - 2 > 0$.
  \end{proof}
\end{document}
