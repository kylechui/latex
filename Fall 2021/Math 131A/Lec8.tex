\documentclass[class=article, crop=false]{standalone}
\usepackage[margin=1in]{geometry}

\usepackage{amsmath}
\usepackage{amssymb}
\usepackage{amsthm}
\usepackage{comment}
\usepackage{enumitem}
\usepackage{fancyhdr}
\usepackage{hyperref}
\usepackage{mathrsfs}
\usepackage{mathtools}
\usepackage{standalone}
\usepackage{tcolorbox}
\usepackage{tikz}
\usepackage{xcolor}

\usetikzlibrary{decorations.pathreplacing}
\tcbuselibrary{skins}

\DeclareMathOperator{\lcm}{lcm}
\DeclareMathOperator{\proj}{proj}
\DeclareMathOperator{\vspan}{span}
\DeclareMathOperator{\im}{im}
\DeclareMathOperator{\range}{range}
\DeclareMathOperator{\Diff}{Diff}
\DeclareMathOperator{\Int}{Int}
\DeclareMathOperator{\fcn}{fcn}
\DeclareMathOperator{\id}{id}

\newcommand{\N}{\ensuremath{\mathbb{N}}}
\newcommand{\Z}{\ensuremath{\mathbb{Z}}}
\newcommand{\Q}{\ensuremath{\mathbb{Q}}}
\newcommand{\R}{\ensuremath{\mathbb{R}}}
\newcommand{\C}{\ensuremath{\mathbb{C}}}
\newcommand{\F}{\ensuremath{\mathbb{F}}}
\newcommand{\lam}{\ensuremath{\lambda}}
\newcommand{\nab}{\ensuremath{\nabla}}
\newcommand{\eps}{\ensuremath{\varepsilon}}
\newcommand{\es}{\ensuremath{\varnothing}}

\newcommand{\abs}[1]{\ensuremath{\left\lvert #1 \right\rvert}}
\newcommand{\bigpar}[1]{\ensuremath{\left( #1 \right)}}
\newcommand{\norm}[1]{\ensuremath{\lVert #1\rVert}}
\newcommand{\set}[1]{\ensuremath{\left\{#1\right\}}}
\newcommand{\tuple}[1]{\ensuremath{\left\langle #1 \right\rangle}}
\newcommand{\floor}[1]{\ensuremath{\left\lfloor #1 \right\rfloor}}
\newcommand{\ceil}[1]{\ensuremath{\left\lceil #1 \right\rceil}}

\newcommand{\chapternum}{}
\newcommand{\ex}[1]{\noindent\textbf{Exercise \chapternum.{#1}.}}

\newcommand{\dx}[1]{\,\mathrm{d}#1}
\newcommand{\inv}{\ensuremath{^{-1}}}
\newcommand{\sm}{\setminus}
\newcommand{\sse}{\subseteq}
\newcommand{\ceq}{\coloneqq}

\definecolor{problemBackground}{RGB}{212,232,246}
\newenvironment{problem}[1]
  {
    \begin{tcolorbox}[
      boxrule=.5pt,
      titlerule=.5pt,
      sharp corners,
      colback=problemBackground
    ]
    \ifx &#1& \textbf{Problem. }
    \else \textbf{Problem #1.} \fi
  }
  {
    \end{tcolorbox}
  }

\definecolor{exampleBackground}{RGB}{255,249,248}
\definecolor{exampleAccent}{RGB}{158,60,14}
\newenvironment{example}[1]
  {
    \begin{tcolorbox}[
      boxrule=.5pt,
      sharp corners,
      colback=exampleBackground,
      colframe=exampleAccent,
    ]
    \color{exampleAccent}\textbf{Example.} \emph{#1}\color{black}
  }
  {
    \end{tcolorbox}
  }

\definecolor{theoremBackground}{RGB}{234,243,251}
\definecolor{theoremAccent}{RGB}{0,116,183}
\newenvironment{theorem}[1]
  {
    \begin{tcolorbox}[
      boxrule=.5pt,
      titlerule=.5pt,
      sharp corners,
      colback=theoremBackground,
      colframe=theoremAccent
    ]
      \color{theoremAccent}\textbf{Theorem --- }\emph{#1}\\\color{black}
  }
  {
    \end{tcolorbox}
  }

\definecolor{noteBackground}{RGB}{244,249,244}
\definecolor{noteAccent}{RGB}{34,139,34}
\newenvironment{note}[1]
  {
  \begin{tcolorbox}[
    enhanced,
    boxrule=0pt,
    frame hidden,
    sharp corners,
    colback=noteBackground,
    borderline west={3pt}{-1.5pt}{noteAccent}
    ]
    \ifx &#1& \color{noteAccent}\textbf{Note. }\color{black}
    \else \color{noteAccent}\textbf{Note (#1). }\color{black} \fi
    }
    {
  \end{tcolorbox}
  }

\definecolor{definitionBackground}{RGB}{246,246,246}
\newenvironment{definition}[1]
  {
    \begin{tcolorbox}[
      enhanced,
      boxrule=0pt,
      frame hidden,
      sharp corners,
      colback=definitionBackground,
      borderline west={3pt}{-1.5pt}{black}
    ]
    \textbf{Definition. }\emph{#1}\\
  }
  {
    \end{tcolorbox}
  }
\newenvironment{amatrix}[2]{
    \left[
      \begin{array}{*{#1}{c}|*{#2}c}
  }
  {
      \end{array}
    \right]
  }
\date{\the\year-\the\month-\the\day}
\author{Kyle Chui}


\fancyhf{}
\lhead{Kyle Chui}
\rhead{Page \thepage}
\pagestyle{fancy}

\begin{document}
  \section{Lecture 8}
  \subsection{A Construction of the Real Numbers}
  We want to show that there exists some field $\R$ such that $\Q$ is a sub-field and $\R$ has the least upper bound property. This construction will be via Dedekind cuts.
  \begin{definition}{2.17---Cut}
    A \emph{cut} in $\Q$ is a pair of subsets $A, B\subset \Q$ such that:
    \begin{enumerate}[label=(\roman*)]
      \item $A\cap B = \es$, $A\cup B = \Q$, $A, B\neq \es$ (They partition $\Q$)
      \item If $a\in A$ and $b\in B$, then $a < b$
      \item $A$ contains no largest element
    \end{enumerate}
  \end{definition}
  \begin{example}{}
    An example of such a cut could be $A = \set{p\in\Q\mid p < 1}$ and $B = \set{p\in\Q\mid p\geq 1}$. We say that $A\mid B$ is a cut. Another such example is $A = \set{p\in\Q\mid p\leq 0 \text{ or } p^2 < 2}$ and $B = \set{p\in\Q\mid p > 0 \text{ and }p^2\geq 2}$
  \end{example}
  \begin{definition}{2.18---The Reals}
    We may define
    \[
      \R = \set{X\mid X \text{ is a cut in }\Q}.
    \]
  \end{definition}
  In order to show that the above is a valid definition for $\R$, we must show:
  \begin{enumerate}[label=(\roman*)]
    \item $\Q$ is contained in $\R$ in some ``natural way''
    \item $\R$ is an ordered field
    \item $\R$ has the least upper bound property
  \end{enumerate}
  \begin{definition}{2.19---Partial Order for the Reals}
    We define a \emph{partial order} on $\R$ as follows:
    If $X = A\mid B$ and $Y = C\mid D$, then we say $X < Y$ if and only if $A\subset C$ and $X\leq Y$ if and only if $A\sse C$.
  \end{definition}
  We will show that the reals contain the rationals.
  \begin{proof}
    We will begin by showing that $\Q\sse\R$. We say that $A\mid B$ is a \emph{rational cut} if for some $c\in\Q$, we have $A = \set{p\in\Q\mid p < c}$ and $B = \Q\sm A$. We will use $c^*$ to denote the rational cut at $c$. Then we may associate every $c\in\Q$ with a corresponding rational cut $c^*\in\R$.
  \end{proof}
  \begin{theorem}{2.20 (The Reals have the LUBP)}
    With respect to the partial order $\leq$ defined earlier, we may show that $\R$ has the least upper bound property.
    \begin{proof}
      Let $\mathscr{C}$ be any non-empty collection of cuts which are bounded above, say by the cut $X$. We want to show that $\sup \mathscr{C}$ exists in $\R$, so $\sup \mathscr{C}$ is itself a cut. A candidate for $\sup \mathscr{C}$ is $C\mid D$, where
      \[
        C = \set{a\in\Q\mid \text{There exists a cut } A\mid B\in \mathscr{C} \text{ such that }a\in A}.
      \]
      Now, $Z = C\mid D$ is a cut in $\Q$. We claim that $C$ has no largest element. If $a\in C$, then there exists a cut $A\mid B\in \mathscr{C}$ such that $a\in A$. Since $A$ is a part of a cut, it has no largest element, so there exists some $a'\in A$ such that $a < a'$. Thus $a'\in C$ so $C$ has no largest element. \par
      We must now show that $Z$ is the least upper bound of $\mathscr{C}$. For any $A\mid B\in \mathscr{C}, A\sse C$. Therefore $A\mid B\leq C\mid D = Z$ and $Z$ is an upper bound for $\mathscr{C}$. We must now show that it is the \emph{least} upper bound. Let $Z' = C'\mid D'$ be an upper bound for $\mathscr{C}$. Then $A\mid B\leq C'\mid D'$ and $A\sse C'$ for any $A\mid B\in \mathscr{C}$. Now by definition of $C$, we have $C\sse C'$. Therefore $C\mid D\leq C'\mid D'$, so $Z\leq Z'$. We have $Z = \sup \mathscr{C}\in \R$, so $\R$ has the least upper bound property.
    \end{proof}
  \end{theorem}
  We now show that $\R$ is a field, namely an ordered field. We define the binary relations $+$ and $\cdot$ as follows: \\
  Given two cuts $A\mid B$ and $C\mid D$, we may define
  \[
    E = A + C = \set{p\in \Q\mid p = a + c \text{ for some }a\in A, c\in C}.
  \]
  \begin{note}{}
    This set summation is known as the Minkowski sum.
  \end{note}
  We must check that $E\mid F$ is a cut. We must also show that the additive identity for $\R$ is $0$, i.e. we must show that $0 + x = x + 0 = x$ for all $x\in\R$. To show the existence of additive inverses, show that for any cut $A\mid B$, there exists some $C\mid D\in \R$ such that $A\mid B + C\mid D = 0$. \par
  Similarly, we define multiplication by
  \[
    E = \set{p\in\Q\mid p = ac \text{ for some }a\in A, c\in C}.
  \]
  \subsection{Interesting Questions}
  \begin{enumerate}[label=(\alph*)]
    \item Can we cut $\R$ to get something larger? No, because every possible cut in $\R$ is an element of $\R$. This is because $\R$ has the least upper bound property and every cut in $\R$ would just be a ``real cut'' at $\sup A$.
    \item Is $\R$ unique in some natural way? Yes. If you take any other ordered field $\F$ such that $\Q\sse \F$ and $\F$ has the least upper bound property, then there exists some bijection between $\F$ and $\R$.
    \item What about $+\infty$ and $-\infty$? We can't treat these as real numbers using Dedekind cuts, as either $A$ or $B$ would have to be empty.
  \end{enumerate}
\end{document}
