\documentclass[class=article, crop=false]{standalone}
% Import packages
\usepackage[margin=1in]{geometry}

\usepackage{amssymb, amsthm}
\usepackage{comment}
\usepackage{enumitem}
\usepackage{fancyhdr}
\usepackage{hyperref}
\usepackage{import}
\usepackage{mathrsfs, mathtools}
\usepackage{multicol}
\usepackage{pdfpages}
\usepackage{standalone}
\usepackage{transparent}
\usepackage{xcolor}

\usepackage{minted}
\usepackage[many]{tcolorbox}
\tcbuselibrary{minted}

% Declare math operators
\DeclareMathOperator{\lcm}{lcm}
\DeclareMathOperator{\proj}{proj}
\DeclareMathOperator{\vspan}{span}
\DeclareMathOperator{\im}{im}
\DeclareMathOperator{\Diff}{Diff}
\DeclareMathOperator{\Int}{Int}
\DeclareMathOperator{\fcn}{fcn}
\DeclareMathOperator{\id}{id}
\DeclareMathOperator{\rank}{rank}
\DeclareMathOperator{\tr}{tr}
\DeclareMathOperator{\dive}{div}
\DeclareMathOperator{\row}{row}
\DeclareMathOperator{\col}{col}
\DeclareMathOperator{\dom}{dom}
\DeclareMathOperator{\ran}{ran}
\DeclareMathOperator{\Dom}{Dom}
\DeclareMathOperator{\Ran}{Ran}
\DeclareMathOperator{\fl}{fl}
% Macros for letters/variables
\newcommand\ttilde{\kern -.15em\lower .7ex\hbox{\~{}}\kern .04em}
\newcommand{\N}{\ensuremath{\mathbb{N}}}
\newcommand{\Z}{\ensuremath{\mathbb{Z}}}
\newcommand{\Q}{\ensuremath{\mathbb{Q}}}
\newcommand{\R}{\ensuremath{\mathbb{R}}}
\newcommand{\C}{\ensuremath{\mathbb{C}}}
\newcommand{\F}{\ensuremath{\mathbb{F}}}
\newcommand{\M}{\ensuremath{\mathbb{M}}}
\renewcommand{\P}{\ensuremath{\mathbb{P}}}
\newcommand{\lam}{\ensuremath{\lambda}}
\newcommand{\nab}{\ensuremath{\nabla}}
\newcommand{\eps}{\ensuremath{\varepsilon}}
\newcommand{\es}{\ensuremath{\varnothing}}
% Macros for math symbols
\newcommand{\dx}[1]{\,\mathrm{d}#1}
\newcommand{\inv}{\ensuremath{^{-1}}}
\newcommand{\sm}{\setminus}
\newcommand{\sse}{\subseteq}
\newcommand{\ceq}{\coloneqq}
% Macros for pairs of math symbols
\newcommand{\abs}[1]{\ensuremath{\left\lvert #1 \right\rvert}}
\newcommand{\paren}[1]{\ensuremath{\left( #1 \right)}}
\newcommand{\norm}[1]{\ensuremath{\left\lVert #1\right\rVert}}
\newcommand{\set}[1]{\ensuremath{\left\{#1\right\}}}
\newcommand{\tup}[1]{\ensuremath{\left\langle #1 \right\rangle}}
\newcommand{\floor}[1]{\ensuremath{\left\lfloor #1 \right\rfloor}}
\newcommand{\ceil}[1]{\ensuremath{\left\lceil #1 \right\rceil}}
\newcommand{\eclass}[1]{\ensuremath{\left[ #1 \right]}}

\newcommand{\chapternum}{}
\newcommand{\ex}[1]{\noindent\textbf{Exercise \chapternum.{#1}.}}

\newcommand{\tsub}[1]{\textsubscript{#1}}
\newcommand{\tsup}[1]{\textsuperscript{#1}}

\renewcommand{\choose}[2]{\ensuremath{\phantom{}_{#1}C_{#2}}}
\newcommand{\perm}[2]{\ensuremath{\phantom{}_{#1}P_{#2}}}

% Include figures
\newcommand{\incfig}[2][1]{%
    \def\svgwidth{#1\columnwidth}
    \import{./figures/}{#2.pdf_tex}
}

\definecolor{problemBackground}{RGB}{212,232,246}
\newenvironment{problem}[1]
  {
    \begin{tcolorbox}[
      boxrule=.5pt,
      titlerule=.5pt,
      sharp corners,
      colback=problemBackground,
      breakable
    ]
    \ifx &#1& \textbf{Problem. }
    \else \textbf{Problem #1.} \fi
  }
  {
    \end{tcolorbox}
  }
\definecolor{exampleBackground}{RGB}{255,249,248}
\definecolor{exampleAccent}{RGB}{158,60,14}
\newenvironment{example}[1]
  {
    \begin{tcolorbox}[
      boxrule=.5pt,
      sharp corners,
      colback=exampleBackground,
      colframe=exampleAccent,
    ]
    \color{exampleAccent}\textbf{Example.} \emph{#1}\color{black}
  }
  {
    \end{tcolorbox}
  }
\definecolor{theoremBackground}{RGB}{234,243,251}
\definecolor{theoremAccent}{RGB}{0,116,183}
\newenvironment{theorem}[1]
  {
    \begin{tcolorbox}[
      boxrule=.5pt,
      titlerule=.5pt,
      sharp corners,
      colback=theoremBackground,
      colframe=theoremAccent,
      breakable
    ]
      % \color{theoremAccent}\textbf{Theorem --- }\emph{#1}\\\color{black}
    \ifx &#1& \color{theoremAccent}\textbf{Theorem. }\color{black}
    \else \color{theoremAccent}\textbf{Theorem --- }\emph{#1}\\\color{black} \fi
  }
  {
    \end{tcolorbox}
  }
\definecolor{noteBackground}{RGB}{244,249,244}
\definecolor{noteAccent}{RGB}{34,139,34}
\newenvironment{note}[1]
  {
  \begin{tcolorbox}[
    enhanced,
    boxrule=0pt,
    frame hidden,
    sharp corners,
    colback=noteBackground,
    borderline west={3pt}{-1.5pt}{noteAccent},
    breakable
    ]
    \ifx &#1& \color{noteAccent}\textbf{Note. }\color{black}
    \else \color{noteAccent}\textbf{Note (#1). }\color{black} \fi
    }
    {
  \end{tcolorbox}
  }
\definecolor{lemmaBackground}{RGB}{255,247,234}
\definecolor{lemmaAccent}{RGB}{255,153,0}
\newenvironment{lemma}[1]
  {
    \begin{tcolorbox}[
      enhanced,
      boxrule=0pt,
      frame hidden,
      sharp corners,
      colback=lemmaBackground,
      borderline west={3pt}{-1.5pt}{lemmaAccent},
      breakable
    ]
    \ifx &#1& \color{lemmaAccent}\textbf{Lemma. }\color{black}
    \else \color{lemmaAccent}\textbf{Lemma #1. }\color{black} \fi
  }
  {
    \end{tcolorbox}
  }
\definecolor{definitionBackground}{RGB}{246,246,246}
\newenvironment{definition}[1]
  {
    \begin{tcolorbox}[
      enhanced,
      boxrule=0pt,
      frame hidden,
      sharp corners,
      colback=definitionBackground,
      borderline west={3pt}{-1.5pt}{black},
      breakable
    ]
    \textbf{Definition. }\emph{#1}\\
  }
  {
    \end{tcolorbox}
  }

\newenvironment{amatrix}[2]{
    \left[
      \begin{array}{*{#1}{c}|*{#2}c}
  }
  {
      \end{array}
    \right]
  }

\definecolor{codeBackground}{RGB}{253,246,227}
\renewcommand\theFancyVerbLine{\arabic{FancyVerbLine}}
\newtcblisting{cpp}[1][]{
  boxrule=1pt,
  sharp corners,
  colback=codeBackground,
  listing only,
  breakable,
  minted language=cpp,
  minted style=material,
  minted options={
    xleftmargin=15pt,
    baselinestretch=1.2,
    linenos,
  }
}

\author{Kyle Chui}


\fancyhf{}
\lhead{Kyle Chui}
\rhead{Page \thepage}
\pagestyle{fancy}

\begin{document}
  \section{Lecture 13}
  \subsection{Cauchy Sequences}
  The notion of a Cauchy sequence is a sequence that doesn't necessarily converge, but whose terms get ``arbitrarily close'' to each other.
  \begin{definition}{Cauchy Sequence}
    A sequence $(x_n)$ is \emph{Cauchy} if for all $\eps > 0$, there exists some $N\in\N$ such that
    \[
      \abs{x_n - x_m} < \eps\quad \text{for all }n,m > N.
    \]
  \end{definition}
  \begin{note}{}
    We don't need to know any limit $x$ in order to determine if a sequence is Cauchy.
  \end{note}
  \begin{example}{Cauchy Sequences}
    \begin{enumerate}[label=\arabic*)]
      \item $x_n = \frac{1}{n}$ is Cauchy.
      \item $x_n = (-1)^n$ is not Cauchy.
      \item $x_n = n$ is not Cauchy.
    \end{enumerate}
  \end{example}
  \textbf{Proposition 3.16} Every convergent sequence is Cauchy.
  \begin{proof}
    Let $(x_n)$ be a convergent sequence. Then for all $\eps > 0$, there exists $N\in\N$ such that for all $n > N$, we have $\abs{x_n - x} < \frac{\eps}{2}$. Let $m > N$. Thus
    \begin{align*}
      \abs{x_n - x_m} &\leq \abs{x_n - x} + \abs{x - x_m} \\
                      &= \abs{x_n - x} + \abs{x_m - x} \\
                      &< \eps.
    \end{align*}
    Therefore $(x_n)$ is Cauchy.
  \end{proof}
  \textbf{Proposition 3.17} Every Cauchy sequence is bounded.
  \begin{proof}
    Let $(x_n)$ be Cauchy. For $\eps = 1$, we know there exists $N\in\N$ such that
    \[
      \abs{x_n - x_m} < 1\quad \text{for all }n, m > N.
    \]
    Thus $\abs{x_n}\leq 1 + \abs{x_{N + 1}}$ for all $n > N$. Suppose we choose
    \[
      M = \max(\abs{x_1}, \abs{x_2},\dotsc,\abs{x_N}, 1 + \abs{x_{N + 1}}).
    \]
    Then $\abs{x_n}\leq M$ for all $n\in\N$, and so $(x_n)$ is bounded.
  \end{proof}
  \subsection{Completeness}
  \begin{theorem}{3.18 (Completeness of $\R$)}
    $\R$ is \emph{complete}, in other words every Cauchy sequence in $\R$ converges.
    \begin{proof}
      Let $(x_n)$ be Cauchy in $\R$. We have two steps:
      \begin{enumerate}[label=(\arabic*)]
        \item Identify some candidate for the limit.
        \item Show that the sequence converges to the candidate.
      \end{enumerate}
      (1) By Proposition $3.17$, we know that $(x_n)$ is bounded. By Bolzano--Weierstrass, there exists some subsequence $x_{n_k}\to x\in\R$ as $k\to\infty$. This is our candidate for our limit. \par
      (2) We now show that our sequence converges to $x$. Bolzano--Weierstrass tells us that for a subsequence of $(x_n)$, we have the terms getting arbitrarily close to $x$. Because $(x_n)$ is Cauchy, we ``control'' the remaining terms by ``proximity'' to $(x_{n_k})$. \par
      Fix $\eps > 0$, $x_{n_k}\to x$, there exists $N_1\in\N$ such that
      \[
        x_{\frac{n}{2}} < \frac{\eps}{2}\quad \text{for all }n_k > N_1.
      \]
      Since $(x_n)$ is Cauchy, there exists some $N_2\in\N$ such that
      \[
        \abs{x_n - x_m} < \frac{\eps}{2}\quad \text{for all }n,m > N_2.
      \]
      So for any $n > \max(N_1, N_2)$, and $n_k > \max(N_1, N_2)$,
      \begin{align*}
        \abs{x_n - x} &\leq \abs{x_n - x_{n_k}} + \abs{x_{n_k} - x} \\
                      &< \frac{\eps}{2} + \frac{\eps}{2} \\
                      &= \eps.
      \end{align*}
      Therefore $x_n\to x$.
    \end{proof}
  \end{theorem}
  \begin{note}{}
    The above proof makes use of Bolzano--Weierstrass, which applies to the reals, not rationals. Thus we think that the rationals are not complete.
  \end{note}
  \begin{theorem}{3.19 (Incompleteness of $\Q$)}
    The rationals are \emph{not} complete. More precisely, there exists a Cauchy sequence of rational numbers that \emph{does not} converge.
    \begin{proof}
      Consider the sequence given by
      \[
        x_1 = 2, x_{n + 1} = \frac{1}{2}\paren{x_n + \frac{2}{x_n}}.
      \]
      We see that $(x_n)$ is a sequence of rationals. We know that it does not converge in the rationals, because if it did it would have to converge to $\sqrt{2}\notin\Q$. It remains to show that it is Cauchy. From one of the homeworks we know that $\abs{x_n - \abs{x_{n + 1}}} < \frac{1}{2^n}$ for all $n\in\N$. We can take telescoping sums by writing (for $n > m$):
      \begin{align*}
        \abs{x_n - x_m} &= \abs{\sum_{k=m}^{n - 1}(x_{k + 1} - x_k)} \\
                        &\leq \sum_{k=m}^{n-1}\abs{x_{k + 1} - x_k} \\
                        &< \abs{\sum_{k=m}^{n - 1} \frac{1}{2^k}}.
        \intertext{This is a geometric series that sums to} 
                        &= \frac{\paren{\frac{1}{2}}^m-\paren{\frac{1}{2}}^n}{1 - \frac{1}{2}} \\
                        &= \frac{4}{2^m},
      \end{align*}
      which converges to $0$ as $m\to\infty$. Thus $(x_n)$ is Cauchy in $\Q$.
    \end{proof}
  \end{theorem}
  \subsection{Bonus: Diverging to Infinity}
  \begin{definition}{3.20---Diverging to $\pm\infty$}
    Let $(x_n)$ be a sequence in $\R$. We say that $x_n\to\infty$ as $n\to\infty$ if:
    for all $M > 0$, there exists $N\in\N$ such that $x_n > M$ for all $n > N$. \par
    Similarly, we say $x_n\to-\infty$ if for all $M > 0$, there exists $N\in\N$ such that $x_n < -M$ for all $n > N$.
  \end{definition}
\end{document}
