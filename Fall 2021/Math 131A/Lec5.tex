\documentclass[class=article, crop=false]{standalone}
% Import packages
\usepackage[margin=1in]{geometry}

\usepackage{amssymb, amsthm}
\usepackage{comment}
\usepackage{enumitem}
\usepackage{fancyhdr}
\usepackage{hyperref}
\usepackage{import}
\usepackage{mathrsfs, mathtools}
\usepackage{multicol}
\usepackage{pdfpages}
\usepackage{standalone}
\usepackage{transparent}
\usepackage{xcolor}

\usepackage{minted}
\usepackage[many]{tcolorbox}
\tcbuselibrary{minted}

% Declare math operators
\DeclareMathOperator{\lcm}{lcm}
\DeclareMathOperator{\proj}{proj}
\DeclareMathOperator{\vspan}{span}
\DeclareMathOperator{\im}{im}
\DeclareMathOperator{\Diff}{Diff}
\DeclareMathOperator{\Int}{Int}
\DeclareMathOperator{\fcn}{fcn}
\DeclareMathOperator{\id}{id}
\DeclareMathOperator{\rank}{rank}
\DeclareMathOperator{\tr}{tr}
\DeclareMathOperator{\dive}{div}
\DeclareMathOperator{\row}{row}
\DeclareMathOperator{\col}{col}
\DeclareMathOperator{\dom}{dom}
\DeclareMathOperator{\ran}{ran}
\DeclareMathOperator{\Dom}{Dom}
\DeclareMathOperator{\Ran}{Ran}
\DeclareMathOperator{\fl}{fl}
% Macros for letters/variables
\newcommand\ttilde{\kern -.15em\lower .7ex\hbox{\~{}}\kern .04em}
\newcommand{\N}{\ensuremath{\mathbb{N}}}
\newcommand{\Z}{\ensuremath{\mathbb{Z}}}
\newcommand{\Q}{\ensuremath{\mathbb{Q}}}
\newcommand{\R}{\ensuremath{\mathbb{R}}}
\newcommand{\C}{\ensuremath{\mathbb{C}}}
\newcommand{\F}{\ensuremath{\mathbb{F}}}
\newcommand{\M}{\ensuremath{\mathbb{M}}}
\renewcommand{\P}{\ensuremath{\mathbb{P}}}
\newcommand{\lam}{\ensuremath{\lambda}}
\newcommand{\nab}{\ensuremath{\nabla}}
\newcommand{\eps}{\ensuremath{\varepsilon}}
\newcommand{\es}{\ensuremath{\varnothing}}
% Macros for math symbols
\newcommand{\dx}[1]{\,\mathrm{d}#1}
\newcommand{\inv}{\ensuremath{^{-1}}}
\newcommand{\sm}{\setminus}
\newcommand{\sse}{\subseteq}
\newcommand{\ceq}{\coloneqq}
% Macros for pairs of math symbols
\newcommand{\abs}[1]{\ensuremath{\left\lvert #1 \right\rvert}}
\newcommand{\paren}[1]{\ensuremath{\left( #1 \right)}}
\newcommand{\norm}[1]{\ensuremath{\left\lVert #1\right\rVert}}
\newcommand{\set}[1]{\ensuremath{\left\{#1\right\}}}
\newcommand{\tup}[1]{\ensuremath{\left\langle #1 \right\rangle}}
\newcommand{\floor}[1]{\ensuremath{\left\lfloor #1 \right\rfloor}}
\newcommand{\ceil}[1]{\ensuremath{\left\lceil #1 \right\rceil}}
\newcommand{\eclass}[1]{\ensuremath{\left[ #1 \right]}}

\newcommand{\chapternum}{}
\newcommand{\ex}[1]{\noindent\textbf{Exercise \chapternum.{#1}.}}

\newcommand{\tsub}[1]{\textsubscript{#1}}
\newcommand{\tsup}[1]{\textsuperscript{#1}}

\renewcommand{\choose}[2]{\ensuremath{\phantom{}_{#1}C_{#2}}}
\newcommand{\perm}[2]{\ensuremath{\phantom{}_{#1}P_{#2}}}

% Include figures
\newcommand{\incfig}[2][1]{%
    \def\svgwidth{#1\columnwidth}
    \import{./figures/}{#2.pdf_tex}
}

\definecolor{problemBackground}{RGB}{212,232,246}
\newenvironment{problem}[1]
  {
    \begin{tcolorbox}[
      boxrule=.5pt,
      titlerule=.5pt,
      sharp corners,
      colback=problemBackground,
      breakable
    ]
    \ifx &#1& \textbf{Problem. }
    \else \textbf{Problem #1.} \fi
  }
  {
    \end{tcolorbox}
  }
\definecolor{exampleBackground}{RGB}{255,249,248}
\definecolor{exampleAccent}{RGB}{158,60,14}
\newenvironment{example}[1]
  {
    \begin{tcolorbox}[
      boxrule=.5pt,
      sharp corners,
      colback=exampleBackground,
      colframe=exampleAccent,
    ]
    \color{exampleAccent}\textbf{Example.} \emph{#1}\color{black}
  }
  {
    \end{tcolorbox}
  }
\definecolor{theoremBackground}{RGB}{234,243,251}
\definecolor{theoremAccent}{RGB}{0,116,183}
\newenvironment{theorem}[1]
  {
    \begin{tcolorbox}[
      boxrule=.5pt,
      titlerule=.5pt,
      sharp corners,
      colback=theoremBackground,
      colframe=theoremAccent,
      breakable
    ]
      % \color{theoremAccent}\textbf{Theorem --- }\emph{#1}\\\color{black}
    \ifx &#1& \color{theoremAccent}\textbf{Theorem. }\color{black}
    \else \color{theoremAccent}\textbf{Theorem --- }\emph{#1}\\\color{black} \fi
  }
  {
    \end{tcolorbox}
  }
\definecolor{noteBackground}{RGB}{244,249,244}
\definecolor{noteAccent}{RGB}{34,139,34}
\newenvironment{note}[1]
  {
  \begin{tcolorbox}[
    enhanced,
    boxrule=0pt,
    frame hidden,
    sharp corners,
    colback=noteBackground,
    borderline west={3pt}{-1.5pt}{noteAccent},
    breakable
    ]
    \ifx &#1& \color{noteAccent}\textbf{Note. }\color{black}
    \else \color{noteAccent}\textbf{Note (#1). }\color{black} \fi
    }
    {
  \end{tcolorbox}
  }
\definecolor{lemmaBackground}{RGB}{255,247,234}
\definecolor{lemmaAccent}{RGB}{255,153,0}
\newenvironment{lemma}[1]
  {
    \begin{tcolorbox}[
      enhanced,
      boxrule=0pt,
      frame hidden,
      sharp corners,
      colback=lemmaBackground,
      borderline west={3pt}{-1.5pt}{lemmaAccent},
      breakable
    ]
    \ifx &#1& \color{lemmaAccent}\textbf{Lemma. }\color{black}
    \else \color{lemmaAccent}\textbf{Lemma #1. }\color{black} \fi
  }
  {
    \end{tcolorbox}
  }
\definecolor{definitionBackground}{RGB}{246,246,246}
\newenvironment{definition}[1]
  {
    \begin{tcolorbox}[
      enhanced,
      boxrule=0pt,
      frame hidden,
      sharp corners,
      colback=definitionBackground,
      borderline west={3pt}{-1.5pt}{black},
      breakable
    ]
    \textbf{Definition. }\emph{#1}\\
  }
  {
    \end{tcolorbox}
  }

\newenvironment{amatrix}[2]{
    \left[
      \begin{array}{*{#1}{c}|*{#2}c}
  }
  {
      \end{array}
    \right]
  }

\definecolor{codeBackground}{RGB}{253,246,227}
\renewcommand\theFancyVerbLine{\arabic{FancyVerbLine}}
\newtcblisting{cpp}[1][]{
  boxrule=1pt,
  sharp corners,
  colback=codeBackground,
  listing only,
  breakable,
  minted language=cpp,
  minted style=material,
  minted options={
    xleftmargin=15pt,
    baselinestretch=1.2,
    linenos,
  }
}

\author{Kyle Chui}


\fancyhf{}
\lhead{Kyle Chui}
\rhead{Page \thepage}
\pagestyle{fancy}

\begin{document}
  \section{Lecture 5}
  \subsection{The Natural Numbers}
  We know that the natural numbers are defined via:
  \[
    \N = \set{1, 2, 3, 4, \dotsc}.
  \]
  \subsubsection{Properties of the Natural Numbers}
  \begin{enumerate}[label=(P\arabic*)]
    \item $1\in\N$
    \item If $n\in \N$, then it has a \emph{successor}, $n + 1\in\N$
    \item $1$ is \emph{not} the successor of any element of $\N$
    \item If $m, n$ have the same successor, then $m = n$
    \begin{note}{}
      The properties above can be abstracted to become:
      \begin{enumerate}[label=(P\arabic*)]
        \item $1\in \N$
        \item There exists some $S\colon \N\to \N$ where $S(n)$ is the successor $n$
        \item $1\notin \range S$
        \item $S$ is injective
        \item Suppose $A\sse \N$ with the properties:
        \begin{enumerate}[label=(\roman*)]
          \item $1\in A$
          \item If $n\in A$, then $S(n)\in A$
        \end{enumerate}
        Then $A = \N$.
      \end{enumerate}
    \end{note}
  \end{enumerate}
  \begin{theorem}{1.1 (Induction)}
    Let $\set{P(n)\mid n\in\N}$ be a set of logical propositions. Suppose that
    \begin{enumerate}[label=(\roman*)]
      \item $P(1)$ is true.
      \item If $P(n)$ is true, then $P(n + 1)$ is true.
    \end{enumerate}
    Then $P(n)$ is true for all $n\in\N$.
    \begin{proof}
      Let $A = \set{n\mid P(n) \text{ is true}}\sse\N$. By our first assumption, $1\in A$. By (ii), if $n\in A$, then $n + 1\in A$. So by (P5) of the natural numbers, we know that $A = \N$.
    \end{proof}
  \end{theorem}
  \begin{definition}{1.2 (Peano Axioms)}
    A triplet $(\N, 1, S)$ is said to be a \emph{system of the naturals} if it satisfies:
    \begin{enumerate}[label=\arabic*)]
      \item $\N$ is a set and $1\in\N$
      \item $S\colon \N\to \N$ is a function
      \item $1\notin \range S$
      \item $S$ is injective
      \item $\forall A\sse N$ such that $1\in A$ and $S(A)\sse A$, then $A = \N$
    \end{enumerate}
  \end{definition}
  \begin{definition}{Addition}
    We define the binary relation $+$ over $\N$:
    \begin{enumerate}[label=(\roman*)]
      \item $\forall n\in\N$, $n + 1\ceq S(n)$
      \item $\forall m, n\in \N$, we have $m + S(n) = S(m + n)$
    \end{enumerate}
  \end{definition}
  The following properties can be proven from the above definition of addition:
  \begin{enumerate}[label=(\alph*)]
    \item Associativity: $\forall x,y,z\in \N$, we have $(x + y) + z = x + (y + z)$
    \item Commutativity: $\forall x,y\in \N$, we have $x + y = y + x$
    \item Cancellative Law: $\forall x, y, z\in \N$, we have $x + y = y + z\implies x = z$
  \end{enumerate}
  \begin{theorem}{1.3 (Existence of the Naturals)}
    There exists a unique system of the naturals. In other words, there is a bijection between different systems of the naturals and so we may write:
    \[
      (\N, 1, S) \iff (\N', 1', S').
    \]
  \end{theorem}


  \subsection{Fields}
  \begin{definition}{Field}
    A \emph{field} is a set with two binary operations,
    \begin{itemize}
      \item $+$, or `addition'
      \item $\cdot$, or `multiplication'
    \end{itemize}
  \end{definition}
  \subsubsection{Axioms for Addition}
  \begin{enumerate}[label=(A\arabic*)]
    \item $x\in \F\land y\in \F\implies x + y\in\F$
    \item $\forall x,y\in\F$, we have $x + y = y + x$
    \item $\forall x,y,z\in \F$, we have $(x + y) + z = x + (y + z)$
    \item There exists some $0\in\F$ such that $0 + x = x$ for all $x\in\F$
    \item $\forall x\in\F$, there exists $-x\in\F$ such that $x + (-x) = 0$
  \end{enumerate}
  \subsubsection{Axioms for Multiplication}
  \begin{enumerate}[label=(M\arabic*)]
    \item $x\in\F\land y\in\F\implies x\cdot y\in\F$
    \item $\forall x,y\in\F$, we have $x\cdot y = y\cdot x$
    \item $\forall x, y, z\in\F$, we have $(x\cdot y)\cdot z = x\cdot (y\cdot z)$
    \item There exists some $1\in\F$ such that $1\neq 0$ and $1\cdot x = x$ for all $x\in\F$
    \item $\forall x\in\F$, there exists some $\frac{1}{x}\in\F$ such that $x\cdot \frac{1}{x} = 1$
  \end{enumerate}
  \subsubsection{Distributive Law}
  \begin{enumerate}[label=(D\arabic*)]
    \item $\forall x, y, z\in\F$, we have $x\cdot (y + z) = x\cdot y + x\cdot z$
  \end{enumerate}
\end{document}
