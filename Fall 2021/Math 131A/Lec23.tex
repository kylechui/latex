\documentclass[class=article, crop=false]{standalone}
% Import packages
\usepackage[margin=1in]{geometry}

\usepackage{amssymb, amsthm}
\usepackage{comment}
\usepackage{enumitem}
\usepackage{fancyhdr}
\usepackage{hyperref}
\usepackage{import}
\usepackage{mathrsfs, mathtools}
\usepackage{multicol}
\usepackage{pdfpages}
\usepackage{standalone}
\usepackage{transparent}
\usepackage{xcolor}

\usepackage{minted}
\usepackage[many]{tcolorbox}
\tcbuselibrary{minted}

% Declare math operators
\DeclareMathOperator{\lcm}{lcm}
\DeclareMathOperator{\proj}{proj}
\DeclareMathOperator{\vspan}{span}
\DeclareMathOperator{\im}{im}
\DeclareMathOperator{\Diff}{Diff}
\DeclareMathOperator{\Int}{Int}
\DeclareMathOperator{\fcn}{fcn}
\DeclareMathOperator{\id}{id}
\DeclareMathOperator{\rank}{rank}
\DeclareMathOperator{\tr}{tr}
\DeclareMathOperator{\dive}{div}
\DeclareMathOperator{\row}{row}
\DeclareMathOperator{\col}{col}
\DeclareMathOperator{\dom}{dom}
\DeclareMathOperator{\ran}{ran}
\DeclareMathOperator{\Dom}{Dom}
\DeclareMathOperator{\Ran}{Ran}
\DeclareMathOperator{\fl}{fl}
% Macros for letters/variables
\newcommand\ttilde{\kern -.15em\lower .7ex\hbox{\~{}}\kern .04em}
\newcommand{\N}{\ensuremath{\mathbb{N}}}
\newcommand{\Z}{\ensuremath{\mathbb{Z}}}
\newcommand{\Q}{\ensuremath{\mathbb{Q}}}
\newcommand{\R}{\ensuremath{\mathbb{R}}}
\newcommand{\C}{\ensuremath{\mathbb{C}}}
\newcommand{\F}{\ensuremath{\mathbb{F}}}
\newcommand{\M}{\ensuremath{\mathbb{M}}}
\renewcommand{\P}{\ensuremath{\mathbb{P}}}
\newcommand{\lam}{\ensuremath{\lambda}}
\newcommand{\nab}{\ensuremath{\nabla}}
\newcommand{\eps}{\ensuremath{\varepsilon}}
\newcommand{\es}{\ensuremath{\varnothing}}
% Macros for math symbols
\newcommand{\dx}[1]{\,\mathrm{d}#1}
\newcommand{\inv}{\ensuremath{^{-1}}}
\newcommand{\sm}{\setminus}
\newcommand{\sse}{\subseteq}
\newcommand{\ceq}{\coloneqq}
% Macros for pairs of math symbols
\newcommand{\abs}[1]{\ensuremath{\left\lvert #1 \right\rvert}}
\newcommand{\paren}[1]{\ensuremath{\left( #1 \right)}}
\newcommand{\norm}[1]{\ensuremath{\left\lVert #1\right\rVert}}
\newcommand{\set}[1]{\ensuremath{\left\{#1\right\}}}
\newcommand{\tup}[1]{\ensuremath{\left\langle #1 \right\rangle}}
\newcommand{\floor}[1]{\ensuremath{\left\lfloor #1 \right\rfloor}}
\newcommand{\ceil}[1]{\ensuremath{\left\lceil #1 \right\rceil}}
\newcommand{\eclass}[1]{\ensuremath{\left[ #1 \right]}}

\newcommand{\chapternum}{}
\newcommand{\ex}[1]{\noindent\textbf{Exercise \chapternum.{#1}.}}

\newcommand{\tsub}[1]{\textsubscript{#1}}
\newcommand{\tsup}[1]{\textsuperscript{#1}}

\renewcommand{\choose}[2]{\ensuremath{\phantom{}_{#1}C_{#2}}}
\newcommand{\perm}[2]{\ensuremath{\phantom{}_{#1}P_{#2}}}

% Include figures
\newcommand{\incfig}[2][1]{%
    \def\svgwidth{#1\columnwidth}
    \import{./figures/}{#2.pdf_tex}
}

\definecolor{problemBackground}{RGB}{212,232,246}
\newenvironment{problem}[1]
  {
    \begin{tcolorbox}[
      boxrule=.5pt,
      titlerule=.5pt,
      sharp corners,
      colback=problemBackground,
      breakable
    ]
    \ifx &#1& \textbf{Problem. }
    \else \textbf{Problem #1.} \fi
  }
  {
    \end{tcolorbox}
  }
\definecolor{exampleBackground}{RGB}{255,249,248}
\definecolor{exampleAccent}{RGB}{158,60,14}
\newenvironment{example}[1]
  {
    \begin{tcolorbox}[
      boxrule=.5pt,
      sharp corners,
      colback=exampleBackground,
      colframe=exampleAccent,
    ]
    \color{exampleAccent}\textbf{Example.} \emph{#1}\color{black}
  }
  {
    \end{tcolorbox}
  }
\definecolor{theoremBackground}{RGB}{234,243,251}
\definecolor{theoremAccent}{RGB}{0,116,183}
\newenvironment{theorem}[1]
  {
    \begin{tcolorbox}[
      boxrule=.5pt,
      titlerule=.5pt,
      sharp corners,
      colback=theoremBackground,
      colframe=theoremAccent,
      breakable
    ]
      % \color{theoremAccent}\textbf{Theorem --- }\emph{#1}\\\color{black}
    \ifx &#1& \color{theoremAccent}\textbf{Theorem. }\color{black}
    \else \color{theoremAccent}\textbf{Theorem --- }\emph{#1}\\\color{black} \fi
  }
  {
    \end{tcolorbox}
  }
\definecolor{noteBackground}{RGB}{244,249,244}
\definecolor{noteAccent}{RGB}{34,139,34}
\newenvironment{note}[1]
  {
  \begin{tcolorbox}[
    enhanced,
    boxrule=0pt,
    frame hidden,
    sharp corners,
    colback=noteBackground,
    borderline west={3pt}{-1.5pt}{noteAccent},
    breakable
    ]
    \ifx &#1& \color{noteAccent}\textbf{Note. }\color{black}
    \else \color{noteAccent}\textbf{Note (#1). }\color{black} \fi
    }
    {
  \end{tcolorbox}
  }
\definecolor{lemmaBackground}{RGB}{255,247,234}
\definecolor{lemmaAccent}{RGB}{255,153,0}
\newenvironment{lemma}[1]
  {
    \begin{tcolorbox}[
      enhanced,
      boxrule=0pt,
      frame hidden,
      sharp corners,
      colback=lemmaBackground,
      borderline west={3pt}{-1.5pt}{lemmaAccent},
      breakable
    ]
    \ifx &#1& \color{lemmaAccent}\textbf{Lemma. }\color{black}
    \else \color{lemmaAccent}\textbf{Lemma #1. }\color{black} \fi
  }
  {
    \end{tcolorbox}
  }
\definecolor{definitionBackground}{RGB}{246,246,246}
\newenvironment{definition}[1]
  {
    \begin{tcolorbox}[
      enhanced,
      boxrule=0pt,
      frame hidden,
      sharp corners,
      colback=definitionBackground,
      borderline west={3pt}{-1.5pt}{black},
      breakable
    ]
    \textbf{Definition. }\emph{#1}\\
  }
  {
    \end{tcolorbox}
  }

\newenvironment{amatrix}[2]{
    \left[
      \begin{array}{*{#1}{c}|*{#2}c}
  }
  {
      \end{array}
    \right]
  }

\definecolor{codeBackground}{RGB}{253,246,227}
\renewcommand\theFancyVerbLine{\arabic{FancyVerbLine}}
\newtcblisting{cpp}[1][]{
  boxrule=1pt,
  sharp corners,
  colback=codeBackground,
  listing only,
  breakable,
  minted language=cpp,
  minted style=material,
  minted options={
    xleftmargin=15pt,
    baselinestretch=1.2,
    linenos,
  }
}

\author{Kyle Chui}


\fancyhf{}
\lhead{Kyle Chui}
\rhead{Page \thepage}
\pagestyle{fancy}

\begin{document}
  \section{Lecture 23}
  \subsection{Differentiation}
  The derivative is just the slope of a tangent line to a function at a point. We can get this slope by taking the limit of the slopes of a bunch of secant lines.
  \begin{definition}{5.1---Derivative at a Point}
    Let $I = (a, b)$ be an interval, $a < b$, $f\colon I\to \R$ and let $c\in I$. The \emph{derivative of $f$ at $c$} is defined by:
    \[
      f'(c) = \lim_{x\to c} \frac{f(x) - f(c)}{x - c},
    \]
    provided the limit exists (i.e. it converges to a real number). If $f'(c)$ exists for all $c\in I$, then we say that $f$ is \emph{differentiable} on $I$.
  \end{definition}
  \begin{note}{}
    \begin{enumerate}[label=\arabic*)]
      \item If we put $h = x - c$, we may rewrite the limit as
      \[
        f'(c) = \lim_{h\to 0} \frac{f(c + h) - f(h)}{h}.
      \]
      \item It is often useful to think of $f'$ as a function in its own right: $\dom(f')\sse\dom(f)$. We may write $f'\colon \dom(f')\to \R$, where
      \[
        f'(c) = \lim_{h\to 0} \frac{f(c + h) - f(h)}{h}.
      \]
    \end{enumerate}
  \end{note}
  \begin{example}{Derivatives}
    \begin{enumerate}[label=\arabic*)]
      \item Let $f(x) = x^n$, $n\in\N$. We claim that $f$ is differentiable on $\R$ and $f'(x) = nx^{n - 1}$.
      \begin{proof}
        Fix $c\in\R$, so
        \begin{align*}
          \frac{f(x) - f(c)}{x - c} &= \frac{x^n - c^n}{x - c} \\
                                    &= \frac{(x - c)(x^{n - 1} + \dotsb + c^{n - 1})}{x - c} \\
                                    &= \sum_{j=0}^{n - 1}x^jc^{n - 1 - j}.
        \end{align*}
        Since polynomials are continuous, we may take the limit of this as $x$ goes to $c$, which is
        \begin{align*}
          f'(c) &= \lim_{x\to c} \paren{\sum_{j=0}^{n - 1}x^jc^{n - 1 - j}} \\
                &= \sum_{j=0}^{n - 1}c^j\cdot c^{n - 1 - j} \\
                &= \sum_{j=0}^{n - 1}c^{n - 1} \\
                &= nc^{n - 1}.
        \end{align*}
      \end{proof}
      \item The function $f(x) = \abs{x}$ is differentiable on $\R\sm\set{0}$, and not differentiable at $x = 0$. At $x\neq 0$, we have $f'(x) = \begin{cases}1 & x > 0 \\ -1 & x < 0\end{cases}$. We see that
      \[
        \lim_{x\to 0^-} \frac{f(x)}{x} = -1\neq 1 = \lim_{x\to 0^+} \frac{f(x)}{x}.
      \]
      In particular, this is an example of a function that is continuous everywhere but not necessarily continuous everywhere.
      \item Bolzano's function (Weierstrass function): A function that is continuous everywhere but differentiable nowhere.
    \end{enumerate}
  \end{example}
  \textbf{Proposition 5.2} If a function $f\colon I\to \R$ is differentiable at $c\in I$, then $f$ is continuous at $c$.
  \begin{proof}
    For $x\neq c$, we may write
    \[
      f(x) = f(c) + \frac{f(x) - f(c)}{x - c}\cdot (x - c).
    \]
    Thus we have
    \[
      \lim_{x\to c} f(x) = f(c) + f'(c)\cdot 0 = f(c).
    \]
    Therefore $f$ is continuous at $c$.
  \end{proof}
  \begin{theorem}{5.3}
    Let $f,g\colon I\to \R$ and $f, g$ are differentiable at $c\in I$. Then:
    \begin{enumerate}[label=(\roman*)]
      \item For all $a, b\in\R$, we have $af + bg$ is differentiable at $c$, and
      \[
        (af + bg)'(c) = af'(c) + bg'(c).
      \]
      \item Leibniz Rule: We know that $fg$ is differentiable at $c$, in particular
      \[
        (fg)'(c) = f'(c)g(c) + f(c)g'(c).
      \]
      \begin{proof}
        When $x\neq c$, we have
        \[
          \frac{f(x)g(x) - f(c)g(c)}{x - c} = f(x)\cdot \frac{g(x) - g(c)}{x - c} + g(x)\cdot \frac{f(x) - f(c)}{x - c}.
        \]
        Taking the limit, we have what we are looking for.
      \end{proof}
      \item If $g(c)\neq 0$, then $\frac{f}{g}$ is differentiable at $c$, and
      \[
        \paren{\frac{f}{g}}'(c) = \frac{g(c)f'(c) - f(c)g'(c)}{g^2(c)}.
      \]
    \end{enumerate}
  \end{theorem}
  \textbf{Proposition 5.4} (Chain Rule): Let $f\colon I\to \R$, $g\colon J\to \R$ such that $f(I)\sse J$, so $g\circ f$ is well-defined on $I$. Then if $f$ is differentiable at $c\in I$ and $g$ is differentiable at $g(c)\in J$, then $g\circ f\colon I\to \R$ is differentiable at $c$ and
  \[
    (g\circ f)'(c) = f'(c)\cdot g'(f(c)).
  \]
\end{document}
