\documentclass[class=article, crop=false]{standalone}
\usepackage[margin=1in]{geometry}

\usepackage{amsmath}
\usepackage{amssymb}
\usepackage{amsthm}
\usepackage{comment}
\usepackage{enumitem}
\usepackage{fancyhdr}
\usepackage{hyperref}
\usepackage{mathrsfs}
\usepackage{mathtools}
\usepackage{standalone}
\usepackage{tcolorbox}
\usepackage{tikz}
\usepackage{xcolor}

\usetikzlibrary{decorations.pathreplacing}
\tcbuselibrary{skins}

\DeclareMathOperator{\lcm}{lcm}
\DeclareMathOperator{\proj}{proj}
\DeclareMathOperator{\vspan}{span}
\DeclareMathOperator{\im}{im}
\DeclareMathOperator{\range}{range}
\DeclareMathOperator{\Diff}{Diff}
\DeclareMathOperator{\Int}{Int}
\DeclareMathOperator{\fcn}{fcn}
\DeclareMathOperator{\id}{id}

\newcommand{\N}{\ensuremath{\mathbb{N}}}
\newcommand{\Z}{\ensuremath{\mathbb{Z}}}
\newcommand{\Q}{\ensuremath{\mathbb{Q}}}
\newcommand{\R}{\ensuremath{\mathbb{R}}}
\newcommand{\C}{\ensuremath{\mathbb{C}}}
\newcommand{\F}{\ensuremath{\mathbb{F}}}
\newcommand{\lam}{\ensuremath{\lambda}}
\newcommand{\nab}{\ensuremath{\nabla}}
\newcommand{\eps}{\ensuremath{\varepsilon}}
\newcommand{\es}{\ensuremath{\varnothing}}

\newcommand{\abs}[1]{\ensuremath{\left\lvert #1 \right\rvert}}
\newcommand{\bigpar}[1]{\ensuremath{\left( #1 \right)}}
\newcommand{\norm}[1]{\ensuremath{\lVert #1\rVert}}
\newcommand{\set}[1]{\ensuremath{\left\{#1\right\}}}
\newcommand{\tuple}[1]{\ensuremath{\left\langle #1 \right\rangle}}
\newcommand{\floor}[1]{\ensuremath{\left\lfloor #1 \right\rfloor}}
\newcommand{\ceil}[1]{\ensuremath{\left\lceil #1 \right\rceil}}

\newcommand{\chapternum}{}
\newcommand{\ex}[1]{\noindent\textbf{Exercise \chapternum.{#1}.}}

\newcommand{\dx}[1]{\,\mathrm{d}#1}
\newcommand{\inv}{\ensuremath{^{-1}}}
\newcommand{\sm}{\setminus}
\newcommand{\sse}{\subseteq}
\newcommand{\ceq}{\coloneqq}

\definecolor{problemBackground}{RGB}{212,232,246}
\newenvironment{problem}[1]
  {
    \begin{tcolorbox}[
      boxrule=.5pt,
      titlerule=.5pt,
      sharp corners,
      colback=problemBackground
    ]
    \ifx &#1& \textbf{Problem. }
    \else \textbf{Problem #1.} \fi
  }
  {
    \end{tcolorbox}
  }

\definecolor{exampleBackground}{RGB}{255,249,248}
\definecolor{exampleAccent}{RGB}{158,60,14}
\newenvironment{example}[1]
  {
    \begin{tcolorbox}[
      boxrule=.5pt,
      sharp corners,
      colback=exampleBackground,
      colframe=exampleAccent,
    ]
    \color{exampleAccent}\textbf{Example.} \emph{#1}\color{black}
  }
  {
    \end{tcolorbox}
  }

\definecolor{theoremBackground}{RGB}{234,243,251}
\definecolor{theoremAccent}{RGB}{0,116,183}
\newenvironment{theorem}[1]
  {
    \begin{tcolorbox}[
      boxrule=.5pt,
      titlerule=.5pt,
      sharp corners,
      colback=theoremBackground,
      colframe=theoremAccent
    ]
      \color{theoremAccent}\textbf{Theorem --- }\emph{#1}\\\color{black}
  }
  {
    \end{tcolorbox}
  }

\definecolor{noteBackground}{RGB}{244,249,244}
\definecolor{noteAccent}{RGB}{34,139,34}
\newenvironment{note}[1]
  {
  \begin{tcolorbox}[
    enhanced,
    boxrule=0pt,
    frame hidden,
    sharp corners,
    colback=noteBackground,
    borderline west={3pt}{-1.5pt}{noteAccent}
    ]
    \ifx &#1& \color{noteAccent}\textbf{Note. }\color{black}
    \else \color{noteAccent}\textbf{Note (#1). }\color{black} \fi
    }
    {
  \end{tcolorbox}
  }

\definecolor{definitionBackground}{RGB}{246,246,246}
\newenvironment{definition}[1]
  {
    \begin{tcolorbox}[
      enhanced,
      boxrule=0pt,
      frame hidden,
      sharp corners,
      colback=definitionBackground,
      borderline west={3pt}{-1.5pt}{black}
    ]
    \textbf{Definition. }\emph{#1}\\
  }
  {
    \end{tcolorbox}
  }
\newenvironment{amatrix}[2]{
    \left[
      \begin{array}{*{#1}{c}|*{#2}c}
  }
  {
      \end{array}
    \right]
  }
\date{\the\year-\the\month-\the\day}
\author{Kyle Chui}


\fancyhf{}
\lhead{Kyle Chui}
\rhead{Page \thepage}
\pagestyle{fancy}

\begin{document}
  \section{Lecture 7}
  \begin{definition}{2.11---Upper and Lower Bounds}
    Let $E$ be an ordered set and $A\subset E$.
    \begin{enumerate}[label=(\alph*)]
      \item If there exists $x\in E$ such that for all $a\in A$, $a\leq x$, we say $A$ is \emph{bounded above} by $x$ and call $x$ an \emph{upper bound} for $A$.
      \item Suppose $A\subset E$ is non-empty and bounded above and there exists some $x^*\in E$ such that
      \begin{enumerate}[label=\roman*)]
        \item $x^*$ is an upper bound for $A$
        \item If $y$ is any upper bound for $A^*$, then $x^*\leq y$
      \end{enumerate}
    \end{enumerate}
    Then we call $x^*$ the \emph{least upper bound for $A$}, and we write
    \[
      x^* = \sup A. \tag{sup meaning supremum}
    \]
    The \emph{greatest lower bound} or \emph{infimum} of a set $B$, which is bounded below and non-empty, satisfies
    \begin{enumerate}[label=\roman*)]
      \item $\inf B$ is a lower bound for $B$
      \item If $y$ is any lower bound for $B$, then $y\leq \inf B$
    \end{enumerate}
  \end{definition}
  \begin{example}{}
    Suppose
    \begin{align*}
      A &= \set{p\in \Q\mid p\geq 0, p^2 < 2}, \\
      B &= \set{p\in \Q\mid p\geq 0, p^2 > 2}.
    \end{align*}
    Then $A$ is bounded above (say by $2$) and $B$ is bounded below (say by $0$). In the last lecture we proved that neither $\sup A$ nor $\inf B$ exist in $\Q$ (because the values would have been $\sqrt 2$).
  \end{example}
  \begin{example}{}
    Let
    \begin{align*}
      C &= \set{p\in\Q\mid p < 0}, \\
      D &= \set{p\in\Q\mid p \leq 0}.
    \end{align*}
    Then
    \[
      \sup C = \sup D = 0.
    \]
    However, notice that $\sup C\notin C$ and $\sup D\in D$.
  \end{example}
  \begin{definition}{Maximum}
    We define $\max A$ to be the largest element of $A$, which satisfies:
    \begin{enumerate}[label=\roman*)]
      \item $\max A\in A$
      \item For all $a\in A$, $a \leq \max A$
    \end{enumerate}
    The definition for minimum is similar.
  \end{definition}
  \begin{definition}{2.12---Least Upper Bound Property (LUBP)}
    An \emph{ordered set} $E$ has the \emph{least upper bound property} if the following is true:
    \begin{enumerate}[label=\roman*)]
      \item If $A\sse E$, $A\neq\es$, $A$ is bounded above, then $\sup A$ exists and $\sup A\in E$.
    \end{enumerate}
  \end{definition}
  \begin{note}{}
    $\Q$ does not have the least upper bound property.
  \end{note}
  \begin{theorem}{2.13 (Existence of $\R$)}
    There exists an ordered field $\R$ which has
    \begin{enumerate}[label=\roman*)]
      \item $\Q$ as a sub-field
      \item The least upper bound property
    \end{enumerate}
  \end{theorem}
  \subsection{Fundamental Properties of the Real Numbers (because of LUBP)}
  \begin{theorem}{2.14 (Archimedean Property of $\R$)}
    If $x, y\in\R$, and $x > 0$, then $\exists n\in \N$ such that $n\cdot x > y$.
    \begin{proof}
      Let $A = \set{nx\mid n\in\N}\sse \R$. Suppose towards a contradiction that there is no such $n$ that satisfies the statement above. In other words, for all $n\in\N$, $nx \leq y$. Thus $A$ is bounded above by $y$. Since $A$ is nonempty and is a subset of $\R$, we know that $\sup A$ exists. Consider the value given by $\sup A - x$, which is not an upper bound for $A$. Then we know there exists some $z\in A$ such that $\sup A - x < z$. Since $z = mx$ (because $z\in A$), we have
      \begin{align*}
        \sup A - x &< z \\
        \sup A - x &< mx \\
        \sup A &< (m + 1)x.
      \end{align*}
      We know that $(m + 1)x\in A$, which contradicts the definition of $\sup A$.
    \end{proof}
  \end{theorem}
  Some remarks:
  \begin{enumerate}
    \item Let $x = 1$. Then $\forall y\in\R, \exists n\in\N$ such that $n > y$.
    \item Let $y = 1$. Then $\forall x\in\R, \exists n\in\N$ such that $x > \frac{1}{n} > 0$.
  \end{enumerate}
  \begin{theorem}{2.15 (Density of $\Q$ in $\R$)}
    For all $x,y\in \R$, $x < y$, there exists some $p\in\Q$ such that $x < p < y$.
    \begin{proof}
      Fix $x < y$. Then by the Archimedean property we have some $n\in\N$ such that $n(y - x) > 1$, or $y - x > \frac{1}{n}$. We may suppose $x > 0$, because otherwise we either have $x < 0 < y$ or $x < y < 0$ (multiply all sides by $-1$). \par
      We want to show that
      \[
        nx < m < ny.
      \]
      Since $nx + 1 < ny$, we have $nx < m < nx + 1$, or $m - 1 < nx < m$. If $nx\in\Z$, we can take $m = nx + 1$. Thus
      \[
        x < x + \frac{1}{n} = \frac{nx + 1}{n} = \frac{m}{n} < \frac{ny}{n} < y.
      \]
      Otherwise we have $nx\notin \Z$. We then apply the following lemma:
      \begin{lemma}{}
        If $x\in\R$, there exists a $k\in\Z$ such that $k - 1\leq x\leq k$.
      \end{lemma}
      Then $m - 1 < nx < m$, as desired.
    \end{proof}
  \end{theorem}
\end{document}
