\documentclass[class=article, crop=false]{standalone}
\usepackage[margin=1in]{geometry}

\usepackage{amsmath}
\usepackage{amssymb}
\usepackage{amsthm}
\usepackage{comment}
\usepackage{enumitem}
\usepackage{fancyhdr}
\usepackage{hyperref}
\usepackage{mathrsfs}
\usepackage{mathtools}
\usepackage{standalone}
\usepackage{tcolorbox}
\usepackage{tikz}
\usepackage{xcolor}

\usetikzlibrary{decorations.pathreplacing}
\tcbuselibrary{skins}

\DeclareMathOperator{\lcm}{lcm}
\DeclareMathOperator{\proj}{proj}
\DeclareMathOperator{\vspan}{span}
\DeclareMathOperator{\im}{im}
\DeclareMathOperator{\range}{range}
\DeclareMathOperator{\Diff}{Diff}
\DeclareMathOperator{\Int}{Int}
\DeclareMathOperator{\fcn}{fcn}
\DeclareMathOperator{\id}{id}

\newcommand{\N}{\ensuremath{\mathbb{N}}}
\newcommand{\Z}{\ensuremath{\mathbb{Z}}}
\newcommand{\Q}{\ensuremath{\mathbb{Q}}}
\newcommand{\R}{\ensuremath{\mathbb{R}}}
\newcommand{\C}{\ensuremath{\mathbb{C}}}
\newcommand{\F}{\ensuremath{\mathbb{F}}}
\newcommand{\lam}{\ensuremath{\lambda}}
\newcommand{\nab}{\ensuremath{\nabla}}
\newcommand{\eps}{\ensuremath{\varepsilon}}
\newcommand{\es}{\ensuremath{\varnothing}}

\newcommand{\abs}[1]{\ensuremath{\left\lvert #1 \right\rvert}}
\newcommand{\bigpar}[1]{\ensuremath{\left( #1 \right)}}
\newcommand{\norm}[1]{\ensuremath{\lVert #1\rVert}}
\newcommand{\set}[1]{\ensuremath{\left\{#1\right\}}}
\newcommand{\tuple}[1]{\ensuremath{\left\langle #1 \right\rangle}}
\newcommand{\floor}[1]{\ensuremath{\left\lfloor #1 \right\rfloor}}
\newcommand{\ceil}[1]{\ensuremath{\left\lceil #1 \right\rceil}}

\newcommand{\chapternum}{}
\newcommand{\ex}[1]{\noindent\textbf{Exercise \chapternum.{#1}.}}

\newcommand{\dx}[1]{\,\mathrm{d}#1}
\newcommand{\inv}{\ensuremath{^{-1}}}
\newcommand{\sm}{\setminus}
\newcommand{\sse}{\subseteq}
\newcommand{\ceq}{\coloneqq}

\definecolor{problemBackground}{RGB}{212,232,246}
\newenvironment{problem}[1]
  {
    \begin{tcolorbox}[
      boxrule=.5pt,
      titlerule=.5pt,
      sharp corners,
      colback=problemBackground
    ]
    \ifx &#1& \textbf{Problem. }
    \else \textbf{Problem #1.} \fi
  }
  {
    \end{tcolorbox}
  }

\definecolor{exampleBackground}{RGB}{255,249,248}
\definecolor{exampleAccent}{RGB}{158,60,14}
\newenvironment{example}[1]
  {
    \begin{tcolorbox}[
      boxrule=.5pt,
      sharp corners,
      colback=exampleBackground,
      colframe=exampleAccent,
    ]
    \color{exampleAccent}\textbf{Example.} \emph{#1}\color{black}
  }
  {
    \end{tcolorbox}
  }

\definecolor{theoremBackground}{RGB}{234,243,251}
\definecolor{theoremAccent}{RGB}{0,116,183}
\newenvironment{theorem}[1]
  {
    \begin{tcolorbox}[
      boxrule=.5pt,
      titlerule=.5pt,
      sharp corners,
      colback=theoremBackground,
      colframe=theoremAccent
    ]
      \color{theoremAccent}\textbf{Theorem --- }\emph{#1}\\\color{black}
  }
  {
    \end{tcolorbox}
  }

\definecolor{noteBackground}{RGB}{244,249,244}
\definecolor{noteAccent}{RGB}{34,139,34}
\newenvironment{note}[1]
  {
  \begin{tcolorbox}[
    enhanced,
    boxrule=0pt,
    frame hidden,
    sharp corners,
    colback=noteBackground,
    borderline west={3pt}{-1.5pt}{noteAccent}
    ]
    \ifx &#1& \color{noteAccent}\textbf{Note. }\color{black}
    \else \color{noteAccent}\textbf{Note (#1). }\color{black} \fi
    }
    {
  \end{tcolorbox}
  }

\definecolor{definitionBackground}{RGB}{246,246,246}
\newenvironment{definition}[1]
  {
    \begin{tcolorbox}[
      enhanced,
      boxrule=0pt,
      frame hidden,
      sharp corners,
      colback=definitionBackground,
      borderline west={3pt}{-1.5pt}{black}
    ]
    \textbf{Definition. }\emph{#1}\\
  }
  {
    \end{tcolorbox}
  }
\newenvironment{amatrix}[2]{
    \left[
      \begin{array}{*{#1}{c}|*{#2}c}
  }
  {
      \end{array}
    \right]
  }
\date{\the\year-\the\month-\the\day}
\author{Kyle Chui}


\fancyhf{}
\lhead{Kyle Chui}
\rhead{Page \thepage}
\pagestyle{fancy}

\begin{document}
  \section{Lecture 15}
  \subsection{Infinite Series}
  Infinite series are \emph{extremely} useful in applications, i.e. Taylor and Fourier series.
  \begin{definition}{3.24---Infinite Series}
    Let $(x_n)_{n = m}^{\infty}$ be a sequence in $\R$ ($m\in\Z$). We define the sequence $(S_n)_{n = m}^{\infty}$ to be defined by
    \[
      S_n = \sum_{k=m}^{n}x_k.
    \]
    We call $(S_n)$ the sequence of \emph{partial sums} of $(x_n)$. The \emph{infinite series} given by
    \[
      \sum_{n=m}^{\infty}x_n
    \]
    is said to converge if the sequence of partial sums converges, and we write
    \[
      \sum_{n=m}^{\infty}x_n \ceq \lim_{n\to \infty} S_n.
    \]
    A series that \emph{doesn't converge} is said to \emph{diverge}. If $S_n\to\pm\infty$, we say that
    \[
      \sum_{n=m}^{\infty}x_n = \pm\infty.
    \]
  \end{definition}
  \subsection{Important Examples of Infinite Series}
  \begin{enumerate}[label=\arabic*)]
    \item Geometric Series---Series of the form
    \[
      \sum_{n=0}^{\infty}x^n = 1 + x + x^2 + \dotsb,
    \]
    where $x\in\R$. We know that
    \begin{align*}
      S_n &= \sum_{k=0}^{n}x^k \\
          &= 1 + x + x^2 + \dotsb + x^k.
    \end{align*}
    With some algebra we find that for all $n\in\N$, $x\neq 1$,
    \[
      S_n = \frac{1-x^{n + 1}}{1 - x}.
    \]
    \begin{proof}
      Observe that
      \begin{align*}
        (1 - x) S_n &= (1 - x)(1 + x + x^2 + \dotsb + x^n) \\
                    &= 1 + x + \dotsb + x^n - x - x^2 - \dotsb - x^{n + 1} \\
                    &= 1 - x^{n + 1}.
      \end{align*}
    \end{proof}
    In the edge case where $x = 1$, we have that
    \[
      \sum_{n=0}^{\infty}1 = 1 + 1 + \dotsb \to+\infty.
    \]
    If $\abs{x} < 1$, then $x^{n + 1}\to 0$, and so
    \[
      S_n\to \frac{1}{1 - x}.
    \]
    If $\abs{x} > 1$, then $(S_n)$ diverges. If $x = 1$ we have $(S_n)$ diverges. When $x = -1$ we have
    \[
      S_n = \frac{1-(-1)^{n + 1}}{2},
    \]
    which does not converge. Thus we have that $(S_n)$ converges if and only if $\abs{x} < 1$, and
    \[
      \sum_{n=0}^{\infty}x^n = \frac{1}{1 - x}.
    \]
    \item Inverse Squares---Series of the form
    \[
      \sum_{n=1}^{\infty} \frac{1}{n^2} = 1 + \frac{1}{4} + \frac{1}{9} + \frac{1}{16} + \dotsb
    \]
    Then we have
    \[
      S_n = \sum_{k=1}^{n} \frac{1}{k^2}.
    \]
    Since this sequence is increasing, it suffices to find an upper bound to show that it converges. Observe that for all $n\in\N$,
    \begin{align*}
      S_n &= 1 + \frac{1}{2\cdot 2} + \frac{1}{3\cdot 3} + \frac{1}{4\cdot 4} + \dotsb \\
          &\leq 1 + \frac{1}{2\cdot 1} + \frac{1}{3\cdot 2} + \dotsb + \frac{1}{n(n - 1)} \\
          &= 1 + \paren{1 - \frac{1}{2}} + \paren{\frac{1}{2} - \frac{1}{3}} + \dotsb + \paren{\frac{1}{n - 1} + \frac{1}{n}} \\
          &\leq 2 - \frac{1}{n} \\
          &\leq 2.
    \end{align*}
    Thus $(S_n)$ is bounded above and so by Monotone Convergence Theorem, we know that
    \[
      \lim_{n\to \infty} S_n
    \]
    exists.
    \newpage
    \item The Harmonic Series---The infinite series given by
    \[
      \sum_{n=1}^{\infty} \frac{1}{n} = 1 + \frac{1}{2} + \frac{1}{3} + \frac{1}{4} + \dotsb
    \]
    \begin{theorem}{Divergence of the Harmonic Series}
      The Harmonic Series diverges!
      \begin{proof}
        Let us write the harmonic series by
        \[
          S_n = \sum_{k=1}^{n} \frac{1}{k}.
        \]
        Since $S_n$ is increasing, it suffices to show that the series is unbounded from above. The idea here is that we will group the terms by powers of $2$. Observe that for all $n\in\N$, there exists some $m\in\N$ such that $2^{m} \leq n$, and so
        \begin{align*}
          S_n &\geq S_{2^m} \\
              &= \sum_{k=1}^{2^m} \frac{1}{k} \\
              &= 1 + \frac{1}{2} + \paren{\frac{1}{3} + \frac{1}{4}} + \paren{\frac{1}{5} + \frac{1}{6} + \frac{1}{7} + \frac{1}{8}} + \dotsb + \paren{\frac{1}{2^{m - 1} + 1} + \dotsb + \frac{1}{2^m}} \\
              &\geq 1 + \frac{1}{2} + \paren{\frac{1}{4} + \frac{1}{4}} + \paren{\frac{1}{8} + \dotsb + \frac{1}{8}} + \dotsb + \paren{\frac{1}{2^m} + \dotsb + \frac{1}{2^m}} \\
              &= 1 + \frac{1}{2} + \frac{2}{4} + \dotsb + \frac{2^{m - 1}}{2^m} \\
              &= 1 + \underbrace{\frac{1}{2} + \dotsb + \frac{1}{2}}_{m\text{ times}} \\
              &= 1 + \frac{m}{2}\to+\infty.
        \end{align*}
      \end{proof}
    \end{theorem}
  \end{enumerate}
  \newpage
  \subsection{Infinite Series Tests}
  \begin{theorem}{3.23 (Cauchy condensation test)}
    Suppose $(x_n)$ is \emph{decreasing} and $x_n\geq 0$ for all $n\in\N$. Then, the series $\sum_{n=1}^{\infty} x_n$ converges if and only if the series
    \[
      \sum_{n=0}^{\infty}\underbrace{2^n x_{2^n}}_{y_n}
    \]
    converges.
    \begin{proof}
      ($\Leftarrow$) Suppose $\sum 2^n x_{2^n}$ converges. Then
      \[
        t_n = \sum_{k=1}^{n} 2^kx_{2^k}
      \]
      converges, and so is bounded. Let
      \[
        S_n = \sum_{k=0}^{n} x_k. 
      \]
      We want to show that $S_n < t_m$ (bounded above). The idea here is to partition $[1, n]$ into dyadic intervals. Fix $n\in\N$ and let $m$ be large enough so that $n\leq 2^{m + 1} - 1$. Then we have
      \begin{align*}
        S_n &\leq S_{2^{m + 1} - 1} \\
            &= \sum_{k=1}^{2^{m + 1} - 1}x_k \\
            &= \sum_{\ell=0}^{m} \sum_{k=2^\ell}^{2^{\ell + 1} - 1}x_k \\
            &\leq \sum_{\ell=0}^{m} \max_{2^\ell\leq k\leq 2^{\ell + 1} - 1} x_k \cdot \sum_{k=2^\ell}^{2^{\ell + 1} - 1}1 \\
            &\leq \sum_{\ell=0}^{m} x_{2^\ell} (2^{\ell + 1} - 1 - 2^\ell + 1) \\
            &\leq \sum_{\ell=0}^{m}2^\ell x_{2^\ell} \\
            &= t_m.
      \end{align*}
    \end{proof}
  \end{theorem}
  \textbf{Corollary 3.24} The series
  \[
    \sum_{n=1}^{\infty} \frac{1}{n^p}
  \]
  converges if and only if $p > 1$.
  \begin{proof}
    \begin{align*}
      \sum_{n=0}^{N} 2^nx_{2^n} &= \sum_{n=0}^{N}2^n \paren{\frac{1}{2^n}}^p \\
                                &= \sum_{n=0}^{N} \frac{2^n}{2^{np}} \\
                                &= \sum_{n=0}^{N} \paren{\frac{1}{2^{p - 1}}}^N,
    \end{align*}
    which converges by Geometric series.
  \end{proof}
\end{document}
