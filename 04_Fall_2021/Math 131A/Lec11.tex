\documentclass[class=article, crop=false]{standalone}
% Import packages
\usepackage[margin=1in]{geometry}

\usepackage{amssymb, amsthm}
\usepackage{comment}
\usepackage{enumitem}
\usepackage{fancyhdr}
\usepackage{hyperref}
\usepackage{import}
\usepackage{mathrsfs, mathtools}
\usepackage{multicol}
\usepackage{pdfpages}
\usepackage{standalone}
\usepackage{transparent}
\usepackage{xcolor}

\usepackage{minted}
\usepackage[many]{tcolorbox}
\tcbuselibrary{minted}

% Declare math operators
\DeclareMathOperator{\lcm}{lcm}
\DeclareMathOperator{\proj}{proj}
\DeclareMathOperator{\vspan}{span}
\DeclareMathOperator{\im}{im}
\DeclareMathOperator{\Diff}{Diff}
\DeclareMathOperator{\Int}{Int}
\DeclareMathOperator{\fcn}{fcn}
\DeclareMathOperator{\id}{id}
\DeclareMathOperator{\rank}{rank}
\DeclareMathOperator{\tr}{tr}
\DeclareMathOperator{\dive}{div}
\DeclareMathOperator{\row}{row}
\DeclareMathOperator{\col}{col}
\DeclareMathOperator{\dom}{dom}
\DeclareMathOperator{\ran}{ran}
\DeclareMathOperator{\Dom}{Dom}
\DeclareMathOperator{\Ran}{Ran}
\DeclareMathOperator{\fl}{fl}
% Macros for letters/variables
\newcommand\ttilde{\kern -.15em\lower .7ex\hbox{\~{}}\kern .04em}
\newcommand{\N}{\ensuremath{\mathbb{N}}}
\newcommand{\Z}{\ensuremath{\mathbb{Z}}}
\newcommand{\Q}{\ensuremath{\mathbb{Q}}}
\newcommand{\R}{\ensuremath{\mathbb{R}}}
\newcommand{\C}{\ensuremath{\mathbb{C}}}
\newcommand{\F}{\ensuremath{\mathbb{F}}}
\newcommand{\M}{\ensuremath{\mathbb{M}}}
\renewcommand{\P}{\ensuremath{\mathbb{P}}}
\newcommand{\lam}{\ensuremath{\lambda}}
\newcommand{\nab}{\ensuremath{\nabla}}
\newcommand{\eps}{\ensuremath{\varepsilon}}
\newcommand{\es}{\ensuremath{\varnothing}}
% Macros for math symbols
\newcommand{\dx}[1]{\,\mathrm{d}#1}
\newcommand{\inv}{\ensuremath{^{-1}}}
\newcommand{\sm}{\setminus}
\newcommand{\sse}{\subseteq}
\newcommand{\ceq}{\coloneqq}
% Macros for pairs of math symbols
\newcommand{\abs}[1]{\ensuremath{\left\lvert #1 \right\rvert}}
\newcommand{\paren}[1]{\ensuremath{\left( #1 \right)}}
\newcommand{\norm}[1]{\ensuremath{\left\lVert #1\right\rVert}}
\newcommand{\set}[1]{\ensuremath{\left\{#1\right\}}}
\newcommand{\tup}[1]{\ensuremath{\left\langle #1 \right\rangle}}
\newcommand{\floor}[1]{\ensuremath{\left\lfloor #1 \right\rfloor}}
\newcommand{\ceil}[1]{\ensuremath{\left\lceil #1 \right\rceil}}
\newcommand{\eclass}[1]{\ensuremath{\left[ #1 \right]}}

\newcommand{\chapternum}{}
\newcommand{\ex}[1]{\noindent\textbf{Exercise \chapternum.{#1}.}}

\newcommand{\tsub}[1]{\textsubscript{#1}}
\newcommand{\tsup}[1]{\textsuperscript{#1}}

\renewcommand{\choose}[2]{\ensuremath{\phantom{}_{#1}C_{#2}}}
\newcommand{\perm}[2]{\ensuremath{\phantom{}_{#1}P_{#2}}}

% Include figures
\newcommand{\incfig}[2][1]{%
    \def\svgwidth{#1\columnwidth}
    \import{./figures/}{#2.pdf_tex}
}

\definecolor{problemBackground}{RGB}{212,232,246}
\newenvironment{problem}[1]
  {
    \begin{tcolorbox}[
      boxrule=.5pt,
      titlerule=.5pt,
      sharp corners,
      colback=problemBackground,
      breakable
    ]
    \ifx &#1& \textbf{Problem. }
    \else \textbf{Problem #1.} \fi
  }
  {
    \end{tcolorbox}
  }
\definecolor{exampleBackground}{RGB}{255,249,248}
\definecolor{exampleAccent}{RGB}{158,60,14}
\newenvironment{example}[1]
  {
    \begin{tcolorbox}[
      boxrule=.5pt,
      sharp corners,
      colback=exampleBackground,
      colframe=exampleAccent,
    ]
    \color{exampleAccent}\textbf{Example.} \emph{#1}\color{black}
  }
  {
    \end{tcolorbox}
  }
\definecolor{theoremBackground}{RGB}{234,243,251}
\definecolor{theoremAccent}{RGB}{0,116,183}
\newenvironment{theorem}[1]
  {
    \begin{tcolorbox}[
      boxrule=.5pt,
      titlerule=.5pt,
      sharp corners,
      colback=theoremBackground,
      colframe=theoremAccent,
      breakable
    ]
      % \color{theoremAccent}\textbf{Theorem --- }\emph{#1}\\\color{black}
    \ifx &#1& \color{theoremAccent}\textbf{Theorem. }\color{black}
    \else \color{theoremAccent}\textbf{Theorem --- }\emph{#1}\\\color{black} \fi
  }
  {
    \end{tcolorbox}
  }
\definecolor{noteBackground}{RGB}{244,249,244}
\definecolor{noteAccent}{RGB}{34,139,34}
\newenvironment{note}[1]
  {
  \begin{tcolorbox}[
    enhanced,
    boxrule=0pt,
    frame hidden,
    sharp corners,
    colback=noteBackground,
    borderline west={3pt}{-1.5pt}{noteAccent},
    breakable
    ]
    \ifx &#1& \color{noteAccent}\textbf{Note. }\color{black}
    \else \color{noteAccent}\textbf{Note (#1). }\color{black} \fi
    }
    {
  \end{tcolorbox}
  }
\definecolor{lemmaBackground}{RGB}{255,247,234}
\definecolor{lemmaAccent}{RGB}{255,153,0}
\newenvironment{lemma}[1]
  {
    \begin{tcolorbox}[
      enhanced,
      boxrule=0pt,
      frame hidden,
      sharp corners,
      colback=lemmaBackground,
      borderline west={3pt}{-1.5pt}{lemmaAccent},
      breakable
    ]
    \ifx &#1& \color{lemmaAccent}\textbf{Lemma. }\color{black}
    \else \color{lemmaAccent}\textbf{Lemma #1. }\color{black} \fi
  }
  {
    \end{tcolorbox}
  }
\definecolor{definitionBackground}{RGB}{246,246,246}
\newenvironment{definition}[1]
  {
    \begin{tcolorbox}[
      enhanced,
      boxrule=0pt,
      frame hidden,
      sharp corners,
      colback=definitionBackground,
      borderline west={3pt}{-1.5pt}{black},
      breakable
    ]
    \textbf{Definition. }\emph{#1}\\
  }
  {
    \end{tcolorbox}
  }

\newenvironment{amatrix}[2]{
    \left[
      \begin{array}{*{#1}{c}|*{#2}c}
  }
  {
      \end{array}
    \right]
  }

\definecolor{codeBackground}{RGB}{253,246,227}
\renewcommand\theFancyVerbLine{\arabic{FancyVerbLine}}
\newtcblisting{cpp}[1][]{
  boxrule=1pt,
  sharp corners,
  colback=codeBackground,
  listing only,
  breakable,
  minted language=cpp,
  minted style=material,
  minted options={
    xleftmargin=15pt,
    baselinestretch=1.2,
    linenos,
  }
}

\author{Kyle Chui}


\fancyhf{}
\lhead{Kyle Chui}
\rhead{Page \thepage}
\pagestyle{fancy}

\begin{document}
  \section{Lecture 11}
  \subsection{Proof for Algebraic Limit Theorem}
  We want to show that
  \[
    \abs{x_ny_n - xy} < \eps
  \]
  for all $\eps > 0$.
  \begin{proof}
    Observe that
    \begin{align*}
      \abs{x_ny_n - xy} &= \abs{x_ny_n - x_ny + x_ny - xy} \\
                        &= \abs{x_n(y_n - y) + (x_n - x)y} \\
                        &\leq \abs{x_n(y_n - y)} + \abs{y(x_n - x)} \\
                        &= \abs{x_n}\abs{y_n - y} + \abs{y}\abs{x_n - x} \\
      \intertext{Since convergent sequences are bounded, we know that there exist some $M > 0$ such that $\abs{x_n}\leq M$ for all $n\in\N$.}
                        &\leq M\abs{y_n - y} + \abs{y}\abs{x_n - x} \\
      \intertext{Fix $\eps > 0$. Then there exists $N_1$ such that $\abs{x_n - x} < \frac{\eps}{2(1 + \abs{y})}$ for all $n > N_1$. Furthermore, we know there exists some $N_2$ such that $\abs{y_n - y} < \frac{\eps}{2(1 + M)}$ for all $n > N_2$.}
                        &< \frac{M\eps}{2(1 + M)} + \frac{\abs{y}\eps}{2(1 + \abs{y})} \\
                        &< \frac{\eps}{2} + \frac{\eps}{2} \\
                        &< \eps,
    \end{align*}
    for all $\eps > 0$. Therefore the limit exists and
    \[
      \lim_{n\to \infty} x_ny_n = xy.
    \]
  \end{proof}
  We now want to show that
  \[
    \lim_{n\to \infty} \paren{\frac{x_n}{y_n}} = \frac{x}{y}.
  \]
  \begin{proof}
    Observe that
    \begin{align*}
      \abs{\frac{x_n}{y_n} - \frac{x}{y}} &= \abs{\frac{x_ny - xy_n}{y_ny}} \\
                                          &= \frac{1}{\abs{y_n}\abs{y}}\abs{x_ny - xy_n} \\
                                          &= \frac{1}{\abs{y_n}\abs{y}} \abs{x_ny - xy + xy - xy_n} \\
                                          &\leq \frac{1}{\abs{y_n}\abs{y}} \abs{y(x_n - x) + x(y - y_n)} \\
                                          &= \frac{1}{\abs{y_n}\abs{y}} \paren{\abs{y}\abs{x_n - x}+ \abs{x}\abs{y - y_n}} \\
      \intertext{We now need to show that $\frac{1}{\abs{y_n}}$ is upper bounded by some value. Since $(y_n)$ converges, we know that for all $\eps > 0$, there exists some $N\in \N$ such that $\abs{y_n - y} < \eps$. Suppose we choose $\eps = \frac{y}{2}$, so we have $\frac{1}{y_n} \leq \frac{2}{y}$. In a fashion similar to the previous part, we may manipulate the above expression to complete the proof.}
    \end{align*}
  \end{proof}
  \begin{theorem}{3.8 (Order Limit Theorem)}
    Assume $x_n\to x$ and $y_n\to y$.
    \begin{enumerate}[label=(\roman*)]
      \item If $x_n\geq 0$ for all $n\in\N$, then $x\geq 0$.
      \item If $x_n\leq y_n$ for all $n\in\N$, then $x\leq y$.
      \item If there exists $a\in\R$ such that $a\leq x_n$ for all $n\in\N$, then $a\leq x$. If there exists $b\in\R$ such that $b\geq x_n$ for all $n\in\N$, then $b\geq x$.
    \end{enumerate}
  \end{theorem}
  \begin{note}{}
    Strict inequalities are not necessarily respected! If $x_n\to x$ and $x_n > 0$, then the most we can say is that $x_n\geq 0$. For example, consider the case where $x_n = \frac{1}{n} > 0$, and $x = 0$.
  \end{note}
  \subsection{Monotone Sequences}
  \begin{definition}{3.9---Monotonicity}
    A sequence $(x_n)$ is \emph{increasing} if $x_n\leq x_{n + 1}$ for all $n\in \N$. A sequence $(x_n)$ is \emph{decreasing} if $x_n\geq x_{n + 1}$ for all $n\in \N$. We call both increasing and decreasing sequences \emph{monotone} or \emph{monotonic}.
  \end{definition}
  \begin{note}{}
    If you iterate the definition for an increasing sequence, you get $x_n\leq x_m$ for $n\leq m$. A similar result comes from iterating the definition for a decreasing sequence.
  \end{note}
  \begin{example}{Monotone Sequences}
    \begin{itemize}
      \item $x_n = 1-\frac{1}{n}$ is increasing (and converges to $1$).
      \item $x_n = \frac{1}{2^n}$ is decreasing (and converges to $0$).
      \item $x_n = n$ is increasing (but is neither convergent nor bounded).
      \item $x_n = (-1)^n$ is not monotonic (and does not converge).
    \end{itemize}
  \end{example}
  \begin{note}{}
    In the above examples, if a sequence is monotonic, then if it is bounded it converges, otherwise diverges. Can we prove this?
  \end{note}
  \begin{theorem}{3.10 (Monotone Convergence Theorem)}
    Every monotonic and bounded sequence in $\R$ necessarily converges.
    \begin{proof}
      Let $(x_n)$ be a sequence in $\R$ which is both increasing and bounded (same argument works for decreasing). If we look at the set
      \[
        S = \set{x_n\mid n\in\N}\sse \R,
      \]
      we see that this set is non-empty and bounded above (because $(x_n)$ is bounded). Thus by the least upper bound property of $\R$ we know that $\sup S$ exists. We claim that $x = \sup S$ is the limit for $(x_n)$. Let $\eps > 0$. Then there exists some $x_N\in S$ such that $x_N > x - \eps$. We know that $(x_n)$ is increasing, then $x - \eps < x_N < x_n$ for all $n > N$. Furthermore, since $x$ is the supremum of $S$, we know that $x_n < x < x + \eps$. Thus we have
      \[
        x - \eps < x_n < x + \eps,
      \]
      so $\abs{x_n - x} < \eps$ for all $\eps > 0$ and all $n > N$.
    \end{proof}
  \end{theorem}
\end{document}
