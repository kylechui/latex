\documentclass[class=article, crop=false]{standalone}
% Import packages
\usepackage[margin=1in]{geometry}

\usepackage{amssymb, amsthm}
\usepackage{comment}
\usepackage{enumitem}
\usepackage{fancyhdr}
\usepackage{hyperref}
\usepackage{import}
\usepackage{mathrsfs, mathtools}
\usepackage{multicol}
\usepackage{pdfpages}
\usepackage{standalone}
\usepackage{transparent}
\usepackage{xcolor}

\usepackage{minted}
\usepackage[many]{tcolorbox}
\tcbuselibrary{minted}

% Declare math operators
\DeclareMathOperator{\lcm}{lcm}
\DeclareMathOperator{\proj}{proj}
\DeclareMathOperator{\vspan}{span}
\DeclareMathOperator{\im}{im}
\DeclareMathOperator{\Diff}{Diff}
\DeclareMathOperator{\Int}{Int}
\DeclareMathOperator{\fcn}{fcn}
\DeclareMathOperator{\id}{id}
\DeclareMathOperator{\rank}{rank}
\DeclareMathOperator{\tr}{tr}
\DeclareMathOperator{\dive}{div}
\DeclareMathOperator{\row}{row}
\DeclareMathOperator{\col}{col}
\DeclareMathOperator{\dom}{dom}
\DeclareMathOperator{\ran}{ran}
\DeclareMathOperator{\Dom}{Dom}
\DeclareMathOperator{\Ran}{Ran}
\DeclareMathOperator{\fl}{fl}
% Macros for letters/variables
\newcommand\ttilde{\kern -.15em\lower .7ex\hbox{\~{}}\kern .04em}
\newcommand{\N}{\ensuremath{\mathbb{N}}}
\newcommand{\Z}{\ensuremath{\mathbb{Z}}}
\newcommand{\Q}{\ensuremath{\mathbb{Q}}}
\newcommand{\R}{\ensuremath{\mathbb{R}}}
\newcommand{\C}{\ensuremath{\mathbb{C}}}
\newcommand{\F}{\ensuremath{\mathbb{F}}}
\newcommand{\M}{\ensuremath{\mathbb{M}}}
\renewcommand{\P}{\ensuremath{\mathbb{P}}}
\newcommand{\lam}{\ensuremath{\lambda}}
\newcommand{\nab}{\ensuremath{\nabla}}
\newcommand{\eps}{\ensuremath{\varepsilon}}
\newcommand{\es}{\ensuremath{\varnothing}}
% Macros for math symbols
\newcommand{\dx}[1]{\,\mathrm{d}#1}
\newcommand{\inv}{\ensuremath{^{-1}}}
\newcommand{\sm}{\setminus}
\newcommand{\sse}{\subseteq}
\newcommand{\ceq}{\coloneqq}
% Macros for pairs of math symbols
\newcommand{\abs}[1]{\ensuremath{\left\lvert #1 \right\rvert}}
\newcommand{\paren}[1]{\ensuremath{\left( #1 \right)}}
\newcommand{\norm}[1]{\ensuremath{\left\lVert #1\right\rVert}}
\newcommand{\set}[1]{\ensuremath{\left\{#1\right\}}}
\newcommand{\tup}[1]{\ensuremath{\left\langle #1 \right\rangle}}
\newcommand{\floor}[1]{\ensuremath{\left\lfloor #1 \right\rfloor}}
\newcommand{\ceil}[1]{\ensuremath{\left\lceil #1 \right\rceil}}
\newcommand{\eclass}[1]{\ensuremath{\left[ #1 \right]}}

\newcommand{\chapternum}{}
\newcommand{\ex}[1]{\noindent\textbf{Exercise \chapternum.{#1}.}}

\newcommand{\tsub}[1]{\textsubscript{#1}}
\newcommand{\tsup}[1]{\textsuperscript{#1}}

\renewcommand{\choose}[2]{\ensuremath{\phantom{}_{#1}C_{#2}}}
\newcommand{\perm}[2]{\ensuremath{\phantom{}_{#1}P_{#2}}}

% Include figures
\newcommand{\incfig}[2][1]{%
    \def\svgwidth{#1\columnwidth}
    \import{./figures/}{#2.pdf_tex}
}

\definecolor{problemBackground}{RGB}{212,232,246}
\newenvironment{problem}[1]
  {
    \begin{tcolorbox}[
      boxrule=.5pt,
      titlerule=.5pt,
      sharp corners,
      colback=problemBackground,
      breakable
    ]
    \ifx &#1& \textbf{Problem. }
    \else \textbf{Problem #1.} \fi
  }
  {
    \end{tcolorbox}
  }
\definecolor{exampleBackground}{RGB}{255,249,248}
\definecolor{exampleAccent}{RGB}{158,60,14}
\newenvironment{example}[1]
  {
    \begin{tcolorbox}[
      boxrule=.5pt,
      sharp corners,
      colback=exampleBackground,
      colframe=exampleAccent,
    ]
    \color{exampleAccent}\textbf{Example.} \emph{#1}\color{black}
  }
  {
    \end{tcolorbox}
  }
\definecolor{theoremBackground}{RGB}{234,243,251}
\definecolor{theoremAccent}{RGB}{0,116,183}
\newenvironment{theorem}[1]
  {
    \begin{tcolorbox}[
      boxrule=.5pt,
      titlerule=.5pt,
      sharp corners,
      colback=theoremBackground,
      colframe=theoremAccent,
      breakable
    ]
      % \color{theoremAccent}\textbf{Theorem --- }\emph{#1}\\\color{black}
    \ifx &#1& \color{theoremAccent}\textbf{Theorem. }\color{black}
    \else \color{theoremAccent}\textbf{Theorem --- }\emph{#1}\\\color{black} \fi
  }
  {
    \end{tcolorbox}
  }
\definecolor{noteBackground}{RGB}{244,249,244}
\definecolor{noteAccent}{RGB}{34,139,34}
\newenvironment{note}[1]
  {
  \begin{tcolorbox}[
    enhanced,
    boxrule=0pt,
    frame hidden,
    sharp corners,
    colback=noteBackground,
    borderline west={3pt}{-1.5pt}{noteAccent},
    breakable
    ]
    \ifx &#1& \color{noteAccent}\textbf{Note. }\color{black}
    \else \color{noteAccent}\textbf{Note (#1). }\color{black} \fi
    }
    {
  \end{tcolorbox}
  }
\definecolor{lemmaBackground}{RGB}{255,247,234}
\definecolor{lemmaAccent}{RGB}{255,153,0}
\newenvironment{lemma}[1]
  {
    \begin{tcolorbox}[
      enhanced,
      boxrule=0pt,
      frame hidden,
      sharp corners,
      colback=lemmaBackground,
      borderline west={3pt}{-1.5pt}{lemmaAccent},
      breakable
    ]
    \ifx &#1& \color{lemmaAccent}\textbf{Lemma. }\color{black}
    \else \color{lemmaAccent}\textbf{Lemma #1. }\color{black} \fi
  }
  {
    \end{tcolorbox}
  }
\definecolor{definitionBackground}{RGB}{246,246,246}
\newenvironment{definition}[1]
  {
    \begin{tcolorbox}[
      enhanced,
      boxrule=0pt,
      frame hidden,
      sharp corners,
      colback=definitionBackground,
      borderline west={3pt}{-1.5pt}{black},
      breakable
    ]
    \textbf{Definition. }\emph{#1}\\
  }
  {
    \end{tcolorbox}
  }

\newenvironment{amatrix}[2]{
    \left[
      \begin{array}{*{#1}{c}|*{#2}c}
  }
  {
      \end{array}
    \right]
  }

\definecolor{codeBackground}{RGB}{253,246,227}
\renewcommand\theFancyVerbLine{\arabic{FancyVerbLine}}
\newtcblisting{cpp}[1][]{
  boxrule=1pt,
  sharp corners,
  colback=codeBackground,
  listing only,
  breakable,
  minted language=cpp,
  minted style=material,
  minted options={
    xleftmargin=15pt,
    baselinestretch=1.2,
    linenos,
  }
}

\author{Kyle Chui}


\fancyhf{}
\lhead{Kyle Chui}
\rhead{Page \thepage}
\pagestyle{fancy}

\begin{document}
  \section{Lecture 3}
  \subsection{More Logic}
  \subsubsection{Proof by Contradiction}
  To do a proof by contradiction for a statement $P\implies Q$, we assume $P\land \neg Q$. We aim to show that $P\land \neg Q$ is false (a contradiction).
  \begin{theorem}{Irrationality of $\sqrt{2}$}
    There is no rational number $x$ such that $x^2 = 2$. In other words, if $x\in \Q$, then $x^2\neq 2$.
    \begin{proof}
      Suppose towards a contradiction that there exists some $x\in \Q$ such that $x^2 = 2$. Since $x$ is rational, there exist integers $p, q$ such that $q\neq 0$, $\frac{p}{q} = x$, and $p$ and $q$ have no common divisors (other than 1). Then
      \begin{align*}
        x^2 &= 2 \\
        \frac{p^2}{q^2} &= 2 \\
        p^2 &= 2q^2.
        \intertext{Since $p^2$ is even, there exists some integer $k$ such that $p = 2k$. Thus}
        (2k)^2 &= 2q^2 \\
        4k^2 &= 2q^2 \\
        2k^2 &= q^2.
      \end{align*}
      By the same logic as before, we know that $q$ must also be even (they share a common factor of $2$). However, this contradicts our original assumption that $p$ and $q$ share no common factors, and this completes the proof.
    \end{proof}
  \end{theorem}
  \subsection{Set Theory}
  We write $x\in A$ when we want to say that ``$x$ is an element of $A$'', and $x\notin A$ when we want to say that ``$x$ is not an element of $A$''.
  \subsubsection{Set Combinations}
  \begin{itemize}
    \item Union: $A\cup B = \set{x\mid x\in A\lor x\in B}$.
    \item Intersection: $A\cap B = \set{x\mid x\in A\land x\in B}$.
    \item Difference: $A\sm B = \set{x\mid x\in A\land x\notin B}$.
    \item Subset (Inclusion): $A\sse B$ if and only if $x\in A\implies x\in B$.
    \begin{definition}{Proper Subset}
      A set $A$ is a \emph{proper subset} of a set $B$ if $A\sse B$ and there exists some $x\in B$ such that $x\notin A$. We denote this as $A\subset B$.
    \end{definition}
    \item Equality: $A = B$ if and only if $A\sse B$ and $B\sse A$.
    \begin{note}{Showing Equality of Sets}
      If you want to show $A = B$, you need to show both $A\sse B$ and $B\sse A$. In other words, you must show that for all $x\in A$, we have $x\in B$, and vice versa.
    \end{note}
  \end{itemize}
  \begin{example}{}
    We have $\N = \set{1, 2, 3, \dotsc}$.
    \begin{itemize}
      \item Let $E$ be the set of even natural numbers. Note that $E\sse \N$.
      \item Let $S = \set{p\in \Q\mid p^2 < 2}\sse \Q$.
    \end{itemize}
    Then we have:
    \begin{itemize}
      \item $\N\cup E = \N$.
      \item $\N\cap E = E$.
      \item $\N\cap S = \set{n\in \N\mid n^2 < 2} = \set{1}$.
      \item $E\cap S = \es$.
      \begin{definition}{Disjoint Sets}
        If $A\cap B = \es$, we call $A$ and $B$ \emph{disjoint} sets.
      \end{definition}
      \begin{proof}
        Suppose towards a contradiction that there exists some $x\in E\cap S$, which is to say $x\in E$ and $x\in S$. Since $x\in E$, we know that $x$ is even, and so there exists some integer $k$ such that $x = 2k$. Then
        \[
          x^2 = (2k)^2 = 4k^2,
        \]
        so $4\mid x^2$. Therefore $x\geq 4$, which contradicts the condition for $x\in S$, namely $x^2 < 2$.
      \end{proof}
      \item Given some $n\in \N$, we define $A_n = \N\sm \set{1,\dotsc,n - 1}$.
      \begin{itemize}
        \item $\bigcup_{n\in \N} A_n = \N$.
        \item $\bigcap_{n\in \N} A_n = \es$.
      \end{itemize}
    \end{itemize}
  \end{example}
  \begin{definition}{Set Complement}
    If $A\sse B$, then we define the \emph{complement} of $A$ in $B$ to be $A^c = B\sm A$.
  \end{definition}
  \subsubsection{De Morgan's Laws}
  If $I$ is an index set and $\set{A_j}_{j\in I}$ are subsets of $B$, then
  \[
    \paren{\bigcup_{j\in I}A_j}^c = \bigcap_{j\in I}A_j^c, \quad\text{and}\quad \paren{\bigcap_{j\in I}A_j}^c = \bigcup_{j\in I}A_j^c
  \]
\end{document}
