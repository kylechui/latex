\documentclass[class=article, crop=false]{standalone}
% Import packages
\usepackage[margin=1in]{geometry}

\usepackage{amssymb, amsthm}
\usepackage{comment}
\usepackage{enumitem}
\usepackage{fancyhdr}
\usepackage{hyperref}
\usepackage{import}
\usepackage{mathrsfs, mathtools}
\usepackage{multicol}
\usepackage{pdfpages}
\usepackage{standalone}
\usepackage{transparent}
\usepackage{xcolor}

\usepackage{minted}
\usepackage[many]{tcolorbox}
\tcbuselibrary{minted}

% Declare math operators
\DeclareMathOperator{\lcm}{lcm}
\DeclareMathOperator{\proj}{proj}
\DeclareMathOperator{\vspan}{span}
\DeclareMathOperator{\im}{im}
\DeclareMathOperator{\Diff}{Diff}
\DeclareMathOperator{\Int}{Int}
\DeclareMathOperator{\fcn}{fcn}
\DeclareMathOperator{\id}{id}
\DeclareMathOperator{\rank}{rank}
\DeclareMathOperator{\tr}{tr}
\DeclareMathOperator{\dive}{div}
\DeclareMathOperator{\row}{row}
\DeclareMathOperator{\col}{col}
\DeclareMathOperator{\dom}{dom}
\DeclareMathOperator{\ran}{ran}
\DeclareMathOperator{\Dom}{Dom}
\DeclareMathOperator{\Ran}{Ran}
\DeclareMathOperator{\fl}{fl}
% Macros for letters/variables
\newcommand\ttilde{\kern -.15em\lower .7ex\hbox{\~{}}\kern .04em}
\newcommand{\N}{\ensuremath{\mathbb{N}}}
\newcommand{\Z}{\ensuremath{\mathbb{Z}}}
\newcommand{\Q}{\ensuremath{\mathbb{Q}}}
\newcommand{\R}{\ensuremath{\mathbb{R}}}
\newcommand{\C}{\ensuremath{\mathbb{C}}}
\newcommand{\F}{\ensuremath{\mathbb{F}}}
\newcommand{\M}{\ensuremath{\mathbb{M}}}
\renewcommand{\P}{\ensuremath{\mathbb{P}}}
\newcommand{\lam}{\ensuremath{\lambda}}
\newcommand{\nab}{\ensuremath{\nabla}}
\newcommand{\eps}{\ensuremath{\varepsilon}}
\newcommand{\es}{\ensuremath{\varnothing}}
% Macros for math symbols
\newcommand{\dx}[1]{\,\mathrm{d}#1}
\newcommand{\inv}{\ensuremath{^{-1}}}
\newcommand{\sm}{\setminus}
\newcommand{\sse}{\subseteq}
\newcommand{\ceq}{\coloneqq}
% Macros for pairs of math symbols
\newcommand{\abs}[1]{\ensuremath{\left\lvert #1 \right\rvert}}
\newcommand{\paren}[1]{\ensuremath{\left( #1 \right)}}
\newcommand{\norm}[1]{\ensuremath{\left\lVert #1\right\rVert}}
\newcommand{\set}[1]{\ensuremath{\left\{#1\right\}}}
\newcommand{\tup}[1]{\ensuremath{\left\langle #1 \right\rangle}}
\newcommand{\floor}[1]{\ensuremath{\left\lfloor #1 \right\rfloor}}
\newcommand{\ceil}[1]{\ensuremath{\left\lceil #1 \right\rceil}}
\newcommand{\eclass}[1]{\ensuremath{\left[ #1 \right]}}

\newcommand{\chapternum}{}
\newcommand{\ex}[1]{\noindent\textbf{Exercise \chapternum.{#1}.}}

\newcommand{\tsub}[1]{\textsubscript{#1}}
\newcommand{\tsup}[1]{\textsuperscript{#1}}

\renewcommand{\choose}[2]{\ensuremath{\phantom{}_{#1}C_{#2}}}
\newcommand{\perm}[2]{\ensuremath{\phantom{}_{#1}P_{#2}}}

% Include figures
\newcommand{\incfig}[2][1]{%
    \def\svgwidth{#1\columnwidth}
    \import{./figures/}{#2.pdf_tex}
}

\definecolor{problemBackground}{RGB}{212,232,246}
\newenvironment{problem}[1]
  {
    \begin{tcolorbox}[
      boxrule=.5pt,
      titlerule=.5pt,
      sharp corners,
      colback=problemBackground,
      breakable
    ]
    \ifx &#1& \textbf{Problem. }
    \else \textbf{Problem #1.} \fi
  }
  {
    \end{tcolorbox}
  }
\definecolor{exampleBackground}{RGB}{255,249,248}
\definecolor{exampleAccent}{RGB}{158,60,14}
\newenvironment{example}[1]
  {
    \begin{tcolorbox}[
      boxrule=.5pt,
      sharp corners,
      colback=exampleBackground,
      colframe=exampleAccent,
    ]
    \color{exampleAccent}\textbf{Example.} \emph{#1}\color{black}
  }
  {
    \end{tcolorbox}
  }
\definecolor{theoremBackground}{RGB}{234,243,251}
\definecolor{theoremAccent}{RGB}{0,116,183}
\newenvironment{theorem}[1]
  {
    \begin{tcolorbox}[
      boxrule=.5pt,
      titlerule=.5pt,
      sharp corners,
      colback=theoremBackground,
      colframe=theoremAccent,
      breakable
    ]
      % \color{theoremAccent}\textbf{Theorem --- }\emph{#1}\\\color{black}
    \ifx &#1& \color{theoremAccent}\textbf{Theorem. }\color{black}
    \else \color{theoremAccent}\textbf{Theorem --- }\emph{#1}\\\color{black} \fi
  }
  {
    \end{tcolorbox}
  }
\definecolor{noteBackground}{RGB}{244,249,244}
\definecolor{noteAccent}{RGB}{34,139,34}
\newenvironment{note}[1]
  {
  \begin{tcolorbox}[
    enhanced,
    boxrule=0pt,
    frame hidden,
    sharp corners,
    colback=noteBackground,
    borderline west={3pt}{-1.5pt}{noteAccent},
    breakable
    ]
    \ifx &#1& \color{noteAccent}\textbf{Note. }\color{black}
    \else \color{noteAccent}\textbf{Note (#1). }\color{black} \fi
    }
    {
  \end{tcolorbox}
  }
\definecolor{lemmaBackground}{RGB}{255,247,234}
\definecolor{lemmaAccent}{RGB}{255,153,0}
\newenvironment{lemma}[1]
  {
    \begin{tcolorbox}[
      enhanced,
      boxrule=0pt,
      frame hidden,
      sharp corners,
      colback=lemmaBackground,
      borderline west={3pt}{-1.5pt}{lemmaAccent},
      breakable
    ]
    \ifx &#1& \color{lemmaAccent}\textbf{Lemma. }\color{black}
    \else \color{lemmaAccent}\textbf{Lemma #1. }\color{black} \fi
  }
  {
    \end{tcolorbox}
  }
\definecolor{definitionBackground}{RGB}{246,246,246}
\newenvironment{definition}[1]
  {
    \begin{tcolorbox}[
      enhanced,
      boxrule=0pt,
      frame hidden,
      sharp corners,
      colback=definitionBackground,
      borderline west={3pt}{-1.5pt}{black},
      breakable
    ]
    \textbf{Definition. }\emph{#1}\\
  }
  {
    \end{tcolorbox}
  }

\newenvironment{amatrix}[2]{
    \left[
      \begin{array}{*{#1}{c}|*{#2}c}
  }
  {
      \end{array}
    \right]
  }

\definecolor{codeBackground}{RGB}{253,246,227}
\renewcommand\theFancyVerbLine{\arabic{FancyVerbLine}}
\newtcblisting{cpp}[1][]{
  boxrule=1pt,
  sharp corners,
  colback=codeBackground,
  listing only,
  breakable,
  minted language=cpp,
  minted style=material,
  minted options={
    xleftmargin=15pt,
    baselinestretch=1.2,
    linenos,
  }
}

\author{Kyle Chui}


\fancyhf{}
\lhead{Kyle Chui}
\rhead{Page \thepage}
\pagestyle{fancy}

\begin{document}
  \section{Lecture 19}
  \begin{theorem}{4.6 (Algebraic Limit Theorem for Functional Limits)}
    Let $f, g$ be defined on $A$, $c$ a limit point of $A$, and
    \[
      \lim_{x\to c} f(x) = L, \lim_{x\to c} g(x) = M.
    \]
    Then
    \begin{enumerate}[label=(\roman*)]
      \item $\displaystyle \lim_{x\to c} [af(x) + bg(x)] = aL + bM$ for all $a, b\in\R$.
      \item $\displaystyle \lim_{x\to c} [f(x)\cdot g(x)] = L\cdot M$.
      \item $\displaystyle \lim_{x\to c} \left[\frac{f(x)}{g(x)}\right] = \frac{L}{M}$, provided $M\neq 0$.
    \end{enumerate}
  \end{theorem}
  \textbf{Proposition 4.7} (Divergence Criterion for Functional Limits) Let $f\colon A\to \R$, $c$ a limit point of $A$. If $(x_n), (y_n)\sse A$ such that $x_n\neq c, y_n\neq c$ and $x_n\to c, y_n\to c$, and
  \[
    \lim_{n\to \infty} f(x_n)\neq \lim_{n\to \infty} f(y_n),
  \]
  then $\displaystyle\lim_{x\to c} f(x)$ does not exist.
  \begin{example}{}
    Consider the function $f(x) = \sin (\frac{1}{x})$, where $A = \R\sm\set{0}$. We claim that $\displaystyle\lim_{x\to 0} f(x)$ does not exist. \par
    If we choose $x_n = \frac{1}{2n\pi}$, we have $f(x_n) = \sin(2\pi n)\to 0$. If we choose $y_n = \frac{1}{2\pi n + \frac{\pi}{2}}$, then we get $f(y_n) = \sin(2\pi n + \frac{\pi}{2})\to 1$. Therefore the functional limit does not exist.
  \end{example}
  \begin{theorem}{4.8 (Quantitative Version of Functional Limits)}
    Let $f\colon A\to \R$, $c$ a limit point of $A$ ($c\in \R, L\in\R$). Then the following are equivalent:
    \begin{enumerate}[label=(\roman*)]
      \item $\displaystyle \lim_{x\to c} f(x) = L$
      \item For all $\eps > 0$, there exists some $\delta > 0$ such that whenever $0 < \abs{x - c} < \delta$, we have $\abs{f(x) - L} < \eps$
    \end{enumerate}
    \begin{proof}
      (i) $\Rightarrow$ (ii) We proceed via contrapositive. Suppose that (ii) does not hold. Then there exists $\eps_0 > 0$ for all $\delta > 0$, such that $\abs{f(x) - L}\geq \eps_0$ whenever $0 < \abs{x - c} < \delta$. Letting $\delta = 1$, we know there exists $\eps_0 > 0$ such that whenever $0 < \abs{x - c} < \delta = 1$, we have $\abs{f(x) - L} \geq \eps_0$. Pick $x_1$ such that $0 < \abs{x_1 - c} < 1$. Inductively choose $\delta = \frac{1}{n}$, to get a sequence $(x_n)\sse A$ where $x_n\neq c$ and $x_n\to c$ (since $\abs{x_n - c} < \frac{1}{n}$). Notice that $f(x_n) - L\geq \eps_0$ for all $n\in\N$ because $0 < \abs{x_n - c} < \frac{1}{n} = \delta$. This contradicts $f(x)\to L$, so $f(x)\not\to L$ and
      \[
        \lim_{x\to c} f(x) \neq L.
      \]
      \par
      (ii) $\Rightarrow$ (i) Let $(x_n)\sse A$, $x_n\neq c$, $x_n\to c$. Fix $\eps > 0$. Then there exists $\delta > 0$, such that $\abs{f(x) - L} < \eps$ whenever $0 < \abs{x - c} < \delta$. Since $x_n\to c$, there exists $N = N(\delta)$ such that $\abs{x_n - c} < \delta$ for all $n > N$. Thus $\abs{f(x_n) - L} < \eps$ for all $n > N$. Hence
      \[
        \lim_{n\to \infty} f(x_n) = L,
      \]
      and so
      \[
        \lim_{x\to c} f(x) = L.
      \]
    \end{proof}
  \end{theorem}
  \begin{note}{}
    The above is also commonly referred to as the $(\eps, \delta)$ definition for functional limits.
  \end{note}
  \begin{example}{}
    Let $f(x) = x^2$, $A = \R$. We wish to show that $\displaystyle\lim_{x\to c} f(x) = c^2$ for any $c\in\R$. \par
    \textbf{Scratch Work.} We see that
    \begin{align*}
      \abs{f(x) - c^2} &= \abs{x^2 - c^2} \\
                       &= \abs{x + c}\cdot \abs{x - c}
      \intertext{If we choose $\delta < 1$, then we have $\abs{x + c} \leq \abs{x - c} + 2\abs{c} < \delta + 2\abs{c} < 1 + 2\abs{c}$. Thus}
                       &< \abs{x + c} \delta \\
                       &< (1 + 2\abs{c})\delta \\
                       &< \eps.
    \end{align*}
    Thus we know that we should choose $\delta = \min(1, \frac{\eps}{1 + 2\abs{c}})$.
    \begin{proof}
      Given $\eps > 0$, pick $\delta = \min (1, \frac{\eps}{1 + 2\abs{c}})$. Then
      \begin{align*}
        \abs{f(x) - c^2} &\leq \abs{x + c}\abs{x - c} \\
                         &< (1 + 2\abs{c})\delta \\
                         &< \eps,
      \end{align*}
      whenever $0 < \abs{x - c} < \delta$. Hence
      \[
        \lim_{x\to c} f(x) = c^2.
      \]
    \end{proof}
  \end{example}
  \subsection{Continuity}
  \begin{definition}{4.9---Continuity}
    We say that $f\colon A\to \R$ is \emph{continuous at $c\in A$} if for all $\eps > 0$, there exists $\delta > 0$ such that $\abs{f(x) - f(c)} < \eps$ whenever $0 < \abs{x - c} < \delta$.
  \end{definition}
  \begin{note}{} How is this different from Theorem $4.8$?
    \begin{itemize}
      \item We need $f$ defined at $c$, i.e. $f(c)$ exists.
      \item We are allowing $x = c$.
      \item We don't say that $c$ is a limit point of $A$.
      \begin{itemize}
        \item Every point is continuous at isolated points of a set, since it trivially satisfies the conditions.
      \end{itemize}
    \end{itemize}
  \end{note}
  \begin{theorem}{4.11}
    Let $f\colon A\to \R$ be a limit point of $A$. Then $f$ is continuous at $c$ if and only if
    \[
      \lim_{x\to c} f(x) = f(c).
    \]
    Moreover, for any point $c\in A$, $f$ is continuous at $c$ if and only if for any $(x_n)\sse A$ such that $x_n\to c$, $f(x_n)\to f(c)$.
  \end{theorem}
\end{document}
