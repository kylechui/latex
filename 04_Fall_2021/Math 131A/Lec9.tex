\documentclass[class=article, crop=false]{standalone}
\usepackage[margin=1in]{geometry}

\usepackage{amsmath}
\usepackage{amssymb}
\usepackage{amsthm}
\usepackage{comment}
\usepackage{enumitem}
\usepackage{fancyhdr}
\usepackage{hyperref}
\usepackage{mathrsfs}
\usepackage{mathtools}
\usepackage{standalone}
\usepackage{tcolorbox}
\usepackage{tikz}
\usepackage{xcolor}

\usetikzlibrary{decorations.pathreplacing}
\tcbuselibrary{skins}

\DeclareMathOperator{\lcm}{lcm}
\DeclareMathOperator{\proj}{proj}
\DeclareMathOperator{\vspan}{span}
\DeclareMathOperator{\im}{im}
\DeclareMathOperator{\range}{range}
\DeclareMathOperator{\Diff}{Diff}
\DeclareMathOperator{\Int}{Int}
\DeclareMathOperator{\fcn}{fcn}
\DeclareMathOperator{\id}{id}

\newcommand{\N}{\ensuremath{\mathbb{N}}}
\newcommand{\Z}{\ensuremath{\mathbb{Z}}}
\newcommand{\Q}{\ensuremath{\mathbb{Q}}}
\newcommand{\R}{\ensuremath{\mathbb{R}}}
\newcommand{\C}{\ensuremath{\mathbb{C}}}
\newcommand{\F}{\ensuremath{\mathbb{F}}}
\newcommand{\lam}{\ensuremath{\lambda}}
\newcommand{\nab}{\ensuremath{\nabla}}
\newcommand{\eps}{\ensuremath{\varepsilon}}
\newcommand{\es}{\ensuremath{\varnothing}}

\newcommand{\abs}[1]{\ensuremath{\left\lvert #1 \right\rvert}}
\newcommand{\bigpar}[1]{\ensuremath{\left( #1 \right)}}
\newcommand{\norm}[1]{\ensuremath{\lVert #1\rVert}}
\newcommand{\set}[1]{\ensuremath{\left\{#1\right\}}}
\newcommand{\tuple}[1]{\ensuremath{\left\langle #1 \right\rangle}}
\newcommand{\floor}[1]{\ensuremath{\left\lfloor #1 \right\rfloor}}
\newcommand{\ceil}[1]{\ensuremath{\left\lceil #1 \right\rceil}}

\newcommand{\chapternum}{}
\newcommand{\ex}[1]{\noindent\textbf{Exercise \chapternum.{#1}.}}

\newcommand{\dx}[1]{\,\mathrm{d}#1}
\newcommand{\inv}{\ensuremath{^{-1}}}
\newcommand{\sm}{\setminus}
\newcommand{\sse}{\subseteq}
\newcommand{\ceq}{\coloneqq}

\definecolor{problemBackground}{RGB}{212,232,246}
\newenvironment{problem}[1]
  {
    \begin{tcolorbox}[
      boxrule=.5pt,
      titlerule=.5pt,
      sharp corners,
      colback=problemBackground
    ]
    \ifx &#1& \textbf{Problem. }
    \else \textbf{Problem #1.} \fi
  }
  {
    \end{tcolorbox}
  }

\definecolor{exampleBackground}{RGB}{255,249,248}
\definecolor{exampleAccent}{RGB}{158,60,14}
\newenvironment{example}[1]
  {
    \begin{tcolorbox}[
      boxrule=.5pt,
      sharp corners,
      colback=exampleBackground,
      colframe=exampleAccent,
    ]
    \color{exampleAccent}\textbf{Example.} \emph{#1}\color{black}
  }
  {
    \end{tcolorbox}
  }

\definecolor{theoremBackground}{RGB}{234,243,251}
\definecolor{theoremAccent}{RGB}{0,116,183}
\newenvironment{theorem}[1]
  {
    \begin{tcolorbox}[
      boxrule=.5pt,
      titlerule=.5pt,
      sharp corners,
      colback=theoremBackground,
      colframe=theoremAccent
    ]
      \color{theoremAccent}\textbf{Theorem --- }\emph{#1}\\\color{black}
  }
  {
    \end{tcolorbox}
  }

\definecolor{noteBackground}{RGB}{244,249,244}
\definecolor{noteAccent}{RGB}{34,139,34}
\newenvironment{note}[1]
  {
  \begin{tcolorbox}[
    enhanced,
    boxrule=0pt,
    frame hidden,
    sharp corners,
    colback=noteBackground,
    borderline west={3pt}{-1.5pt}{noteAccent}
    ]
    \ifx &#1& \color{noteAccent}\textbf{Note. }\color{black}
    \else \color{noteAccent}\textbf{Note (#1). }\color{black} \fi
    }
    {
  \end{tcolorbox}
  }

\definecolor{definitionBackground}{RGB}{246,246,246}
\newenvironment{definition}[1]
  {
    \begin{tcolorbox}[
      enhanced,
      boxrule=0pt,
      frame hidden,
      sharp corners,
      colback=definitionBackground,
      borderline west={3pt}{-1.5pt}{black}
    ]
    \textbf{Definition. }\emph{#1}\\
  }
  {
    \end{tcolorbox}
  }
\newenvironment{amatrix}[2]{
    \left[
      \begin{array}{*{#1}{c}|*{#2}c}
  }
  {
      \end{array}
    \right]
  }
\date{\the\year-\the\month-\the\day}
\author{Kyle Chui}


\fancyhf{}
\lhead{Kyle Chui}
\rhead{Page \thepage}
\pagestyle{fancy}

\begin{document}
  \section{Lecture 9}
  \subsection{Sequences}
  \begin{example}{Approximating $\pi$} \\
    We know that the area of the unit circle should be $\pi$. We can approximate the area of a unit circle by inscribing various shapes in the circle and finding their areas (giving us a sequence of lower bounds). \par
    If we inscribe an equilateral triangle in the circle, we find that its side length is $\sqrt{3}$, so the area of the triangle is $a_1 = \frac{3\sqrt{3}}{4}\approx 1.299$. \par
    If we use a square instead, its side length is $\sqrt{2}$, so we have a new lower bound of $a_2 = 2 < \pi$. \par
    Using a regular pentagon, we have a new approximation as given by $a_3 = \frac{5}{2}\sin \paren{\frac{2\pi}{5}} < \pi$. \par
    Continuing this pattern, we get
    \[
      a_n = \frac{n+2}{2}\sin \paren{\frac{2\pi}{n + 2}} < \pi.
    \]
    Notice that for all $n\in\N$, we have $a_n < a_{n + 1} < \pi$. Computing this value for larger $n$, we have
    \begin{align*}
      a_{100}&\approx 3.1396 \\
      a_{1000}&\approx 3.141572 \\
      a_{10000}&\approx 3.14159245.
    \end{align*}
    Similarly, we may obtain upper bounds for $\pi$ by circumscribing regular polygons around the unit circle. By circumscribing a square, we know that $\pi < 4$, so we know that there is some upper bound for what $\pi$ is equal to. Thus we know that there exists some $a\in\R$ such that $a_n\approx a$ for some sufficiently large $n$. \par
    In other words, there exists some $a\in\R$ such that $a_n$ converges to $a$ as $n$ approaches $\infty$. \par
    One issue to note is that we are using $\pi$ in our formula for $a_n$ to approximate the value of $\pi$. Using some trigonometric identities, we may circumvent this by evaluating
    \[
      a_{2n} = \frac{n}{2} \paren{n - \sqrt{n - 4a_n^2}}.
    \]
  \end{example}
  \begin{definition}{3.1---Sequences}
    A \emph{sequence} is a function $f\colon \N\to \R$. Instead of writing $f(1), f(2), \dotsc, f(n)$, we tend to write a sequence as $f_1, f_2, \dotsc, f_n$.
  \end{definition}
  \begin{note}{Notations for Sequences}
    There are many different notations for expressing sequences, a few popular notations being used below:
    \[
      \paren{1, \frac{1}{2}, \frac{1}{3}, \dotsc} = \paren{\frac{1}{n}}_{n = 1}^\infty = \paren{\frac{1}{n}}_{n\in\N} = \set{\frac{1}{n}\mid n\in\N}.
    \]
  \end{note}
  \subsubsection{Behaviors of Sequences}
  \begin{enumerate}[label=(\alph*)]
    \item ``Convergence''---Getting closer and closer to a given point.
    \item ``Divergence''---Getting closer and closer to $\pm\infty$.
    \item ``Oscillation''---The sequence does not approach any value in particular.
  \end{enumerate}
  \subsubsection{Facts From the Homework}
  We define the absolute value function
  \[
    \abs{\cdot}\colon \R\to[0, \infty) = \begin{cases}
      x & \text{if }x\geq 0, \\
      -x & \text{if }x < 0.
    \end{cases}
  \]
  We have the following properties of the absolute value function:
  \begin{enumerate}[label=(\roman*)]
    \item $\abs{xy} = \abs{x}\abs{y}$ for all $x,y\in\R$.
    \item $\abs{x - y}\leq \ell$ if and only if $y - \ell\leq x\leq y + \ell$, where $\ell\geq 0$ and $x, y\in\R$.
    \item \textbf{Triangle Inequality:} $\abs{x + y}\leq \abs{x} + \abs{y}$.
    \begin{note}{}
      One consequence of this is
      \begin{align*}
        \abs{x - y} &= \abs{x - z + z - y} \\
                    &\leq \abs{x - z} + \abs{z - y}.
      \end{align*}
      In other words, the distance between two points $x$ and $y$ is always less than or equal to the sum of the distances of $x$ and $y$ to a third point, $z$.
    \end{note}
  \end{enumerate}
  We say that $a_n$ approaches $a$ if ``$\abs{a_n - a}$ gets arbitrarily small as $n$ gets arbitrarily large''.
  \subsubsection{Convergence}
  \begin{definition}{3.2---Convergence}
    A sequence $(x_n)$ of real numbers is said to \emph{converge} to an $x\in\R$ if for all $\eps > 0$, there exists $N > 0$ such that
    \[
      \abs{x_n - x} < \eps
    \]
    for all $n > N$. If $(x_n)$ converges to $x$, we also write:
    \[
      x_n\to x \text{ as }n\to \infty \quad \text{or}\quad \lim_{n\to\infty}x_n = x.
    \]
    We call $x$ the \emph{limit} of the sequence $(x_n)$. $W$e say that a sequence \emph{diverges} if it does not converge.
  \end{definition}
  \begin{note}{}
    By the Archimedean property, we can always take $N\in\N$. In general, $N$ is a function of $\eps$. We fix $\eps > 0$, and use that to find some sufficient $N$ so that the sequence converges.
  \end{note}
  \begin{example}{}
    Consider the sequence $x_n = \frac{1}{n^2}$. We suspect that $x_n\to 0\in\R$ as $n\to\infty$. Fix $\eps > 0$. If we let $N = \ceil{\frac{1}{\sqrt{\eps}}}$ and $n > N$, we have
    \begin{align*}
      \abs{x_n - 0} &= \abs{\frac{1}{n^2}} \\
                    &= \frac{1}{n^2} \\
                    &< \frac{1}{N^2} \\
                    &< \eps.
    \end{align*}
    Thus, $x_n\to 0$ as $n\to\infty$, so
    \[
      \lim_{n\to \infty}x_n = 0.
    \]
  \end{example}
\end{document}
