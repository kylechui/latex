\documentclass[class=article, crop=false]{standalone}
\usepackage[margin=1in]{geometry}

\usepackage{amsmath}
\usepackage{amssymb}
\usepackage{amsthm}
\usepackage{comment}
\usepackage{enumitem}
\usepackage{fancyhdr}
\usepackage{hyperref}
\usepackage{mathrsfs}
\usepackage{mathtools}
\usepackage{standalone}
\usepackage{tcolorbox}
\usepackage{tikz}
\usepackage{xcolor}

\usetikzlibrary{decorations.pathreplacing}
\tcbuselibrary{skins}

\DeclareMathOperator{\lcm}{lcm}
\DeclareMathOperator{\proj}{proj}
\DeclareMathOperator{\vspan}{span}
\DeclareMathOperator{\im}{im}
\DeclareMathOperator{\range}{range}
\DeclareMathOperator{\Diff}{Diff}
\DeclareMathOperator{\Int}{Int}
\DeclareMathOperator{\fcn}{fcn}
\DeclareMathOperator{\id}{id}

\newcommand{\N}{\ensuremath{\mathbb{N}}}
\newcommand{\Z}{\ensuremath{\mathbb{Z}}}
\newcommand{\Q}{\ensuremath{\mathbb{Q}}}
\newcommand{\R}{\ensuremath{\mathbb{R}}}
\newcommand{\C}{\ensuremath{\mathbb{C}}}
\newcommand{\F}{\ensuremath{\mathbb{F}}}
\newcommand{\lam}{\ensuremath{\lambda}}
\newcommand{\nab}{\ensuremath{\nabla}}
\newcommand{\eps}{\ensuremath{\varepsilon}}
\newcommand{\es}{\ensuremath{\varnothing}}

\newcommand{\abs}[1]{\ensuremath{\left\lvert #1 \right\rvert}}
\newcommand{\bigpar}[1]{\ensuremath{\left( #1 \right)}}
\newcommand{\norm}[1]{\ensuremath{\lVert #1\rVert}}
\newcommand{\set}[1]{\ensuremath{\left\{#1\right\}}}
\newcommand{\tuple}[1]{\ensuremath{\left\langle #1 \right\rangle}}
\newcommand{\floor}[1]{\ensuremath{\left\lfloor #1 \right\rfloor}}
\newcommand{\ceil}[1]{\ensuremath{\left\lceil #1 \right\rceil}}

\newcommand{\chapternum}{}
\newcommand{\ex}[1]{\noindent\textbf{Exercise \chapternum.{#1}.}}

\newcommand{\dx}[1]{\,\mathrm{d}#1}
\newcommand{\inv}{\ensuremath{^{-1}}}
\newcommand{\sm}{\setminus}
\newcommand{\sse}{\subseteq}
\newcommand{\ceq}{\coloneqq}

\definecolor{problemBackground}{RGB}{212,232,246}
\newenvironment{problem}[1]
  {
    \begin{tcolorbox}[
      boxrule=.5pt,
      titlerule=.5pt,
      sharp corners,
      colback=problemBackground
    ]
    \ifx &#1& \textbf{Problem. }
    \else \textbf{Problem #1.} \fi
  }
  {
    \end{tcolorbox}
  }

\definecolor{exampleBackground}{RGB}{255,249,248}
\definecolor{exampleAccent}{RGB}{158,60,14}
\newenvironment{example}[1]
  {
    \begin{tcolorbox}[
      boxrule=.5pt,
      sharp corners,
      colback=exampleBackground,
      colframe=exampleAccent,
    ]
    \color{exampleAccent}\textbf{Example.} \emph{#1}\color{black}
  }
  {
    \end{tcolorbox}
  }

\definecolor{theoremBackground}{RGB}{234,243,251}
\definecolor{theoremAccent}{RGB}{0,116,183}
\newenvironment{theorem}[1]
  {
    \begin{tcolorbox}[
      boxrule=.5pt,
      titlerule=.5pt,
      sharp corners,
      colback=theoremBackground,
      colframe=theoremAccent
    ]
      \color{theoremAccent}\textbf{Theorem --- }\emph{#1}\\\color{black}
  }
  {
    \end{tcolorbox}
  }

\definecolor{noteBackground}{RGB}{244,249,244}
\definecolor{noteAccent}{RGB}{34,139,34}
\newenvironment{note}[1]
  {
  \begin{tcolorbox}[
    enhanced,
    boxrule=0pt,
    frame hidden,
    sharp corners,
    colback=noteBackground,
    borderline west={3pt}{-1.5pt}{noteAccent}
    ]
    \ifx &#1& \color{noteAccent}\textbf{Note. }\color{black}
    \else \color{noteAccent}\textbf{Note (#1). }\color{black} \fi
    }
    {
  \end{tcolorbox}
  }

\definecolor{definitionBackground}{RGB}{246,246,246}
\newenvironment{definition}[1]
  {
    \begin{tcolorbox}[
      enhanced,
      boxrule=0pt,
      frame hidden,
      sharp corners,
      colback=definitionBackground,
      borderline west={3pt}{-1.5pt}{black}
    ]
    \textbf{Definition. }\emph{#1}\\
  }
  {
    \end{tcolorbox}
  }
\newenvironment{amatrix}[2]{
    \left[
      \begin{array}{*{#1}{c}|*{#2}c}
  }
  {
      \end{array}
    \right]
  }
\date{\the\year-\the\month-\the\day}
\author{Kyle Chui}


\fancyhf{}
\lhead{Kyle Chui}
\rhead{Page \thepage}
\pagestyle{fancy}

\begin{document}
  \section{Lecture 28}
  \begin{example}{Integrating a function with a jump discontinuity} \par
    Consider the function given by
    \[
      f(x) = \begin{cases}1 & x\neq 1 \\ \frac{1}{2} & x = 1\end{cases}, \text{ on } [0, 2].
    \]
    We claim that $f$ is integrable on $[0, 2]$ and $\int_{0}^{2}f(x) \,\mathrm dx = 2$. Observe that for any partition $\mathcal{P}$, we know that $\mathcal{U}(f, \mathcal{P}) = 2$. The issue arises when we take the lower Darboux sum, namely near the point of discontinuity. The idea here is to choose a partition such that the interval containing $x = 1$ is \emph{small}, so the difference between upper and lower Darboux sums is minimized. Given $\eps > 0$ small, consider
    \[
      \mathcal{P}_\eps = \set{0, 1 - \frac{\eps}{3}, 1 + \frac{\eps}{3}, 2}.
    \]
    The lower Darboux sum is thus:
    \begin{align*}
      \mathcal{L}(f, \mathcal{P}_\eps) &= 1\cdot \paren{1 - \frac{\eps}{3}} + \frac{1}{2}\cdot \paren{\frac{2\eps}{3}} + 1\cdot \paren{1 - \frac{\eps}{3}} \\
                                       &= 2 - \frac{\eps}{3}.
    \end{align*}
    Hence if we take the difference of our Darboux sums, we have
    \[
      \mathcal{U}(f, \mathcal{P}_\eps) - \mathcal{L}(f, \mathcal{P}_\eps) = 2 - \paren{2 - \frac{\eps}{3}} = \frac{\eps}{3} < \eps.
    \]
    Thus $f$ is Riemann integrable on $[0, 2]$. Since $\mathcal{U}(f, \mathcal{P}) = 2$ for all partitions $\mathcal{P}$, we know that
    \[
      \int_{0}^{2}f(x) \,\mathrm dx = \overline{\int_{0}^{2}f(x) \,\mathrm dx} = 2.
    \]
  \end{example}
  \subsection{Properties of the Riemann Integral}
  \textbf{Proposition 6.12} Additivity---If $f, g\colon [a, b]\to \R$ are integrable, then so is $f + g$ and
  \[
    \int_{a}^{b}(f + g)(x) \,\mathrm dx = \int_{a}^{b}f(x) \,\mathrm dx + \int_{a}^{b}g(x) \,\mathrm dx.
  \]
  \begin{proof}
    We know that $f + g$ is bounded since both $f$ and $g$ are bounded (since they are integrable). Thus all Darboux sums for $f + g$ are finite. We will show that the upper integrals are \emph{sub-additive}, i.e.
    \[
      \overline{\int_{a}^{b}(f + g)(x) \,\mathrm dx}\leq \overline{\int_{a}^{b}f(x) \,\mathrm dx} + \overline{\int_{a}^{b}g(x) \,\mathrm dx}.
    \]
    Similarly, the lower integrals are \emph{super-additive}, i.e.
    \[
      \underline{\int_{a}^{b}(f + g)(x) \,\mathrm dx}\geq \underline{\int_{a}^{b}f(x) \,\mathrm dx} + \underline{\int_{a}^{b}g(x) \,\mathrm dx}.
    \]
    For $I\sse [a, b]$,
    \[
      \sup_{x\in I} [f(x) + g(x)]\leq \sup_{x\in I} [f(x)] + \sup_{x\in I} [g(x)].
    \]
    Thus we know that for all partitions $\mathcal{P}$,
    \[
      \mathcal{U}(f + g, \mathcal{P})\leq \mathcal{U}(f, \mathcal{P}) + \mathcal{U}(g, \mathcal{P}).
    \]
    Let $\eps > 0$, so there exist $\mathcal{P}_\eps', \mathcal{P}_\eps''$ such that
    \begin{align*}
      \mathcal{U}(f, \mathcal{P}_\eps')&\leq \overline{\int_{a}^{b}f(x) \,\mathrm dx} + \frac{\eps}{2} \\
      \mathcal{U}(g, \mathcal{P}_\eps'')&\leq \overline{\int_{a}^{b}g(x) \,\mathrm dx} + \frac{\eps}{2}.
    \end{align*}
    Taking the common refinement, we have $\mathcal{P}_\eps = \mathcal{P}_\eps'\cup \mathcal{P}_\eps''$. Hence
    \begin{align*}
      \overline{\int_{a}^{b}(f + g)(x) \,\mathrm dx} &\leq \mathcal{U}(f + g, \mathcal{P}_\eps) \\
                                                     &\leq \mathcal{U}(f, \mathcal{P}_\eps) + \mathcal{U}(g, \mathcal{P}_\eps) \\
                                                     &\leq \overline{\int_{a}^{b}f(x) \,\mathrm dx} + \overline{\int_{a}^{b}g(x) \,\mathrm dx} + \eps.
    \end{align*}
    If we map $f\mapsto -f$ and $g\mapsto -g$, we have the same super-additivity of the lower integrals. Therefore
    \begin{align*}
      \int_{a}^{b}f(x) \,\mathrm dx + \int_{a}^{b}g(x) \,\mathrm dx &= \underline{\int_{a}^{b}f(x) \,\mathrm dx} + \underline{\int_{a}^{b}g(x) \,\mathrm dx} \\
                                                                    &\leq \underline{\int_{a}^{b}(f + g)(x) \,\mathrm dx} \\
                                                                    &\leq \overline{\int_{a}^{b}(f + g)(x) \,\mathrm dx} \\
                                                                    &\leq \overline{\int_{a}^{b}f(x) \,\mathrm dx} + \overline{\int_{a}^{b}g(x) \,\mathrm dx} \\
                                                                    &= \int_{a}^{b}f(x) \,\mathrm dx + \int_{a}^{b}g(x) \,\mathrm dx.
    \end{align*}
    Since the expressions at the beginning and at the end are equal, we know that all of the terms in between are also equal, and so
    \[
      \int_{a}^{b}f(x) \,\mathrm dx + \int_{a}^{b}g(x) \,\mathrm dx = \int_{a}^{b}(f + g)(x) \,\mathrm dx.
    \]
  \end{proof}
  \textbf{Lemma 6.13} Homogeneity \par
  If $f$ is integrable on $[a, b]$ and $\lam\in\R$, then $\lam f$ is integrable and
  \[
    \int_{a}^{b}(\lam f)(x) \,\mathrm dx = \lam \int_{a}^{b}f(x) \,\mathrm dx.
  \]
  \begin{note}{}
    This implies that integration is \emph{linear}, i.e.
    \[
      \int_{a}^{b}(\lam f + \mu g) \,\mathrm dx = \lam \int_{a}^{b}f(x) \,\mathrm dx + \mu \int_{a}^{b}g(x) \,\mathrm dx.
    \]
  \end{note}
  \textbf{Lemma 6.14} Monotonicity \par
  If $f, g$ are integrable and $f(x)\leq g(x)$ for all $x\in [a, b]$, then
  \[
    \int_{a}^{b}f(x) \,\mathrm dx \leq \int_{a}^{b}g(x) \,\mathrm dx.
  \]
  We can prove this by showing that
  \[
    \int_{a}^{b}(g - f)(x) \,\mathrm dx\geq \mathcal{L}(g - f, \mathcal{P})\geq 0.
  \]
  \textbf{Lemma 6.15} Let $f$ be integrable on $[a, b]$. Then $f$ is integrable on any interval $I\sse [a, b]$ and for any $c\in (a, b)$, we have
  \[
    \int_{a}^{b}f(x) \,\mathrm dx = \int_{a}^{c}f(x) \,\mathrm dx + \int_{c}^{b}f(x) \,\mathrm dx.
  \]
  \subsection{Fundamental Theorem of Calculus}
  \begin{theorem}{6.17 (Fundamental Theorem of Calculus I)}
    Let $f\colon [a, b]\to \R$ be continuous on $[a, b]$, and let $F\colon [a, b]\to \R$ be defined by $F(a) = 0$ and $F(x) = \int_{a}^{x}f(t) \,\mathrm dt$. The claim is that $F$ is differentiable on $[a, b]$ and $F'(x) = f(x)$ for all $x\in (a, b)$.
    \begin{proof}
      We claim that for all $x_0\in (a, b)$, we have that if $f$ is continuous at $x_0$, then $F$ is differentiable at $x_0$ and $F'(x_0) = f(x_0)$. Let $\eps > 0$. Since $f$ is continuous at $x_0$, there exists $\delta > 0$ such that
      \[
        \abs{f(x) - f(x_0)} < \eps
      \]
      whenever
      \[
        \abs{x - x_0} < \delta.
      \]
      Now let $x\in (a, b)$ such that $0 < \abs{x - x_0} < \delta$. We consider the case where $x > x_0$, so we have
      \begin{align*}
        \abs{\frac{F(x) - F(x_0)}{x - x_0}} &= \abs{\frac{1}{x - x_0}\paren{\int_{a}^{x}f(t) \,\mathrm dt - \int_{a}^{x_0}f(t) \,\mathrm dt} - f(x_0)} \\
                                            &= \abs{\frac{1}{x - x_0}\int_{x_0}^{x}f(t) \,\mathrm dt - f(x_0)} \\
                                            &= \abs{\frac{1}{x - x_0} \int_{x_0}^{x}(f(t) - f(x_0)) \,\mathrm dt} \\
                                            &\leq \frac{1}{\abs{x - x_0}} \int_{x_0}^{x}\underbrace{\abs{f(t) - f(x_0)}}_{< \eps} \,\mathrm dt \\
                                            &< \frac{1}{\abs{x - x_0}}\cdot (x - x_0)\eps \\
                                            &= \eps.
      \end{align*}
      Therefore $F$ is differentiable at $x_0$ and $F'(x_0) = f(x_0)$.
    \end{proof}
  \end{theorem}
  \begin{theorem}{6.18 (Fundamental Theorem of Calculus II)}
    If $f\colon [a, b]\to \R$ is \emph{continuous} on $[a, b]$ and \emph{differentiable} on $(a, b)$ and $f'$ is integrable on $[a, b]$, then
    \[
      \int_{a}^{b}f'(t) \,\mathrm dt = f(b) - f(a).
    \]
    \begin{proof}
      We proceed by showing that
      \[
        \int_{a}^{b}f'(t) \,\mathrm dt - (f(b) - f(a)) < \eps
      \]
      for any $\eps > 0$. Since $f'$ is integrable, by Proposition 6.10 we have that for all $\eps > 0$, there exists a partition $\mathcal{P}_\eps$ such that
      \[
        \mathcal{U}(f, \mathcal{P}_\eps) - \mathcal{L}(f, \mathcal{P}_\eps) < \eps.
      \]
      Then
      \begin{align*}
        f(b) - f(a) &= \sum_{j=1}^{n} (f(t_j) - f(t_{j - 1})) \\
                    &= \sum_{j=1}^{n} \frac{f(t_j) - f(t_{j - 1})}{t_j - t_{j - 1}}(t_j - t_{j - 1}) \\
                    &= \sum_{j=1}^{n} f'(t_j^*)(t_j - t_{j - 1}). \tag{MVT on $[t_{j - 1}, t_j], t_j^*\in [t_{j - 1}, t_j]$}
      \end{align*}
      We know that
      \[
        \mathcal{L}(f', \mathcal{P}_\eps) \leq \sum_{j=1}^{n} f'(t_j^*)(t_j - t_{j - 1})\leq \mathcal{U}(f', \mathcal{P}_\eps).
      \]
      Thus
      \begin{align*}
        \abs{\int_{a}^{b}f'(t) \,\mathrm dt - \sum_{j=1}^{n}f'(t_j^*)(t_j - t_{j - 1})} &\leq \mathcal{U}(f', \mathcal{P}_\eps) - \mathcal{L}(f', \mathcal{P}_\eps) \\
                                                                                            &< \eps,
      \end{align*}
      for all $\eps > 0$. Therefore
      \[
        \int_{a}^{b}f'(t) \,\mathrm dt = f(b) - f(a).
      \]
    \end{proof}
  \end{theorem}
\end{document}
