\documentclass[class=article, crop=false]{standalone}
% Import packages
\usepackage[margin=1in]{geometry}

\usepackage{amssymb, amsthm}
\usepackage{comment}
\usepackage{enumitem}
\usepackage{fancyhdr}
\usepackage{hyperref}
\usepackage{import}
\usepackage{mathrsfs, mathtools}
\usepackage{multicol}
\usepackage{pdfpages}
\usepackage{standalone}
\usepackage{transparent}
\usepackage{xcolor}

\usepackage{minted}
\usepackage[many]{tcolorbox}
\tcbuselibrary{minted}

% Declare math operators
\DeclareMathOperator{\lcm}{lcm}
\DeclareMathOperator{\proj}{proj}
\DeclareMathOperator{\vspan}{span}
\DeclareMathOperator{\im}{im}
\DeclareMathOperator{\Diff}{Diff}
\DeclareMathOperator{\Int}{Int}
\DeclareMathOperator{\fcn}{fcn}
\DeclareMathOperator{\id}{id}
\DeclareMathOperator{\rank}{rank}
\DeclareMathOperator{\tr}{tr}
\DeclareMathOperator{\dive}{div}
\DeclareMathOperator{\row}{row}
\DeclareMathOperator{\col}{col}
\DeclareMathOperator{\dom}{dom}
\DeclareMathOperator{\ran}{ran}
\DeclareMathOperator{\Dom}{Dom}
\DeclareMathOperator{\Ran}{Ran}
\DeclareMathOperator{\fl}{fl}
% Macros for letters/variables
\newcommand\ttilde{\kern -.15em\lower .7ex\hbox{\~{}}\kern .04em}
\newcommand{\N}{\ensuremath{\mathbb{N}}}
\newcommand{\Z}{\ensuremath{\mathbb{Z}}}
\newcommand{\Q}{\ensuremath{\mathbb{Q}}}
\newcommand{\R}{\ensuremath{\mathbb{R}}}
\newcommand{\C}{\ensuremath{\mathbb{C}}}
\newcommand{\F}{\ensuremath{\mathbb{F}}}
\newcommand{\M}{\ensuremath{\mathbb{M}}}
\renewcommand{\P}{\ensuremath{\mathbb{P}}}
\newcommand{\lam}{\ensuremath{\lambda}}
\newcommand{\nab}{\ensuremath{\nabla}}
\newcommand{\eps}{\ensuremath{\varepsilon}}
\newcommand{\es}{\ensuremath{\varnothing}}
% Macros for math symbols
\newcommand{\dx}[1]{\,\mathrm{d}#1}
\newcommand{\inv}{\ensuremath{^{-1}}}
\newcommand{\sm}{\setminus}
\newcommand{\sse}{\subseteq}
\newcommand{\ceq}{\coloneqq}
% Macros for pairs of math symbols
\newcommand{\abs}[1]{\ensuremath{\left\lvert #1 \right\rvert}}
\newcommand{\paren}[1]{\ensuremath{\left( #1 \right)}}
\newcommand{\norm}[1]{\ensuremath{\left\lVert #1\right\rVert}}
\newcommand{\set}[1]{\ensuremath{\left\{#1\right\}}}
\newcommand{\tup}[1]{\ensuremath{\left\langle #1 \right\rangle}}
\newcommand{\floor}[1]{\ensuremath{\left\lfloor #1 \right\rfloor}}
\newcommand{\ceil}[1]{\ensuremath{\left\lceil #1 \right\rceil}}
\newcommand{\eclass}[1]{\ensuremath{\left[ #1 \right]}}

\newcommand{\chapternum}{}
\newcommand{\ex}[1]{\noindent\textbf{Exercise \chapternum.{#1}.}}

\newcommand{\tsub}[1]{\textsubscript{#1}}
\newcommand{\tsup}[1]{\textsuperscript{#1}}

\renewcommand{\choose}[2]{\ensuremath{\phantom{}_{#1}C_{#2}}}
\newcommand{\perm}[2]{\ensuremath{\phantom{}_{#1}P_{#2}}}

% Include figures
\newcommand{\incfig}[2][1]{%
    \def\svgwidth{#1\columnwidth}
    \import{./figures/}{#2.pdf_tex}
}

\definecolor{problemBackground}{RGB}{212,232,246}
\newenvironment{problem}[1]
  {
    \begin{tcolorbox}[
      boxrule=.5pt,
      titlerule=.5pt,
      sharp corners,
      colback=problemBackground,
      breakable
    ]
    \ifx &#1& \textbf{Problem. }
    \else \textbf{Problem #1.} \fi
  }
  {
    \end{tcolorbox}
  }
\definecolor{exampleBackground}{RGB}{255,249,248}
\definecolor{exampleAccent}{RGB}{158,60,14}
\newenvironment{example}[1]
  {
    \begin{tcolorbox}[
      boxrule=.5pt,
      sharp corners,
      colback=exampleBackground,
      colframe=exampleAccent,
    ]
    \color{exampleAccent}\textbf{Example.} \emph{#1}\color{black}
  }
  {
    \end{tcolorbox}
  }
\definecolor{theoremBackground}{RGB}{234,243,251}
\definecolor{theoremAccent}{RGB}{0,116,183}
\newenvironment{theorem}[1]
  {
    \begin{tcolorbox}[
      boxrule=.5pt,
      titlerule=.5pt,
      sharp corners,
      colback=theoremBackground,
      colframe=theoremAccent,
      breakable
    ]
      % \color{theoremAccent}\textbf{Theorem --- }\emph{#1}\\\color{black}
    \ifx &#1& \color{theoremAccent}\textbf{Theorem. }\color{black}
    \else \color{theoremAccent}\textbf{Theorem --- }\emph{#1}\\\color{black} \fi
  }
  {
    \end{tcolorbox}
  }
\definecolor{noteBackground}{RGB}{244,249,244}
\definecolor{noteAccent}{RGB}{34,139,34}
\newenvironment{note}[1]
  {
  \begin{tcolorbox}[
    enhanced,
    boxrule=0pt,
    frame hidden,
    sharp corners,
    colback=noteBackground,
    borderline west={3pt}{-1.5pt}{noteAccent},
    breakable
    ]
    \ifx &#1& \color{noteAccent}\textbf{Note. }\color{black}
    \else \color{noteAccent}\textbf{Note (#1). }\color{black} \fi
    }
    {
  \end{tcolorbox}
  }
\definecolor{lemmaBackground}{RGB}{255,247,234}
\definecolor{lemmaAccent}{RGB}{255,153,0}
\newenvironment{lemma}[1]
  {
    \begin{tcolorbox}[
      enhanced,
      boxrule=0pt,
      frame hidden,
      sharp corners,
      colback=lemmaBackground,
      borderline west={3pt}{-1.5pt}{lemmaAccent},
      breakable
    ]
    \ifx &#1& \color{lemmaAccent}\textbf{Lemma. }\color{black}
    \else \color{lemmaAccent}\textbf{Lemma #1. }\color{black} \fi
  }
  {
    \end{tcolorbox}
  }
\definecolor{definitionBackground}{RGB}{246,246,246}
\newenvironment{definition}[1]
  {
    \begin{tcolorbox}[
      enhanced,
      boxrule=0pt,
      frame hidden,
      sharp corners,
      colback=definitionBackground,
      borderline west={3pt}{-1.5pt}{black},
      breakable
    ]
    \textbf{Definition. }\emph{#1}\\
  }
  {
    \end{tcolorbox}
  }

\newenvironment{amatrix}[2]{
    \left[
      \begin{array}{*{#1}{c}|*{#2}c}
  }
  {
      \end{array}
    \right]
  }

\definecolor{codeBackground}{RGB}{253,246,227}
\renewcommand\theFancyVerbLine{\arabic{FancyVerbLine}}
\newtcblisting{cpp}[1][]{
  boxrule=1pt,
  sharp corners,
  colback=codeBackground,
  listing only,
  breakable,
  minted language=cpp,
  minted style=material,
  minted options={
    xleftmargin=15pt,
    baselinestretch=1.2,
    linenos,
  }
}

\author{Kyle Chui}


\fancyhf{}
\lhead{Kyle Chui}
\rhead{Page \thepage}
\pagestyle{fancy}

\begin{document}
  \section{Lecture 24}
  \textbf{Proposition 5.5} Let $I = (a, b)$, $f\colon I\to \R$, $c\in I$. Then
  \[
    \lim_{x\to c} \frac{f(x) - f(c)}{x - c} \text{ exists in }\R,
  \]
  if and only if there exists $L\in\R$, $R\colon I\to \R$ such that $\displaystyle \lim_{x\to c} R(x) = 0$ and
  \[
    f(x) = f(c) + (x - c)L + (x - c)R(x).
  \]
  \begin{note}{}
    The above more or less states that the derivative of $f$ at a point $c$ exists if and only if when we approximate $f$ with a line, the remainder function goes to $0$. Additionally, it says that if 
    \[
      \lim_{x\to c} \frac{f(x) - f(c)}{x - c} = f'(c),
    \]
    then $f'(c) = L$.
  \end{note}
  \begin{proof}
    ($\Rightarrow$) Assume $\displaystyle \frac{f(x) - f(c)}{x - c} = L\in\R$. Define
    \[
      R(x)\ceq \begin{cases}\frac{f(x) - f(c)}{x - c} - L & \text{if }x\neq c,\\0 & \text{if $x$ = c}.\end{cases}
    \]
    Then
    \[
      \lim_{x\to c} R(x) = \lim_{x\to c} \paren{\frac{f(x) - f(c)}{x - c} - L} = L - L = 0.
    \]
    Thus for $x\neq c$, we have $(x - c)(L + R(x)) = f(x) - f(c)$. \par
    ($\Leftarrow$) If $x\neq c$, we write
    \[
      \frac{f(x) - f(c)}{x - c} = L + R(x).
    \]
    Thus by Algebraic Limit Theorem,
    \[
      \lim_{x\to c} \frac{f(x) - f(c)}{x - c} = L + 0 = L.
    \]
  \end{proof}
  \begin{note}{}
    We can also rewrite the linear approximation by
    \[
      f(c + h) = f(c) + hL + hR(h),
    \]
    for some other remainder function where
    \[
      \lim_{h\to 0} R(h) = 0.
    \]
  \end{note}
  We now have the tools necessary to prove the Chain Rule.
  \begin{proof}
    By Proposition $5.5$, we may write
    \begin{align*}
      f(c + h) &= f(c) + f'(c)h + hR_1(h) \\
      g(f(c) + \widetilde h) &= g(f(c)) + \widetilde h g'(f(c)) + \widetilde h R_2(\widetilde h).
    \end{align*}
    Thus we have
    \begin{align*}
      \frac{g(f(c + h)) - g(f(c))}{h} &= \frac{1}{h}\paren{g(f(c) + \underbrace{hf'(c) + hR_1(h)}_{\widetilde h}) - g(f(c))} \\
                                      &= \frac{1}{h}\paren{g(f(c) + \widetilde h) - g(f(c))} \\
                                      &= \frac{1}{h} \paren{g(f(c)) + \widetilde h g'(f(c)) + \widetilde h R_2(\widetilde h) - g(f(c))} \\
                                      &= \frac{\widetilde h}{h} \paren{g'(f(c)) + R_2(\widetilde h)} \\
                                      &= (f'(c) + R_1(h))\cdot (g'(f(c)) + R_2(\widetilde h)) \\
                                      &= (f'(c) + R_1(h))\cdot (g'(f(c)) + R_2(hf'(c) + hR_1(h))).
    \end{align*}
    Note that by taking the limit as $h\to 0$, we may apply the Algebraic Limit Theorem, getting:
    \begin{align*}
      \lim_{h\to 0} \frac{(g\circ f)(c + h) - (g\circ f)(c)}{h} &= \lim_{h\to 0} (f'(c) + R_1(h))\cdot (g'(f(c)) + R_2(hf'(c) + hR_1(h))) \\
                                                                &= f'(c)\cdot (g'(f(c)) + 0) \\
                                                                &= f'(c)\cdot g'(f(c)).
    \end{align*}
    Hence $(g\circ f)'(c)$ exists and is equal to $f'(c)\cdot g'(f(c))$.
  \end{proof}
  \textbf{Question.} Is the derivative of a function a continuous function? \par
  In general, no. Consider the function defined for all $n\in\N\cup \set{0}$ by
  \[
    f_n(x) = \begin{cases}x^n\sin \paren{\frac{1}{x}} & x\neq 0, \\ 0 & x = 0.\end{cases}
  \]
  When $n = 0$, we see that $f_0$ is not even continuous at $x = 0$. \par
  When $n = 1$, $f_1$ is continuous at $x = 0$ because of Squeeze Theorem. When $x\neq 0$, $f_1'(x) = \sin(\frac{1}{x}) - \frac{1}{x}\cos(\frac{1}{x})$, which doesn't have a limit as $x$ goes to $0$. Similarly, if we take a difference quotient, we still get that the limit doesn't exist so $f_1$ is not differentiable at $x = 0$. \par
  When $n = 2$, we have that $f_2$ is continuous at $x = 0$ by Squeeze Theorem. Through some algebra, we see that $f_2$ is differentiable on $\R\sm \set{0}$, and by taking a difference quotient we are able to see that $f_2$ \emph{is} differentiable at $x = 0$. However, $f_2'$ is not continuous on $\R$.
  \begin{note}{}
    It is possible to have a function that is differentiable everywhere, but that derivative is \emph{not} necessarily continuous everywhere.
  \end{note}
  \subsection{Properties of Derivatives}
  We can use derivatives to help us locate extrema. \\
  \textbf{Proposition 5.6} (Interior Extremum Theorem) Let $I = (a, b)$, $f\colon I\to \R$, $c\in I$. If $c$ is an extremum point for $f$ (i.e. a max or min), and $f$ is differentiable at $c$, then $f'(c) = 0$.
  \begin{proof}
    Since $I$ is an interval, we can find sequences $x_n < c < y_n$ such that $x_n\to c$ and $y_n\to c$. Suppose $f$ has a maximum at $c$. Then $f(y_n)\leq f(c)$ for all $n\in\N$, so $f(y_n) - f(c)\leq 0$. Thus
    \[
      f'(c) = \lim_{n\to \infty} \frac{f(y_n) - f(c)}{y_n - c} \leq 0.
    \]
    Similarly, we have $f(x_n)\leq f(c)$ for all $n\in \N$, so $f(x_n) - f(c)\leq 0$. Thus
    \[
      f'(c) = \lim_{n\to \infty} \frac{f(x_n) - f(c)}{x_n - c} \geq 0.
    \]
    Hence $f'(c) = 0$.
  \end{proof}
  \begin{note}{}
    A local maxima of $f$ at $c\in (a, b)$ exists if there exists $\delta > 0$ such that $f(x)\leq f(c)$ for all $x\in (c - \delta, c + \delta)$.
  \end{note}
  \begin{note}{}
    The converse of Proposition $5.6$ is false, i.e. if there exists $c$ such that $f'(c) = 0$, it \emph{is not necessarily true} that $f(c)$ is an extremum. For instance, consider $f(x) = x^3$ at $c = 0$.
  \end{note}
  \textbf{Proposition 5.7} (Location of Extrema) Let $f\colon [a, b]\to \R$, continuous on $[a, b]$, differentiable on $(a, b)$. Then $f$ has extrema in $[a, b]$ and they occur:
  \begin{enumerate}[label=(\roman*)]
    \item At $a$.
    \item At $b$.
    \item At some $c\in (a, b)$ where $f'(c) = 0$.
  \end{enumerate}
  \begin{example}{}
    Consider the function $f(x) = x^2$ over the interval $[-1, 1]$. Then we have maxima at the endpoints, $c=\pm 1$. The minima is at $0\in (-1, 1)$, and $f'(0) = 0$.
  \end{example}
\end{document}
