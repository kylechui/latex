\documentclass[class=article, crop=false]{standalone}
\usepackage[margin=1in]{geometry}

\usepackage{amsmath}
\usepackage{amssymb}
\usepackage{amsthm}
\usepackage{comment}
\usepackage{enumitem}
\usepackage{fancyhdr}
\usepackage{hyperref}
\usepackage{mathrsfs}
\usepackage{mathtools}
\usepackage{standalone}
\usepackage{tcolorbox}
\usepackage{tikz}
\usepackage{xcolor}

\usetikzlibrary{decorations.pathreplacing}
\tcbuselibrary{skins}

\DeclareMathOperator{\lcm}{lcm}
\DeclareMathOperator{\proj}{proj}
\DeclareMathOperator{\vspan}{span}
\DeclareMathOperator{\im}{im}
\DeclareMathOperator{\range}{range}
\DeclareMathOperator{\Diff}{Diff}
\DeclareMathOperator{\Int}{Int}
\DeclareMathOperator{\fcn}{fcn}
\DeclareMathOperator{\id}{id}

\newcommand{\N}{\ensuremath{\mathbb{N}}}
\newcommand{\Z}{\ensuremath{\mathbb{Z}}}
\newcommand{\Q}{\ensuremath{\mathbb{Q}}}
\newcommand{\R}{\ensuremath{\mathbb{R}}}
\newcommand{\C}{\ensuremath{\mathbb{C}}}
\newcommand{\F}{\ensuremath{\mathbb{F}}}
\newcommand{\lam}{\ensuremath{\lambda}}
\newcommand{\nab}{\ensuremath{\nabla}}
\newcommand{\eps}{\ensuremath{\varepsilon}}
\newcommand{\es}{\ensuremath{\varnothing}}

\newcommand{\abs}[1]{\ensuremath{\left\lvert #1 \right\rvert}}
\newcommand{\bigpar}[1]{\ensuremath{\left( #1 \right)}}
\newcommand{\norm}[1]{\ensuremath{\lVert #1\rVert}}
\newcommand{\set}[1]{\ensuremath{\left\{#1\right\}}}
\newcommand{\tuple}[1]{\ensuremath{\left\langle #1 \right\rangle}}
\newcommand{\floor}[1]{\ensuremath{\left\lfloor #1 \right\rfloor}}
\newcommand{\ceil}[1]{\ensuremath{\left\lceil #1 \right\rceil}}

\newcommand{\chapternum}{}
\newcommand{\ex}[1]{\noindent\textbf{Exercise \chapternum.{#1}.}}

\newcommand{\dx}[1]{\,\mathrm{d}#1}
\newcommand{\inv}{\ensuremath{^{-1}}}
\newcommand{\sm}{\setminus}
\newcommand{\sse}{\subseteq}
\newcommand{\ceq}{\coloneqq}

\definecolor{problemBackground}{RGB}{212,232,246}
\newenvironment{problem}[1]
  {
    \begin{tcolorbox}[
      boxrule=.5pt,
      titlerule=.5pt,
      sharp corners,
      colback=problemBackground
    ]
    \ifx &#1& \textbf{Problem. }
    \else \textbf{Problem #1.} \fi
  }
  {
    \end{tcolorbox}
  }

\definecolor{exampleBackground}{RGB}{255,249,248}
\definecolor{exampleAccent}{RGB}{158,60,14}
\newenvironment{example}[1]
  {
    \begin{tcolorbox}[
      boxrule=.5pt,
      sharp corners,
      colback=exampleBackground,
      colframe=exampleAccent,
    ]
    \color{exampleAccent}\textbf{Example.} \emph{#1}\color{black}
  }
  {
    \end{tcolorbox}
  }

\definecolor{theoremBackground}{RGB}{234,243,251}
\definecolor{theoremAccent}{RGB}{0,116,183}
\newenvironment{theorem}[1]
  {
    \begin{tcolorbox}[
      boxrule=.5pt,
      titlerule=.5pt,
      sharp corners,
      colback=theoremBackground,
      colframe=theoremAccent
    ]
      \color{theoremAccent}\textbf{Theorem --- }\emph{#1}\\\color{black}
  }
  {
    \end{tcolorbox}
  }

\definecolor{noteBackground}{RGB}{244,249,244}
\definecolor{noteAccent}{RGB}{34,139,34}
\newenvironment{note}[1]
  {
  \begin{tcolorbox}[
    enhanced,
    boxrule=0pt,
    frame hidden,
    sharp corners,
    colback=noteBackground,
    borderline west={3pt}{-1.5pt}{noteAccent}
    ]
    \ifx &#1& \color{noteAccent}\textbf{Note. }\color{black}
    \else \color{noteAccent}\textbf{Note (#1). }\color{black} \fi
    }
    {
  \end{tcolorbox}
  }

\definecolor{definitionBackground}{RGB}{246,246,246}
\newenvironment{definition}[1]
  {
    \begin{tcolorbox}[
      enhanced,
      boxrule=0pt,
      frame hidden,
      sharp corners,
      colback=definitionBackground,
      borderline west={3pt}{-1.5pt}{black}
    ]
    \textbf{Definition. }\emph{#1}\\
  }
  {
    \end{tcolorbox}
  }
\newenvironment{amatrix}[2]{
    \left[
      \begin{array}{*{#1}{c}|*{#2}c}
  }
  {
      \end{array}
    \right]
  }
\date{\the\year-\the\month-\the\day}
\author{Kyle Chui}


\fancyhf{}
\lhead{Kyle Chui}
\rhead{Page \thepage}
\pagestyle{fancy}

\begin{document}
  \section{Lecture 20}
  \begin{theorem}{4.12}
    Let $f, g\colon A\to \R$ be continuous at $c\in A$. Then:
    \begin{enumerate}[label=(\roman*)]
      \item $af + bg$ is continuous at $c$, for any $a, b\in\R$.
      \item $f\cdot g$ is continuous at $c$.
      \item $\frac{f}{g}$ is continuous at $c$, whenever the quotient is defined.
    \end{enumerate}
  \end{theorem}
  \begin{example}{}
    \begin{enumerate}[label=\arabic*)]
      \item $g(x) = x$ is continuous on $\R$, which is to say that for all $c\in\R$, $g$ is continuous at $c$. Additionally, where $a\in\R$, is continuous on $\R$.
      \item By Theorem $4.12$ and the examples given above, we have that \emph{any polynomial} is continuous on $\R$:
      \[
        p(x) = a_nx^n + \dotsb + a_1x + a_0,
      \]
      for $a_0,\dotsc,a_n\in\R, n\in\N$.
      \item Every rational function is continuous wherever they are defined. We have for polynomials $p, q$, that
      \[
        r(x) = \frac{p(x)}{q(x)}.
      \]
      \item $f(x) = \sin (\frac{1}{x})$ on $\R\sm\set{0}$. Since $\displaystyle\lim_{x\to 0} f(x)$ does not exist, $f$ is not continuous at $x = 0$.
      \item Consider the function
      \[
        g(x) = \begin{cases} x\sin (\frac{1}{x}) & x\neq 0, \\ 0 & x = 0. \end{cases}
      \]
      This function is continuous at $x = 0$ since the function gets squeezed to $0$.
      \item $f(x) = \sqrt{x}$, $A = [0, \infty)$. We know that $f$ is continuous on $A$, but we need to consider the cases where $c = 0$ and $c\neq 0$. For the former, we have
      \begin{align*}
        \abs{f(x) - f(0)} &= \abs{\sqrt{x}} \\
                          &= \sqrt{x} \\
                          &< \eps,
      \end{align*}
      so we choose $\delta = \eps^2$. When $c\neq 0$, we have
      \begin{align*}
        \abs{f(x) - f(c)} &= \abs{\sqrt{x} - \sqrt{c}} \\
                          &= \frac{\abs{x - c}}{\sqrt{x} + \sqrt{c}} \\
                          &\leq \frac{1}{\sqrt{c}}\abs{x - c} \\
                          &< \frac{\delta}{\sqrt{c}},
      \end{align*}
      so we choose $\delta = \eps\sqrt{c}$.
    \end{enumerate}
  \end{example}
  \newpage
  \textbf{Proposition 4.13} Let $f\colon A\to \R$, $g\colon B\to \R$, where $A, B\sse \R$. Assume $\Ran(f)\sse B$ so that the composition $(g\circ f)(x)$ is well defined. Assume that $f$ is continuous at $c\in A$ and $g$ is continuous at $f(c)\in B$. Then $g\circ f$ is continuous at $c\in A$.
  \begin{proof}
    We use sequences. Let $(x_n)\sse A$, $x_n\to c$. Since $f$ is continuous at $c$, we know that $f(x_n)\to f(c)$. Since $g$ is continuous at $f(c)$, we know that $g(f(x_n))\to g(f(c))$, so $(g\circ f)(x_n)\to (g\circ f)(c)$. Therefore $g\circ f$ is continuous at $c\in A$.
  \end{proof}
  \begin{example}{}
    We are going to assume for this class that $f(x) = \cos x, \sin x, e^x, \log x$ are all continuous functions wherever they are defined. Thus $g(x) = \cos (e^x), e^{\sin x}, \sin(\sin(e^x))$ are all continuous on $\R$.
  \end{example}
  \subsection{The Extreme Value Theorem}
  \begin{definition}{4.14---Compactness}
    We say that a set $K\sse\R$ is \emph{compact} if for any sequence $(x_n)\sse K$ there exists a subsequence $(x_{n_k})$ which converges to $x\in K$.
  \end{definition}
  \begin{note}{}
    This is a generalization of the notion of closed sets, which says that a set is closed if it contains all of its limit points.
  \end{note}
  \begin{example}{Compact Sets}
    \begin{enumerate}[label=\arabic*)]
      \item The set $K = [a, b]$ is compact, where $a\leq b$. \par
      Let $(x_n)\sse K$. We see that $K$ is bounded, so $(x_n)$ is bounded. Thus by Bolzano--Weierstrass, we have there exists some subsequence $(x_{n_k})$ such that $x_{n_k}\to x\in\R$. Since $a\leq x_{n_k}\leq b$, we know that $a\leq x\leq b$, so $x\in K$.
      \item The set $(a, b)$ is not compact. Observe that $x_n = b - \frac{b - a}{2n}\sse (a, b)$, but $x_n\to b\notin (a, b)$. Thus it is not compact.
      \item $[0, \infty)$ is not compact, because $x_n = n$ diverges to $+\infty$, and so every subsequence $(x_{n_k})$ must also diverge to $+\infty$.
    \end{enumerate}
  \end{example}
  \begin{theorem}{4.15 (Heine--Borel Theorem)}
    A set $K\sse \R$ is compact if and only if it is closed and bounded.
    \begin{proof}
      ($\Leftarrow$) Suppose $K\sse\R$ is closed and bounded. We claim that $K$ is compact. Let $(x_n)\sse K$. Since $K$ is bounded, Bolzano--Weierstrass says that there exists some subsequence $(x_{n_k})\sse K$ such that $x_{n_k}\to x\in\R$. Since $K$ is closed, we know that $x\in K$, and so $K$ is compact. \par
      ($\Rightarrow$) Suppose $K$ is compact. We first show that $K$ is bounded. Suppose towards a contradiction that $K$ is not bounded, so there exists some sequence $(x_n)\sse K$ such that $\abs{x_n}\to +\infty$. However, this means that every subsequence $\abs{x_{n_k}}$ must also diverge to $+\infty$, which contradicts compactness. \par
      We now show that $K$ is closed, which is to say that if $(x_n)\sse K$ such that $x_n\to x\in \R$, then $x\in K$. By compactness, there exists a subsequence $(x_{n_k})$ such that $x_{n_k}\to y\in K$. However, since every subsequence of a convergent sequence must converge to the same value, we have $x = y\in K$ and so $K$ is closed.
    \end{proof}
  \end{theorem}
\end{document}
