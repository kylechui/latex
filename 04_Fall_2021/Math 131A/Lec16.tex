\documentclass[class=article, crop=false]{standalone}
\usepackage[margin=1in]{geometry}

\usepackage{amsmath}
\usepackage{amssymb}
\usepackage{amsthm}
\usepackage{comment}
\usepackage{enumitem}
\usepackage{fancyhdr}
\usepackage{hyperref}
\usepackage{mathrsfs}
\usepackage{mathtools}
\usepackage{standalone}
\usepackage{tcolorbox}
\usepackage{tikz}
\usepackage{xcolor}

\usetikzlibrary{decorations.pathreplacing}
\tcbuselibrary{skins}

\DeclareMathOperator{\lcm}{lcm}
\DeclareMathOperator{\proj}{proj}
\DeclareMathOperator{\vspan}{span}
\DeclareMathOperator{\im}{im}
\DeclareMathOperator{\range}{range}
\DeclareMathOperator{\Diff}{Diff}
\DeclareMathOperator{\Int}{Int}
\DeclareMathOperator{\fcn}{fcn}
\DeclareMathOperator{\id}{id}

\newcommand{\N}{\ensuremath{\mathbb{N}}}
\newcommand{\Z}{\ensuremath{\mathbb{Z}}}
\newcommand{\Q}{\ensuremath{\mathbb{Q}}}
\newcommand{\R}{\ensuremath{\mathbb{R}}}
\newcommand{\C}{\ensuremath{\mathbb{C}}}
\newcommand{\F}{\ensuremath{\mathbb{F}}}
\newcommand{\lam}{\ensuremath{\lambda}}
\newcommand{\nab}{\ensuremath{\nabla}}
\newcommand{\eps}{\ensuremath{\varepsilon}}
\newcommand{\es}{\ensuremath{\varnothing}}

\newcommand{\abs}[1]{\ensuremath{\left\lvert #1 \right\rvert}}
\newcommand{\bigpar}[1]{\ensuremath{\left( #1 \right)}}
\newcommand{\norm}[1]{\ensuremath{\lVert #1\rVert}}
\newcommand{\set}[1]{\ensuremath{\left\{#1\right\}}}
\newcommand{\tuple}[1]{\ensuremath{\left\langle #1 \right\rangle}}
\newcommand{\floor}[1]{\ensuremath{\left\lfloor #1 \right\rfloor}}
\newcommand{\ceil}[1]{\ensuremath{\left\lceil #1 \right\rceil}}

\newcommand{\chapternum}{}
\newcommand{\ex}[1]{\noindent\textbf{Exercise \chapternum.{#1}.}}

\newcommand{\dx}[1]{\,\mathrm{d}#1}
\newcommand{\inv}{\ensuremath{^{-1}}}
\newcommand{\sm}{\setminus}
\newcommand{\sse}{\subseteq}
\newcommand{\ceq}{\coloneqq}

\definecolor{problemBackground}{RGB}{212,232,246}
\newenvironment{problem}[1]
  {
    \begin{tcolorbox}[
      boxrule=.5pt,
      titlerule=.5pt,
      sharp corners,
      colback=problemBackground
    ]
    \ifx &#1& \textbf{Problem. }
    \else \textbf{Problem #1.} \fi
  }
  {
    \end{tcolorbox}
  }

\definecolor{exampleBackground}{RGB}{255,249,248}
\definecolor{exampleAccent}{RGB}{158,60,14}
\newenvironment{example}[1]
  {
    \begin{tcolorbox}[
      boxrule=.5pt,
      sharp corners,
      colback=exampleBackground,
      colframe=exampleAccent,
    ]
    \color{exampleAccent}\textbf{Example.} \emph{#1}\color{black}
  }
  {
    \end{tcolorbox}
  }

\definecolor{theoremBackground}{RGB}{234,243,251}
\definecolor{theoremAccent}{RGB}{0,116,183}
\newenvironment{theorem}[1]
  {
    \begin{tcolorbox}[
      boxrule=.5pt,
      titlerule=.5pt,
      sharp corners,
      colback=theoremBackground,
      colframe=theoremAccent
    ]
      \color{theoremAccent}\textbf{Theorem --- }\emph{#1}\\\color{black}
  }
  {
    \end{tcolorbox}
  }

\definecolor{noteBackground}{RGB}{244,249,244}
\definecolor{noteAccent}{RGB}{34,139,34}
\newenvironment{note}[1]
  {
  \begin{tcolorbox}[
    enhanced,
    boxrule=0pt,
    frame hidden,
    sharp corners,
    colback=noteBackground,
    borderline west={3pt}{-1.5pt}{noteAccent}
    ]
    \ifx &#1& \color{noteAccent}\textbf{Note. }\color{black}
    \else \color{noteAccent}\textbf{Note (#1). }\color{black} \fi
    }
    {
  \end{tcolorbox}
  }

\definecolor{definitionBackground}{RGB}{246,246,246}
\newenvironment{definition}[1]
  {
    \begin{tcolorbox}[
      enhanced,
      boxrule=0pt,
      frame hidden,
      sharp corners,
      colback=definitionBackground,
      borderline west={3pt}{-1.5pt}{black}
    ]
    \textbf{Definition. }\emph{#1}\\
  }
  {
    \end{tcolorbox}
  }
\newenvironment{amatrix}[2]{
    \left[
      \begin{array}{*{#1}{c}|*{#2}c}
  }
  {
      \end{array}
    \right]
  }
\date{\the\year-\the\month-\the\day}
\author{Kyle Chui}


\fancyhf{}
\lhead{Kyle Chui}
\rhead{Page \thepage}
\pagestyle{fancy}

\begin{document}
  \section{Lecture 16}
  \begin{definition}{3.25---Cauchy Criterion for Series}
    We say a series $\sum x_n$ satisfies the \emph{Cauchy Criterion} if the sequence of partial sums $(S_n)$ is a Cauchy sequence. In other words, for all $\eps > 0$, there exists some $N\in\N$ such that
    \[
      \abs{S_n - S_m} < \eps,
    \]
    for all $m, n > N$. Written another way, we have that for all $\eps > 0$, there exists some $N\in\N$ such that
    \[
      \abs{\sum_{k=m + 1}^{n}x_k} < \eps
    \]
    for all $n\geq m > N$.
  \end{definition}
  \begin{theorem}{3.26}
    A series $\sum x_n$ satisfies the Cauchy criterion if and only if the series converges.
    \begin{proof}
      ($\Rightarrow$) We know that the sequence of partial sums is Cauchy. Since the reals are complete, we have that the partial sums converge, and so the series converges. \par
      ($\Leftarrow$) If the series converges, the sequence of partial sums is converges, and so is Cauchy. Therefore the series satisfies the Cauchy criterion.
    \end{proof}
  \end{theorem}
  \textbf{Corollary 3.27} If $\sum x_n$ converges, then $\lim_{n\to \infty} x_n = 0$.
  \begin{proof}
    Let $(S_n)$ denote the sequence of partial sums of $(x_n)$. Since $(S_n)$ converges, we know that it satisfies the Cauchy criterion. Since $(S_n)$ is Cauchy, we have that for all $\eps > 0$, there exists some $N\in\N$ such that
    \[
      \abs{\sum_{k=1}^{n}x_k - \sum_{k=1}^{m}x_k} = \sum_{k=m + 1}^{n}x_k < \eps,
    \]
    for all $m,n > N$. Letting $n = m + 1$, we have
    \begin{align*}
      \abs{\sum_{k=m + 1}^{n}x_k} &< \eps \\
      \abs{\sum_{k=m + 1}^{m + 1}x_k} &< \eps \\
      \abs{x_{m + 1}} &< \eps.
    \end{align*}
    Thus for all $n > N$, we have $\abs{x_n} < \eps$, and so
    \[
      \lim_{n\to \infty} x_n = 0.
    \]
  \end{proof}
  \subsection{Various Convergence Tests}
  \begin{note}{}
    This gives us the ``divergence test''. If the limit of the summand of a series does not converge to $0$, then the series diverges.
  \end{note}
  \begin{example}{Divergence Test} \\
    Consider the series given by
    \[
      \sum_{n=1}^{\infty} \frac{n^2 - 1}{n^2 + 1}.
    \]
    The above series diverges because the summand converges to $1$ (by Algebraic Limit Theorem). However, the sequence converging a necessary and non-sufficient condition to show that the series converges, i.e. consider $\sum x_n = \frac{1}{n}$.
  \end{example}
  \begin{theorem}{3.28 (Comparison Test)}
    Assume $(x_n)$, $(y_n)$ are sequences in $\R$, $y_n\geq 0$ for all $n\in\N$. Then
    \begin{enumerate}[label=(\roman*)]
      \item If the series $\sum y_n$ converges and $\abs{x_n}\leq y_n$ for all $n\in\N$, then $\sum x_n$ converges.
      \item If the series $\sum y_n$ diverges and $\abs{x_n}\geq y_n$ for all $n\in\N$, then $\sum x_n$ diverges.
    \end{enumerate}
    \begin{proof}
      \begin{enumerate}[label=(\roman*)]
        \item Suppose $n > m$. Then
        \begin{align*}
          \abs{\sum_{k=m+1}^{n}x_k} &\leq \sum_{k=m+1}^{n}\abs{x_k} \\
                                    &\leq \sum_{k=m+1}^{n}y_k.
        \end{align*}
        Since $\sum y_n$ converges, it satisfies the Cauchy criterion. Thus we fix some $\eps > 0$, and there exists some $N\in\N$ such that
        \[
          \sum_{k=m+1}^{n}y_k < \eps,
        \]
        for all $n > m > N$. Continuing the inequality from before we get
        \[
          \abs{\sum_{k=m+1}^{n}x_k} < \eps,
        \]
        so $\sum x_k$ is Cauchy, and so converges.
        \item Let $(X_n)$ be partial sums for $\sum x_n$, and $(Y_n)$ be partial sums for $\sum y_n$. By domination, we know that $X_n\geq Y_n$ for all $n\in\N$. Since $\sum y_n$ diverges, we have $Y_n\to+\infty$. This forces $X_n$ to diverge to $+\infty$ as well.
      \end{enumerate}
    \end{proof}
  \end{theorem}
  \begin{note}{}
    The domination conditions in the previous theorem need only apply for all but finitely many $n$, rather than for all $n\in\N$. In other words, we only need those conditions to hold ``from some point onwards''. This is because a series' convergence is dictated by the convergence/divergence of its ``tail'':
    \[
      \sum_{n=1}^{\infty} x_n = \underbrace{\sum_{n=1}^{N}x_n}_{\text{Finite}} + \underbrace{\sum_{n=N+1}^{\infty}x_n}_{\text{``Tail''}}.
    \]
  \end{note}
  \subsection{Convergent Series and the Harmonics}
  \textbf{Question.} How close can we get to the Harmonic Series and still converge? \par
  By the $p$-series test, we know that $\sum \frac{1}{n^p} < +\infty$ if and only if $p > 1$. Thus for any $\eps > 0$, we have $\sum \frac{1}{n}\cdot \frac{1}{n^{\eps}} = \sum \frac{1}{n^{1 + \eps}}$, which converges. The question remains whether we can replace $\frac{1}{n^\eps}$ by something that decays \emph{slower}. \par
  We know that for any $a > 0$, there exists $N\in\N$ such that $\log n\leq n^a$ for all $n > N$. In other words, logarithms grow slower than any exponential. We know that the series
  \[
    \sum \frac{1}{n^p\log (1 + n)}
  \]
  converges when $p > 1$, by comparison with $\sum \frac{1}{n^p}$. Here we try replacing the $n^\eps$ from earlier with $\log (1 + n)$. Then
  \[
    \sum_{n=2}^{\infty} \frac{1}{n (\log n)^\beta}
  \]
  converges if and only if $\beta > 1$. We can continue to get closer to the $\beta = 1$ case by adding more logs, i.e.
  \[
    \sum_{n=2}^{\infty} \frac{1}{n (\log n)(\log\log n)^\alpha}
  \]
  converges if and only if $\alpha > 1$.
  \subsection{Absolute and Conditional Convergence}
  \begin{theorem}{3.29 (Absolute Convergence Test)}
    If the series $\sum \abs{x_n}$ converges, then $\sum x_n$ converges.
    \begin{proof}
      Comparison test---we have $\abs{x_n}\leq \abs{x_n}$.
    \end{proof}
  \end{theorem}
  \begin{note}{}
    The converse of the above is generally false, as $\sum\frac{1}{n}$ diverges but $\sum \frac{(-1)^n}{n}$ converges.
  \end{note}
  \begin{definition}{3.30---Absolute/Conditional Convergence}
    A series $\sum x_n$ converges \emph{absolutely} if $\sum \abs{x_n}$ converges. If the series $\sum x_n$ converges and $\sum \abs{x_n}$ diverges, we say $\sum x_n$ converges \emph{conditionally}.
  \end{definition}
\end{document}
