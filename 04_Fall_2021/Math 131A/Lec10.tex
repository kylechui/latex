\documentclass[class=article, crop=false]{standalone}
% Import packages
\usepackage[margin=1in]{geometry}

\usepackage{amssymb, amsthm}
\usepackage{comment}
\usepackage{enumitem}
\usepackage{fancyhdr}
\usepackage{hyperref}
\usepackage{import}
\usepackage{mathrsfs, mathtools}
\usepackage{multicol}
\usepackage{pdfpages}
\usepackage{standalone}
\usepackage{transparent}
\usepackage{xcolor}

\usepackage{minted}
\usepackage[many]{tcolorbox}
\tcbuselibrary{minted}

% Declare math operators
\DeclareMathOperator{\lcm}{lcm}
\DeclareMathOperator{\proj}{proj}
\DeclareMathOperator{\vspan}{span}
\DeclareMathOperator{\im}{im}
\DeclareMathOperator{\Diff}{Diff}
\DeclareMathOperator{\Int}{Int}
\DeclareMathOperator{\fcn}{fcn}
\DeclareMathOperator{\id}{id}
\DeclareMathOperator{\rank}{rank}
\DeclareMathOperator{\tr}{tr}
\DeclareMathOperator{\dive}{div}
\DeclareMathOperator{\row}{row}
\DeclareMathOperator{\col}{col}
\DeclareMathOperator{\dom}{dom}
\DeclareMathOperator{\ran}{ran}
\DeclareMathOperator{\Dom}{Dom}
\DeclareMathOperator{\Ran}{Ran}
\DeclareMathOperator{\fl}{fl}
% Macros for letters/variables
\newcommand\ttilde{\kern -.15em\lower .7ex\hbox{\~{}}\kern .04em}
\newcommand{\N}{\ensuremath{\mathbb{N}}}
\newcommand{\Z}{\ensuremath{\mathbb{Z}}}
\newcommand{\Q}{\ensuremath{\mathbb{Q}}}
\newcommand{\R}{\ensuremath{\mathbb{R}}}
\newcommand{\C}{\ensuremath{\mathbb{C}}}
\newcommand{\F}{\ensuremath{\mathbb{F}}}
\newcommand{\M}{\ensuremath{\mathbb{M}}}
\renewcommand{\P}{\ensuremath{\mathbb{P}}}
\newcommand{\lam}{\ensuremath{\lambda}}
\newcommand{\nab}{\ensuremath{\nabla}}
\newcommand{\eps}{\ensuremath{\varepsilon}}
\newcommand{\es}{\ensuremath{\varnothing}}
% Macros for math symbols
\newcommand{\dx}[1]{\,\mathrm{d}#1}
\newcommand{\inv}{\ensuremath{^{-1}}}
\newcommand{\sm}{\setminus}
\newcommand{\sse}{\subseteq}
\newcommand{\ceq}{\coloneqq}
% Macros for pairs of math symbols
\newcommand{\abs}[1]{\ensuremath{\left\lvert #1 \right\rvert}}
\newcommand{\paren}[1]{\ensuremath{\left( #1 \right)}}
\newcommand{\norm}[1]{\ensuremath{\left\lVert #1\right\rVert}}
\newcommand{\set}[1]{\ensuremath{\left\{#1\right\}}}
\newcommand{\tup}[1]{\ensuremath{\left\langle #1 \right\rangle}}
\newcommand{\floor}[1]{\ensuremath{\left\lfloor #1 \right\rfloor}}
\newcommand{\ceil}[1]{\ensuremath{\left\lceil #1 \right\rceil}}
\newcommand{\eclass}[1]{\ensuremath{\left[ #1 \right]}}

\newcommand{\chapternum}{}
\newcommand{\ex}[1]{\noindent\textbf{Exercise \chapternum.{#1}.}}

\newcommand{\tsub}[1]{\textsubscript{#1}}
\newcommand{\tsup}[1]{\textsuperscript{#1}}

\renewcommand{\choose}[2]{\ensuremath{\phantom{}_{#1}C_{#2}}}
\newcommand{\perm}[2]{\ensuremath{\phantom{}_{#1}P_{#2}}}

% Include figures
\newcommand{\incfig}[2][1]{%
    \def\svgwidth{#1\columnwidth}
    \import{./figures/}{#2.pdf_tex}
}

\definecolor{problemBackground}{RGB}{212,232,246}
\newenvironment{problem}[1]
  {
    \begin{tcolorbox}[
      boxrule=.5pt,
      titlerule=.5pt,
      sharp corners,
      colback=problemBackground,
      breakable
    ]
    \ifx &#1& \textbf{Problem. }
    \else \textbf{Problem #1.} \fi
  }
  {
    \end{tcolorbox}
  }
\definecolor{exampleBackground}{RGB}{255,249,248}
\definecolor{exampleAccent}{RGB}{158,60,14}
\newenvironment{example}[1]
  {
    \begin{tcolorbox}[
      boxrule=.5pt,
      sharp corners,
      colback=exampleBackground,
      colframe=exampleAccent,
    ]
    \color{exampleAccent}\textbf{Example.} \emph{#1}\color{black}
  }
  {
    \end{tcolorbox}
  }
\definecolor{theoremBackground}{RGB}{234,243,251}
\definecolor{theoremAccent}{RGB}{0,116,183}
\newenvironment{theorem}[1]
  {
    \begin{tcolorbox}[
      boxrule=.5pt,
      titlerule=.5pt,
      sharp corners,
      colback=theoremBackground,
      colframe=theoremAccent,
      breakable
    ]
      % \color{theoremAccent}\textbf{Theorem --- }\emph{#1}\\\color{black}
    \ifx &#1& \color{theoremAccent}\textbf{Theorem. }\color{black}
    \else \color{theoremAccent}\textbf{Theorem --- }\emph{#1}\\\color{black} \fi
  }
  {
    \end{tcolorbox}
  }
\definecolor{noteBackground}{RGB}{244,249,244}
\definecolor{noteAccent}{RGB}{34,139,34}
\newenvironment{note}[1]
  {
  \begin{tcolorbox}[
    enhanced,
    boxrule=0pt,
    frame hidden,
    sharp corners,
    colback=noteBackground,
    borderline west={3pt}{-1.5pt}{noteAccent},
    breakable
    ]
    \ifx &#1& \color{noteAccent}\textbf{Note. }\color{black}
    \else \color{noteAccent}\textbf{Note (#1). }\color{black} \fi
    }
    {
  \end{tcolorbox}
  }
\definecolor{lemmaBackground}{RGB}{255,247,234}
\definecolor{lemmaAccent}{RGB}{255,153,0}
\newenvironment{lemma}[1]
  {
    \begin{tcolorbox}[
      enhanced,
      boxrule=0pt,
      frame hidden,
      sharp corners,
      colback=lemmaBackground,
      borderline west={3pt}{-1.5pt}{lemmaAccent},
      breakable
    ]
    \ifx &#1& \color{lemmaAccent}\textbf{Lemma. }\color{black}
    \else \color{lemmaAccent}\textbf{Lemma #1. }\color{black} \fi
  }
  {
    \end{tcolorbox}
  }
\definecolor{definitionBackground}{RGB}{246,246,246}
\newenvironment{definition}[1]
  {
    \begin{tcolorbox}[
      enhanced,
      boxrule=0pt,
      frame hidden,
      sharp corners,
      colback=definitionBackground,
      borderline west={3pt}{-1.5pt}{black},
      breakable
    ]
    \textbf{Definition. }\emph{#1}\\
  }
  {
    \end{tcolorbox}
  }

\newenvironment{amatrix}[2]{
    \left[
      \begin{array}{*{#1}{c}|*{#2}c}
  }
  {
      \end{array}
    \right]
  }

\definecolor{codeBackground}{RGB}{253,246,227}
\renewcommand\theFancyVerbLine{\arabic{FancyVerbLine}}
\newtcblisting{cpp}[1][]{
  boxrule=1pt,
  sharp corners,
  colback=codeBackground,
  listing only,
  breakable,
  minted language=cpp,
  minted style=material,
  minted options={
    xleftmargin=15pt,
    baselinestretch=1.2,
    linenos,
  }
}

\author{Kyle Chui}


\fancyhf{}
\lhead{Kyle Chui}
\rhead{Page \thepage}
\pagestyle{fancy}

\begin{document}
  \section{Lecture 10}
  \subsection{Convergence of Sequences Using the Definition of Convergence}
  From last lecture, we have a basic definition of convergence. However, this is not very helpful for determining whether a sequence converges or not because we have to actually know what the sequence converges to.
  \begin{example}{}
    Consider the sequence $x_n = \frac{4n^3 + n}{n^3 - 6}$. We have an intuition that this converges to $4$ by dividing the leading coefficients of the numerator and denominator. \\[10pt]
    We have
    \begin{align*}
      \abs{x_n - 4} &= \abs{\frac{4n^3 + n}{n^3 - 6} - 4} \\
                    &= \abs{\frac{(4n^3 + n) - 4(n^3 - 6)}{n^3 - 6}} \\
                    &= \abs{\frac{n + 24}{n^3 - 6}} \\
                    &= \frac{\abs{n + 24}}{\abs{n^3 - 6}} \\
                    &= \frac{n + 24}{n^3 - 6}
    \end{align*}
    We want to find some $N$ such that $\frac{n+24}{\abs{n^3 - 6}} < \eps$ for all $n > N$. Thus we want to show
    \[
      \frac{n+24}{\abs{n^3 - 6}}\leq \frac{C}{n^2}\leq \eps.
    \]
    If $n\geq 24$, we have $n + 24\leq 2n$. Suppose we wish for $\abs{n^3 - 6} > 0$, so $n \geq 2$. Furthermore, note that $n^3 - 6 \geq \frac{1}{2}n^3$ when $n\geq 12^{\frac{1}{3}}$. Thus for $n\geq 24$, we have
    \begin{align*}
      \frac{n + 24}{\abs{n^3 -6}} &\leq \frac{2n}{\frac{1}{2}n^3} \\
                                  &= \frac{4}{n^2}.
    \end{align*}
    Thus we take 
    \[
      N = \max\paren{24, \ceil{\sqrt{\frac{4}{\eps}}}} = \max\paren{24, \ceil{\frac{2}{\sqrt{\eps}}}}.
    \]
    Thus for $n > N$, we have
    \[
      \frac{n + 24}{\abs{n^3 - 6}} < \eps.
    \]
    \begin{proof}
      Fix $\eps > 0$ and choose $N = \max\paren{24, \ceil{\frac{2}{\sqrt{\eps}}}}$. Then
      \begin{align*}
        \abs{x_n - 4} &= \frac{n + 24}{\abs{n^3 - 6}} \\
                      &\leq \frac{4}{n^2} \\
                      &< \eps
      \end{align*}
      for all $n > N$. Thus $\lim_{n\to \infty}x_n = 4$.
    \end{proof}
  \end{example}
  \begin{example}{}
    Show that $x_n = (-1)^n$ diverges. \\[10pt]
    Note that $x_{2k} = 1$, and $x_{2k + 1} = -1$ for some integer $k$.
    \begin{proof}
      Suppose towards a contradiction that $x_n$ converges to a point $x$. Since we know that the sequence only takes on the values $1$ and $-1$, we know that
      \begin{align*}
        2 &= \abs{(1 - x) + (x + 1)} \\
          &\leq \abs{1 - x} + \abs{x + 1} \\
        2 &\leq \abs{x_{2n} - x} + \abs{x - x_{2n + 1}}.
      \end{align*}
      Since $x_n\to x$, given $\eps > 0$, there exists $N$ such that $\abs{x_n - 1} < \eps$ for all $n > N$. For $n > N$, we have $\abs{x_{2n} - x} < \eps$ and $\abs{x - x_{2n + 1}} < \eps$. Thus we have $2\leq 2\eps$, so $1\leq \eps$, which contradicts that $\eps$ may be arbitrarily small. Hence the sequence diverges.
    \end{proof}
  \end{example}
  \subsection{Limit Laws}
  \textbf{Proposition 3.4} (Limits are Unique)---If $x_n\to x$ and $x_n\to y$, then $x = y$. \par
  Idea: The points in the sequence have to be arbitrarily close to $x$ and $y$ simultaneously. \par
  We see that
  \begin{align*}
    \abs{x - y}&\leq \abs{x - x_n} + \abs{x_n - y} \\
               &= \eps,
  \end{align*}
  for any $\eps > 0$. By a previous theorem, we have $x = y$.
  \begin{proof}
    Fix $\eps > 0$. Since $x_n\to x$, there exists $N_1\in\N$ such that $\abs{x_n - x} < \frac{\eps}{2}$ for all $n > N_1$. Since $x_n\to y$, there exists $N_2\in\N$ such that $\abs{x_n - y} < \frac{\eps}{2}$ for all $n > N_2$. Then for $n > \max(N_1, N_2)$, the triangle inequality implies that
    \begin{align*}
      \abs{x - y} &\leq \abs{x_n - x} + \abs{y - x_n} \\
                  &< \frac{\eps}{2} + \frac{\eps}{2} \\
                  &= \eps.
    \end{align*}
    Since we have $\abs{x - y} < \eps$ for any $\eps > 0$, we have $x = y$.
  \end{proof}
  \begin{definition}{3.5---Boundedness for Sequences}
    A sequence $(x_n)$ is \emph{bounded} if there exists some real number $M > 0$ such that $\abs{x_n}\leq M$ for all $n\in\N$.
  \end{definition}
  \begin{note}{}
    In the above, our choice of $M$ does not depend on $n$. Additionally, all convergent sequences are bounded (since we may just choose an arbitrary $\eps$ and we know that $x_n$ will at some point rest between $x - \eps$ and $x + \eps$).
  \end{note}
  \begin{theorem}{3.6 (Convergent Sequences are Bounded)}
    Suppose $x_n\to x$. Then for $\eps = 1$, we may find a $N\in\N$ such that $\abs{x_n - x} < 1$ for all $n > N$. Then when $n > N$, we have
    \begin{align*}
      \abs{x_n} &\leq \abs{x_n - x} + \abs{x} \\
                &\leq 1 + \abs{x}.
    \end{align*}
    Hence, for $M = \max(\abs{x_1}, \abs{x_2},\dotsc, \abs{x_N}, 1 + \abs{x})$, we see that $\abs{x_n} \leq M$ for all $n\in \N$.
  \end{theorem}
  \begin{theorem}{3.7 (Algebraic Limit Theorem)}
    Let $\lim_{n\to \infty}x_n = x$ and $\lim_{n\to \infty}y_n = y$. Then
    \begin{enumerate}[label=(\roman*)]
      \item $\lim_{n\to \infty}(ax_n + by_n) = ax + by$ for all $a, b\in\R$.
      \item $\lim_{n\to \infty}(x_n\cdot y_n) = xy$.
      \item $\lim_{n\to \infty} \frac{x_n}{y_n} = \frac{x}{y}$, provided $y\neq 0$.
    \end{enumerate}
  \end{theorem}
\end{document}
