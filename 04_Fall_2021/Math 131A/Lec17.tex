\documentclass[class=article, crop=false]{standalone}
% Import packages
\usepackage[margin=1in]{geometry}

\usepackage{amssymb, amsthm}
\usepackage{comment}
\usepackage{enumitem}
\usepackage{fancyhdr}
\usepackage{hyperref}
\usepackage{import}
\usepackage{mathrsfs, mathtools}
\usepackage{multicol}
\usepackage{pdfpages}
\usepackage{standalone}
\usepackage{transparent}
\usepackage{xcolor}

\usepackage{minted}
\usepackage[many]{tcolorbox}
\tcbuselibrary{minted}

% Declare math operators
\DeclareMathOperator{\lcm}{lcm}
\DeclareMathOperator{\proj}{proj}
\DeclareMathOperator{\vspan}{span}
\DeclareMathOperator{\im}{im}
\DeclareMathOperator{\Diff}{Diff}
\DeclareMathOperator{\Int}{Int}
\DeclareMathOperator{\fcn}{fcn}
\DeclareMathOperator{\id}{id}
\DeclareMathOperator{\rank}{rank}
\DeclareMathOperator{\tr}{tr}
\DeclareMathOperator{\dive}{div}
\DeclareMathOperator{\row}{row}
\DeclareMathOperator{\col}{col}
\DeclareMathOperator{\dom}{dom}
\DeclareMathOperator{\ran}{ran}
\DeclareMathOperator{\Dom}{Dom}
\DeclareMathOperator{\Ran}{Ran}
\DeclareMathOperator{\fl}{fl}
% Macros for letters/variables
\newcommand\ttilde{\kern -.15em\lower .7ex\hbox{\~{}}\kern .04em}
\newcommand{\N}{\ensuremath{\mathbb{N}}}
\newcommand{\Z}{\ensuremath{\mathbb{Z}}}
\newcommand{\Q}{\ensuremath{\mathbb{Q}}}
\newcommand{\R}{\ensuremath{\mathbb{R}}}
\newcommand{\C}{\ensuremath{\mathbb{C}}}
\newcommand{\F}{\ensuremath{\mathbb{F}}}
\newcommand{\M}{\ensuremath{\mathbb{M}}}
\renewcommand{\P}{\ensuremath{\mathbb{P}}}
\newcommand{\lam}{\ensuremath{\lambda}}
\newcommand{\nab}{\ensuremath{\nabla}}
\newcommand{\eps}{\ensuremath{\varepsilon}}
\newcommand{\es}{\ensuremath{\varnothing}}
% Macros for math symbols
\newcommand{\dx}[1]{\,\mathrm{d}#1}
\newcommand{\inv}{\ensuremath{^{-1}}}
\newcommand{\sm}{\setminus}
\newcommand{\sse}{\subseteq}
\newcommand{\ceq}{\coloneqq}
% Macros for pairs of math symbols
\newcommand{\abs}[1]{\ensuremath{\left\lvert #1 \right\rvert}}
\newcommand{\paren}[1]{\ensuremath{\left( #1 \right)}}
\newcommand{\norm}[1]{\ensuremath{\left\lVert #1\right\rVert}}
\newcommand{\set}[1]{\ensuremath{\left\{#1\right\}}}
\newcommand{\tup}[1]{\ensuremath{\left\langle #1 \right\rangle}}
\newcommand{\floor}[1]{\ensuremath{\left\lfloor #1 \right\rfloor}}
\newcommand{\ceil}[1]{\ensuremath{\left\lceil #1 \right\rceil}}
\newcommand{\eclass}[1]{\ensuremath{\left[ #1 \right]}}

\newcommand{\chapternum}{}
\newcommand{\ex}[1]{\noindent\textbf{Exercise \chapternum.{#1}.}}

\newcommand{\tsub}[1]{\textsubscript{#1}}
\newcommand{\tsup}[1]{\textsuperscript{#1}}

\renewcommand{\choose}[2]{\ensuremath{\phantom{}_{#1}C_{#2}}}
\newcommand{\perm}[2]{\ensuremath{\phantom{}_{#1}P_{#2}}}

% Include figures
\newcommand{\incfig}[2][1]{%
    \def\svgwidth{#1\columnwidth}
    \import{./figures/}{#2.pdf_tex}
}

\definecolor{problemBackground}{RGB}{212,232,246}
\newenvironment{problem}[1]
  {
    \begin{tcolorbox}[
      boxrule=.5pt,
      titlerule=.5pt,
      sharp corners,
      colback=problemBackground,
      breakable
    ]
    \ifx &#1& \textbf{Problem. }
    \else \textbf{Problem #1.} \fi
  }
  {
    \end{tcolorbox}
  }
\definecolor{exampleBackground}{RGB}{255,249,248}
\definecolor{exampleAccent}{RGB}{158,60,14}
\newenvironment{example}[1]
  {
    \begin{tcolorbox}[
      boxrule=.5pt,
      sharp corners,
      colback=exampleBackground,
      colframe=exampleAccent,
    ]
    \color{exampleAccent}\textbf{Example.} \emph{#1}\color{black}
  }
  {
    \end{tcolorbox}
  }
\definecolor{theoremBackground}{RGB}{234,243,251}
\definecolor{theoremAccent}{RGB}{0,116,183}
\newenvironment{theorem}[1]
  {
    \begin{tcolorbox}[
      boxrule=.5pt,
      titlerule=.5pt,
      sharp corners,
      colback=theoremBackground,
      colframe=theoremAccent,
      breakable
    ]
      % \color{theoremAccent}\textbf{Theorem --- }\emph{#1}\\\color{black}
    \ifx &#1& \color{theoremAccent}\textbf{Theorem. }\color{black}
    \else \color{theoremAccent}\textbf{Theorem --- }\emph{#1}\\\color{black} \fi
  }
  {
    \end{tcolorbox}
  }
\definecolor{noteBackground}{RGB}{244,249,244}
\definecolor{noteAccent}{RGB}{34,139,34}
\newenvironment{note}[1]
  {
  \begin{tcolorbox}[
    enhanced,
    boxrule=0pt,
    frame hidden,
    sharp corners,
    colback=noteBackground,
    borderline west={3pt}{-1.5pt}{noteAccent},
    breakable
    ]
    \ifx &#1& \color{noteAccent}\textbf{Note. }\color{black}
    \else \color{noteAccent}\textbf{Note (#1). }\color{black} \fi
    }
    {
  \end{tcolorbox}
  }
\definecolor{lemmaBackground}{RGB}{255,247,234}
\definecolor{lemmaAccent}{RGB}{255,153,0}
\newenvironment{lemma}[1]
  {
    \begin{tcolorbox}[
      enhanced,
      boxrule=0pt,
      frame hidden,
      sharp corners,
      colback=lemmaBackground,
      borderline west={3pt}{-1.5pt}{lemmaAccent},
      breakable
    ]
    \ifx &#1& \color{lemmaAccent}\textbf{Lemma. }\color{black}
    \else \color{lemmaAccent}\textbf{Lemma #1. }\color{black} \fi
  }
  {
    \end{tcolorbox}
  }
\definecolor{definitionBackground}{RGB}{246,246,246}
\newenvironment{definition}[1]
  {
    \begin{tcolorbox}[
      enhanced,
      boxrule=0pt,
      frame hidden,
      sharp corners,
      colback=definitionBackground,
      borderline west={3pt}{-1.5pt}{black},
      breakable
    ]
    \textbf{Definition. }\emph{#1}\\
  }
  {
    \end{tcolorbox}
  }

\newenvironment{amatrix}[2]{
    \left[
      \begin{array}{*{#1}{c}|*{#2}c}
  }
  {
      \end{array}
    \right]
  }

\definecolor{codeBackground}{RGB}{253,246,227}
\renewcommand\theFancyVerbLine{\arabic{FancyVerbLine}}
\newtcblisting{cpp}[1][]{
  boxrule=1pt,
  sharp corners,
  colback=codeBackground,
  listing only,
  breakable,
  minted language=cpp,
  minted style=material,
  minted options={
    xleftmargin=15pt,
    baselinestretch=1.2,
    linenos,
  }
}

\author{Kyle Chui}


\fancyhf{}
\lhead{Kyle Chui}
\rhead{Page \thepage}
\pagestyle{fancy}

\begin{document}
  \section{Lecture 17}
  \begin{theorem}{3.31 (Alternating Series Test)}
    Let $(x_n)$ be a sequence such that
    \begin{enumerate}[label=(\roman*)]
      \item $(x_n)$ is decreasing
      \item $\lim_{n\to \infty} x_n = 0$
    \end{enumerate}
    Then
    \[
      \sum_{n=1}^{\infty}(-1)^{n + 1}x_n
    \]
    converges.
    \begin{proof}
      Let
      \[
        S_n = \sum_{k=1}^{n} (-1)^{k + 1}x_k.
      \]
      Then we have
      \begin{align*}
        S_1 &= x_1 \\
        S_2 &= x_1 - x_2 \\
        S_3 &= x_1 - x_2 + x_3 \\
        S_4 &= x_1 - x_2 + x_3 - x_4.
      \end{align*}
      Since $(x_n)$ is decreasing, we have that $(S_{2n})$ is increasing (as $x_{2n - 1} - x_{2n - 2}$ is non-negative). Furthermore, we have $(S_{2n+1})$ is decreasing, as $x_{2n} - x_{2n - 1}$ is non-positive. Thus we have the inequality
      \[
        S_2\leq S_{2n}\leq S_{2n+1}\leq S_1.
      \]
      By the Monotone Convergence Theorem, we can say that both partial sum subsequences converge, say $S_{2n}\to a$ and $S_{2n+1}\to b$, for some $a,b\in\R$. By the Ordered Limit Theorem we know that $a\leq b$. \par
      It remains to show that $a = b$ and that any subsequence of $(S_n)$ converges to $a$. Notice that
      \[
        S_{2n + 1} - S_{2n} = (-1)^{2n+ 2}x_{2n + 1},
      \]
      and so by Algebraic Limit Theorem we have $b - a = 0$ (since $(x_n)$ converges to $0$). Thus $a = b$. Fixing $m\in\N$, we know there exists some $n\in\N$ such that
      \[
        S_{2n}\leq S_n\leq S_{2n+1}.
      \]
      By Squeeze Theorem, we have $S_n\to a$ as $n\to\infty$.
    \end{proof}
  \end{theorem}
  \begin{note}{}
    This theorem is quite helpful for helping us test for conditional convergence.
  \end{note}
  \begin{example}{}
    Consider the series given by $\frac{1}{n^p}$, where $p > 0$. Observe that this sequence is both decreasing and converges to zero. Hence
    \[
      \sum_{n=1}^{\infty} \frac{(-1)^{n+1}}{n^p}
    \]
    converges. Furthermore, even something that decays much slower like
    \[
      \sum_{n=1}^{\infty} \frac{(-1)^{n+1}}{\log\log\log\log(1 + n)}
    \]
    converges.
  \end{example}
  \begin{theorem}{3.32 (Ratio Test)}
    Suppose $x_n\neq 0$ for all $n\in\N$ and
    \[
      L\ceq \lim_{n\to \infty} \abs{\frac{x_{n+1}}{x_n}}.
    \]
    Then the series $\sum x_n$:
    \begin{enumerate}[label=(\roman*)]
      \item If $L < 1$, the series converges absolutely.
      \item If $L > 1$, the series diverges.
      \item If $L = 1$, the test is inconclusive.
    \end{enumerate}
    \begin{proof}
      \begin{enumerate}[label=(\roman*)]
        \item For all $\eps > 0$, there exists some $N$ such that for all $n > N$,
        \begin{align*}
          \abs{\abs{\frac{x_{n+1}}{x_n}} - L} &< \eps \\
          \abs{x_{n + 1}} &< (L + \eps)\abs{x_n}.
        \end{align*}
        If $L < 1$, we may choose $\eps > 0$ sufficiently small so that $L+\eps = L' < 1$. Thus we have
        \begin{align*}
          \abs{x_2} &< L'x_1 \\
          \abs{x_3} &< L'x_2 < (L')^2x_1 \\
          \abs{x_4} &< L'x_3 < (L')^3x_1.
        \end{align*}
        Iterating inductively, we see $\abs{x_{n + k}}\leq (L')^k\abs{x_n}$ for all $n > N$. We then may define
        \[
          \abs{x_n}\leq \begin{cases}
            \max(\abs{x_1}, \dotsb, \abs{x_N}) & \text{for } 1\leq n\leq N, \\
            (L')^{n - N} \abs{x_N} & \text{for }n > N.
          \end{cases}
        \]
        Summing this up, we get
        \[
          \sum_{n=1}^{\infty}\abs{x_n} = \underbrace{\sum_{n=1}^{N} \max(\abs{x_1}, \dotsb, \abs{x_N})}_{\text{finite}} + \underbrace{\sum_{n=N+1}^{\infty}(L')^{n - N}\abs{x_N}}_{\text{converges geometrically}}
        \]
        converges.
        \item We want to show that for all $\eps > 0$, there exists $N$ such that $\abs{x_{n + 1}} > (L - \eps)\abs{x_n}$ for all $n > N$. Since $L > 1$, we may choose some $\eps > 0$ small such that $L' = L - \eps > 1$. Iterating again, we get
        \[
          \abs{x_n} > (L')^{n - N} \abs{x_N}
        \]
        for all $n > N$. Since $L' > 1$, we have that the series diverges by Geometric Series.
      \end{enumerate}
    \end{proof}
  \end{theorem}
  \begin{note}{}
    The test fails when $L = 1$ because there's no way to choose an $\eps > 0$ such that $L' < 1$ or $L' > 1$, so we can't claim that the series converges/diverges by Geometric Series.
  \end{note}
  \begin{note}{}
    Alternatively, we may actually replace the $L$ in parts (i) and (ii) with $\limsup$ and $\liminf$, respectively. This makes a stronger statement since we know that $\limsup$ and $\liminf$ always exist.
  \end{note}
  \begin{theorem}{3.33 (Root Test)}
    Let $(x_n)$ be a sequence such that
    \[
      \lim_{n\to \infty} \abs{x_n}^{\frac{1}{n}} = L.
    \]
    Then the series $\sum x_n$:
    \begin{enumerate}[label=(\roman*)]
      \item If $L < 1$ then $\sum x_n$ converges absolutely.
      \item If $L > 1$ then $\sum x_n$ diverges.
      \item If $L = 1$ then the test is inconclusive.
    \end{enumerate}
    \begin{proof}
      Since we have that
      \[
        \lim_{n\to \infty} \abs{x_n}^{\frac{1}{n}} = L,
      \]
      we have that for all $\eps > 0$, there exists $N$ such that $\abs{\abs{x_n}^{\frac{1}{n}} - L} < \eps$ for all $n > N$. Thus we have
      \[
        L - \eps < \abs{x_n}^{\frac{1}{n}} < L + \eps.
      \]
      If $L < 1$, we may choose $\eps > 0$ such that $L' = L + \eps < 1$, so
      \begin{align*}
        \abs{x_n}^{\frac{1}{n}} &< L' \\
        \abs{x_n} < (L')^n.
      \end{align*}
      Thus by the comparison test with a geometric series we see that $\sum x_n$ converges absolutely. If $L > 1$, we may choose $\eps > 0$ such that $L' = L - \eps > 1$, so
      \begin{align*}
        \abs{x_n}^{\frac{1}{n}} &> L' \\
        \abs{x_n} &> (L')^n.
      \end{align*}
      Thus by the comparison test with a geometric series we see that $\sum x_n$ diverges.
    \end{proof}
  \end{theorem}
  \begin{note}{}
    In a fashion similar to the Ratio Test, if $L = 1$, we can't compare $\sum x_n$ to a geometric series, and so the test is inconclusive.
  \end{note}
  \begin{note}{}
    If we consider the Ratio and Root Tests in the more general setting (with $\limsup$ and $\liminf$ instead of $\lim$), then the Root Test is ``better'' than the Ratio Test. However, if we just look at regular limits, then the tests yield the \emph{same information}. From Ross:
    \[
      \liminf \abs{\frac{x_{n + 1}}{x_n}}\leq \limsup \abs{x_n}^{\frac{1}{n}}\leq \limsup \abs{\frac{x_{n + 1}}{x_n}}.
    \]
    For the above, we ``lose a little bit'' when we use the Ratio Test, and have fewer cases where we can be bounded by a geometric series.
  \end{note}
  \begin{example}{}
    An example where the Root Test works but the Ratio Test fails is for the series
    \[
      \sum_{n=1}^{\infty}2^{(-1)^n}2^{-n}.
    \]
  \end{example}
\end{document}
