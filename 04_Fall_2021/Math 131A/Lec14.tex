\documentclass[class=article, crop=false]{standalone}
\usepackage[margin=1in]{geometry}

\usepackage{amsmath}
\usepackage{amssymb}
\usepackage{amsthm}
\usepackage{comment}
\usepackage{enumitem}
\usepackage{fancyhdr}
\usepackage{hyperref}
\usepackage{mathrsfs}
\usepackage{mathtools}
\usepackage{standalone}
\usepackage{tcolorbox}
\usepackage{tikz}
\usepackage{xcolor}

\usetikzlibrary{decorations.pathreplacing}
\tcbuselibrary{skins}

\DeclareMathOperator{\lcm}{lcm}
\DeclareMathOperator{\proj}{proj}
\DeclareMathOperator{\vspan}{span}
\DeclareMathOperator{\im}{im}
\DeclareMathOperator{\range}{range}
\DeclareMathOperator{\Diff}{Diff}
\DeclareMathOperator{\Int}{Int}
\DeclareMathOperator{\fcn}{fcn}
\DeclareMathOperator{\id}{id}

\newcommand{\N}{\ensuremath{\mathbb{N}}}
\newcommand{\Z}{\ensuremath{\mathbb{Z}}}
\newcommand{\Q}{\ensuremath{\mathbb{Q}}}
\newcommand{\R}{\ensuremath{\mathbb{R}}}
\newcommand{\C}{\ensuremath{\mathbb{C}}}
\newcommand{\F}{\ensuremath{\mathbb{F}}}
\newcommand{\lam}{\ensuremath{\lambda}}
\newcommand{\nab}{\ensuremath{\nabla}}
\newcommand{\eps}{\ensuremath{\varepsilon}}
\newcommand{\es}{\ensuremath{\varnothing}}

\newcommand{\abs}[1]{\ensuremath{\left\lvert #1 \right\rvert}}
\newcommand{\bigpar}[1]{\ensuremath{\left( #1 \right)}}
\newcommand{\norm}[1]{\ensuremath{\lVert #1\rVert}}
\newcommand{\set}[1]{\ensuremath{\left\{#1\right\}}}
\newcommand{\tuple}[1]{\ensuremath{\left\langle #1 \right\rangle}}
\newcommand{\floor}[1]{\ensuremath{\left\lfloor #1 \right\rfloor}}
\newcommand{\ceil}[1]{\ensuremath{\left\lceil #1 \right\rceil}}

\newcommand{\chapternum}{}
\newcommand{\ex}[1]{\noindent\textbf{Exercise \chapternum.{#1}.}}

\newcommand{\dx}[1]{\,\mathrm{d}#1}
\newcommand{\inv}{\ensuremath{^{-1}}}
\newcommand{\sm}{\setminus}
\newcommand{\sse}{\subseteq}
\newcommand{\ceq}{\coloneqq}

\definecolor{problemBackground}{RGB}{212,232,246}
\newenvironment{problem}[1]
  {
    \begin{tcolorbox}[
      boxrule=.5pt,
      titlerule=.5pt,
      sharp corners,
      colback=problemBackground
    ]
    \ifx &#1& \textbf{Problem. }
    \else \textbf{Problem #1.} \fi
  }
  {
    \end{tcolorbox}
  }

\definecolor{exampleBackground}{RGB}{255,249,248}
\definecolor{exampleAccent}{RGB}{158,60,14}
\newenvironment{example}[1]
  {
    \begin{tcolorbox}[
      boxrule=.5pt,
      sharp corners,
      colback=exampleBackground,
      colframe=exampleAccent,
    ]
    \color{exampleAccent}\textbf{Example.} \emph{#1}\color{black}
  }
  {
    \end{tcolorbox}
  }

\definecolor{theoremBackground}{RGB}{234,243,251}
\definecolor{theoremAccent}{RGB}{0,116,183}
\newenvironment{theorem}[1]
  {
    \begin{tcolorbox}[
      boxrule=.5pt,
      titlerule=.5pt,
      sharp corners,
      colback=theoremBackground,
      colframe=theoremAccent
    ]
      \color{theoremAccent}\textbf{Theorem --- }\emph{#1}\\\color{black}
  }
  {
    \end{tcolorbox}
  }

\definecolor{noteBackground}{RGB}{244,249,244}
\definecolor{noteAccent}{RGB}{34,139,34}
\newenvironment{note}[1]
  {
  \begin{tcolorbox}[
    enhanced,
    boxrule=0pt,
    frame hidden,
    sharp corners,
    colback=noteBackground,
    borderline west={3pt}{-1.5pt}{noteAccent}
    ]
    \ifx &#1& \color{noteAccent}\textbf{Note. }\color{black}
    \else \color{noteAccent}\textbf{Note (#1). }\color{black} \fi
    }
    {
  \end{tcolorbox}
  }

\definecolor{definitionBackground}{RGB}{246,246,246}
\newenvironment{definition}[1]
  {
    \begin{tcolorbox}[
      enhanced,
      boxrule=0pt,
      frame hidden,
      sharp corners,
      colback=definitionBackground,
      borderline west={3pt}{-1.5pt}{black}
    ]
    \textbf{Definition. }\emph{#1}\\
  }
  {
    \end{tcolorbox}
  }
\newenvironment{amatrix}[2]{
    \left[
      \begin{array}{*{#1}{c}|*{#2}c}
  }
  {
      \end{array}
    \right]
  }
\date{\the\year-\the\month-\the\day}
\author{Kyle Chui}


\fancyhf{}
\lhead{Kyle Chui}
\rhead{Page \thepage}
\pagestyle{fancy}

\begin{document}
  \section{Lecture 14}
  \subsection{Diverging Limits and Multiplication}
  \textbf{Proposition 3.21} Let $x_n\to+\infty$ and $y_n\to y > 0$. Then
  \[
    \lim_{n\to \infty} (x_ny_n) = +\infty.
  \]
  The intuition via the Algebraic Limit Theorem tells us that this limit should approach ``$x\cdot y = +\infty\cdot y = +\infty$''.
  \begin{note}{}
    We can't always allow the case when $y = 0$, i.e. consider the case where $y_n = 0$, so 
    \[
      \lim_{n\to \infty} x_n\cdot 0 = \lim_{n\to \infty} 0 = 0.
    \]
  \end{note}
  \textbf{Proposition 3.22} Let $(x_n)$, $x_n\geq 0$. Then $x_n\to+\infty$ if and only if $\frac{1}{x_n}\to 0$.
  \begin{example}{}
    Consider sequences of the form $x_n = n^p$, where $x_n\geq 0$ and $p > 0$. Then $x_n\to+\infty$ and $\frac{1}{x_n} = \frac{1}{n^p}\to 0$.
  \end{example}
  \subsection{Diverging Limits and Addition}
  \begin{note}{}
    Limits diverging to infinity don't behave nearly as well when subjected to addition and subtraction, in comparison to multiplication and division. For example, if $x_n\to+\infty$ and $y_n\to\-\infty$, we can't say for sure what $x_n + y_n$ converges to, if anything at all. \par
  \end{note}
  \subsection{Notion of Infinity}
  We introduce symbols $+\infty$ and $-\infty$, and define the \emph{extended real numbers}:
  \[
    \overline{\R} = \R\cup \set{-\infty, +\infty}.
  \]
  We extend the notion of $\leq$ to $\overline\R$ as follows: For all $x\in\overline\R$, we have
  \[
    -\infty < x < +\infty.
  \]
  We aren't going to extend arithmetic (addition, subtraction, multiplication, division, etc.) to our new symbols. We also introduce the notation:
  \begin{align*}
    [a, +\infty) &= \set{x\in\R\mid a\leq x}. \\
    (-\infty, b) &= \set{x\in\R\mid x\leq b}. \\
    (-\infty, +\infty) &= \R.
  \end{align*}
  \begin{note}{}
    If $A\neq \es$ and $A\sse\R$ which is unbounded above, then $\sup A = +\infty$. We should always write $+\infty$ and $-\infty$, not just $\infty$.
  \end{note}
  \subsection{More on Sequences}
  Let $(x_n)$ be a sequence in $\R$. Define
  \[
    L = \set{x_0\in\overline\R\mid \text{There exists a subsequence }(x_{n_k})\text{ such that }x_{n_k}\to x_0}\sse\overline\R.
  \]
  We may think of the above by ``the set of all possible sub sequential limits of $(x_n)$''. We continue to study the three possible behaviors:
  \begin{itemize}
    \item Case 1: $x_n\to x$ (Converging) \par
    Then we know that $L = \set{x}\in\R$ because if the sequence converges to a point $x$, then all sub sequences must converge to the same $x$.
    \item Case 2: $x_n\to+\infty$. Then $L = \set{+\infty}$. \par
    In other words, any subsequence of a sequence diverging to $+\infty$ must also diverge to $+\infty$.
    \begin{proof}
      Suppose $(x_{n_k})$ is a subsequence of $(x_n)$, which diverges to $+\infty$. Thus for all $M > 0$, there exists some $N$ such that $x_n > M$ for all $n > N$. We know that $n_k\geq k$, and so fixing $M > 0$, there exists $N$ such that
      \[
        n_k\geq k > N,
      \]
      so $x_{n_k} > M$. Thus $x_{n_k}\to+\infty$ as $k\to+\infty$.
    \end{proof}
    A similar result can be shown for if $x_n\to-\infty$.
    \item Case 3: ``Oscillating Sequences'' \par
    Characterized by $L$ having more than one element in the extended reals.
    \begin{theorem}{3.22}
      A sequence of real numbers which does not converge in $\overline\R$ has at least two distinct limit points, i.e. there exists two subsequences converging to different things.
      \begin{proof}
        Case 1: $(x_n)$ is unbounded from above. \par
        We can extract a subsequence $(x_{n_k})$ such that $x_{n_k}\to+\infty$. Since $x_n$ is unbounded above, for all $M > 0$, there exists $n\in\N$ such that $x_n > M$.  We consider the subsequence
        \[
          \paren{x_n}_{n = n_1 + 1}^{\infty},
        \]
        which is also unbounded above. Let $M = 1$, so there exists $n_1\in\N$ such that $x_{n_1} > 1$.
        We then choose $M = 2$, and get $n_2 > n_1$ such that
        \[
          \paren{x_n}_{n = n_2 + 1}^{\infty},
        \]
        and continue this pattern to construct a subsequence $(x_{n_k})$ such that $x_{n_k}\geq k\to+\infty$, so $+\infty\in L$. \par
        Case 1.1: $(x_n)$ is also unbounded from below. Same arguments give you a subsequence $(x_{n_\ell})$ such that $x_{n_\ell}\to-\infty$, and so $-\infty\in L$. \par
        Case 1.2: $(x_n)$ is bounded from below. \par
        Since $(x_n)$ does not converge in the extended real numbers, it does \emph{not} diverge to $+\infty$. Then there exists $M > 0$ for all $N\in\N$, such that $x_n\leq M$ for some $n > N$. Thus we may extract a bounded subsequence. If $N = 1$, then there exists $n_1 > 1$ such that $x_{n_1} \leq M$. If $N = n_1$, there exists $n_2 > n_1$ such that $x_{n_2}\leq M$. We continue this process until we have constructed our subsequence, i.e.
        \[
          x_{n_k}\leq M\quad \text{for all }k\in \N.
        \]
        Bolzano--Weierstrass tells us that there exists a further subsequence $(x_{n_{k_\ell}})$ such that
        \[
          x_{n_{k_\ell}}\to x\in\R.
        \]
        We know that $(x_{n_{k_\ell}})$ is a subsequence of our subsequence $(x_{n_k})$, and so it must be a subsequence of our original sequence $(x_n)$.
        Case 3: $(x_n)$ is bounded. From the homework we know that
        \[
          \liminf_{n\to\infty} x_n, \limsup_{x\to\infty}x_n\in L.
        \]
        Since $(x_n)$ does not converge in $\overline\R$, we have that the limits are distinct and so $L$ has more than $1$ element.
      \end{proof}
    \end{theorem}
  \end{itemize}
\end{document}
