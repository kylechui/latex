\documentclass[class=article, crop=false]{standalone}
\usepackage[margin=1in]{geometry}

\usepackage{amsmath}
\usepackage{amssymb}
\usepackage{amsthm}
\usepackage{comment}
\usepackage{enumitem}
\usepackage{fancyhdr}
\usepackage{hyperref}
\usepackage{mathrsfs}
\usepackage{mathtools}
\usepackage{standalone}
\usepackage{tcolorbox}
\usepackage{tikz}
\usepackage{xcolor}

\usetikzlibrary{decorations.pathreplacing}
\tcbuselibrary{skins}

\DeclareMathOperator{\lcm}{lcm}
\DeclareMathOperator{\proj}{proj}
\DeclareMathOperator{\vspan}{span}
\DeclareMathOperator{\im}{im}
\DeclareMathOperator{\range}{range}
\DeclareMathOperator{\Diff}{Diff}
\DeclareMathOperator{\Int}{Int}
\DeclareMathOperator{\fcn}{fcn}
\DeclareMathOperator{\id}{id}

\newcommand{\N}{\ensuremath{\mathbb{N}}}
\newcommand{\Z}{\ensuremath{\mathbb{Z}}}
\newcommand{\Q}{\ensuremath{\mathbb{Q}}}
\newcommand{\R}{\ensuremath{\mathbb{R}}}
\newcommand{\C}{\ensuremath{\mathbb{C}}}
\newcommand{\F}{\ensuremath{\mathbb{F}}}
\newcommand{\lam}{\ensuremath{\lambda}}
\newcommand{\nab}{\ensuremath{\nabla}}
\newcommand{\eps}{\ensuremath{\varepsilon}}
\newcommand{\es}{\ensuremath{\varnothing}}

\newcommand{\abs}[1]{\ensuremath{\left\lvert #1 \right\rvert}}
\newcommand{\bigpar}[1]{\ensuremath{\left( #1 \right)}}
\newcommand{\norm}[1]{\ensuremath{\lVert #1\rVert}}
\newcommand{\set}[1]{\ensuremath{\left\{#1\right\}}}
\newcommand{\tuple}[1]{\ensuremath{\left\langle #1 \right\rangle}}
\newcommand{\floor}[1]{\ensuremath{\left\lfloor #1 \right\rfloor}}
\newcommand{\ceil}[1]{\ensuremath{\left\lceil #1 \right\rceil}}

\newcommand{\chapternum}{}
\newcommand{\ex}[1]{\noindent\textbf{Exercise \chapternum.{#1}.}}

\newcommand{\dx}[1]{\,\mathrm{d}#1}
\newcommand{\inv}{\ensuremath{^{-1}}}
\newcommand{\sm}{\setminus}
\newcommand{\sse}{\subseteq}
\newcommand{\ceq}{\coloneqq}

\definecolor{problemBackground}{RGB}{212,232,246}
\newenvironment{problem}[1]
  {
    \begin{tcolorbox}[
      boxrule=.5pt,
      titlerule=.5pt,
      sharp corners,
      colback=problemBackground
    ]
    \ifx &#1& \textbf{Problem. }
    \else \textbf{Problem #1.} \fi
  }
  {
    \end{tcolorbox}
  }

\definecolor{exampleBackground}{RGB}{255,249,248}
\definecolor{exampleAccent}{RGB}{158,60,14}
\newenvironment{example}[1]
  {
    \begin{tcolorbox}[
      boxrule=.5pt,
      sharp corners,
      colback=exampleBackground,
      colframe=exampleAccent,
    ]
    \color{exampleAccent}\textbf{Example.} \emph{#1}\color{black}
  }
  {
    \end{tcolorbox}
  }

\definecolor{theoremBackground}{RGB}{234,243,251}
\definecolor{theoremAccent}{RGB}{0,116,183}
\newenvironment{theorem}[1]
  {
    \begin{tcolorbox}[
      boxrule=.5pt,
      titlerule=.5pt,
      sharp corners,
      colback=theoremBackground,
      colframe=theoremAccent
    ]
      \color{theoremAccent}\textbf{Theorem --- }\emph{#1}\\\color{black}
  }
  {
    \end{tcolorbox}
  }

\definecolor{noteBackground}{RGB}{244,249,244}
\definecolor{noteAccent}{RGB}{34,139,34}
\newenvironment{note}[1]
  {
  \begin{tcolorbox}[
    enhanced,
    boxrule=0pt,
    frame hidden,
    sharp corners,
    colback=noteBackground,
    borderline west={3pt}{-1.5pt}{noteAccent}
    ]
    \ifx &#1& \color{noteAccent}\textbf{Note. }\color{black}
    \else \color{noteAccent}\textbf{Note (#1). }\color{black} \fi
    }
    {
  \end{tcolorbox}
  }

\definecolor{definitionBackground}{RGB}{246,246,246}
\newenvironment{definition}[1]
  {
    \begin{tcolorbox}[
      enhanced,
      boxrule=0pt,
      frame hidden,
      sharp corners,
      colback=definitionBackground,
      borderline west={3pt}{-1.5pt}{black}
    ]
    \textbf{Definition. }\emph{#1}\\
  }
  {
    \end{tcolorbox}
  }
\newenvironment{amatrix}[2]{
    \left[
      \begin{array}{*{#1}{c}|*{#2}c}
  }
  {
      \end{array}
    \right]
  }
\date{\the\year-\the\month-\the\day}
\author{Kyle Chui}


\fancyhf{}
\lhead{Kyle Chui}
\rhead{Page \thepage}
\pagestyle{fancy}

\begin{document}
  \section{Lecture 2}
  \subsection{Continuation of Logic}
  \subsubsection{De Morgan's Laws}
  \begin{align*}
    \neg(P\lor Q) &= \neg P \land \neg Q \\
    \neg(P\land Q) &= \neg P \lor \neg Q
  \end{align*}
  \begin{note}{}
    Negations turn ``and'' into ``or'' and vice versa.
  \end{note}
  \begin{example}{}
    Suppose we have the following statement:
    \begin{center}
      $P$: $x$ is even and $x > 0$.
    \end{center}
    Then the negation of $P$ would be:
    \begin{center}
      $\neg P$: $x$ is odd or $x\leq 0$.
    \end{center}
  \end{example}
  \subsubsection{Converse}
  \begin{definition}{Converse}
    The \emph{converse} of a statement $P\implies Q$ is the statement $Q\implies P$. In general, the converse of a statement says nothing about the original statement.
  \end{definition}
  \begin{example}{}
    Consider the statement
    \begin{center}
      If $x > 0$, then $x^3\neq 0$.
    \end{center}
    The converse is then
    \begin{center}
      If $x^3 \neq 0$, then $x > 0$.
    \end{center}
    Note that the converse is false even though the original statement is true.
  \end{example}
  If we know both the original statement and the converse are true, then we may write $P\iff Q$ instead of $(P\implies Q)\land (Q\implies P)$. In this case, we call $P$ and $Q$ \emph{logically equivalent}. In writing, we say ``$P$ if and only if $Q$''.
  \subsubsection{Proof by Contrapositive}
  We can show that
  \[
    (\neg Q)\implies (\neg P) \iff P\implies Q.
  \]
  This gives us another way of proving the original statement, since the contrapositive has the same truth values.
  \begin{lemma}{1}
    Let $a$ be an integer. If $a^2$ is even, then $a$ is even.
    \begin{proof}
      Suppose $a$ is odd, so $a = 2k + 1$ for some integer $k$. Then
      \begin{align*}
        a^2 &= (2k + 1)^2 \\
            &= 2(2k^2 + 2k) + 1.
      \end{align*}
      Thus $a^2$ is odd and this completes the proof.
    \end{proof}
  \end{lemma}
  \subsubsection{Variables and Quantifiers}
  We have a value $x$ that varies over some values, so we use $P(x)$ to denote a statement that depends on the value of $x$.
  \begin{example}{}
    Consider the statement
    \[
      P(x): x + 2 = 3.
    \]
    The statement is true if and only if $x = 1$.
  \end{example}
  We have two quantifiers---$\forall =$ ``for all'', and $\exists =$ ``there exists''.
  \begin{itemize}
    \item 
      $\forall x : P(x)$ is true if $P(x)$ is true for all $x$.
    \item 
      $\exists x : P(x)$ is true if there exists at least one $x$ such that $P(x)$ is true.
  \end{itemize}
  We can also have nested quantifiers,
  \[
    \forall m\in\N, \exists n\in\N : m < n.
  \]
  \begin{note}{}
    The order of quantifiers matters. In the above example, the quantifiers tell us that our choice of $n$ depends on $m$.
  \end{note}
  \subsubsection{Proof by Counterexample}
  After ``simplifying'' the statement $\neg(\forall x : P(x))$, we get $\exists x : \neg P(x)$. We simply need to find a single counterexample to show that a statement is false for all $x$.
  \begin{example}{}
    Consider the statement $\forall x\in \R : x + 2 = 3$. All we need to do is show that there exists some $x\in \R$ such that $x + 2\neq 3$. This occurs when $x = 0$, so the statement is false.
  \end{example}
  \subsubsection{Proof by Contradiction}
  \textbf{Key Idea.} We want to show that $P\implies Q$ indirectly.
  \begin{lemma}{2}
    We can show that
    \[
      P\implies Q = (\neg P)\lor Q.
    \]
    Then $P\implies Q$ is true if and only if $\neg(P\implies Q)$ is false, and so by Lemma $2$  and De Morgan's Laws, $P\land \neg Q$ is false.
  \end{lemma}
  For proof by contradiction, we assume $P$ is true and $\neg Q$ is true, and try to show that $P\land \neg Q$ is false (a contradiction).
\end{document}
