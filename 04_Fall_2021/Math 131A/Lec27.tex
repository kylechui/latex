\documentclass[class=article, crop=false]{standalone}
% Import packages
\usepackage[margin=1in]{geometry}

\usepackage{amssymb, amsthm}
\usepackage{comment}
\usepackage{enumitem}
\usepackage{fancyhdr}
\usepackage{hyperref}
\usepackage{import}
\usepackage{mathrsfs, mathtools}
\usepackage{multicol}
\usepackage{pdfpages}
\usepackage{standalone}
\usepackage{transparent}
\usepackage{xcolor}

\usepackage{minted}
\usepackage[many]{tcolorbox}
\tcbuselibrary{minted}

% Declare math operators
\DeclareMathOperator{\lcm}{lcm}
\DeclareMathOperator{\proj}{proj}
\DeclareMathOperator{\vspan}{span}
\DeclareMathOperator{\im}{im}
\DeclareMathOperator{\Diff}{Diff}
\DeclareMathOperator{\Int}{Int}
\DeclareMathOperator{\fcn}{fcn}
\DeclareMathOperator{\id}{id}
\DeclareMathOperator{\rank}{rank}
\DeclareMathOperator{\tr}{tr}
\DeclareMathOperator{\dive}{div}
\DeclareMathOperator{\row}{row}
\DeclareMathOperator{\col}{col}
\DeclareMathOperator{\dom}{dom}
\DeclareMathOperator{\ran}{ran}
\DeclareMathOperator{\Dom}{Dom}
\DeclareMathOperator{\Ran}{Ran}
\DeclareMathOperator{\fl}{fl}
% Macros for letters/variables
\newcommand\ttilde{\kern -.15em\lower .7ex\hbox{\~{}}\kern .04em}
\newcommand{\N}{\ensuremath{\mathbb{N}}}
\newcommand{\Z}{\ensuremath{\mathbb{Z}}}
\newcommand{\Q}{\ensuremath{\mathbb{Q}}}
\newcommand{\R}{\ensuremath{\mathbb{R}}}
\newcommand{\C}{\ensuremath{\mathbb{C}}}
\newcommand{\F}{\ensuremath{\mathbb{F}}}
\newcommand{\M}{\ensuremath{\mathbb{M}}}
\renewcommand{\P}{\ensuremath{\mathbb{P}}}
\newcommand{\lam}{\ensuremath{\lambda}}
\newcommand{\nab}{\ensuremath{\nabla}}
\newcommand{\eps}{\ensuremath{\varepsilon}}
\newcommand{\es}{\ensuremath{\varnothing}}
% Macros for math symbols
\newcommand{\dx}[1]{\,\mathrm{d}#1}
\newcommand{\inv}{\ensuremath{^{-1}}}
\newcommand{\sm}{\setminus}
\newcommand{\sse}{\subseteq}
\newcommand{\ceq}{\coloneqq}
% Macros for pairs of math symbols
\newcommand{\abs}[1]{\ensuremath{\left\lvert #1 \right\rvert}}
\newcommand{\paren}[1]{\ensuremath{\left( #1 \right)}}
\newcommand{\norm}[1]{\ensuremath{\left\lVert #1\right\rVert}}
\newcommand{\set}[1]{\ensuremath{\left\{#1\right\}}}
\newcommand{\tup}[1]{\ensuremath{\left\langle #1 \right\rangle}}
\newcommand{\floor}[1]{\ensuremath{\left\lfloor #1 \right\rfloor}}
\newcommand{\ceil}[1]{\ensuremath{\left\lceil #1 \right\rceil}}
\newcommand{\eclass}[1]{\ensuremath{\left[ #1 \right]}}

\newcommand{\chapternum}{}
\newcommand{\ex}[1]{\noindent\textbf{Exercise \chapternum.{#1}.}}

\newcommand{\tsub}[1]{\textsubscript{#1}}
\newcommand{\tsup}[1]{\textsuperscript{#1}}

\renewcommand{\choose}[2]{\ensuremath{\phantom{}_{#1}C_{#2}}}
\newcommand{\perm}[2]{\ensuremath{\phantom{}_{#1}P_{#2}}}

% Include figures
\newcommand{\incfig}[2][1]{%
    \def\svgwidth{#1\columnwidth}
    \import{./figures/}{#2.pdf_tex}
}

\definecolor{problemBackground}{RGB}{212,232,246}
\newenvironment{problem}[1]
  {
    \begin{tcolorbox}[
      boxrule=.5pt,
      titlerule=.5pt,
      sharp corners,
      colback=problemBackground,
      breakable
    ]
    \ifx &#1& \textbf{Problem. }
    \else \textbf{Problem #1.} \fi
  }
  {
    \end{tcolorbox}
  }
\definecolor{exampleBackground}{RGB}{255,249,248}
\definecolor{exampleAccent}{RGB}{158,60,14}
\newenvironment{example}[1]
  {
    \begin{tcolorbox}[
      boxrule=.5pt,
      sharp corners,
      colback=exampleBackground,
      colframe=exampleAccent,
    ]
    \color{exampleAccent}\textbf{Example.} \emph{#1}\color{black}
  }
  {
    \end{tcolorbox}
  }
\definecolor{theoremBackground}{RGB}{234,243,251}
\definecolor{theoremAccent}{RGB}{0,116,183}
\newenvironment{theorem}[1]
  {
    \begin{tcolorbox}[
      boxrule=.5pt,
      titlerule=.5pt,
      sharp corners,
      colback=theoremBackground,
      colframe=theoremAccent,
      breakable
    ]
      % \color{theoremAccent}\textbf{Theorem --- }\emph{#1}\\\color{black}
    \ifx &#1& \color{theoremAccent}\textbf{Theorem. }\color{black}
    \else \color{theoremAccent}\textbf{Theorem --- }\emph{#1}\\\color{black} \fi
  }
  {
    \end{tcolorbox}
  }
\definecolor{noteBackground}{RGB}{244,249,244}
\definecolor{noteAccent}{RGB}{34,139,34}
\newenvironment{note}[1]
  {
  \begin{tcolorbox}[
    enhanced,
    boxrule=0pt,
    frame hidden,
    sharp corners,
    colback=noteBackground,
    borderline west={3pt}{-1.5pt}{noteAccent},
    breakable
    ]
    \ifx &#1& \color{noteAccent}\textbf{Note. }\color{black}
    \else \color{noteAccent}\textbf{Note (#1). }\color{black} \fi
    }
    {
  \end{tcolorbox}
  }
\definecolor{lemmaBackground}{RGB}{255,247,234}
\definecolor{lemmaAccent}{RGB}{255,153,0}
\newenvironment{lemma}[1]
  {
    \begin{tcolorbox}[
      enhanced,
      boxrule=0pt,
      frame hidden,
      sharp corners,
      colback=lemmaBackground,
      borderline west={3pt}{-1.5pt}{lemmaAccent},
      breakable
    ]
    \ifx &#1& \color{lemmaAccent}\textbf{Lemma. }\color{black}
    \else \color{lemmaAccent}\textbf{Lemma #1. }\color{black} \fi
  }
  {
    \end{tcolorbox}
  }
\definecolor{definitionBackground}{RGB}{246,246,246}
\newenvironment{definition}[1]
  {
    \begin{tcolorbox}[
      enhanced,
      boxrule=0pt,
      frame hidden,
      sharp corners,
      colback=definitionBackground,
      borderline west={3pt}{-1.5pt}{black},
      breakable
    ]
    \textbf{Definition. }\emph{#1}\\
  }
  {
    \end{tcolorbox}
  }

\newenvironment{amatrix}[2]{
    \left[
      \begin{array}{*{#1}{c}|*{#2}c}
  }
  {
      \end{array}
    \right]
  }

\definecolor{codeBackground}{RGB}{253,246,227}
\renewcommand\theFancyVerbLine{\arabic{FancyVerbLine}}
\newtcblisting{cpp}[1][]{
  boxrule=1pt,
  sharp corners,
  colback=codeBackground,
  listing only,
  breakable,
  minted language=cpp,
  minted style=material,
  minted options={
    xleftmargin=15pt,
    baselinestretch=1.2,
    linenos,
  }
}

\author{Kyle Chui}


\fancyhf{}
\lhead{Kyle Chui}
\rhead{Page \thepage}
\pagestyle{fancy}

\begin{document}
  \section{Lecture 27}
  \textbf{Question} Which functions can we integrate? \par
  \textbf{Lemma 6.9} If $f\colon [a, b]\to \R$ is Riemann, then $f$ is bounded on $[a, b]$.
  \begin{proof}
    We proceed via proof by contrapositive. Suppose $f$ is not bounded on $[a, b]$, so
    \[
      \sup_{x\in [a, b]}\abs{f(x)} = +\infty.
    \]
    Hence $f$ is either unbounded from above or from below. We wish to show that $f$ is not Riemann integrable. Observe that since $f$ is unbounded (say, from above), for any partition $\mathcal{P} = \set{t_j}_{j = 1}^n$ there exists some $j\in [1, n]$ such that
    \[
      \sup_{x\in [t_{j - 1}, t_j]}f(x) = +\infty.
    \]
    Thus $\mathcal{U}(f, \mathcal{P}) = +\infty$, and since $\mathcal{P}$ is arbitrary we have
    \[
      \overline{\int_{a}^{b}f(x) \,\mathrm dx} = +\infty.
    \]
    Thus $f$ is not Riemann integrable. Using similar logic, we may show that $f$ is not integrable when $f$ is unbounded from below.
  \end{proof}
  \begin{note}{}
    We have that
    \[
      \set{f\colon [a, b]\to \R\mid f\text{ is Riemann integrable}}\subset \set{f\colon [a, b]\to \R\mid f\text{ is bounded}}.
    \]
  \end{note}
  \begin{note}{}
    Not every bounded function is integrable! Consider Dirichlet's function over the interval $[0, 1]$:
    \[
      f(x) = \begin{cases}1 & x\in \Q, \\ 0 & x\notin \Q.\end{cases}
    \]
    It is clearly true that $f$ is bounded. Notice that by the density of the rationals and irrationals, we have
    \begin{align*}
      \mathcal{U}(f, \mathcal{P}) &= \sum_{i=1}^{n}\sup_{x\in [t_{j - 1}, t_j]}f(x)\cdot (t_j - t_{j - 1}) & \mathcal{L}(f, \mathcal{P}) &= \sum_{i=1}^{n}\inf_{x\in [t_{j - 1}, t_j]}f(x)\cdot (t_j - t_{j - 1}) \\
                                  &= \sum_{i=1}^{n} (t_j - t_{j - 1}) & &= \sum_{i=1}^{n} 0 \\
                                  &= t_n - t_0 & &= 0. \\
                                  &= 1.
    \end{align*}
    Since this holds for any partition $\mathcal{P}$, we have that
    \[
      \overline{\int_{0}^{1}f(x) \,\mathrm dx} = 1\neq 0 = \underline{\int_{0}^{1}f(x) \,\mathrm dx},
    \]
    so the $f$ is non-integrable over $[0, 1]$.
  \end{note}
  \textbf{Lemma 6.9} Suppose $f\colon [a, b]\to \R$ is \emph{bounded}, i.e. there exists $M > 0$ such that $\sup_{x\in [a, b]}\abs{f(x)}\leq M$. Then for every partition $\mathcal{P}$ of $[a, b]$, 
  \[
    -M(b - a)\leq \mathcal{L}(f, \mathcal{P})\leq \mathcal{U}(f, \mathcal{P})\leq M(b - a).
  \]
  Furthermore, if $f$ is Riemann integrable, then
  \[
    \abs{\int_{a}^{b}f(x) \,\mathrm dx}\leq M(b - a).
  \]
  \begin{proof}
    Since $\mathcal{P}$ is a partition of $[a, b]$, we have
    \[
      \sum_{j=1}^{n}(t_j - t_{j - 1}) = t_n - t_0 = b - a.
    \]
    Furthermore, we have
    \begin{align*}
      -M &\leq \inf_{x\in [a, b]} f(x) \\
         &\leq \inf_{x\in [t_{j - 1}, t_j]}f(x) \\
         &\leq \sup{x\in [t_{j - 1}, t_j]}f(x) \\
         &\leq \sup_{x\in [a, b]} f(x) \\
         &= M.
    \end{align*}
    Combining these two statements yields the desired result.
  \end{proof}
  \begin{note}{}
    If $f$ is Riemann integrable, then $\abs{f}$ is as well and
    \[
      \abs{\int_{a}^{b}f(x) \,\mathrm dx} \leq \int_{a}^{b}\abs{f(x)} \,\mathrm dx.
    \]
    Furthermore, if $f, g$ are Riemann integrable over $[a, b]$ where $f\leq g$, then
    \[
      \int_{a}^{b}f(x) \,\mathrm dx\leq \int_{a}^{b}g(x) \,\mathrm dx.
    \]
    Thus if $f$ is Riemann integrable, we have $\sup_{x\in [a, b]}\abs{f(x)}\leq M$ and
    \begin{align*}
      \abs{\int_{a}^{b}f(x) \,\mathrm dx} &\leq \int_{a}^{b}\abs{f(x)} \,\mathrm dx \\
                                          &\leq \int_{a}^{b}M \,\mathrm dx \\
                                          &= M(b - a).
    \end{align*}
  \end{note}
  \textbf{Proposition 6.10} Let $f\colon [a, b]\to \R$ be \emph{bounded}. Then $f$ is Riemann integrable on $[a, b]$ if and only if for any $\eps > 0$, there exists a partition $\mathcal{P}_\eps$ of $[a, b]$ such that
  \[
    0\leq \mathcal{U}(f, \mathcal{P}_\eps) - \mathcal{L}(f, \mathcal{P}_\eps)\leq \eps.
  \]
  \begin{proof}
    ($\Rightarrow$) Let $\eps > 0$, and suppose $f$ is Riemann integrable on $[a, b]$ so
    \[
      \int_{a}^{b}f(x) \,\mathrm dx = \overline{\int_{a}^{b}f(x) \,\mathrm dx} = \underline{\int_{a}^{b}f(x) \,\mathrm dx}\in\R.
    \]
    Since $\overline{\int_{a}^{b}f(x) \,\mathrm dx}$ is an infimum, there exists some $\mathcal{P}_\eps'$ such that
    \[
      \mathcal{U}(f, \mathcal{P}_\eps') < \overline{\int_{a}^{b}f(x) \,\mathrm dx} + \frac{\eps}{2}.
    \]
    Since $\underline{\int_{a}^{b}f(x) \,\mathrm dx}$ is a supremum, there exists $\mathcal{P}_\eps''$ such that
    \[
      \mathcal{L}(f, \mathcal{P}_\eps'') > \underline{\int_{a}^{b}f(x) \,\mathrm dx} - \frac{\eps}{2}.
    \]
    Taking the common refinement $\mathcal{P}_\eps = \mathcal{P}_\eps' \cup \mathcal{P}_\eps''$, we have
    \[
      \mathcal{U}(f, \mathcal{P}_\eps)\leq \mathcal{U}(f, \mathcal{P}_\eps'),
    \]
    and
    \[
      \mathcal{L}(f, \mathcal{P}_\eps')\leq \mathcal{L}(f, \mathcal{P}_\eps).
    \]
    Hence we have that
    \begin{align*}
      \mathcal{U}(f, \mathcal{P}_\eps) - \mathcal{L}(f, \mathcal{P}_\eps) &< \paren{\overline{\int_{a}^{b}f(x) \,\mathrm dx} + \frac{\eps}{2}} - \paren{\underline{\int_{a}^{b}f(x) \,\mathrm dx} - \frac{\eps}{2}} \\
                                                                          &= \frac{\eps}{2} + \frac{\eps}{2} \tag{$f$ is Riemann integrable}\\
                                                                          &= \eps.
    \end{align*}
    ($\Leftarrow$) We proceed via proof by contrapositive. Suppose that $f$ is \emph{not} Riemann integrable, so
    \[
      \overline{\int_{a}^{b}f(x) \,\mathrm dx} - \underline{\int_{a}^{b}f(x) \,\mathrm dx} > 0.
    \]
    Let us pick
    \[
      \eps = \frac{1}{2}\paren{\overline{\int_{a}^{b}f(x) \,\mathrm dx} - \underline{\int_{a}^{b}f(x) \,\mathrm dx}}.
    \]
    Then we can never have a partition $\mathcal{P}_\eps$ such that
    \[
      0\leq \mathcal{U}(f, \mathcal{P}_\eps) - \mathcal{L}(f, \mathcal{P}_\eps) < \eps.
    \]
  \end{proof}
  \textbf{Proposition 6.11} Every continuous function $f\colon [a, b]\to \R$ is Riemann integrable on $[a, b]$.
  \begin{proof}
    We will use Proposition $6.10$ to prove this statement. Let $\eps > 0$. Since $f$ is continuous on compact $[a, b]$, we have that $f$ is bounded on $[a, b]$. Moreover, $f$ is \emph{uniformly} continuous on $[a, b]$. Hence there exists $\delta > 0$ such that
    \[
      \sup_{\substack{x, y\in [a, b]\\ \abs{x - y}<\delta}} \abs{f(x) - f(y)} < \frac{\eps}{b - a}.
    \]
    Let $\mathcal{P}$ be a partition of $[a, b]$ such that $t_j - t_{j - 1} < \delta$ for all $j = 1,\dotsc,n$. Then on $I_j = [t_{j - 1}, t_j]$, the Extreme Value Theorem implies that $f$ has a maximum and minimum (say at $x_j^*$ and $y_j^*$). Hence
    \[
      \sup_{x\in I_j}f(x) - \inf_{x\in I_j} = f(x_j^*) - f(y_j^*) < \frac{\eps}{b - a}.
    \]
    Therefore
    \begin{align*}
      \mathcal{U}(f, \mathcal{P}) &= \sum_{j=1}^{n} \paren{\sup_{I_j} f - \inf_{I_j}f}\cdot (t_j - t_{j - 1}) \\
                                  &< \frac{\eps}{b - a}\cdot \sum_{j=1}^{n}(t_j - t_{j - 1}) \\
                                  &= \eps.
    \end{align*}
    Since $\eps > 0$ and arbitrary, Proposition $6.10$ implies that $f$ is Riemann integrable over $[a, b]$.
  \end{proof}
  \begin{note}{}
    We can also integrate functions that are continuous, but not ``too discontinuous'', i.e. any function with a finite number of:
    \begin{enumerate}[label=(\alph*)]
      \item Removable discontinuities: there exists $\displaystyle \lim_{x\to c} f(x)\neq f(c)$.
      \item Jump discontinuities: $\displaystyle \lim_{x\to c^-} f(x)\neq \lim_{x\to c^+} f(x)$.
    \end{enumerate}
  \end{note}
\end{document}
