\documentclass[class=article, crop=false]{standalone}
\usepackage[margin=1in]{geometry}

\usepackage{amsmath}
\usepackage{amssymb}
\usepackage{amsthm}
\usepackage{comment}
\usepackage{enumitem}
\usepackage{fancyhdr}
\usepackage{hyperref}
\usepackage{mathrsfs}
\usepackage{mathtools}
\usepackage{standalone}
\usepackage{tcolorbox}
\usepackage{tikz}
\usepackage{xcolor}

\usetikzlibrary{decorations.pathreplacing}
\tcbuselibrary{skins}

\DeclareMathOperator{\lcm}{lcm}
\DeclareMathOperator{\proj}{proj}
\DeclareMathOperator{\vspan}{span}
\DeclareMathOperator{\im}{im}
\DeclareMathOperator{\range}{range}
\DeclareMathOperator{\Diff}{Diff}
\DeclareMathOperator{\Int}{Int}
\DeclareMathOperator{\fcn}{fcn}
\DeclareMathOperator{\id}{id}

\newcommand{\N}{\ensuremath{\mathbb{N}}}
\newcommand{\Z}{\ensuremath{\mathbb{Z}}}
\newcommand{\Q}{\ensuremath{\mathbb{Q}}}
\newcommand{\R}{\ensuremath{\mathbb{R}}}
\newcommand{\C}{\ensuremath{\mathbb{C}}}
\newcommand{\F}{\ensuremath{\mathbb{F}}}
\newcommand{\lam}{\ensuremath{\lambda}}
\newcommand{\nab}{\ensuremath{\nabla}}
\newcommand{\eps}{\ensuremath{\varepsilon}}
\newcommand{\es}{\ensuremath{\varnothing}}

\newcommand{\abs}[1]{\ensuremath{\left\lvert #1 \right\rvert}}
\newcommand{\bigpar}[1]{\ensuremath{\left( #1 \right)}}
\newcommand{\norm}[1]{\ensuremath{\lVert #1\rVert}}
\newcommand{\set}[1]{\ensuremath{\left\{#1\right\}}}
\newcommand{\tuple}[1]{\ensuremath{\left\langle #1 \right\rangle}}
\newcommand{\floor}[1]{\ensuremath{\left\lfloor #1 \right\rfloor}}
\newcommand{\ceil}[1]{\ensuremath{\left\lceil #1 \right\rceil}}

\newcommand{\chapternum}{}
\newcommand{\ex}[1]{\noindent\textbf{Exercise \chapternum.{#1}.}}

\newcommand{\dx}[1]{\,\mathrm{d}#1}
\newcommand{\inv}{\ensuremath{^{-1}}}
\newcommand{\sm}{\setminus}
\newcommand{\sse}{\subseteq}
\newcommand{\ceq}{\coloneqq}

\definecolor{problemBackground}{RGB}{212,232,246}
\newenvironment{problem}[1]
  {
    \begin{tcolorbox}[
      boxrule=.5pt,
      titlerule=.5pt,
      sharp corners,
      colback=problemBackground
    ]
    \ifx &#1& \textbf{Problem. }
    \else \textbf{Problem #1.} \fi
  }
  {
    \end{tcolorbox}
  }

\definecolor{exampleBackground}{RGB}{255,249,248}
\definecolor{exampleAccent}{RGB}{158,60,14}
\newenvironment{example}[1]
  {
    \begin{tcolorbox}[
      boxrule=.5pt,
      sharp corners,
      colback=exampleBackground,
      colframe=exampleAccent,
    ]
    \color{exampleAccent}\textbf{Example.} \emph{#1}\color{black}
  }
  {
    \end{tcolorbox}
  }

\definecolor{theoremBackground}{RGB}{234,243,251}
\definecolor{theoremAccent}{RGB}{0,116,183}
\newenvironment{theorem}[1]
  {
    \begin{tcolorbox}[
      boxrule=.5pt,
      titlerule=.5pt,
      sharp corners,
      colback=theoremBackground,
      colframe=theoremAccent
    ]
      \color{theoremAccent}\textbf{Theorem --- }\emph{#1}\\\color{black}
  }
  {
    \end{tcolorbox}
  }

\definecolor{noteBackground}{RGB}{244,249,244}
\definecolor{noteAccent}{RGB}{34,139,34}
\newenvironment{note}[1]
  {
  \begin{tcolorbox}[
    enhanced,
    boxrule=0pt,
    frame hidden,
    sharp corners,
    colback=noteBackground,
    borderline west={3pt}{-1.5pt}{noteAccent}
    ]
    \ifx &#1& \color{noteAccent}\textbf{Note. }\color{black}
    \else \color{noteAccent}\textbf{Note (#1). }\color{black} \fi
    }
    {
  \end{tcolorbox}
  }

\definecolor{definitionBackground}{RGB}{246,246,246}
\newenvironment{definition}[1]
  {
    \begin{tcolorbox}[
      enhanced,
      boxrule=0pt,
      frame hidden,
      sharp corners,
      colback=definitionBackground,
      borderline west={3pt}{-1.5pt}{black}
    ]
    \textbf{Definition. }\emph{#1}\\
  }
  {
    \end{tcolorbox}
  }
\newenvironment{amatrix}[2]{
    \left[
      \begin{array}{*{#1}{c}|*{#2}c}
  }
  {
      \end{array}
    \right]
  }
\date{\the\year-\the\month-\the\day}
\author{Kyle Chui}


\fancyhf{}
\lhead{Kyle Chui}
\rhead{Page \thepage}
\pagestyle{fancy}

\begin{document}
  \section{Electric Potential Energy and Voltage}
  You can store energy as electric potential energy and convert it back to kinetic energy on demand. To move a charge against the electric force generated by a capacitor, you need to perform work: $W = F\cdot d = (q\cdot E)\cdot d$. If you let the charge go, we have conservation of energy: $qEd = \frac{1}{2}mv^2$. \par
  All this is nice, but if you double the charge then you double the work and stored energy. To get a quantity that describes the energy storage capability of the field (that does not depend on $q$), we must define a new quantity. We say that \emph{electric potential} is defined by $V = \frac{U}{q}$. It is measured in $\frac{\mathrm{J}}{\mathrm{C}}$, or in volts ($\mathrm{V}$). For a homogeneous electric field, we have
  \[
    V = \frac{U}{q} = \frac{qEd}{q} = E\cdot d.
  \]
  \begin{note}{}
    Since we have $E = \frac{V}{d}$, we can measure electric field in $\frac{\mathrm{V}}{\mathrm{m}}$ and $\frac{\mathrm{N}}{\mathrm{C}}$.
  \end{note}
  \subsection{Gravity Analogue to Electric Potential}
  \begin{center}\begin{tabular}{c|c}
    Gravitational Potential Energy & Electric Potential Energy \\
    \hline
    $U = mgh$ & $U = qEd$
  \end{tabular}\end{center}
  \begin{itemize}
    \item Move a mass/charge against the field/force increases $U$ and $V$.
    \item Moving a mass/charge perpendicular to the field does not change $U$ and $V$.
    \item $V$ is a measure of how much energy a mass/charge would have at that point.
    \item Equipotential lines are perpendicular to the field. No work is done along those lines.
  \end{itemize}
  We define \emph{voltage} to be the potential difference between two points: $\Delta V = V_A - V_B$. Potential is a relative number but voltage is not. You must set the potential to zero somewhere (typically ground, one plate, or far away from a charge).
  \begin{note}{}
    Work performed by moving a charge $q$ in an electric field $E$ from point $A$ to point $B$ does not depend on the path taken. This is because both $F_g$ and $F_e$ are conservative forces.
  \end{note}
  \begin{definition}{Electric Potential Energy}
    We denote \emph{electric potential energy} by $U$ and it is given by
    \[
      U = W = -\int_{i}^{f}\vec{F} \dx \ell = -q \int_{i}^{f}\vec{E} \dx \ell.
    \]
  \end{definition}
  \begin{definition}{Electric Potential}
    We denote \emph{electric potential} by $V$ and it is given by
    \[
      V = -\int_{i}^{f}\vec{E} \dx \ell.
    \]
    It can also be found using $V = \frac{U}{q}$, but $V$ depends on the shape of the field.
  \end{definition}
  \begin{example}{Potential of a Point Charge} \\
    For a point charge (or sphere), the potential energy increases as you get closer to it. We can find the potential energy by integrating the force with respect to distance, yielding
    \begin{align*}
      W &= U \\
        &= -\int F\dx r \\
        &= -\int k\frac{Q_1Q_2}{r^2}\dx r \\
        &= \frac{kQ_1Q_2}{r} + C.
    \end{align*}
    Thus the electric potential is $V = \frac{W}{q} = k\frac{Q}{r} + C$. We define the constant of integration to be zero when the radius approaches infinity.
  \end{example}
  Inside a conducting sphere, the field is zero so the potential stays constant. Outside the sphere however, it behaves just like a point charge so we have $V = k\frac{q}{r}$. \par
  We draw equipotential lines to designate what paths have constant electric potential. They are always perpendicular to the electric field lines.
  \begin{note}{}
    $V$ is a scalar field and only depends on $r$, but $E$ is a vector field---the direction of it depends on the angle.
  \end{note}
  \subsection{General Rules for Potential}
  \begin{itemize}
    \item Moving a charge perpendicular to $E$ does not require work so $V$ does not change.
    \item $V$ increases as you move against $E$, and decreases as you move with $E$.
    \item If the field is zero then the potential remains constant.
    \item $V$ is only a relative number. We need to define where $V = 0$ (by grounding).
    \item A positive test charge wants to move to regions of smaller $V$, and a negative charge wants to move to regions of larger $V$.
  \end{itemize}
  \begin{example}{Mass Spectrometer} \\
    A mass spectrometer functions by ionizing particles and seeing how much they are deflected by a constant magnetic field (this will tell you the mass).
  \end{example}
  \subsection{Electric Potential In and Around a Conducting Cylinder}
  Outside of the charge, we have $\vec{E} = \frac{2k\lam}{\rho}\vec{a}_\rho$, where $\lam$ is the linear charge density. Inside the cylinder, the electric field is zero. Integrating along the radial path, we have
  \begin{align*}
    \Delta V &= -\int_{i}^{f}\vec{E}\cdot  \dx \vec{\rho} \\
             &= -2k\lam\int_{i}^{f}\frac{\mathrm{d}\rho}{\rho} \\
             &= -2k\lam\cdot \ln(\rho)|_{\rho_i}^{\rho_f} \\
             &= -2k\lam\cdot (\ln \rho_f - \ln\rho_i) \\
             &= \boxed{2k\lam\cdot \ln \paren{\frac{\rho_i}{\rho_f}}.}
  \end{align*}
  We set $V_i = 0$ at some arbitrary distance $\rho_0$. Then we have
  \[
    \boxed{V(\rho) = k\lam\cdot \ln \paren{\frac{\rho_0}{\rho}}.}
  \]
  \subsection{The Potential Gradient}
  We know that $\Delta V = -\int_{i}^{f}\vec{E}\cdot  \dx \vec{\ell}$, so we can find $E$ from $V$. For short distances $\Delta\ell$, it turns out that
  \[
    \Delta V = -\vec{E}\cdot \Delta\vec{\ell}.
  \]
  Thus $\Delta V = -E\cdot \Delta \ell\cdot \cos\theta$, so we have
  \[
    \boxed{\frac{\mathrm{d}V}{\mathrm{d}\ell} = -E\cdot \cos\theta.}
  \]
  In general for three-dimensions, we have
  \[
    \vec{\nab}V = -\vec{E}.
  \]
  \subsection{The Electric Field of a Dipole}
  We apply the principle of superposition. Thus the potential at point $P$ is 
  \[
    V = kQ \paren{\frac{1}{R_1}-\frac{1}{R_2}} = kQ \paren{\frac{R_2-R_1}{R_1R_2}}.
  \]
  If $P$ is very very far away from the dipole ($R \gg d$), then
  \[
    R_2 - R_1 = d\cdot \cos\theta.
  \]
  Thus we have
  \[
    V = \frac{kQd\cos\theta}{R_1R_2} = \frac{kQd\cos\theta}{r^2}.
  \]
  Plugging this into the formula we found in the earlier section, we have
  \[
    \vec{E} = \underbrace{\frac{kQd}{r^3}}_{\text{magnitude}}\underbrace{(2\cos\theta\cdot \vec{a}_r+\sin\theta\cdot \vec{a}_\theta)}_{\text{direction}}
  \]
  From this, we can see that $E\sim \frac{1}{r^3}$.
  \newpage
  \section{Capacitors}
  \begin{definition}{Capacitor}
    Any two conductors separated by an insulator (or vacuum) for a \emph{capacitor}.
  \end{definition}
  Charges can be stored indefinitely on the conductors, and can be discharged quickly. This is in contrast to batteries, which provide a consistent, steady flow of charges.
  \begin{definition}{Capacitance}
    \emph{Capacitance} is a measure of how much charge a capacitor can store for a given voltage, and is given by the equation
    \[
      C = \frac{Q}{V}.
    \]
    \begin{itemize}
      \item Capacitance is measured in Coulombs per Volt or Farads (F).
      \item $1$ F is huge capacitance.
      \item $Q$ (on the plates) causes $E$ (between the plates), which causes $V$. So $Q\sim E\sim V$.
      \item The actual ratio of $\frac{Q}{V}$ depends only on the shape and material because $Q$ and $V$ are proportional to each other and cancel out. In other words, capacitance depends only on the shape/material of the capacitor.
    \end{itemize}
  \end{definition}
  \subsection{Parallel Plate Capacitor}
  The electric field for a parallel plate capacitor is $E = \frac{\sigma}{\eps_0} = \frac{Q}{\eps_0A}$. We know that
  \[
    V = E\cdot d = \frac{Qd}{\eps_0A},
  \]
  so 
  \[
    \boxed{C = \frac{Q}{V} = \frac{\eps_0A}{d}.}
  \]
  \subsection{Energy Stored in a Plane Capacitor}
  We know that the charge $q$ placed in an electric potential $V$ stores energy: $U = q\cdot V$, so
  \[
    U_C = \int_{0}^{Q}V \dx q = \int_{0}^{Q}\frac{q}{C} \dx q = \frac{1}{C}\int_{0}^{Q}q \dx q = \frac{1}{2}\frac{Q^2}{C}
  \]
  Thus we have
  \[
    \boxed{U_C = \frac{1}{2}CV^2 = \frac{1}{2}QV = \frac{1}{2}\frac{Q^2}{C}.}
  \]
  \subsection{Spherical Capacitor}
  Solving for the voltage difference between two points, we have
  \[
    \Delta V = kQ \paren{\frac{1}{r_a} - \frac{1}{r_b}} = kQ \paren{\frac{r_b - r_a}{r_ar_b}}.
  \]
  Thus the capacitance is
  \[
    C = \frac{Q}{V} = \frac{1}{k}\frac{r_ar_b}{r_b-r_a}
  \]
  \subsection{Cylindrical Capacitor}
  From earlier, we found that the potential difference for a cylinder is
  \[
    \Delta V = 2k\lam\ln \paren{\frac{r_b}{r_a}}.
  \]
  Thus we have
  \[
    C = \frac{Q}{\Delta V} = \frac{\lam L}{2k\lam\ln \paren{\frac{r_b}{r_a}}} = \frac{2\pi \eps_0 L}{\ln \paren{\frac{r_b}{r_a}}}.
  \]
  In other words, we may write
  \[
    \frac{C}{L} = \frac{2\pi\eps_0}{\ln \paren{\frac{r_b}{r_a}}}.
  \]
  \subsection{Capacitors in Series or Parallel}
  \begin{itemize}
    \item Capacitors can be combined in ``series" or ``parallel".
    \item All capacitors then act as one single capacitor with an ``effective capacitance".
    \item Adding capacitors in parallel makes a larger capacitor.
    \item Adding capacitors in series allows you to operate at much higher voltages.
  \end{itemize}
  \textbf{Series Capacitors.} In series, the voltages add up, and the charge is constant across the capacitors. We write $\frac{1}{C_{\text{eq}}} = \sum \frac{1}{C_i}$. Thus the equivalent capacitance is smaller than all of the capacitances of the components. Each capacitor only sees a fraction of the total voltage. Putting capacitors in series essentially increases $d$, while $A$ remains constant. \\[10pt]
  \textbf{Parallel Capacitors.} When in parallel, the capacitance adds up linearly, and the voltage is constant across the capacitors. We write $C_{\text{eq}} = \sum C_i$. The equivalent capacitance is larger than all of the capacitances of the components. Each capacitor only sees a fraction of the total charge. Putting capacitors in parallel essentially increases $A$, while $d$ remains constant.
  \subsection{Dielectrics}
  Most real capacitors have an insulator (dielectric) between the plates, which stores additonal energy via electric dipole alignment. This increased breakdown threshold allows for higher voltages between the plates.
  \begin{definition}{Dielectric Constant}
    The \emph{dielectric constant} $K$ is the ratio of the capacitances before and after you put a dielectric in it. It is given by $K = \frac{C}{C_0}$, and is always greater than or equal to one.
  \end{definition}
  \begin{note}{}
    When you align dipoles inside an electric field, they line up anti-parallel to the original field $E_0$ and create a new, weaker field $E$ that is in the opposite direction of the original. The magnitude of this field is $E = \frac{E_0}{K}$, where $K$ is the dielectric constant. More energy is now stored in the aligned dipoles. 
  \end{note}
  When you fill a capacitor with a dielectric, the opposing electric field decreases the overall electric field. Let the induced charge inside a capacitor be denoted by $\sigma_i$. Then we have that 
  \[
    \sigma_i = \sigma \paren{1 - \frac{1}{K}}.
  \]
  So if $K$ is very large, then $\sigma_i$ is almost as large as $\sigma$. Furthermore, 
  \[
    E = \frac{1}{\eps_0}(\sigma - \sigma_i) = \frac{1}{\eps_0}\paren{\sigma -\sigma + \frac{\sigma}{K}} = \frac{\sigma}{\eps_0K}.
  \]
  \begin{definition}{Permittivity}
    The product $\eps = K\eps_0$ is called the \emph{permittivity} of the dielectric.
  \end{definition}
  Thus we can write that the field $E$ is given by $E = \frac{\sigma}{\eps}$.
  \subsection{Summary}
  For constant charge on a capacitor, adding a dielectric with constant $K$ between the plates will:
  \begin{itemize}
    \item Decrease the electric field by a factor of $K$.
    \item Decrease the voltage between the plates because $V = Ed$.
    \item Increase the capacitance by a factor of $K$ because $C = \frac{Q}{V}$.
  \end{itemize}
  Aligning dipoles in a dielectric increases the energy density:
  \[
    u = \frac{1}{2}K\eps_0 E^2 = \frac{1}{2}\eps E^2.
  \]
  \begin{theorem}{General Form of Gauss's Law}
    If we have charges inside a dielectric, we modify Gauss's Law to be
    \[
      \oint \vec{E}\dx \vec{A} = \frac{Q_{\text{enclosed}-\text{free}}}{K\cdot \eps_0} = \frac{Q_{\text{enclosed}-\text{free}}}{\eps}.
    \]
    We ignore the induced dipole charges when evaluating the formula.
  \end{theorem}
  \section{Current, Resistance, and Electromotive Force}
  We can have an electric field in a conductor for a split second, but the free electrons will quickly move to the end of the conductor and cancel it out. We bypass this by connecting the conductor to a battery in a circuit, which will keep the charges flowing in a steady ``current".
  \begin{definition}{Current}
    When we have moving charges, we have \emph{current}. It is caused by a potential difference. We define it to be the amount of charge $Q$ that passes in a time $t$. Thus $I = \frac{\mathrm{d}Q}{\mathrm{d}t}$, which is measured in Coulombs per second or Amperes. It flows from $+$ to $-$, which is the same way that positive charges would flow.
  \end{definition}
  The \emph{conventional current} is treated as a flow of positive charges, regardless of what's actually happening. \\[10pt]
  In a conductor with length $L$, the electrons experience a constant acceleration due to the electric field. As they speed up, they randomly collide with fixed ions---on average, they travel at a constant velocity. This velocity, known as the electron drift velocity, is given by 
  \[
    v_{\text{drift}} = -\mu_e\cdot E,
  \]
  where $\mu_e$ is the mobility of an electron, which depends on the material. For a typical copper wire, this drift velocity is about $10^{-5}$ $\frac{\text{m}}{\text{s}}$. \\[10pt]
  The total charge in a volume $V$ in a conductor is
  \[
    Q = -e \cdot n_e\cdot \underbrace{A\cdot L}_{\text{volume}},
  \]
  where $n_e$ is the electron density of the material (number of electrons per volume). Furthermore, the time it takes for this charge to move out of the volume $V$ is given by $t = \frac{L}{v_{\text{drift}}}$, so
  \[
    I = \frac{Q}{t} = \frac{-en_eAL}{\frac{L}{v_{\text{drift}}}} = -en_eA\cdot v_{\text{drift}} = e\cdot n_e\cdot A\cdot \mu_e\cdot E.
  \]
  In other words,
  \[
    I = \underbrace{en_e\mu_e}_{\mathclap{\text{material constant: conductivity $\sigma$}}}\cdot \frac{A}{L}\cdot V
  \]
  \begin{theorem}{Ohm's Law}
    We have a relationship between voltage, current, and resistance given by:
    \begin{align*}
      I &= \frac{V}{R} \\
      R &= \frac{V}{I} \\
      V &= IR
    \end{align*}
  \end{theorem}
  \subsection{Resistivity}
  The resistance $R$ is measured in Volts per Ampere = Ohm = $\Omega$. To calculate the resistance $R$ of a wire, we have
  \[
    \boxed{R = \rho \frac{L}{A}.}
  \]
  The resistance $R$ depends on the material \emph{and} shape. Resistivity is a material constant, given by $\rho = \frac{1}{\sigma}$. \\[10pt]
  The smaller the resistivity the less voltage you need to drive a certain current (if the geometry of the wire stays the same). The resistivity of a material also tells you if the material is a conductor or an insulator. If a material has a resistivity of less than $10^{-8}\,\Omega \mathrm{m}$, then it is a conductor. If it has a resistivity of more than $10^{16}\,\Omega \mathrm{m}$, then it is an insulator.
  If electron flow depends on space or is not confined to a well defined wire it is more useful to define a vector current density.
  \begin{definition}{Current Density}
    The \emph{current density} measures the amount of current passing through a given area, and is given by
    \[
      J = \frac{I}{A} = en_e\cdot v_{\text{drift}},
    \]
    and has units of $\frac{\text{A}}{\text{m}^2}$.
  \end{definition}
  From before, we knew that
  \begin{align*}
    I &= \underbrace{e\cdot n_e\cdot \mu_e}_{\sigma}\cdot \frac{A}{L}\cdot V, \quad\text{so} \\ 
    \frac{I}{A} &= \sigma \frac{V}{L} \\
    J &= \sigma\cdot E \\
    \vec{J} &= \sigma\cdot \vec{E}
  \end{align*}
  \begin{definition}{Ohmic and Nonohmic resistors}
    An \emph{ohmic} resistor (e.g. a typical metal wire) is a resistor where the current is proportional to the voltage, at a given tempterature. For a \emph{nonohmic} resistor, the current scales nonlinearly with the voltage.
  \end{definition}
  The resistivity of a material depends on the temperature:
  \[
    \rho(T) = \rho_0 \paren{1 + \alpha(T - T_0)}.
  \]
  \begin{note}{}
    It should make sense that resistivity goes up with temperature for conductors, because collisions slow electrons down.
  \end{note}
  \subsection{Electric Power}
  Power is measured in Watts, and electric power is given by
  \[
    P = \frac{U}{t} = \frac{Q\cdot V}{t} = \frac{Q}{t}\cdot V = I\cdot V.
  \]
  Using Ohm's law, we also have
  \[
    P = I\cdot V = \frac{V^2}{R} = I^2R.
  \]
  \begin{note}{}
    Unless otherwise stated, assume lightbulbs operate at $120$ Volts.
  \end{note}
  \begin{theorem}{Junction Rule for Parallel Circuits}
    Since \emph{charge is conserved}, whatever current flows into a junction must also flow out on the other side, so
    \[
      I_{\text{total}} = I_1 + I_2 + I_3 + \dotsb
    \]
    In other words, $\sum I = 0$ (the sum of currents entering and leaving a junction).
  \end{theorem}
  \textbf{Resistors in Parallel}
  \begin{itemize}
    \item Each resistor gets the full voltage
    \item The current is split amongst the resistors
    \item The resistors are independent of each other---remove one and it will still function (although differently)
    \item $\frac{1}{R_{\text{total}}} = \sum_{i=1}^{n} \frac{1}{R_i}$
  \end{itemize}
  \begin{note}{}
    When we talk about which lightbulb is brighter, we say that the bulb with the higher power dissipation is brighter.
  \end{note}
  \textbf{Resistors in Series}
  \begin{itemize}
    \item Each resistor only sees a fraction of the total voltage
    \item The bulbs are \emph{dependent} on each other---remove one and the circuit breaks
    \item Each bulb receives the same current
    \item $R_{\text{total}} = \sum_{i=1}^{n}R_i$
  \end{itemize}
  \begin{theorem}{Loop Rule for Series Circuits}
    In any closed circuit loop the voltages must cancel out:
    \[
      V - V_1 - \dotsb - V_n = 0.
    \]
    A battery raises the potential by $V$, and a resistor decreases it by $V$ (in the direction of $I$). In other words, $\sum V = 0$.
  \end{theorem}
  \subsection{How to Analyze an Electric Circuit}
  \begin{enumerate}
    \item Redraw the circuit so you can clearly see series and parallel connections.
    \item Find the equivalent resistance of the circuit.
    \item Calculate the total current.
    \item Find the voltage drop across series resistors.
    \item Use the loop rule to find the voltage drop across any remaining resistors.
    \item Find the current and power through each resistor as needed.
  \end{enumerate}
  \subsection{Internal Resistance of a Battery}
  \begin{itemize}
    \item Any battery has an internal resistance $r$.
    \item The resistance $r$ depends on the chemistry, size, etc.
    \item You could treat $r$ as any other external resistance (in series).
    \item The smaller the $r$, the higher the currents a battery can drive.
    \item Adding batteries in parallel reduces the internal resistance.
  \end{itemize}
  When measuring the current and voltage in a circuit, we don't want out instruments to affect the readings at all. Thus when we put an ammeter inline to measure current, it should have as little resistance as possible. When we measure voltage using a voltmeter, it should have near infinite resistance to prevent current from flowing through it. \\[10pt]
  In an ammeter, the shunt is placed in the circuit, and we have
  \[
    I_{fs}\cdot R_c = (I - I_{fs})\cdot R_s.
  \]
  In a voltmeter, the coil resistance is very small, so we need a much larger ``shunt resistance'' in order to divert as little current as possible from the circuit. We can find the voltage drop in a circuit via
  \[
    V = I_{fs}(R_c + R_s),
  \]
  where $I_{\text{fs}}$ is the full scale current, $R_c$ is the coil resistance, and $R_s$ is the shunt resistance.
\end{document}
