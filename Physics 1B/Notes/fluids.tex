\documentclass[class=article, crop=false]{standalone}
\usepackage[margin=1in]{geometry}

\usepackage{amsmath}
\usepackage{amssymb}
\usepackage{amsthm}
\usepackage{comment}
\usepackage{enumitem}
\usepackage{fancyhdr}
\usepackage{hyperref}
\usepackage{mathrsfs}
\usepackage{mathtools}
\usepackage{standalone}
\usepackage{tcolorbox}
\usepackage{tikz}
\usepackage{xcolor}

\usetikzlibrary{decorations.pathreplacing}
\tcbuselibrary{skins}

\DeclareMathOperator{\lcm}{lcm}
\DeclareMathOperator{\proj}{proj}
\DeclareMathOperator{\vspan}{span}
\DeclareMathOperator{\im}{im}
\DeclareMathOperator{\range}{range}
\DeclareMathOperator{\Diff}{Diff}
\DeclareMathOperator{\Int}{Int}
\DeclareMathOperator{\fcn}{fcn}
\DeclareMathOperator{\id}{id}

\newcommand{\N}{\ensuremath{\mathbb{N}}}
\newcommand{\Z}{\ensuremath{\mathbb{Z}}}
\newcommand{\Q}{\ensuremath{\mathbb{Q}}}
\newcommand{\R}{\ensuremath{\mathbb{R}}}
\newcommand{\C}{\ensuremath{\mathbb{C}}}
\newcommand{\F}{\ensuremath{\mathbb{F}}}
\newcommand{\lam}{\ensuremath{\lambda}}
\newcommand{\nab}{\ensuremath{\nabla}}
\newcommand{\eps}{\ensuremath{\varepsilon}}
\newcommand{\es}{\ensuremath{\varnothing}}

\newcommand{\abs}[1]{\ensuremath{\left\lvert #1 \right\rvert}}
\newcommand{\bigpar}[1]{\ensuremath{\left( #1 \right)}}
\newcommand{\norm}[1]{\ensuremath{\lVert #1\rVert}}
\newcommand{\set}[1]{\ensuremath{\left\{#1\right\}}}
\newcommand{\tuple}[1]{\ensuremath{\left\langle #1 \right\rangle}}
\newcommand{\floor}[1]{\ensuremath{\left\lfloor #1 \right\rfloor}}
\newcommand{\ceil}[1]{\ensuremath{\left\lceil #1 \right\rceil}}

\newcommand{\chapternum}{}
\newcommand{\ex}[1]{\noindent\textbf{Exercise \chapternum.{#1}.}}

\newcommand{\dx}[1]{\,\mathrm{d}#1}
\newcommand{\inv}{\ensuremath{^{-1}}}
\newcommand{\sm}{\setminus}
\newcommand{\sse}{\subseteq}
\newcommand{\ceq}{\coloneqq}

\definecolor{problemBackground}{RGB}{212,232,246}
\newenvironment{problem}[1]
  {
    \begin{tcolorbox}[
      boxrule=.5pt,
      titlerule=.5pt,
      sharp corners,
      colback=problemBackground
    ]
    \ifx &#1& \textbf{Problem. }
    \else \textbf{Problem #1.} \fi
  }
  {
    \end{tcolorbox}
  }

\definecolor{exampleBackground}{RGB}{255,249,248}
\definecolor{exampleAccent}{RGB}{158,60,14}
\newenvironment{example}[1]
  {
    \begin{tcolorbox}[
      boxrule=.5pt,
      sharp corners,
      colback=exampleBackground,
      colframe=exampleAccent,
    ]
    \color{exampleAccent}\textbf{Example.} \emph{#1}\color{black}
  }
  {
    \end{tcolorbox}
  }

\definecolor{theoremBackground}{RGB}{234,243,251}
\definecolor{theoremAccent}{RGB}{0,116,183}
\newenvironment{theorem}[1]
  {
    \begin{tcolorbox}[
      boxrule=.5pt,
      titlerule=.5pt,
      sharp corners,
      colback=theoremBackground,
      colframe=theoremAccent
    ]
      \color{theoremAccent}\textbf{Theorem --- }\emph{#1}\\\color{black}
  }
  {
    \end{tcolorbox}
  }

\definecolor{noteBackground}{RGB}{244,249,244}
\definecolor{noteAccent}{RGB}{34,139,34}
\newenvironment{note}[1]
  {
  \begin{tcolorbox}[
    enhanced,
    boxrule=0pt,
    frame hidden,
    sharp corners,
    colback=noteBackground,
    borderline west={3pt}{-1.5pt}{noteAccent}
    ]
    \ifx &#1& \color{noteAccent}\textbf{Note. }\color{black}
    \else \color{noteAccent}\textbf{Note (#1). }\color{black} \fi
    }
    {
  \end{tcolorbox}
  }

\definecolor{definitionBackground}{RGB}{246,246,246}
\newenvironment{definition}[1]
  {
    \begin{tcolorbox}[
      enhanced,
      boxrule=0pt,
      frame hidden,
      sharp corners,
      colback=definitionBackground,
      borderline west={3pt}{-1.5pt}{black}
    ]
    \textbf{Definition. }\emph{#1}\\
  }
  {
    \end{tcolorbox}
  }
\newenvironment{amatrix}[2]{
    \left[
      \begin{array}{*{#1}{c}|*{#2}c}
  }
  {
      \end{array}
    \right]
  }
\date{\the\year-\the\month-\the\day}
\author{Kyle Chui}


\fancyhf{}
\lhead{Kyle Chui}
\rhead{Page \thepage}
\pagestyle{fancy}

\begin{document}
  \section{Fluids}
  \subsection{Basic Definitions}
  The density of a material is a constant and is given by $\rho = \frac{m}{V}$. \\[10pt]
  Pressure is the average force per area and is given by $p=\frac{F}{A}$. The hydrostatic pressure due to the weight of fluid is $p=\rho gh$. The \emph{gauge pressure} is basically just relative pressure. Pressure is a scalar and has no direction. \\[10pt]
  Pascal's Law states that a pressure change at any point in a confined incompressible fluid is transmitted throughout the fluid such that the same change occurs everywhere.
  \subsection{Archimedes' Principle}
  The buoyant force on an object immersed in a fluid is equal to the weight of the displaced fluid. It is given by the equation $F_{\text{buoyant}} = \rho V_{\text{disp}}g$.
  \begin{note}{Special Cases}
    If an object is floating at the surface of a fluid, not all of its volume is displacing fluid, so the buoyant force is less than you think it is. Moreover, if an object has sunken to the bottom of a fluid and is resting on the floor, don't forget to take into account the normal force on the object.
  \end{note}
  Archimedes' Principle can be explained from the existence of a pressure difference between the top and bottom of an object.
  \subsection{Fluid Dynamics}
  The continuity equation comes from the simple idea that ``what comes in must go out". In other words, the volume of fluid passing through a tube is constant, regardless of where in the tube you are. Thus we have $A_1v_1 = Q = A_2v_2$. \\[10pt]
  Poiseuille's Law explains why there is flow in a pipe to begin with, and is given by 
  \[
    Q = \frac{\pi}{8\eta}r^4\frac{\Delta p}{L},
  \]
  where $\eta$ is the viscosity, $r$ is the radius of the pipe, $\Delta p$ is the pressure difference over the pipe, and $L$ is the length of the pipe. \\[10pt]
  Bernoulli's Law is basically just energy conservation, and helps us find local pressure changes in moving fluids. It is given by:
  \[
    P_0+\frac{1}{2}\rho v_0^2 + \rho gh_0 =P_1 + \frac{1}{2}\rho v_1^2 + \rho gh_1
  \]
  A special case of this is Toricelli's law, which says that for an open tank, the speed of liquid coming out of the hole a distance $h$ below the surface is equal to that acquired by an object falling freely through the same distance $h$. By using conservation energy, we find $v=\sqrt{2gh}$.
  \begin{note}{}
    If the height of a fluid is constant, then by Bernoulli's Law, we have that faster fluids have less pressure.
  \end{note}
  \subsection{Laminar and Turbulent Flows}
  A dimensionless parameter, called Reynold's number $N_r$ reveals whether a flow is laminar or turbulent. We calculate this by $N_r = \frac{2\rho vr}{\eta}$. If $N_r < 2000$, then the flow is laminar. If $N_r > 3000$, then the flow is turbulent. If it is anywhere in between, then it is unstable. Reynold's number is also the ratio of inertial forces to viscous forces (think: high ratio = more inertia, low ratio = high viscosity). A force $F$ is needed to keep a fluid moving at a constant velocity, given by $F = \eta \frac{vA}{L}$. 
\end{document}
